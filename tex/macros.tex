% --- Macros ---

% Blank line
\newcommand{\blankline}{\vspace{\baselineskip}}


% Intervals
\newcommand{\interval}[2]{\ensuremath{[ #1 \,, #2 ]}}


% Norm and absolute value
\newcommand{\norm}[1]{\ensuremath{\left\Vert#1\right\Vert}}
\newcommand{\abs}[1]{\ensuremath{\left\vert#1\right\vert}}


% Dot product and cross product
\newcommand{\dynamicvert}[1]{\,\left\vert\vphantom{#1}\right.}
\newcommand{\dotproduct}[2]{\ensuremath{\left\langle\,#1\,\middle\vert\,#2\,\right\rangle}}
\newcommand{\crossproduct}[2]{\ensuremath{#1\times#2}}



\newcommand{\order}[1]{\mathcal{O}\left(#1\right)}

\newcommand{\expectation}[1]{\mathbb{E}\left[#1\right]}
\newcommand{\deviation}[1]{\mathbb{D}\left[#1\right]}
\newcommand{\variance}[1]{\mathbb{D}^2\left[#1\right]}




% argmin
\DeclareMathOperator*{\argmin}{arg\,min}
\DeclareMathOperator*{\argmax}{arg\,max}


% sign
\newcommand{\sign}[1]{\mathrm{sign}\left(#1\right)}


% XOR
\newcommand{\xor}{\mathbin{\oplus}}


% Functions with brackets
\renewcommand{\min}[1]{\mathrm{min}\left(#1\right)}
\renewcommand{\max}[1]{\mathrm{max}\left(#1\right)}
\renewcommand{\tan}[1]{\mathrm{tan}\left(#1\right)}
\newcommand{\logarithm}[2]{\mathrm{log}_{#1}\left(#2\right)}

\newcommand{\nn}[2][]{\underset{#1}{\mathrm{nn}}\left(#2\right)}

\newcommand{\ME}[1]{\mathrm{ME}\left(#1\right)}
\newcommand{\RMSE}[1]{\mathrm{RMSE}\left(#1\right)}
\newcommand{\NME}[1]{\mathrm{NME}\left(#1\right)}
\newcommand{\NRMSE}[1]{\mathrm{NRMSE}\left(#1\right)}
\newcommand{\PSNR}[1]{\mathrm{PSNR}\left(#1\right)}
\newcommand{\BB}[1]{\mathrm{BB}\left(#1\right)}


% Plus-equal
\newcommand{\pluseq}{\mathrel{+}=}


% Tilde in math mode
\newcommand{\mathtilde}{{\raise.17ex\hbox{$\scriptstyle\mathtt{\sim}$}}}



% Alignment in algorithms
\newcommand{\alignto}[2]{\mathrlap{#2}\phantom{#1}}





% Mathematical blocks
% \newtheorem{lemma}{Lemma}[chapter]
% \newtheorem{theorem}{Theorem}[chapter]
% \newtheorem{definition}{Definition}[chapter]
% \newtheorem{corollary}{Corollary}[chapter]


% Gradient, Divergenz and Laplace
\DeclareMathOperator{\gradient}{\nabla}
\DeclareMathOperator{\divergence}{div}
\DeclareMathOperator{\laplacian}{\Delta}


% Vector with an arrow above
\let\oldvec\vec
\newcommand{\vecarrow}[1]{\oldvec{#1}}

% Vector and Matrix macros		
\renewcommand{\vec}[1]{\bm{#1}}
\newcommand{\mat}[1]{\bm{#1}}
\newcommand{\set}[1]{\mathcal{#1}}
\newcommand{\neighborhood}[1]{\mathcal{N}(#1)}

% More intutive macros for set operations
\newcommand{\intersect}[0]{\cap}
\newcommand{\union}[0]{\cup}
\newcommand{\difference}[0]{\,\backslash\,}

\newcommand{\bigintersect}[0]{\bigcap}
\newcommand{\bigunion}[0]{\bigcup}

% Probability
\newcommand{\probability}[1]{\mathrm{Pr}(#1)}
\newcommand{\probabilitygiven}[2]{\mathrm{Pr}(#1\mid#2)}


\newcommand{\transposed}{\top}


\newcommand{\evalat}[2]{\left.\kern-\nulldelimiterspace#1\right|_{\substack{#2}}}


\newcommand{\const}{\mathrm{const}}


% Table stuff
\newcolumntype{L}[1]{>{\raggedright\arraybackslash}m{#1}}
\newcolumntype{C}[1]{>{\centering\arraybackslash}m{#1}}
\newcolumntype{R}[1]{>{\raggedleft\arraybackslash}m{#1}}

%%%%%%%%%%%%%%%%%%%%%%%
%%%%%% My Macros %%%%%%

% Flip metric
\usepackage{mathtools}
\usepackage{xspace}
\newcommand{\FLIP}{\protect\reflectbox{F}LIP\xspace}

% Estimator with angle brackets
\newcommand{\estimator}[1]{\ensuremath{\left\langle #1 \right\rangle_N}}

% Equalities
\newcommand{\hateq}{\mathrel{\widehat{=}}}
\newcommand\eqhat{\mathrel{\stackon[1.5pt]{=}{\stretchto{\scalerel*[\widthof{=}]{\wedge}{\rule{1ex}{3ex}}}{0.5ex}}}}
\newcommand{\defeq}{\mathrel{\mathop:}=}

\newcommand{\diff}{\mathop{}\!d}

\newcommand{\expectationvar}[2]{\mathbb{E}_{#1}\left[#2\right]}

\newcommand{\x}{\vec{x}}
\newcommand{\wo}{{\vec{\omega}_o}}
\newcommand{\wi}{{\vec{\omega}_i}}
\newcommand{\wm}{\vec{\omega}_m}
\newcommand{\n}{\vec{n}}
\newcommand{\NdotX}{\dotproduct{\n}{\vec{\omega}}}
\newcommand{\NdotV}{\dotproduct{\n}{\wo}}
\newcommand{\NdotL}{\dotproduct{\n}{\wi}}
\newcommand{\NdotH}{\dotproduct{\n}{\wm}}
\newcommand{\LdotH}{\dotproduct{\wm}{\wi}}
\newcommand{\VdotH}{\dotproduct{\wm}{\wo}}
\DeclareMathOperator{\mix}{mix}
\DeclareMathOperator{\reflect}{reflect}
\DeclareMathOperator{\refract}{refract}

\DeclareMathOperator{\supp}{\mathrm{supp}}

% Short arrow from https://tex.stackexchange.com/a/395014
\newcommand{\veryshortarrow}[1][3pt]{\mathrel{%
   \hbox{\rule[\dimexpr\fontdimen22\textfont2-.2pt\relax]{#1}{.4pt}}%
   \mkern-4mu\hbox{\usefont{U}{lasy}{m}{n}\symbol{41}}}}

\newcommand{\pto}{\mathrel{\mkern-1mu\veryshortarrow[5pt]\mkern-1mu}}
\newcommand{\ptrip}[3]{\vec{x}_{#1} \pto \vec{x}_{#2} \pto \vec{x}_{#3}}
\newcommand{\pdir}[2]{\vec{x}_{#1} \pto \vec{x}_{#2}}
\newcommand{\pbidir}[2]{\vec{x}_{#1} \leftrightarrow  \vec{x}_{#2}}

\newcommand{\G}[2]{G(\pbidir{#1}{#2})}
\newcommand{\f}[3]{f(\ptrip{#1}{#2}{#3})}

% \newcommand{\G}[2]{G_{#1 \leftrightarrow #2}}
% \newcommand{\f}[3]{f_{#1 \veryshortarrow #2 \veryshortarrow #3}}