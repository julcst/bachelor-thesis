
\chapter{Related Work}
\label{chap:related}
The research related to this thesis can be divided into three main categories:
First, I will give a short overview over several relevant approaches to the global illumination problem.
After that, I will specifically discuss methods that reuse computed radiance estimates across frames or space.
Finally, I will take a closer look into papers that apply neural networks to rendering.
For a more in-depth overview over the field I refer to \textcite{ritschel2012,kang2016}.

\section{Global Illumination}
In the early days of rendering, lighting was limited to direct illumination and ray traced reflections \bcite{whitted1980} and indirect illumination was often hand-crafted \bcite{christensen2016}.

\paragraph{Finite Element Methods}
\textcite{goral1984} were the first to also compute indirect illumination, which they achieved by dividing the scene into finite elements and solving the resulting system of equations.
Yet, their approach was limited to diffuse surfaces, though later papers extended this idea also to non-diffuse surfaces \bcite{immel1986a}.
However, the separation of the scene into finite elements generally results in visible bias, especially for complex geometry.

\paragraph{Monte Carlo Methods}
The first universal unbiased algorithm to solve the global illumination problem was given by \textcite{kajiya1986}, who formulated a radiance estimator based on Monte Carlo integration.
Simultaneously, \textcite{arvo1986} discovered the potential of light tracing to simulate indirect lighting effects.
\textcite{lafortune1993} later combined light tracing and pathtracing into Bidirectional Pathtracing (BDPT), and \textcite{veach1997} introduced provably optimal weighting strategies to maximize convergence speed.
\textcite{keller1995,owen1995} improved convergence of Monte-Carlo samplers in general by using low discrepancy samples.
Furthermore, \textcite{veach1997a} also proposed a new mutation based sampling strategy which exceeds in finding low-probability high-contribution paths which can cause significant noise in classic Monte Carlo methods.
In addition, Manifold Exploration techniques were developed which excel at discovering highly specular paths \bcite{jakob2012}.

\paragraph{Photon Mapping}
Simultaneously, a number of generally biased radiance estimators emerged that are based on storing outgoing flux \textit{(photons)} in spatial data structures \bcite{jensen1996,kang2016}.
The original estimator of \textcite{jensen1996} was made consistent by \textcite{hachisuka2008,knaus2011} and extended by \textcite{hachisuka2009a} to also handle glossy surfaces efficiently.
\textcite{georgiev2012} further improved the quality by integrating bidirectional path tracing and shadow tests into photon mapping.

\paragraph{Real-time Photon Mapping}
As the original method for photon mapping \bcite{jensen1996} used highly incoherent k-d-tree traversal to perform Fixed-Radius-Near-Neighbor (FRNN) searches, it was not well suited for GPU execution.
To solve this, \textcite{hachisuka2010} proposed to stochastically store photons in a fixed-size hash grid, which improves parallelization and reduces memory access, yet increases noise.
\textcite{mara2013} later published four further GPU-optimized photon mapping algorithms.
However, with the recent advancements in GPU architectures, traversal of bounding volume hierarchies can be hardware accelerated.
\textcite{evangelou2021} successfully leveraged these modern hardware features to perform fast FRNN queries, leading \textcite{kern2023} to apply this to photon mapping and together with culling and stochastic rejection they achieve real-time performance.

\paragraph{Path Guiding}
Alternatively, to obtain unbiased estimators, photon maps \bcite{jensen1996} can also be used only to \textit{guide} path sampling \bcite{jensen1995}.
\textcite{vorba2014} later improved upon this idea by learning Gaussian Mixture Models.

\paragraph{Reservoir Sampling}
The most recent advancements in path tracing are based on spatio-temporal path reuse through reservoir sampling \bcite{bitterli2020} and already achieve acceptable noise at close to real-time performance.
Although the original paper used reservoir sampling only for direct illumination \bcite{bitterli2020}, the idea was recently extended among others to indirect lighting \bcite{ouyang2021}, full path tracing \bcite{lin2022}, photon mapping \bcite{kern2024} and bidirectional path tracing \bcite{hedstrom2025}.

\section{Radiance Caching}
\paragraph{Radiance Caching}
\textcite{ward1988} lay the groundwork for radiance caching with their seminal paper on the topic.

\section{Neural Rendering}