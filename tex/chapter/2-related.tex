
\chapter{Related Work}
\label{chap:related}
The research related to this thesis can be divided into three main categories:
First, I will give a short overview over approaches to the global illumination problem.
Secondly, I will specifically discuss approaches that reuse computed radiance estimates across frames or space.
Finally, I will take a look into approaches that introduce neural networks into rendering.
For a more complete overview over the field I refer to \textcite{ritschel2012,kang2016}.

\section{Global Illumination}
In the early days of rendering, lighting was limited to direct illumination and ray traced reflections \bcite{whitted1980} and indirect illumination was often hand-crafted \bcite{christensen2016}.
\textcite{goral1984} later also introduced indirect illumination based on finite elements, but their approach was limited to diffuse surfaces.

The first universal algorithm to solve the global illumination problem was given by \textcite{kajiya1986}, who formulated an unbiased estimator based on Monte Carlo integration.
Simultaneously, \textcite{arvo1986} discovered the potential of light tracing to simulate indirect lighting effects.
\textcite{lafortune1993} later combined light tracing and pathtracing, and \textcite{veach1997} introduced optimal weighting strategies to maximize convergence speed.
\textcite{keller1995,owen1995} improved convergence for Monte-Carlo samplers by using low discrepancy samples.
Furthermore, \textcite{veach1997} also proposed a new mutation based sampling strategy which exceeds in finding low-probability high-contribution paths which can cause significant noise.
In addition, Manifold Exploration techniques were developed which excel at highly specular paths \bcite{manifold}.

Simultaneously, a number of generally biased radiance estimators emerged that are based on storing outgoing flux \textit{(photons)} in spatial data structures \bcite{jensen1996,hachisuka2008,hachisuka2009a,knaus2011,kang2016,georgiev2012}.
Alternatively, to obtain unbiased estimators, these spatial data structures can also be used to guide path sampling \bcite{jensen1995}.

The most recent advancements in path tracing are based on spatio-temporal path reuse through reservoir sampling \bcite{bitterli2020,kern2024,hedstrom2025} and already achieve acceptable noise at close to real-time performance.

\section{Radiance Caching}
The groundstone of radiance cache was the seminal paper of \textcite{ward1988}.

\section{Neural Rendering}