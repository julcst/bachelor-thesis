
\chapter{Results}
\label{chap:results}

\section{Methodology}
For the purpose of reproducible evaluation, a JSON-configurable CLI tool was used to generate all results presented in this thesis and the exact parametrization is distributed along the source code on the accompanying repository \parencite{stamm2025}.

\paragraph{Network Architecture} For the training and inference, the fully fused network architecture introduced by \textcite{muller2021} and described in \cref{sec:fully_fused} is used.
The implementation uses the \emph{tiny-cuda-nn} library \parencite{muller2021a} that was distributed alongside the original NRC paper and updated to also provide an implementation of the Multiresolution Hash Encoding (MHE) presented by \textcite{muller2022}.
The network architectures are kept identical to those in the original papers.
Specifically, the MHE network uses 3 hidden layers with 64 neurons, while the Triangle Wave Encoding variant uses 5 hidden layers instead to compensate for the lower quality of the encoding.

\paragraph{Performance Measurement}
Cuda Events are used to evaluate performance.
Tone mapping and screen blitting are not included into the measurements, so only the pure rendering time is reported.
If not otherwise stated, all measurements are taken on an NVIDIA RTX 3060 Ti GPU and averaged over at least 10 seconds.

\paragraph{Error Metric}
Reference images are generated using Path Tracing (PT) and Stochastic Progressive Photon Mapping (SPPM) \parencite{hachisuka2009a} in the \textsc{Caustics} scene where PT does not converge in reasonable time.
To compare the quality of different techniques, the Mean Squared Error (MSE) in linear RGB space is reported for numerical error and the HDR variant of the \FLIP metric \parencite{andersson2021} for humanly perceived error.
Furthermore, bias and variance are measured separately by computing the first and second moment over 1,000 independent 1spp renders.
Between the bias and variance samples the configuration and cache are kept constant, but for the inference pass different samples are drawn from a Quasi-Random Low Discrepancy Sequence.

\paragraph{Scenes}
Tests are performed on a set of scenes that are also included in the source code repository \parencite{stamm2025}:
\begin{itemize}
\item The \textsc{Diffuse} scene is an adaptation of the Cornell Box and highlights diffuse indirect illumination.
\item The \textsc{Thinker} scene incorporates glossy and transmissive materials with complex specular highlights.
It contains decimated and remeshed version of the Thinker statue by Auguste Rodin scanned by \textcite{scantheworld2014} and the Stanford Bunny scanned by the \textcite{stanforduniversitycomputergraphicslaboratory1994}.
\item The \textsc{Chess} scene is a complex textured scene composed of public domain assets from \textcite{polyhaven}.
\item The \textsc{Ajar} scene is mostly lit by long indirect light paths, which are difficult to find with eye tracing.
\item The \textsc{Caustics} scene contains complex sharp caustic patterns that are challenging for traditional path tracing.
\end{itemize}

\section{Parametrizing the NRC}
\begin{figure}[htb!]
    \centering
    \tiny
    \begin{tabularx}{\textwidth}{r*{7}{>{\centering\arraybackslash}X}}
        & Reference (PT) & PT & SAH & BTH $1/2$ & BTH $1/10$ & 1st Diffuse & 1st Vertex\\
        &2879spp (1m)
 & 1spp (20.96ms)
 & 1spp (11.46ms)
 & 1spp (11.35ms)
 & 1spp (10.83ms)
 & 1spp (10.77ms)
 & 1spp (\textbf{10.40ms})
\\
\rotatebox{90}{\textsc{Thinker}}\hspace{-1.5em}
&\includegraphics[width=\linewidth]{figures/py/tests/path_termination/ref_1min_thinker.png}
& \includegraphics[width=\linewidth]{figures/py/tests/path_termination/ref_1spp_thinker.png}
& \includegraphics[width=\linewidth]{figures/py/tests/path_termination/sah_1spp_thinker.png}
& \includegraphics[width=\linewidth]{figures/py/tests/path_termination/bth_1spp_thinker.png}
& \includegraphics[width=\linewidth]{figures/py/tests/path_termination/bthk9_1spp_thinker.png}
& \includegraphics[width=\linewidth]{figures/py/tests/path_termination/1stdiff_1spp_thinker.png}
& \includegraphics[width=\linewidth]{figures/py/tests/path_termination/1stvert_1spp_thinker.png}
\\
&& \includegraphics[width=\linewidth]{figures/py/tests/path_termination/ref_1spp_thinker_flip.png}
& \includegraphics[width=\linewidth]{figures/py/tests/path_termination/sah_1spp_thinker_flip.png}
& \includegraphics[width=\linewidth]{figures/py/tests/path_termination/bth_1spp_thinker_flip.png}
& \includegraphics[width=\linewidth]{figures/py/tests/path_termination/bthk9_1spp_thinker_flip.png}
& \includegraphics[width=\linewidth]{figures/py/tests/path_termination/1stdiff_1spp_thinker_flip.png}
& \includegraphics[width=\linewidth]{figures/py/tests/path_termination/1stvert_1spp_thinker_flip.png}
\\
&\FLIP/MSE: & \num{0.531}/\num{0.061}
 & \num{0.633}/\num{0.335}
 & \num{0.491}/\num{0.206}
 & \num{0.360}/\num{0.013}
 & \textbf{\num{0.351}}/\num{0.014}
 & \num{0.355}/\textbf{\num{0.011}}
\\
&$\mathrm{Bias}^2/\mathrm{Variance}$ & \textbf{\num{6.42e-09}}/\num{6.38e-02}
 & \num{5.29e-05}/\num{2.77e-01}
 & \num{1.23e-04}/\num{1.31e-01}
 & \num{1.46e-04}/\num{1.47e-02}
 & \num{1.43e-04}/\num{1.48e-02}
 & \num{1.40e-04}/\textbf{\num{8.81e-03}}
\\

        &2879spp (1m)
 & 1spp (20.97ms)
 & 1spp (13.06ms)
 & 1spp (12.53ms)
 & 1spp (11.49ms)
 & 1spp (11.83ms)
 & 1spp (\textbf{10.69ms})
\\
\rotatebox{90}{\textsc{Thinker+NEE}}\hspace{-1.5em}
&\includegraphics[width=\linewidth]{figures/py/tests/path_termination/ref_1min_thinker.png}
& \includegraphics[width=\linewidth]{figures/py/tests/path_termination/ref_1spp_thinker.png}
& \includegraphics[width=\linewidth]{figures/py/tests/path_termination/sah+nee_1spp_thinker.png}
& \includegraphics[width=\linewidth]{figures/py/tests/path_termination/bth+nee_1spp_thinker.png}
& \includegraphics[width=\linewidth]{figures/py/tests/path_termination/bthk9+nee_1spp_thinker.png}
& \includegraphics[width=\linewidth]{figures/py/tests/path_termination/1stdiff+nee_1spp_thinker.png}
& \includegraphics[width=\linewidth]{figures/py/tests/path_termination/1stvert+nee_1spp_thinker.png}
\\
&& \includegraphics[width=\linewidth]{figures/py/tests/path_termination/ref_1spp_thinker_flip.png}
& \includegraphics[width=\linewidth]{figures/py/tests/path_termination/sah+nee_1spp_thinker_flip.png}
& \includegraphics[width=\linewidth]{figures/py/tests/path_termination/bth+nee_1spp_thinker_flip.png}
& \includegraphics[width=\linewidth]{figures/py/tests/path_termination/bthk9+nee_1spp_thinker_flip.png}
& \includegraphics[width=\linewidth]{figures/py/tests/path_termination/1stdiff+nee_1spp_thinker_flip.png}
& \includegraphics[width=\linewidth]{figures/py/tests/path_termination/1stvert+nee_1spp_thinker_flip.png}
\\
&\FLIP/MSE: & \num{0.531}/\num{0.061}
 & \num{0.506}/\num{0.016}
 & \num{0.410}/\num{0.015}
 & \textbf{\num{0.404}}/\num{0.014}
 & \num{0.408}/\num{0.015}
 & \num{0.410}/\textbf{\num{0.013}}
\\
&$\mathrm{Bias}^2/\mathrm{Variance}$ & \textbf{\num{6.42e-09}}/\num{6.38e-02}
 & \num{1.67e-04}/\num{1.56e-02}
 & \num{5.60e-07}/\num{1.47e-02}
 & \num{1.80e-04}/\num{1.28e-02}
 & \num{2.13e-04}/\num{1.37e-02}
 & \num{1.83e-04}/\textbf{\num{9.41e-03}}
\\

    \end{tabularx}
    \caption{Comparison of the Path Termination strategies from \cref{sec:path_termination}.}
    \label{fig:pathterm_comparison}
\end{figure}
\paragraph{Path Termination Strategies}
Choosing a path termination strategy is a trade-off between bias and variance (see \cref{fig:pathterm_comparison}).
The SAH termination strategy has the highest variance and also the highest performance cost, because it terminates paths the least aggressively.
On the other end of the spectrum, terminating on the first vertex directly has the lowest variance and is also the fastest but has some issues with specular highlights.
All the other strategies lie somewhere in between, with the parametrized BTH strategy bridging the gap.
As the SPPC strategy does not learn glossy highlights at all and since the quality of the NRC seems good enough for early termination, the following tests use termination at the first diffuse vertex.

\begin{figure}[htb!]
    \centering
    \tiny
    \begin{tabularx}{0.4\textwidth}{r*{3}{>{\centering\arraybackslash}X}}
        & Reference & TWE & MHE \\
        &8608spp (3m)
 & 1spp (\textbf{4.96ms})
 & 1spp (10.80ms)
\\
\rotatebox{90}{\textsc{Thinker}}\hspace{-1.5em}
&\includegraphics[width=\linewidth]{figures/py/tests/encodings/../quality_comparison/refpt_3min_thinker.png}
& \includegraphics[width=\linewidth]{figures/py/tests/encodings/nrc+ptTWE_1spp.png}
& \includegraphics[width=\linewidth]{figures/py/tests/encodings/nrc+ptMHE_1spp.png}
\\
&& \includegraphics[width=\linewidth]{figures/py/tests/encodings/nrc+ptTWE_1spp_flip.png}
& \includegraphics[width=\linewidth]{figures/py/tests/encodings/nrc+ptMHE_1spp_flip.png}
\\
&\FLIP/MSE: & \num{0.398}/\num{0.014}
 & \textbf{\num{0.359}}/\textbf{\num{0.014}}
\\
&$\mathrm{Bias}^2/\mathrm{Variance}$ & \num{1.70e-04}/\textbf{\num{1.48e-02}}
 & \textbf{\num{1.60e-04}}/\num{1.48e-02}
\\

    \end{tabularx}
    \caption{Comparison of different input encodings}
    \label{fig:encodings}
\end{figure}
\paragraph{Input Encodings}
The input encoding has a significant impact on both quality and performance of the cache (see \cref{fig:encodings}).
The Multiresolution Hash Encoding (MHE) by \textcite{muller2022} comes at a great performance hit compared to the original Triangle Wave Encoding (TWE) by \textcite{muller2021} (more than $2\times$ in the test), but distinctly enhances the representation of high-frequency features.
Thus, since we aim to capture caustics, the MHE is the natural choice used in the following evaluations.

\begin{figure}[htb!]
    \centering
    \tiny
    \begin{tabularx}{\textwidth}{r*{7}{>{\centering\arraybackslash}X}}
        & Reference & 1 & 5 & 25 & 100 & 500 & 2500 \\
        &8608spp (3m)
 & 1spp (10.92ms)
 & 1spp (9.59ms)
 & 1spp (9.68ms)
 & 1spp (9.08ms)
 & 1spp (11.41ms)
 & 1spp (\textbf{8.88ms})
\\
\rotatebox{90}{\textsc{PT14}}\hspace{-1.5em}
&\includegraphics[width=\linewidth]{figures/py/tests/batch_size/../quality_comparison/refpt_3min_thinker.png}
& \includegraphics[width=\linewidth]{figures/py/tests/batch_size/1+nrc+pt+14_1spp.png}
& \includegraphics[width=\linewidth]{figures/py/tests/batch_size/5+nrc+pt+14_1spp.png}
& \includegraphics[width=\linewidth]{figures/py/tests/batch_size/25+nrc+pt+14_1spp.png}
& \includegraphics[width=\linewidth]{figures/py/tests/batch_size/100+nrc+pt+14_1spp.png}
& \includegraphics[width=\linewidth]{figures/py/tests/batch_size/500+nrc+pt+14_1spp.png}
& \includegraphics[width=\linewidth]{figures/py/tests/batch_size/2500+nrc+pt+14_1spp.png}
\\
&& \includegraphics[width=\linewidth]{figures/py/tests/batch_size/1+nrc+pt+14_1spp_flip.png}
& \includegraphics[width=\linewidth]{figures/py/tests/batch_size/5+nrc+pt+14_1spp_flip.png}
& \includegraphics[width=\linewidth]{figures/py/tests/batch_size/25+nrc+pt+14_1spp_flip.png}
& \includegraphics[width=\linewidth]{figures/py/tests/batch_size/100+nrc+pt+14_1spp_flip.png}
& \includegraphics[width=\linewidth]{figures/py/tests/batch_size/500+nrc+pt+14_1spp_flip.png}
& \includegraphics[width=\linewidth]{figures/py/tests/batch_size/2500+nrc+pt+14_1spp_flip.png}
\\
&\FLIP/MSE: & \num{0.887}/\num{0.044}
 & \num{0.696}/\num{0.021}
 & \num{0.594}/\num{0.017}
 & \num{0.433}/\num{0.015}
 & \num{0.408}/\num{0.014}
 & \textbf{\num{0.392}}/\textbf{\num{0.014}}
\\
&$\mathrm{Bias}^2/\mathrm{Variance}$ & \num{3.73e-03}/\num{1.50e-02}
 & \num{4.84e-04}/\num{1.49e-02}
 & \num{5.08e-04}/\textbf{\num{1.48e-02}}
 & \num{2.66e-04}/\num{1.48e-02}
 & \textbf{\num{1.96e-04}}/\num{1.48e-02}
 & \num{1.98e-04}/\num{1.48e-02}
\\

        &8608spp (3m)
 & 1spp (\textbf{10.85ms})
 & 1spp (14.14ms)
 & 1spp (13.75ms)
 & 1spp (15.91ms)
 & 1spp (14.04ms)
 & 1spp (11.81ms)
\\
\rotatebox{90}{\textsc{PT14@4}}\hspace{-1.5em}
&\includegraphics[width=\linewidth]{figures/py/tests/batch_size/../quality_comparison/refpt_3min_thinker.png}
& \includegraphics[width=\linewidth]{figures/py/tests/batch_size/1+nrc+pt+14@4_1spp.png}
& \includegraphics[width=\linewidth]{figures/py/tests/batch_size/5+nrc+pt+14@4_1spp.png}
& \includegraphics[width=\linewidth]{figures/py/tests/batch_size/25+nrc+pt+14@4_1spp.png}
& \includegraphics[width=\linewidth]{figures/py/tests/batch_size/100+nrc+pt+14@4_1spp.png}
& \includegraphics[width=\linewidth]{figures/py/tests/batch_size/500+nrc+pt+14@4_1spp.png}
& \includegraphics[width=\linewidth]{figures/py/tests/batch_size/2500+nrc+pt+14@4_1spp.png}
\\
&& \includegraphics[width=\linewidth]{figures/py/tests/batch_size/1+nrc+pt+14@4_1spp_flip.png}
& \includegraphics[width=\linewidth]{figures/py/tests/batch_size/5+nrc+pt+14@4_1spp_flip.png}
& \includegraphics[width=\linewidth]{figures/py/tests/batch_size/25+nrc+pt+14@4_1spp_flip.png}
& \includegraphics[width=\linewidth]{figures/py/tests/batch_size/100+nrc+pt+14@4_1spp_flip.png}
& \includegraphics[width=\linewidth]{figures/py/tests/batch_size/500+nrc+pt+14@4_1spp_flip.png}
& \includegraphics[width=\linewidth]{figures/py/tests/batch_size/2500+nrc+pt+14@4_1spp_flip.png}
\\
&\FLIP/MSE: & \num{0.633}/\num{0.018}
 & \num{0.555}/\num{0.016}
 & \num{0.459}/\num{0.015}
 & \num{0.414}/\num{0.014}
 & \num{0.386}/\textbf{\num{0.014}}
 & \textbf{\num{0.371}}/\num{0.014}
\\
&$\mathrm{Bias}^2/\mathrm{Variance}$ & \textbf{\num{4.41e-05}}/\num{1.49e-02}
 & \num{1.89e-04}/\textbf{\num{1.48e-02}}
 & \num{2.51e-04}/\num{1.48e-02}
 & \num{2.28e-04}/\num{1.48e-02}
 & \num{1.49e-04}/\num{1.48e-02}
 & \num{1.45e-04}/\num{1.48e-02}
\\

        \input{figures/py/tests/batch_size/PT16.tex}
        &8608spp (3m)
 & 1spp (15.57ms)
 & 1spp (13.87ms)
 & 1spp (16.10ms)
 & 1spp (15.24ms)
 & 1spp (\textbf{12.92ms})
 & 1spp (13.44ms)
\\
\rotatebox{90}{\textsc{PT16@4}}\hspace{-1.5em}
&\includegraphics[width=\linewidth]{figures/py/tests/batch_size/../quality_comparison/refpt_3min_thinker.png}
& \includegraphics[width=\linewidth]{figures/py/tests/batch_size/1+nrc+pt+16@4_1spp.png}
& \includegraphics[width=\linewidth]{figures/py/tests/batch_size/5+nrc+pt+16@4_1spp.png}
& \includegraphics[width=\linewidth]{figures/py/tests/batch_size/25+nrc+pt+16@4_1spp.png}
& \includegraphics[width=\linewidth]{figures/py/tests/batch_size/100+nrc+pt+16@4_1spp.png}
& \includegraphics[width=\linewidth]{figures/py/tests/batch_size/500+nrc+pt+16@4_1spp.png}
& \includegraphics[width=\linewidth]{figures/py/tests/batch_size/2500+nrc+pt+16@4_1spp.png}
\\
&& \includegraphics[width=\linewidth]{figures/py/tests/batch_size/1+nrc+pt+16@4_1spp_flip.png}
& \includegraphics[width=\linewidth]{figures/py/tests/batch_size/5+nrc+pt+16@4_1spp_flip.png}
& \includegraphics[width=\linewidth]{figures/py/tests/batch_size/25+nrc+pt+16@4_1spp_flip.png}
& \includegraphics[width=\linewidth]{figures/py/tests/batch_size/100+nrc+pt+16@4_1spp_flip.png}
& \includegraphics[width=\linewidth]{figures/py/tests/batch_size/500+nrc+pt+16@4_1spp_flip.png}
& \includegraphics[width=\linewidth]{figures/py/tests/batch_size/2500+nrc+pt+16@4_1spp_flip.png}
\\
&\FLIP/MSE: & \num{0.627}/\num{0.018}
 & \num{0.557}/\num{0.016}
 & \num{0.431}/\num{0.015}
 & \num{0.382}/\num{0.014}
 & \num{0.356}/\num{0.014}
 & \textbf{\num{0.345}}/\textbf{\num{0.014}}
\\
&$\mathrm{Bias}^2/\mathrm{Variance}$ & \textbf{\num{9.08e-05}}/\num{1.49e-02}
 & \num{2.09e-04}/\textbf{\num{1.48e-02}}
 & \num{2.36e-04}/\num{1.48e-02}
 & \num{1.59e-04}/\num{1.48e-02}
 & \num{1.09e-04}/\num{1.48e-02}
 & \num{9.39e-05}/\num{1.48e-02}
\\

    \end{tabularx}
    \caption{Comparison of cache convergence for different training set sizes (14: $2^{14}=16,384$, 16: $2^{16}=65,536$) and training steps per frame (@4 indicates that the training set is split into 4 batches, otherwise it is processed as a single batch). The columns indicate how many frames were rendered to train the cache (1, 5, 25, 100, 500, 2500).}
    \label{fig:batch_size}
\end{figure}
\paragraph{Training} Like recommended by \textcite{muller2022}, training is performed using the Adam optimizer \parencite{kingma2014} with a learning rate of $10^{-2}$, $\beta_1 = 0.9$, $\beta_2 = 0.99$, $\epsilon = 10^{-15}$ and an L2 regularization factor of $10^{-6}$.
Furthermore, they use 4 training steps with a batch size of $2^{14}=16,384$ each.
Tests show, that the usage of multiple training steps per frame leads to faster cache convergence (see \cref{fig:batch_size}) while larger training sets evidently result in lower error.

\section{Optimizations}

\begin{figure}[htb!]
    \centering
    %% Creator: Matplotlib, PGF backend
%%
%% To include the figure in your LaTeX document, write
%%   \input{<filename>.pgf}
%%
%% Make sure the required packages are loaded in your preamble
%%   \usepackage{pgf}
%%
%% Also ensure that all the required font packages are loaded; for instance,
%% the lmodern package is sometimes necessary when using math font.
%%   \usepackage{lmodern}
%%
%% Figures using additional raster images can only be included by \input if
%% they are in the same directory as the main LaTeX file. For loading figures
%% from other directories you can use the `import` package
%%   \usepackage{import}
%%
%% and then include the figures with
%%   \import{<path to file>}{<filename>.pgf}
%%
%% Matplotlib used the following preamble
%%   \def\mathdefault#1{#1}
%%   \everymath=\expandafter{\the\everymath\displaystyle}
%%   \IfFileExists{scrextend.sty}{
%%     \usepackage[fontsize=10.000000pt]{scrextend}
%%   }{
%%     \renewcommand{\normalsize}{\fontsize{10.000000}{12.000000}\selectfont}
%%     \normalsize
%%   }
%%   
%%   \ifdefined\pdftexversion\else  % non-pdftex case.
%%     \usepackage{fontspec}
%%     \setmainfont{DejaVuSerif.ttf}[Path=\detokenize{/opt/homebrew/Cellar/python-matplotlib/3.10.5/libexec/lib/python3.13/site-packages/matplotlib/mpl-data/fonts/ttf/}]
%%     \setsansfont{DejaVuSans.ttf}[Path=\detokenize{/opt/homebrew/Cellar/python-matplotlib/3.10.5/libexec/lib/python3.13/site-packages/matplotlib/mpl-data/fonts/ttf/}]
%%     \setmonofont{DejaVuSansMono.ttf}[Path=\detokenize{/opt/homebrew/Cellar/python-matplotlib/3.10.5/libexec/lib/python3.13/site-packages/matplotlib/mpl-data/fonts/ttf/}]
%%   \fi
%%   \makeatletter\@ifpackageloaded{underscore}{}{\usepackage[strings]{underscore}}\makeatother
%%
\begingroup%
\makeatletter%
\begin{pgfpicture}%
\pgfpathrectangle{\pgfpointorigin}{\pgfqpoint{4.118122in}{2.116660in}}%
\pgfusepath{use as bounding box, clip}%
\begin{pgfscope}%
\pgfsetbuttcap%
\pgfsetmiterjoin%
\definecolor{currentfill}{rgb}{1.000000,1.000000,1.000000}%
\pgfsetfillcolor{currentfill}%
\pgfsetlinewidth{0.000000pt}%
\definecolor{currentstroke}{rgb}{1.000000,1.000000,1.000000}%
\pgfsetstrokecolor{currentstroke}%
\pgfsetdash{}{0pt}%
\pgfpathmoveto{\pgfqpoint{0.000000in}{0.000000in}}%
\pgfpathlineto{\pgfqpoint{4.118122in}{0.000000in}}%
\pgfpathlineto{\pgfqpoint{4.118122in}{2.116660in}}%
\pgfpathlineto{\pgfqpoint{0.000000in}{2.116660in}}%
\pgfpathlineto{\pgfqpoint{0.000000in}{0.000000in}}%
\pgfpathclose%
\pgfusepath{fill}%
\end{pgfscope}%
\begin{pgfscope}%
\pgfsetbuttcap%
\pgfsetmiterjoin%
\definecolor{currentfill}{rgb}{1.000000,1.000000,1.000000}%
\pgfsetfillcolor{currentfill}%
\pgfsetlinewidth{0.000000pt}%
\definecolor{currentstroke}{rgb}{0.000000,0.000000,0.000000}%
\pgfsetstrokecolor{currentstroke}%
\pgfsetstrokeopacity{0.000000}%
\pgfsetdash{}{0pt}%
\pgfpathmoveto{\pgfqpoint{1.983610in}{0.290505in}}%
\pgfpathlineto{\pgfqpoint{4.018122in}{0.290505in}}%
\pgfpathlineto{\pgfqpoint{4.018122in}{2.016660in}}%
\pgfpathlineto{\pgfqpoint{1.983610in}{2.016660in}}%
\pgfpathlineto{\pgfqpoint{1.983610in}{0.290505in}}%
\pgfpathclose%
\pgfusepath{fill}%
\end{pgfscope}%
\begin{pgfscope}%
\pgfpathrectangle{\pgfqpoint{1.983610in}{0.290505in}}{\pgfqpoint{2.034512in}{1.726155in}}%
\pgfusepath{clip}%
\pgfsetbuttcap%
\pgfsetmiterjoin%
\definecolor{currentfill}{rgb}{0.814118,0.883922,0.949804}%
\pgfsetfillcolor{currentfill}%
\pgfsetlinewidth{0.000000pt}%
\definecolor{currentstroke}{rgb}{0.000000,0.000000,0.000000}%
\pgfsetstrokecolor{currentstroke}%
\pgfsetstrokeopacity{0.000000}%
\pgfsetdash{}{0pt}%
\pgfpathmoveto{\pgfqpoint{2.076088in}{0.290505in}}%
\pgfpathlineto{\pgfqpoint{2.445999in}{0.290505in}}%
\pgfpathlineto{\pgfqpoint{2.445999in}{0.361376in}}%
\pgfpathlineto{\pgfqpoint{2.076088in}{0.361376in}}%
\pgfpathlineto{\pgfqpoint{2.076088in}{0.290505in}}%
\pgfpathclose%
\pgfusepath{fill}%
\end{pgfscope}%
\begin{pgfscope}%
\pgfpathrectangle{\pgfqpoint{1.983610in}{0.290505in}}{\pgfqpoint{2.034512in}{1.726155in}}%
\pgfusepath{clip}%
\pgfsetbuttcap%
\pgfsetmiterjoin%
\definecolor{currentfill}{rgb}{0.887059,0.887059,0.887059}%
\pgfsetfillcolor{currentfill}%
\pgfsetlinewidth{0.000000pt}%
\definecolor{currentstroke}{rgb}{0.000000,0.000000,0.000000}%
\pgfsetstrokecolor{currentstroke}%
\pgfsetstrokeopacity{0.000000}%
\pgfsetdash{}{0pt}%
\pgfpathmoveto{\pgfqpoint{2.076088in}{0.361376in}}%
\pgfpathlineto{\pgfqpoint{2.445999in}{0.361376in}}%
\pgfpathlineto{\pgfqpoint{2.445999in}{0.696436in}}%
\pgfpathlineto{\pgfqpoint{2.076088in}{0.696436in}}%
\pgfpathlineto{\pgfqpoint{2.076088in}{0.361376in}}%
\pgfpathclose%
\pgfusepath{fill}%
\end{pgfscope}%
\begin{pgfscope}%
\pgfpathrectangle{\pgfqpoint{1.983610in}{0.290505in}}{\pgfqpoint{2.034512in}{1.726155in}}%
\pgfusepath{clip}%
\pgfsetbuttcap%
\pgfsetmiterjoin%
\definecolor{currentfill}{rgb}{0.710588,0.710588,0.710588}%
\pgfsetfillcolor{currentfill}%
\pgfsetlinewidth{0.000000pt}%
\definecolor{currentstroke}{rgb}{0.000000,0.000000,0.000000}%
\pgfsetstrokecolor{currentstroke}%
\pgfsetstrokeopacity{0.000000}%
\pgfsetdash{}{0pt}%
\pgfpathmoveto{\pgfqpoint{2.076088in}{0.696436in}}%
\pgfpathlineto{\pgfqpoint{2.445999in}{0.696436in}}%
\pgfpathlineto{\pgfqpoint{2.445999in}{0.809748in}}%
\pgfpathlineto{\pgfqpoint{2.076088in}{0.809748in}}%
\pgfpathlineto{\pgfqpoint{2.076088in}{0.696436in}}%
\pgfpathclose%
\pgfusepath{fill}%
\end{pgfscope}%
\begin{pgfscope}%
\pgfpathrectangle{\pgfqpoint{1.983610in}{0.290505in}}{\pgfqpoint{2.034512in}{1.726155in}}%
\pgfusepath{clip}%
\pgfsetbuttcap%
\pgfsetmiterjoin%
\definecolor{currentfill}{rgb}{0.478431,0.478431,0.478431}%
\pgfsetfillcolor{currentfill}%
\pgfsetlinewidth{0.000000pt}%
\definecolor{currentstroke}{rgb}{0.000000,0.000000,0.000000}%
\pgfsetstrokecolor{currentstroke}%
\pgfsetstrokeopacity{0.000000}%
\pgfsetdash{}{0pt}%
\pgfpathmoveto{\pgfqpoint{2.076088in}{0.809748in}}%
\pgfpathlineto{\pgfqpoint{2.445999in}{0.809748in}}%
\pgfpathlineto{\pgfqpoint{2.445999in}{1.125000in}}%
\pgfpathlineto{\pgfqpoint{2.076088in}{1.125000in}}%
\pgfpathlineto{\pgfqpoint{2.076088in}{0.809748in}}%
\pgfpathclose%
\pgfusepath{fill}%
\end{pgfscope}%
\begin{pgfscope}%
\pgfpathrectangle{\pgfqpoint{1.983610in}{0.290505in}}{\pgfqpoint{2.034512in}{1.726155in}}%
\pgfusepath{clip}%
\pgfsetbuttcap%
\pgfsetmiterjoin%
\definecolor{currentfill}{rgb}{1.000000,0.752941,0.796078}%
\pgfsetfillcolor{currentfill}%
\pgfsetlinewidth{0.000000pt}%
\definecolor{currentstroke}{rgb}{0.000000,0.000000,0.000000}%
\pgfsetstrokecolor{currentstroke}%
\pgfsetstrokeopacity{0.000000}%
\pgfsetdash{}{0pt}%
\pgfpathmoveto{\pgfqpoint{2.076088in}{1.125000in}}%
\pgfpathlineto{\pgfqpoint{2.445999in}{1.125000in}}%
\pgfpathlineto{\pgfqpoint{2.445999in}{1.148969in}}%
\pgfpathlineto{\pgfqpoint{2.076088in}{1.148969in}}%
\pgfpathlineto{\pgfqpoint{2.076088in}{1.125000in}}%
\pgfpathclose%
\pgfusepath{fill}%
\end{pgfscope}%
\begin{pgfscope}%
\pgfpathrectangle{\pgfqpoint{1.983610in}{0.290505in}}{\pgfqpoint{2.034512in}{1.726155in}}%
\pgfusepath{clip}%
\pgfsetbuttcap%
\pgfsetmiterjoin%
\definecolor{currentfill}{rgb}{0.854902,0.439216,0.839216}%
\pgfsetfillcolor{currentfill}%
\pgfsetlinewidth{0.000000pt}%
\definecolor{currentstroke}{rgb}{0.000000,0.000000,0.000000}%
\pgfsetstrokecolor{currentstroke}%
\pgfsetstrokeopacity{0.000000}%
\pgfsetdash{}{0pt}%
\pgfpathmoveto{\pgfqpoint{2.076088in}{1.148969in}}%
\pgfpathlineto{\pgfqpoint{2.445999in}{1.148969in}}%
\pgfpathlineto{\pgfqpoint{2.445999in}{1.166815in}}%
\pgfpathlineto{\pgfqpoint{2.076088in}{1.166815in}}%
\pgfpathlineto{\pgfqpoint{2.076088in}{1.148969in}}%
\pgfpathclose%
\pgfusepath{fill}%
\end{pgfscope}%
\begin{pgfscope}%
\pgfpathrectangle{\pgfqpoint{1.983610in}{0.290505in}}{\pgfqpoint{2.034512in}{1.726155in}}%
\pgfusepath{clip}%
\pgfsetbuttcap%
\pgfsetmiterjoin%
\definecolor{currentfill}{rgb}{0.814118,0.883922,0.949804}%
\pgfsetfillcolor{currentfill}%
\pgfsetlinewidth{0.000000pt}%
\definecolor{currentstroke}{rgb}{0.000000,0.000000,0.000000}%
\pgfsetstrokecolor{currentstroke}%
\pgfsetstrokeopacity{0.000000}%
\pgfsetdash{}{0pt}%
\pgfpathmoveto{\pgfqpoint{2.815910in}{0.290505in}}%
\pgfpathlineto{\pgfqpoint{3.185821in}{0.290505in}}%
\pgfpathlineto{\pgfqpoint{3.185821in}{0.349719in}}%
\pgfpathlineto{\pgfqpoint{2.815910in}{0.349719in}}%
\pgfpathlineto{\pgfqpoint{2.815910in}{0.290505in}}%
\pgfpathclose%
\pgfusepath{fill}%
\end{pgfscope}%
\begin{pgfscope}%
\pgfpathrectangle{\pgfqpoint{1.983610in}{0.290505in}}{\pgfqpoint{2.034512in}{1.726155in}}%
\pgfusepath{clip}%
\pgfsetbuttcap%
\pgfsetmiterjoin%
\definecolor{currentfill}{rgb}{0.887059,0.887059,0.887059}%
\pgfsetfillcolor{currentfill}%
\pgfsetlinewidth{0.000000pt}%
\definecolor{currentstroke}{rgb}{0.000000,0.000000,0.000000}%
\pgfsetstrokecolor{currentstroke}%
\pgfsetstrokeopacity{0.000000}%
\pgfsetdash{}{0pt}%
\pgfpathmoveto{\pgfqpoint{2.815910in}{0.349719in}}%
\pgfpathlineto{\pgfqpoint{3.185821in}{0.349719in}}%
\pgfpathlineto{\pgfqpoint{3.185821in}{0.754880in}}%
\pgfpathlineto{\pgfqpoint{2.815910in}{0.754880in}}%
\pgfpathlineto{\pgfqpoint{2.815910in}{0.349719in}}%
\pgfpathclose%
\pgfusepath{fill}%
\end{pgfscope}%
\begin{pgfscope}%
\pgfpathrectangle{\pgfqpoint{1.983610in}{0.290505in}}{\pgfqpoint{2.034512in}{1.726155in}}%
\pgfusepath{clip}%
\pgfsetbuttcap%
\pgfsetmiterjoin%
\definecolor{currentfill}{rgb}{0.710588,0.710588,0.710588}%
\pgfsetfillcolor{currentfill}%
\pgfsetlinewidth{0.000000pt}%
\definecolor{currentstroke}{rgb}{0.000000,0.000000,0.000000}%
\pgfsetstrokecolor{currentstroke}%
\pgfsetstrokeopacity{0.000000}%
\pgfsetdash{}{0pt}%
\pgfpathmoveto{\pgfqpoint{2.815910in}{0.754880in}}%
\pgfpathlineto{\pgfqpoint{3.185821in}{0.754880in}}%
\pgfpathlineto{\pgfqpoint{3.185821in}{0.873203in}}%
\pgfpathlineto{\pgfqpoint{2.815910in}{0.873203in}}%
\pgfpathlineto{\pgfqpoint{2.815910in}{0.754880in}}%
\pgfpathclose%
\pgfusepath{fill}%
\end{pgfscope}%
\begin{pgfscope}%
\pgfpathrectangle{\pgfqpoint{1.983610in}{0.290505in}}{\pgfqpoint{2.034512in}{1.726155in}}%
\pgfusepath{clip}%
\pgfsetbuttcap%
\pgfsetmiterjoin%
\definecolor{currentfill}{rgb}{0.478431,0.478431,0.478431}%
\pgfsetfillcolor{currentfill}%
\pgfsetlinewidth{0.000000pt}%
\definecolor{currentstroke}{rgb}{0.000000,0.000000,0.000000}%
\pgfsetstrokecolor{currentstroke}%
\pgfsetstrokeopacity{0.000000}%
\pgfsetdash{}{0pt}%
\pgfpathmoveto{\pgfqpoint{2.815910in}{0.873203in}}%
\pgfpathlineto{\pgfqpoint{3.185821in}{0.873203in}}%
\pgfpathlineto{\pgfqpoint{3.185821in}{1.529910in}}%
\pgfpathlineto{\pgfqpoint{2.815910in}{1.529910in}}%
\pgfpathlineto{\pgfqpoint{2.815910in}{0.873203in}}%
\pgfpathclose%
\pgfusepath{fill}%
\end{pgfscope}%
\begin{pgfscope}%
\pgfpathrectangle{\pgfqpoint{1.983610in}{0.290505in}}{\pgfqpoint{2.034512in}{1.726155in}}%
\pgfusepath{clip}%
\pgfsetbuttcap%
\pgfsetmiterjoin%
\definecolor{currentfill}{rgb}{1.000000,0.752941,0.796078}%
\pgfsetfillcolor{currentfill}%
\pgfsetlinewidth{0.000000pt}%
\definecolor{currentstroke}{rgb}{0.000000,0.000000,0.000000}%
\pgfsetstrokecolor{currentstroke}%
\pgfsetstrokeopacity{0.000000}%
\pgfsetdash{}{0pt}%
\pgfpathmoveto{\pgfqpoint{2.815910in}{1.529910in}}%
\pgfpathlineto{\pgfqpoint{3.185821in}{1.529910in}}%
\pgfpathlineto{\pgfqpoint{3.185821in}{1.547627in}}%
\pgfpathlineto{\pgfqpoint{2.815910in}{1.547627in}}%
\pgfpathlineto{\pgfqpoint{2.815910in}{1.529910in}}%
\pgfpathclose%
\pgfusepath{fill}%
\end{pgfscope}%
\begin{pgfscope}%
\pgfpathrectangle{\pgfqpoint{1.983610in}{0.290505in}}{\pgfqpoint{2.034512in}{1.726155in}}%
\pgfusepath{clip}%
\pgfsetbuttcap%
\pgfsetmiterjoin%
\definecolor{currentfill}{rgb}{0.854902,0.439216,0.839216}%
\pgfsetfillcolor{currentfill}%
\pgfsetlinewidth{0.000000pt}%
\definecolor{currentstroke}{rgb}{0.000000,0.000000,0.000000}%
\pgfsetstrokecolor{currentstroke}%
\pgfsetstrokeopacity{0.000000}%
\pgfsetdash{}{0pt}%
\pgfpathmoveto{\pgfqpoint{2.815910in}{1.547627in}}%
\pgfpathlineto{\pgfqpoint{3.185821in}{1.547627in}}%
\pgfpathlineto{\pgfqpoint{3.185821in}{1.555087in}}%
\pgfpathlineto{\pgfqpoint{2.815910in}{1.555087in}}%
\pgfpathlineto{\pgfqpoint{2.815910in}{1.547627in}}%
\pgfpathclose%
\pgfusepath{fill}%
\end{pgfscope}%
\begin{pgfscope}%
\pgfpathrectangle{\pgfqpoint{1.983610in}{0.290505in}}{\pgfqpoint{2.034512in}{1.726155in}}%
\pgfusepath{clip}%
\pgfsetbuttcap%
\pgfsetmiterjoin%
\definecolor{currentfill}{rgb}{0.814118,0.883922,0.949804}%
\pgfsetfillcolor{currentfill}%
\pgfsetlinewidth{0.000000pt}%
\definecolor{currentstroke}{rgb}{0.000000,0.000000,0.000000}%
\pgfsetstrokecolor{currentstroke}%
\pgfsetstrokeopacity{0.000000}%
\pgfsetdash{}{0pt}%
\pgfpathmoveto{\pgfqpoint{3.555733in}{0.290505in}}%
\pgfpathlineto{\pgfqpoint{3.925644in}{0.290505in}}%
\pgfpathlineto{\pgfqpoint{3.925644in}{0.349131in}}%
\pgfpathlineto{\pgfqpoint{3.555733in}{0.349131in}}%
\pgfpathlineto{\pgfqpoint{3.555733in}{0.290505in}}%
\pgfpathclose%
\pgfusepath{fill}%
\end{pgfscope}%
\begin{pgfscope}%
\pgfpathrectangle{\pgfqpoint{1.983610in}{0.290505in}}{\pgfqpoint{2.034512in}{1.726155in}}%
\pgfusepath{clip}%
\pgfsetbuttcap%
\pgfsetmiterjoin%
\definecolor{currentfill}{rgb}{0.887059,0.887059,0.887059}%
\pgfsetfillcolor{currentfill}%
\pgfsetlinewidth{0.000000pt}%
\definecolor{currentstroke}{rgb}{0.000000,0.000000,0.000000}%
\pgfsetstrokecolor{currentstroke}%
\pgfsetstrokeopacity{0.000000}%
\pgfsetdash{}{0pt}%
\pgfpathmoveto{\pgfqpoint{3.555733in}{0.349131in}}%
\pgfpathlineto{\pgfqpoint{3.925644in}{0.349131in}}%
\pgfpathlineto{\pgfqpoint{3.925644in}{0.752915in}}%
\pgfpathlineto{\pgfqpoint{3.555733in}{0.752915in}}%
\pgfpathlineto{\pgfqpoint{3.555733in}{0.349131in}}%
\pgfpathclose%
\pgfusepath{fill}%
\end{pgfscope}%
\begin{pgfscope}%
\pgfpathrectangle{\pgfqpoint{1.983610in}{0.290505in}}{\pgfqpoint{2.034512in}{1.726155in}}%
\pgfusepath{clip}%
\pgfsetbuttcap%
\pgfsetmiterjoin%
\definecolor{currentfill}{rgb}{0.710588,0.710588,0.710588}%
\pgfsetfillcolor{currentfill}%
\pgfsetlinewidth{0.000000pt}%
\definecolor{currentstroke}{rgb}{0.000000,0.000000,0.000000}%
\pgfsetstrokecolor{currentstroke}%
\pgfsetstrokeopacity{0.000000}%
\pgfsetdash{}{0pt}%
\pgfpathmoveto{\pgfqpoint{3.555733in}{0.752915in}}%
\pgfpathlineto{\pgfqpoint{3.925644in}{0.752915in}}%
\pgfpathlineto{\pgfqpoint{3.925644in}{0.870487in}}%
\pgfpathlineto{\pgfqpoint{3.555733in}{0.870487in}}%
\pgfpathlineto{\pgfqpoint{3.555733in}{0.752915in}}%
\pgfpathclose%
\pgfusepath{fill}%
\end{pgfscope}%
\begin{pgfscope}%
\pgfpathrectangle{\pgfqpoint{1.983610in}{0.290505in}}{\pgfqpoint{2.034512in}{1.726155in}}%
\pgfusepath{clip}%
\pgfsetbuttcap%
\pgfsetmiterjoin%
\definecolor{currentfill}{rgb}{0.478431,0.478431,0.478431}%
\pgfsetfillcolor{currentfill}%
\pgfsetlinewidth{0.000000pt}%
\definecolor{currentstroke}{rgb}{0.000000,0.000000,0.000000}%
\pgfsetstrokecolor{currentstroke}%
\pgfsetstrokeopacity{0.000000}%
\pgfsetdash{}{0pt}%
\pgfpathmoveto{\pgfqpoint{3.555733in}{0.870487in}}%
\pgfpathlineto{\pgfqpoint{3.925644in}{0.870487in}}%
\pgfpathlineto{\pgfqpoint{3.925644in}{1.527790in}}%
\pgfpathlineto{\pgfqpoint{3.555733in}{1.527790in}}%
\pgfpathlineto{\pgfqpoint{3.555733in}{0.870487in}}%
\pgfpathclose%
\pgfusepath{fill}%
\end{pgfscope}%
\begin{pgfscope}%
\pgfpathrectangle{\pgfqpoint{1.983610in}{0.290505in}}{\pgfqpoint{2.034512in}{1.726155in}}%
\pgfusepath{clip}%
\pgfsetbuttcap%
\pgfsetmiterjoin%
\definecolor{currentfill}{rgb}{1.000000,0.752941,0.796078}%
\pgfsetfillcolor{currentfill}%
\pgfsetlinewidth{0.000000pt}%
\definecolor{currentstroke}{rgb}{0.000000,0.000000,0.000000}%
\pgfsetstrokecolor{currentstroke}%
\pgfsetstrokeopacity{0.000000}%
\pgfsetdash{}{0pt}%
\pgfpathmoveto{\pgfqpoint{3.555733in}{1.527790in}}%
\pgfpathlineto{\pgfqpoint{3.925644in}{1.527790in}}%
\pgfpathlineto{\pgfqpoint{3.925644in}{1.545545in}}%
\pgfpathlineto{\pgfqpoint{3.555733in}{1.545545in}}%
\pgfpathlineto{\pgfqpoint{3.555733in}{1.527790in}}%
\pgfpathclose%
\pgfusepath{fill}%
\end{pgfscope}%
\begin{pgfscope}%
\pgfpathrectangle{\pgfqpoint{1.983610in}{0.290505in}}{\pgfqpoint{2.034512in}{1.726155in}}%
\pgfusepath{clip}%
\pgfsetbuttcap%
\pgfsetmiterjoin%
\definecolor{currentfill}{rgb}{0.854902,0.439216,0.839216}%
\pgfsetfillcolor{currentfill}%
\pgfsetlinewidth{0.000000pt}%
\definecolor{currentstroke}{rgb}{0.000000,0.000000,0.000000}%
\pgfsetstrokecolor{currentstroke}%
\pgfsetstrokeopacity{0.000000}%
\pgfsetdash{}{0pt}%
\pgfpathmoveto{\pgfqpoint{3.555733in}{1.545545in}}%
\pgfpathlineto{\pgfqpoint{3.925644in}{1.545545in}}%
\pgfpathlineto{\pgfqpoint{3.925644in}{1.552636in}}%
\pgfpathlineto{\pgfqpoint{3.555733in}{1.552636in}}%
\pgfpathlineto{\pgfqpoint{3.555733in}{1.545545in}}%
\pgfpathclose%
\pgfusepath{fill}%
\end{pgfscope}%
\begin{pgfscope}%
\pgfsetbuttcap%
\pgfsetroundjoin%
\definecolor{currentfill}{rgb}{0.000000,0.000000,0.000000}%
\pgfsetfillcolor{currentfill}%
\pgfsetlinewidth{0.803000pt}%
\definecolor{currentstroke}{rgb}{0.000000,0.000000,0.000000}%
\pgfsetstrokecolor{currentstroke}%
\pgfsetdash{}{0pt}%
\pgfsys@defobject{currentmarker}{\pgfqpoint{0.000000in}{-0.048611in}}{\pgfqpoint{0.000000in}{0.000000in}}{%
\pgfpathmoveto{\pgfqpoint{0.000000in}{0.000000in}}%
\pgfpathlineto{\pgfqpoint{0.000000in}{-0.048611in}}%
\pgfusepath{stroke,fill}%
}%
\begin{pgfscope}%
\pgfsys@transformshift{2.261044in}{0.290505in}%
\pgfsys@useobject{currentmarker}{}%
\end{pgfscope}%
\end{pgfscope}%
\begin{pgfscope}%
\definecolor{textcolor}{rgb}{0.000000,0.000000,0.000000}%
\pgfsetstrokecolor{textcolor}%
\pgfsetfillcolor{textcolor}%
\pgftext[x=2.261044in,y=0.193282in,,top]{\color{textcolor}{\rmfamily\fontsize{6.940000}{8.328000}\selectfont\catcode`\^=\active\def^{\ifmmode\sp\else\^{}\fi}\catcode`\%=\active\def%{\%}NRC+PT}}%
\end{pgfscope}%
\begin{pgfscope}%
\pgfsetbuttcap%
\pgfsetroundjoin%
\definecolor{currentfill}{rgb}{0.000000,0.000000,0.000000}%
\pgfsetfillcolor{currentfill}%
\pgfsetlinewidth{0.803000pt}%
\definecolor{currentstroke}{rgb}{0.000000,0.000000,0.000000}%
\pgfsetstrokecolor{currentstroke}%
\pgfsetdash{}{0pt}%
\pgfsys@defobject{currentmarker}{\pgfqpoint{0.000000in}{-0.048611in}}{\pgfqpoint{0.000000in}{0.000000in}}{%
\pgfpathmoveto{\pgfqpoint{0.000000in}{0.000000in}}%
\pgfpathlineto{\pgfqpoint{0.000000in}{-0.048611in}}%
\pgfusepath{stroke,fill}%
}%
\begin{pgfscope}%
\pgfsys@transformshift{3.000866in}{0.290505in}%
\pgfsys@useobject{currentmarker}{}%
\end{pgfscope}%
\end{pgfscope}%
\begin{pgfscope}%
\definecolor{textcolor}{rgb}{0.000000,0.000000,0.000000}%
\pgfsetstrokecolor{textcolor}%
\pgfsetfillcolor{textcolor}%
\pgftext[x=3.000866in,y=0.193282in,,top]{\color{textcolor}{\rmfamily\fontsize{6.940000}{8.328000}\selectfont\catcode`\^=\active\def^{\ifmmode\sp\else\^{}\fi}\catcode`\%=\active\def%{\%}+Fused}}%
\end{pgfscope}%
\begin{pgfscope}%
\pgfsetbuttcap%
\pgfsetroundjoin%
\definecolor{currentfill}{rgb}{0.000000,0.000000,0.000000}%
\pgfsetfillcolor{currentfill}%
\pgfsetlinewidth{0.803000pt}%
\definecolor{currentstroke}{rgb}{0.000000,0.000000,0.000000}%
\pgfsetstrokecolor{currentstroke}%
\pgfsetdash{}{0pt}%
\pgfsys@defobject{currentmarker}{\pgfqpoint{0.000000in}{-0.048611in}}{\pgfqpoint{0.000000in}{0.000000in}}{%
\pgfpathmoveto{\pgfqpoint{0.000000in}{0.000000in}}%
\pgfpathlineto{\pgfqpoint{0.000000in}{-0.048611in}}%
\pgfusepath{stroke,fill}%
}%
\begin{pgfscope}%
\pgfsys@transformshift{3.740688in}{0.290505in}%
\pgfsys@useobject{currentmarker}{}%
\end{pgfscope}%
\end{pgfscope}%
\begin{pgfscope}%
\definecolor{textcolor}{rgb}{0.000000,0.000000,0.000000}%
\pgfsetstrokecolor{textcolor}%
\pgfsetfillcolor{textcolor}%
\pgftext[x=3.740688in,y=0.193282in,,top]{\color{textcolor}{\rmfamily\fontsize{6.940000}{8.328000}\selectfont\catcode`\^=\active\def^{\ifmmode\sp\else\^{}\fi}\catcode`\%=\active\def%{\%}+FusedVis}}%
\end{pgfscope}%
\begin{pgfscope}%
\pgfsetbuttcap%
\pgfsetroundjoin%
\definecolor{currentfill}{rgb}{0.000000,0.000000,0.000000}%
\pgfsetfillcolor{currentfill}%
\pgfsetlinewidth{0.803000pt}%
\definecolor{currentstroke}{rgb}{0.000000,0.000000,0.000000}%
\pgfsetstrokecolor{currentstroke}%
\pgfsetdash{}{0pt}%
\pgfsys@defobject{currentmarker}{\pgfqpoint{-0.048611in}{0.000000in}}{\pgfqpoint{-0.000000in}{0.000000in}}{%
\pgfpathmoveto{\pgfqpoint{-0.000000in}{0.000000in}}%
\pgfpathlineto{\pgfqpoint{-0.048611in}{0.000000in}}%
\pgfusepath{stroke,fill}%
}%
\begin{pgfscope}%
\pgfsys@transformshift{1.983610in}{0.290505in}%
\pgfsys@useobject{currentmarker}{}%
\end{pgfscope}%
\end{pgfscope}%
\begin{pgfscope}%
\definecolor{textcolor}{rgb}{0.000000,0.000000,0.000000}%
\pgfsetstrokecolor{textcolor}%
\pgfsetfillcolor{textcolor}%
\pgftext[x=1.816943in, y=0.237743in, left, base]{\color{textcolor}{\rmfamily\fontsize{10.000000}{12.000000}\selectfont\catcode`\^=\active\def^{\ifmmode\sp\else\^{}\fi}\catcode`\%=\active\def%{\%}$\mathdefault{0}$}}%
\end{pgfscope}%
\begin{pgfscope}%
\pgfsetbuttcap%
\pgfsetroundjoin%
\definecolor{currentfill}{rgb}{0.000000,0.000000,0.000000}%
\pgfsetfillcolor{currentfill}%
\pgfsetlinewidth{0.803000pt}%
\definecolor{currentstroke}{rgb}{0.000000,0.000000,0.000000}%
\pgfsetstrokecolor{currentstroke}%
\pgfsetdash{}{0pt}%
\pgfsys@defobject{currentmarker}{\pgfqpoint{-0.048611in}{0.000000in}}{\pgfqpoint{-0.000000in}{0.000000in}}{%
\pgfpathmoveto{\pgfqpoint{-0.000000in}{0.000000in}}%
\pgfpathlineto{\pgfqpoint{-0.048611in}{0.000000in}}%
\pgfusepath{stroke,fill}%
}%
\begin{pgfscope}%
\pgfsys@transformshift{1.983610in}{0.668819in}%
\pgfsys@useobject{currentmarker}{}%
\end{pgfscope}%
\end{pgfscope}%
\begin{pgfscope}%
\definecolor{textcolor}{rgb}{0.000000,0.000000,0.000000}%
\pgfsetstrokecolor{textcolor}%
\pgfsetfillcolor{textcolor}%
\pgftext[x=1.816943in, y=0.616058in, left, base]{\color{textcolor}{\rmfamily\fontsize{10.000000}{12.000000}\selectfont\catcode`\^=\active\def^{\ifmmode\sp\else\^{}\fi}\catcode`\%=\active\def%{\%}$\mathdefault{5}$}}%
\end{pgfscope}%
\begin{pgfscope}%
\pgfsetbuttcap%
\pgfsetroundjoin%
\definecolor{currentfill}{rgb}{0.000000,0.000000,0.000000}%
\pgfsetfillcolor{currentfill}%
\pgfsetlinewidth{0.803000pt}%
\definecolor{currentstroke}{rgb}{0.000000,0.000000,0.000000}%
\pgfsetstrokecolor{currentstroke}%
\pgfsetdash{}{0pt}%
\pgfsys@defobject{currentmarker}{\pgfqpoint{-0.048611in}{0.000000in}}{\pgfqpoint{-0.000000in}{0.000000in}}{%
\pgfpathmoveto{\pgfqpoint{-0.000000in}{0.000000in}}%
\pgfpathlineto{\pgfqpoint{-0.048611in}{0.000000in}}%
\pgfusepath{stroke,fill}%
}%
\begin{pgfscope}%
\pgfsys@transformshift{1.983610in}{1.047134in}%
\pgfsys@useobject{currentmarker}{}%
\end{pgfscope}%
\end{pgfscope}%
\begin{pgfscope}%
\definecolor{textcolor}{rgb}{0.000000,0.000000,0.000000}%
\pgfsetstrokecolor{textcolor}%
\pgfsetfillcolor{textcolor}%
\pgftext[x=1.747499in, y=0.994372in, left, base]{\color{textcolor}{\rmfamily\fontsize{10.000000}{12.000000}\selectfont\catcode`\^=\active\def^{\ifmmode\sp\else\^{}\fi}\catcode`\%=\active\def%{\%}$\mathdefault{10}$}}%
\end{pgfscope}%
\begin{pgfscope}%
\pgfsetbuttcap%
\pgfsetroundjoin%
\definecolor{currentfill}{rgb}{0.000000,0.000000,0.000000}%
\pgfsetfillcolor{currentfill}%
\pgfsetlinewidth{0.803000pt}%
\definecolor{currentstroke}{rgb}{0.000000,0.000000,0.000000}%
\pgfsetstrokecolor{currentstroke}%
\pgfsetdash{}{0pt}%
\pgfsys@defobject{currentmarker}{\pgfqpoint{-0.048611in}{0.000000in}}{\pgfqpoint{-0.000000in}{0.000000in}}{%
\pgfpathmoveto{\pgfqpoint{-0.000000in}{0.000000in}}%
\pgfpathlineto{\pgfqpoint{-0.048611in}{0.000000in}}%
\pgfusepath{stroke,fill}%
}%
\begin{pgfscope}%
\pgfsys@transformshift{1.983610in}{1.425448in}%
\pgfsys@useobject{currentmarker}{}%
\end{pgfscope}%
\end{pgfscope}%
\begin{pgfscope}%
\definecolor{textcolor}{rgb}{0.000000,0.000000,0.000000}%
\pgfsetstrokecolor{textcolor}%
\pgfsetfillcolor{textcolor}%
\pgftext[x=1.747499in, y=1.372687in, left, base]{\color{textcolor}{\rmfamily\fontsize{10.000000}{12.000000}\selectfont\catcode`\^=\active\def^{\ifmmode\sp\else\^{}\fi}\catcode`\%=\active\def%{\%}$\mathdefault{15}$}}%
\end{pgfscope}%
\begin{pgfscope}%
\pgfsetbuttcap%
\pgfsetroundjoin%
\definecolor{currentfill}{rgb}{0.000000,0.000000,0.000000}%
\pgfsetfillcolor{currentfill}%
\pgfsetlinewidth{0.803000pt}%
\definecolor{currentstroke}{rgb}{0.000000,0.000000,0.000000}%
\pgfsetstrokecolor{currentstroke}%
\pgfsetdash{}{0pt}%
\pgfsys@defobject{currentmarker}{\pgfqpoint{-0.048611in}{0.000000in}}{\pgfqpoint{-0.000000in}{0.000000in}}{%
\pgfpathmoveto{\pgfqpoint{-0.000000in}{0.000000in}}%
\pgfpathlineto{\pgfqpoint{-0.048611in}{0.000000in}}%
\pgfusepath{stroke,fill}%
}%
\begin{pgfscope}%
\pgfsys@transformshift{1.983610in}{1.803763in}%
\pgfsys@useobject{currentmarker}{}%
\end{pgfscope}%
\end{pgfscope}%
\begin{pgfscope}%
\definecolor{textcolor}{rgb}{0.000000,0.000000,0.000000}%
\pgfsetstrokecolor{textcolor}%
\pgfsetfillcolor{textcolor}%
\pgftext[x=1.747499in, y=1.751001in, left, base]{\color{textcolor}{\rmfamily\fontsize{10.000000}{12.000000}\selectfont\catcode`\^=\active\def^{\ifmmode\sp\else\^{}\fi}\catcode`\%=\active\def%{\%}$\mathdefault{20}$}}%
\end{pgfscope}%
\begin{pgfscope}%
\definecolor{textcolor}{rgb}{0.000000,0.000000,0.000000}%
\pgfsetstrokecolor{textcolor}%
\pgfsetfillcolor{textcolor}%
\pgftext[x=1.691943in,y=1.153582in,,bottom,rotate=90.000000]{\color{textcolor}{\rmfamily\fontsize{10.000000}{12.000000}\selectfont\catcode`\^=\active\def^{\ifmmode\sp\else\^{}\fi}\catcode`\%=\active\def%{\%}Time (ms)}}%
\end{pgfscope}%
\begin{pgfscope}%
\pgfsetrectcap%
\pgfsetmiterjoin%
\pgfsetlinewidth{0.803000pt}%
\definecolor{currentstroke}{rgb}{0.000000,0.000000,0.000000}%
\pgfsetstrokecolor{currentstroke}%
\pgfsetdash{}{0pt}%
\pgfpathmoveto{\pgfqpoint{1.983610in}{0.290505in}}%
\pgfpathlineto{\pgfqpoint{1.983610in}{2.016660in}}%
\pgfusepath{stroke}%
\end{pgfscope}%
\begin{pgfscope}%
\pgfsetrectcap%
\pgfsetmiterjoin%
\pgfsetlinewidth{0.803000pt}%
\definecolor{currentstroke}{rgb}{0.000000,0.000000,0.000000}%
\pgfsetstrokecolor{currentstroke}%
\pgfsetdash{}{0pt}%
\pgfpathmoveto{\pgfqpoint{4.018122in}{0.290505in}}%
\pgfpathlineto{\pgfqpoint{4.018122in}{2.016660in}}%
\pgfusepath{stroke}%
\end{pgfscope}%
\begin{pgfscope}%
\pgfsetrectcap%
\pgfsetmiterjoin%
\pgfsetlinewidth{0.803000pt}%
\definecolor{currentstroke}{rgb}{0.000000,0.000000,0.000000}%
\pgfsetstrokecolor{currentstroke}%
\pgfsetdash{}{0pt}%
\pgfpathmoveto{\pgfqpoint{1.983610in}{0.290505in}}%
\pgfpathlineto{\pgfqpoint{4.018122in}{0.290505in}}%
\pgfusepath{stroke}%
\end{pgfscope}%
\begin{pgfscope}%
\pgfsetrectcap%
\pgfsetmiterjoin%
\pgfsetlinewidth{0.803000pt}%
\definecolor{currentstroke}{rgb}{0.000000,0.000000,0.000000}%
\pgfsetstrokecolor{currentstroke}%
\pgfsetdash{}{0pt}%
\pgfpathmoveto{\pgfqpoint{1.983610in}{2.016660in}}%
\pgfpathlineto{\pgfqpoint{4.018122in}{2.016660in}}%
\pgfusepath{stroke}%
\end{pgfscope}%
\begin{pgfscope}%
\definecolor{textcolor}{rgb}{0.000000,0.000000,0.000000}%
\pgfsetstrokecolor{textcolor}%
\pgfsetfillcolor{textcolor}%
\pgftext[x=2.261044in,y=0.325940in,,]{\color{textcolor}{\rmfamily\fontsize{5.790000}{6.948000}\selectfont\catcode`\^=\active\def^{\ifmmode\sp\else\^{}\fi}\catcode`\%=\active\def%{\%}0.94 ms}}%
\end{pgfscope}%
\begin{pgfscope}%
\definecolor{textcolor}{rgb}{0.000000,0.000000,0.000000}%
\pgfsetstrokecolor{textcolor}%
\pgfsetfillcolor{textcolor}%
\pgftext[x=2.261044in,y=0.528906in,,]{\color{textcolor}{\rmfamily\fontsize{5.790000}{6.948000}\selectfont\catcode`\^=\active\def^{\ifmmode\sp\else\^{}\fi}\catcode`\%=\active\def%{\%}4.43 ms}}%
\end{pgfscope}%
\begin{pgfscope}%
\definecolor{textcolor}{rgb}{0.000000,0.000000,0.000000}%
\pgfsetstrokecolor{textcolor}%
\pgfsetfillcolor{textcolor}%
\pgftext[x=2.261044in,y=0.753092in,,]{\color{textcolor}{\rmfamily\fontsize{5.790000}{6.948000}\selectfont\catcode`\^=\active\def^{\ifmmode\sp\else\^{}\fi}\catcode`\%=\active\def%{\%}1.50 ms}}%
\end{pgfscope}%
\begin{pgfscope}%
\definecolor{textcolor}{rgb}{0.000000,0.000000,0.000000}%
\pgfsetstrokecolor{textcolor}%
\pgfsetfillcolor{textcolor}%
\pgftext[x=2.261044in,y=0.967374in,,]{\color{textcolor}{\rmfamily\fontsize{5.790000}{6.948000}\selectfont\catcode`\^=\active\def^{\ifmmode\sp\else\^{}\fi}\catcode`\%=\active\def%{\%}4.17 ms}}%
\end{pgfscope}%
\begin{pgfscope}%
\definecolor{textcolor}{rgb}{0.000000,0.000000,0.000000}%
\pgfsetstrokecolor{textcolor}%
\pgfsetfillcolor{textcolor}%
\pgftext[x=2.261044in,y=1.136985in,,]{\color{textcolor}{\rmfamily\fontsize{5.790000}{6.948000}\selectfont\catcode`\^=\active\def^{\ifmmode\sp\else\^{}\fi}\catcode`\%=\active\def%{\%}0.32 ms}}%
\end{pgfscope}%
\begin{pgfscope}%
\definecolor{textcolor}{rgb}{0.000000,0.000000,0.000000}%
\pgfsetstrokecolor{textcolor}%
\pgfsetfillcolor{textcolor}%
\pgftext[x=2.261044in,y=1.157892in,,]{\color{textcolor}{\rmfamily\fontsize{5.790000}{6.948000}\selectfont\catcode`\^=\active\def^{\ifmmode\sp\else\^{}\fi}\catcode`\%=\active\def%{\%}0.24 ms}}%
\end{pgfscope}%
\begin{pgfscope}%
\definecolor{textcolor}{rgb}{0.000000,0.000000,0.000000}%
\pgfsetstrokecolor{textcolor}%
\pgfsetfillcolor{textcolor}%
\pgftext[x=2.261044in,y=1.282509in,,bottom]{\color{textcolor}{\rmfamily\fontsize{8.330000}{9.996000}\bfseries\selectfont\catcode`\^=\active\def^{\ifmmode\sp\else\^{}\fi}\catcode`\%=\active\def%{\%}11.58 ms}}%
\end{pgfscope}%
\begin{pgfscope}%
\definecolor{textcolor}{rgb}{0.000000,0.000000,0.000000}%
\pgfsetstrokecolor{textcolor}%
\pgfsetfillcolor{textcolor}%
\pgftext[x=3.000866in,y=0.320112in,,]{\color{textcolor}{\rmfamily\fontsize{5.790000}{6.948000}\selectfont\catcode`\^=\active\def^{\ifmmode\sp\else\^{}\fi}\catcode`\%=\active\def%{\%}0.78 ms}}%
\end{pgfscope}%
\begin{pgfscope}%
\definecolor{textcolor}{rgb}{0.000000,0.000000,0.000000}%
\pgfsetstrokecolor{textcolor}%
\pgfsetfillcolor{textcolor}%
\pgftext[x=3.000866in,y=0.552300in,,]{\color{textcolor}{\rmfamily\fontsize{5.790000}{6.948000}\selectfont\catcode`\^=\active\def^{\ifmmode\sp\else\^{}\fi}\catcode`\%=\active\def%{\%}5.35 ms}}%
\end{pgfscope}%
\begin{pgfscope}%
\definecolor{textcolor}{rgb}{0.000000,0.000000,0.000000}%
\pgfsetstrokecolor{textcolor}%
\pgfsetfillcolor{textcolor}%
\pgftext[x=3.000866in,y=0.814042in,,]{\color{textcolor}{\rmfamily\fontsize{5.790000}{6.948000}\selectfont\catcode`\^=\active\def^{\ifmmode\sp\else\^{}\fi}\catcode`\%=\active\def%{\%}1.56 ms}}%
\end{pgfscope}%
\begin{pgfscope}%
\definecolor{textcolor}{rgb}{0.000000,0.000000,0.000000}%
\pgfsetstrokecolor{textcolor}%
\pgfsetfillcolor{textcolor}%
\pgftext[x=3.000866in,y=1.201556in,,]{\color{textcolor}{\rmfamily\fontsize{5.790000}{6.948000}\selectfont\catcode`\^=\active\def^{\ifmmode\sp\else\^{}\fi}\catcode`\%=\active\def%{\%}8.68 ms}}%
\end{pgfscope}%
\begin{pgfscope}%
\definecolor{textcolor}{rgb}{0.000000,0.000000,0.000000}%
\pgfsetstrokecolor{textcolor}%
\pgfsetfillcolor{textcolor}%
\pgftext[x=3.000866in,y=1.538768in,,]{\color{textcolor}{\rmfamily\fontsize{5.790000}{6.948000}\selectfont\catcode`\^=\active\def^{\ifmmode\sp\else\^{}\fi}\catcode`\%=\active\def%{\%}0.23 ms}}%
\end{pgfscope}%
\begin{pgfscope}%
\definecolor{textcolor}{rgb}{0.000000,0.000000,0.000000}%
\pgfsetstrokecolor{textcolor}%
\pgfsetfillcolor{textcolor}%
\pgftext[x=3.000866in,y=1.579135in,,]{\color{textcolor}{\rmfamily\fontsize{5.790000}{6.948000}\selectfont\catcode`\^=\active\def^{\ifmmode\sp\else\^{}\fi}\catcode`\%=\active\def%{\%}0.10 ms}}%
\end{pgfscope}%
\begin{pgfscope}%
\definecolor{textcolor}{rgb}{0.000000,0.000000,0.000000}%
\pgfsetstrokecolor{textcolor}%
\pgfsetfillcolor{textcolor}%
\pgftext[x=3.000866in,y=1.670782in,,bottom]{\color{textcolor}{\rmfamily\fontsize{8.330000}{9.996000}\bfseries\selectfont\catcode`\^=\active\def^{\ifmmode\sp\else\^{}\fi}\catcode`\%=\active\def%{\%}16.71 ms}}%
\end{pgfscope}%
\begin{pgfscope}%
\definecolor{textcolor}{rgb}{0.000000,0.000000,0.000000}%
\pgfsetstrokecolor{textcolor}%
\pgfsetfillcolor{textcolor}%
\pgftext[x=3.740688in,y=0.319818in,,]{\color{textcolor}{\rmfamily\fontsize{5.790000}{6.948000}\selectfont\catcode`\^=\active\def^{\ifmmode\sp\else\^{}\fi}\catcode`\%=\active\def%{\%}0.77 ms}}%
\end{pgfscope}%
\begin{pgfscope}%
\definecolor{textcolor}{rgb}{0.000000,0.000000,0.000000}%
\pgfsetstrokecolor{textcolor}%
\pgfsetfillcolor{textcolor}%
\pgftext[x=3.740688in,y=0.551023in,,]{\color{textcolor}{\rmfamily\fontsize{5.790000}{6.948000}\selectfont\catcode`\^=\active\def^{\ifmmode\sp\else\^{}\fi}\catcode`\%=\active\def%{\%}5.34 ms}}%
\end{pgfscope}%
\begin{pgfscope}%
\definecolor{textcolor}{rgb}{0.000000,0.000000,0.000000}%
\pgfsetstrokecolor{textcolor}%
\pgfsetfillcolor{textcolor}%
\pgftext[x=3.740688in,y=0.811701in,,]{\color{textcolor}{\rmfamily\fontsize{5.790000}{6.948000}\selectfont\catcode`\^=\active\def^{\ifmmode\sp\else\^{}\fi}\catcode`\%=\active\def%{\%}1.55 ms}}%
\end{pgfscope}%
\begin{pgfscope}%
\definecolor{textcolor}{rgb}{0.000000,0.000000,0.000000}%
\pgfsetstrokecolor{textcolor}%
\pgfsetfillcolor{textcolor}%
\pgftext[x=3.740688in,y=1.199138in,,]{\color{textcolor}{\rmfamily\fontsize{5.790000}{6.948000}\selectfont\catcode`\^=\active\def^{\ifmmode\sp\else\^{}\fi}\catcode`\%=\active\def%{\%}8.69 ms}}%
\end{pgfscope}%
\begin{pgfscope}%
\definecolor{textcolor}{rgb}{0.000000,0.000000,0.000000}%
\pgfsetstrokecolor{textcolor}%
\pgfsetfillcolor{textcolor}%
\pgftext[x=3.740688in,y=1.536668in,,]{\color{textcolor}{\rmfamily\fontsize{5.790000}{6.948000}\selectfont\catcode`\^=\active\def^{\ifmmode\sp\else\^{}\fi}\catcode`\%=\active\def%{\%}0.23 ms}}%
\end{pgfscope}%
\begin{pgfscope}%
\definecolor{textcolor}{rgb}{0.000000,0.000000,0.000000}%
\pgfsetstrokecolor{textcolor}%
\pgfsetfillcolor{textcolor}%
\pgftext[x=3.740688in,y=1.576868in,,]{\color{textcolor}{\rmfamily\fontsize{5.790000}{6.948000}\selectfont\catcode`\^=\active\def^{\ifmmode\sp\else\^{}\fi}\catcode`\%=\active\def%{\%}0.09 ms}}%
\end{pgfscope}%
\begin{pgfscope}%
\definecolor{textcolor}{rgb}{0.000000,0.000000,0.000000}%
\pgfsetstrokecolor{textcolor}%
\pgfsetfillcolor{textcolor}%
\pgftext[x=3.740688in,y=1.668331in,,bottom]{\color{textcolor}{\rmfamily\fontsize{8.330000}{9.996000}\bfseries\selectfont\catcode`\^=\active\def^{\ifmmode\sp\else\^{}\fi}\catcode`\%=\active\def%{\%}16.68 ms}}%
\end{pgfscope}%
\begin{pgfscope}%
\pgfsetbuttcap%
\pgfsetmiterjoin%
\definecolor{currentfill}{rgb}{1.000000,1.000000,1.000000}%
\pgfsetfillcolor{currentfill}%
\pgfsetfillopacity{0.800000}%
\pgfsetlinewidth{1.003750pt}%
\definecolor{currentstroke}{rgb}{0.800000,0.800000,0.800000}%
\pgfsetstrokecolor{currentstroke}%
\pgfsetstrokeopacity{0.800000}%
\pgfsetdash{}{0pt}%
\pgfpathmoveto{\pgfqpoint{0.056292in}{0.691030in}}%
\pgfpathlineto{\pgfqpoint{1.396854in}{0.691030in}}%
\pgfpathquadraticcurveto{\pgfqpoint{1.412937in}{0.691030in}}{\pgfqpoint{1.412937in}{0.707113in}}%
\pgfpathlineto{\pgfqpoint{1.412937in}{1.409547in}}%
\pgfpathquadraticcurveto{\pgfqpoint{1.412937in}{1.425630in}}{\pgfqpoint{1.396854in}{1.425630in}}%
\pgfpathlineto{\pgfqpoint{0.056292in}{1.425630in}}%
\pgfpathquadraticcurveto{\pgfqpoint{0.040208in}{1.425630in}}{\pgfqpoint{0.040208in}{1.409547in}}%
\pgfpathlineto{\pgfqpoint{0.040208in}{0.707113in}}%
\pgfpathquadraticcurveto{\pgfqpoint{0.040208in}{0.691030in}}{\pgfqpoint{0.056292in}{0.691030in}}%
\pgfpathlineto{\pgfqpoint{0.056292in}{0.691030in}}%
\pgfpathclose%
\pgfusepath{stroke,fill}%
\end{pgfscope}%
\begin{pgfscope}%
\pgfsetbuttcap%
\pgfsetmiterjoin%
\definecolor{currentfill}{rgb}{0.814118,0.883922,0.949804}%
\pgfsetfillcolor{currentfill}%
\pgfsetlinewidth{0.000000pt}%
\definecolor{currentstroke}{rgb}{0.000000,0.000000,0.000000}%
\pgfsetstrokecolor{currentstroke}%
\pgfsetstrokeopacity{0.000000}%
\pgfsetdash{}{0pt}%
\pgfpathmoveto{\pgfqpoint{0.072375in}{1.332366in}}%
\pgfpathlineto{\pgfqpoint{0.233208in}{1.332366in}}%
\pgfpathlineto{\pgfqpoint{0.233208in}{1.388658in}}%
\pgfpathlineto{\pgfqpoint{0.072375in}{1.388658in}}%
\pgfpathlineto{\pgfqpoint{0.072375in}{1.332366in}}%
\pgfpathclose%
\pgfusepath{fill}%
\end{pgfscope}%
\begin{pgfscope}%
\definecolor{textcolor}{rgb}{0.000000,0.000000,0.000000}%
\pgfsetstrokecolor{textcolor}%
\pgfsetfillcolor{textcolor}%
\pgftext[x=0.297542in,y=1.332366in,left,base]{\color{textcolor}{\rmfamily\fontsize{5.790000}{6.948000}\selectfont\catcode`\^=\active\def^{\ifmmode\sp\else\^{}\fi}\catcode`\%=\active\def%{\%}forwardSampleGeneration}}%
\end{pgfscope}%
\begin{pgfscope}%
\pgfsetbuttcap%
\pgfsetmiterjoin%
\definecolor{currentfill}{rgb}{0.887059,0.887059,0.887059}%
\pgfsetfillcolor{currentfill}%
\pgfsetlinewidth{0.000000pt}%
\definecolor{currentstroke}{rgb}{0.000000,0.000000,0.000000}%
\pgfsetstrokecolor{currentstroke}%
\pgfsetstrokeopacity{0.000000}%
\pgfsetdash{}{0pt}%
\pgfpathmoveto{\pgfqpoint{0.072375in}{1.214333in}}%
\pgfpathlineto{\pgfqpoint{0.233208in}{1.214333in}}%
\pgfpathlineto{\pgfqpoint{0.233208in}{1.270625in}}%
\pgfpathlineto{\pgfqpoint{0.072375in}{1.270625in}}%
\pgfpathlineto{\pgfqpoint{0.072375in}{1.214333in}}%
\pgfpathclose%
\pgfusepath{fill}%
\end{pgfscope}%
\begin{pgfscope}%
\definecolor{textcolor}{rgb}{0.000000,0.000000,0.000000}%
\pgfsetstrokecolor{textcolor}%
\pgfsetfillcolor{textcolor}%
\pgftext[x=0.297542in,y=1.214333in,left,base]{\color{textcolor}{\rmfamily\fontsize{5.790000}{6.948000}\selectfont\catcode`\^=\active\def^{\ifmmode\sp\else\^{}\fi}\catcode`\%=\active\def%{\%}training}}%
\end{pgfscope}%
\begin{pgfscope}%
\pgfsetbuttcap%
\pgfsetmiterjoin%
\definecolor{currentfill}{rgb}{0.710588,0.710588,0.710588}%
\pgfsetfillcolor{currentfill}%
\pgfsetlinewidth{0.000000pt}%
\definecolor{currentstroke}{rgb}{0.000000,0.000000,0.000000}%
\pgfsetstrokecolor{currentstroke}%
\pgfsetstrokeopacity{0.000000}%
\pgfsetdash{}{0pt}%
\pgfpathmoveto{\pgfqpoint{0.072375in}{1.095161in}}%
\pgfpathlineto{\pgfqpoint{0.233208in}{1.095161in}}%
\pgfpathlineto{\pgfqpoint{0.233208in}{1.151453in}}%
\pgfpathlineto{\pgfqpoint{0.072375in}{1.151453in}}%
\pgfpathlineto{\pgfqpoint{0.072375in}{1.095161in}}%
\pgfpathclose%
\pgfusepath{fill}%
\end{pgfscope}%
\begin{pgfscope}%
\definecolor{textcolor}{rgb}{0.000000,0.000000,0.000000}%
\pgfsetstrokecolor{textcolor}%
\pgfsetfillcolor{textcolor}%
\pgftext[x=0.297542in,y=1.095161in,left,base]{\color{textcolor}{\rmfamily\fontsize{5.790000}{6.948000}\selectfont\catcode`\^=\active\def^{\ifmmode\sp\else\^{}\fi}\catcode`\%=\active\def%{\%}pathtracing}}%
\end{pgfscope}%
\begin{pgfscope}%
\pgfsetbuttcap%
\pgfsetmiterjoin%
\definecolor{currentfill}{rgb}{0.478431,0.478431,0.478431}%
\pgfsetfillcolor{currentfill}%
\pgfsetlinewidth{0.000000pt}%
\definecolor{currentstroke}{rgb}{0.000000,0.000000,0.000000}%
\pgfsetstrokecolor{currentstroke}%
\pgfsetstrokeopacity{0.000000}%
\pgfsetdash{}{0pt}%
\pgfpathmoveto{\pgfqpoint{0.072375in}{0.975990in}}%
\pgfpathlineto{\pgfqpoint{0.233208in}{0.975990in}}%
\pgfpathlineto{\pgfqpoint{0.233208in}{1.032281in}}%
\pgfpathlineto{\pgfqpoint{0.072375in}{1.032281in}}%
\pgfpathlineto{\pgfqpoint{0.072375in}{0.975990in}}%
\pgfpathclose%
\pgfusepath{fill}%
\end{pgfscope}%
\begin{pgfscope}%
\definecolor{textcolor}{rgb}{0.000000,0.000000,0.000000}%
\pgfsetstrokecolor{textcolor}%
\pgfsetfillcolor{textcolor}%
\pgftext[x=0.297542in,y=0.975990in,left,base]{\color{textcolor}{\rmfamily\fontsize{5.790000}{6.948000}\selectfont\catcode`\^=\active\def^{\ifmmode\sp\else\^{}\fi}\catcode`\%=\active\def%{\%}inference}}%
\end{pgfscope}%
\begin{pgfscope}%
\pgfsetbuttcap%
\pgfsetmiterjoin%
\definecolor{currentfill}{rgb}{1.000000,0.752941,0.796078}%
\pgfsetfillcolor{currentfill}%
\pgfsetlinewidth{0.000000pt}%
\definecolor{currentstroke}{rgb}{0.000000,0.000000,0.000000}%
\pgfsetstrokecolor{currentstroke}%
\pgfsetstrokeopacity{0.000000}%
\pgfsetdash{}{0pt}%
\pgfpathmoveto{\pgfqpoint{0.072375in}{0.857957in}}%
\pgfpathlineto{\pgfqpoint{0.233208in}{0.857957in}}%
\pgfpathlineto{\pgfqpoint{0.233208in}{0.914248in}}%
\pgfpathlineto{\pgfqpoint{0.072375in}{0.914248in}}%
\pgfpathlineto{\pgfqpoint{0.072375in}{0.857957in}}%
\pgfpathclose%
\pgfusepath{fill}%
\end{pgfscope}%
\begin{pgfscope}%
\definecolor{textcolor}{rgb}{0.000000,0.000000,0.000000}%
\pgfsetstrokecolor{textcolor}%
\pgfsetfillcolor{textcolor}%
\pgftext[x=0.297542in,y=0.857957in,left,base]{\color{textcolor}{\rmfamily\fontsize{5.790000}{6.948000}\selectfont\catcode`\^=\active\def^{\ifmmode\sp\else\^{}\fi}\catcode`\%=\active\def%{\%}visualization}}%
\end{pgfscope}%
\begin{pgfscope}%
\pgfsetbuttcap%
\pgfsetmiterjoin%
\definecolor{currentfill}{rgb}{0.854902,0.439216,0.839216}%
\pgfsetfillcolor{currentfill}%
\pgfsetlinewidth{0.000000pt}%
\definecolor{currentstroke}{rgb}{0.000000,0.000000,0.000000}%
\pgfsetstrokecolor{currentstroke}%
\pgfsetstrokeopacity{0.000000}%
\pgfsetdash{}{0pt}%
\pgfpathmoveto{\pgfqpoint{0.072375in}{0.739923in}}%
\pgfpathlineto{\pgfqpoint{0.233208in}{0.739923in}}%
\pgfpathlineto{\pgfqpoint{0.233208in}{0.796215in}}%
\pgfpathlineto{\pgfqpoint{0.072375in}{0.796215in}}%
\pgfpathlineto{\pgfqpoint{0.072375in}{0.739923in}}%
\pgfpathclose%
\pgfusepath{fill}%
\end{pgfscope}%
\begin{pgfscope}%
\definecolor{textcolor}{rgb}{0.000000,0.000000,0.000000}%
\pgfsetstrokecolor{textcolor}%
\pgfsetfillcolor{textcolor}%
\pgftext[x=0.297542in,y=0.739923in,left,base]{\color{textcolor}{\rmfamily\fontsize{5.790000}{6.948000}\selectfont\catcode`\^=\active\def^{\ifmmode\sp\else\^{}\fi}\catcode`\%=\active\def%{\%}other}}%
\end{pgfscope}%
\end{pgfpicture}%
\makeatother%
\endgroup%

    \caption{Applying JIT kernel fusion. \emph{+Fused} indicates fusion of the encoding with the training and inference kernels, \emph{+FusedVis} also fuses the visualization step, where the output of the network is multiplied with the reflectance and accumulated to the render buffer. Unfortunately, enabling JIT-fusion \emph{decreases} performance, probably because of the older GPU architecture.}
    \label{fig:jit}
\end{figure}
\paragraph{Kernel Fusion} Recently, the \emph{tiny-cuda-nn} library of \textcite{muller2021a} was extended by a feature that allows combining the encoding, MLP, loss function, training and inference steps together into fused monolithic kernels by on-the-fly JIT compilation.
Using this approach, \textcite{muller2021a} report a potential speedup of $1.5\times$ up to $2.5\times$ for training and inference, especially for modern GPU architectures.
Measurements on an NVIDIA RTX 3060 Ti, however, show a performance \emph{decrease} in the inference step and similar performance for training (see \cref{fig:jit}).
This could potentially be caused by an older GPU architecture, the following tests are thus performed without JIT-fusion.

\begin{figure}[htb!]
    \centering
    \begin{subfigure}{0.5\textwidth}
        \centering
        \tiny
        \begin{tabularx}{\linewidth}{r*{4}{>{\centering\arraybackslash}X}}
            &Reference (SPPM) & SER & SER, 70\% & SER, 70\%, 30° \\
            &2749spp (2m)
 & 1spp (116.80ms)
 & 1spp (\textbf{93.93ms})
 & 1spp (103.69ms)
\\
\rotatebox{90}{\textsc{Caustics}}\hspace{-1.5em}
&\includegraphics[width=\linewidth]{figures/py/tests/photon_optimization/ref_2min.png}
& \includegraphics[width=\linewidth]{figures/py/tests/photon_optimization/SER_1spp.png}
& \includegraphics[width=\linewidth]{figures/py/tests/photon_optimization/SER+Reject70_1spp.png}
& \includegraphics[width=\linewidth]{figures/py/tests/photon_optimization/SER+Reject70+RejectN_1spp.png}
\\
&& \includegraphics[width=\linewidth]{figures/py/tests/photon_optimization/SER_1spp_flip.png}
& \includegraphics[width=\linewidth]{figures/py/tests/photon_optimization/SER+Reject70_1spp_flip.png}
& \includegraphics[width=\linewidth]{figures/py/tests/photon_optimization/SER+Reject70+RejectN_1spp_flip.png}
\\
&\FLIP/MSE: & \textbf{\num{0.272}}/\textbf{\num{973.693}}
 & \num{0.347}/\num{973.694}
 & \num{0.349}/\num{973.694}
\\
&$\mathrm{Bias}^2/\mathrm{Variance}$ & \textbf{\num{2.11e-07}}/\textbf{\num{4.24e+02}}
 & \num{2.32e-06}/\num{5.24e+02}
 & \num{1.57e-06}/\num{5.07e+02}
\\

        \end{tabularx}
    \end{subfigure}%
    \begin{subfigure}{0.5\textwidth}
        \centering
        \small
        \begin{tabular}{lll}
            \textbf{Technique} & \textbf{Frametime} & \textbf{Change} \\
            \midrule
            \textbf{Baseline} & \textbf{159.50ms} & \\
            IS only & 178.19ms & $+11.7\%$\\
            SER & 158.54ms & $-0.6\%$\\
            SER, 70\% & 124.46ms & $-22\%$\\
            SER, 70\%, 30$^{\circ}$ & 123.58ms & $-22.5\%$
        \end{tabular}
    \end{subfigure}
    \caption{Optimizations to Photon Mapping. \emph{IS only} directly accumulates in the Intersection Shader (IS) and skips the Any Hit Shader (AH). \emph{SER} uses Shader Execution Reordering between the Intersection and Any Hit shader to improve coherence. \emph{SER, 70\%} additionally rejects 70\% of the non-caustic photons in the Any Hit shader \parencite{kern2023}. \emph{SER, 70\%, 30$^{\circ}$} rejects photon hits whose surface normals deviate more than 30° from the surface normals at the query point \parencite{kern2023}. The most notable performance improvement comes from rejecting non-caustic photons, however, this notably increases error in non-caustic areas. Rejecting photons based on normal deviation has a small positive performance impact and decreases bias.}
    \label{fig:photon_optimization}
\end{figure}
\paragraph{Photon Mapping} To optimize SPPC, several optimizations were tested (see \cref{fig:photon_optimization}).

\section{Evaluation}

\begin{figure}[htb!]
    \centering
    \tiny
    \begin{tabularx}{\textwidth}{r*{9}{>{\centering\arraybackslash}X}}
        &Reference & PT & NRC+PT & NRC+PT+SL & NRC+BT & NRC+LT & NRC+LT+Bal & NRC+SPPC & PM \\
        \input{figures/py/tests/quality_comparison/Diffuse}\\
        &8608spp (3m)
 & 1spp (\textbf{4.96ms})
 & 1spp (10.80ms)
\\
\rotatebox{90}{\textsc{Thinker}}\hspace{-1.5em}
&\includegraphics[width=\linewidth]{figures/py/tests/encodings/../quality_comparison/refpt_3min_thinker.png}
& \includegraphics[width=\linewidth]{figures/py/tests/encodings/nrc+ptTWE_1spp.png}
& \includegraphics[width=\linewidth]{figures/py/tests/encodings/nrc+ptMHE_1spp.png}
\\
&& \includegraphics[width=\linewidth]{figures/py/tests/encodings/nrc+ptTWE_1spp_flip.png}
& \includegraphics[width=\linewidth]{figures/py/tests/encodings/nrc+ptMHE_1spp_flip.png}
\\
&\FLIP/MSE: & \num{0.398}/\num{0.014}
 & \textbf{\num{0.359}}/\textbf{\num{0.014}}
\\
&$\mathrm{Bias}^2/\mathrm{Variance}$ & \num{1.70e-04}/\textbf{\num{1.48e-02}}
 & \textbf{\num{1.60e-04}}/\num{1.48e-02}
\\
\\
        \input{figures/py/tests/quality_comparison/Chess}\\
        \input{figures/py/tests/quality_comparison/Ajar}\\
        &2749spp (2m)
 & 1spp (116.80ms)
 & 1spp (\textbf{93.93ms})
 & 1spp (103.69ms)
\\
\rotatebox{90}{\textsc{Caustics}}\hspace{-1.5em}
&\includegraphics[width=\linewidth]{figures/py/tests/photon_optimization/ref_2min.png}
& \includegraphics[width=\linewidth]{figures/py/tests/photon_optimization/SER_1spp.png}
& \includegraphics[width=\linewidth]{figures/py/tests/photon_optimization/SER+Reject70_1spp.png}
& \includegraphics[width=\linewidth]{figures/py/tests/photon_optimization/SER+Reject70+RejectN_1spp.png}
\\
&& \includegraphics[width=\linewidth]{figures/py/tests/photon_optimization/SER_1spp_flip.png}
& \includegraphics[width=\linewidth]{figures/py/tests/photon_optimization/SER+Reject70_1spp_flip.png}
& \includegraphics[width=\linewidth]{figures/py/tests/photon_optimization/SER+Reject70+RejectN_1spp_flip.png}
\\
&\FLIP/MSE: & \textbf{\num{0.272}}/\textbf{\num{973.693}}
 & \num{0.347}/\num{973.694}
 & \num{0.349}/\num{973.694}
\\
&$\mathrm{Bias}^2/\mathrm{Variance}$ & \textbf{\num{2.11e-07}}/\textbf{\num{4.24e+02}}
 & \num{2.32e-06}/\num{5.24e+02}
 & \num{1.57e-06}/\num{5.07e+02}
\\

    \end{tabularx}
    \caption{Comparison of the different radiance estimators from \cref{chap:bidirectional_caching}. To isolate training quality, inference is terminated after the first diffuse vertex and is not combined with NEE.}
    \label{fig:quality_comparison}
\end{figure}

\begin{figure}[htb!]
    \centering
    \tiny
    \begin{tabularx}{\textwidth}{r*{7}{>{\centering\arraybackslash}X}}
        &Reference & NRC+LT & NRC+LT+Bal & NRC+LT+BalCam & NRC+Naïve & NRC+Naïve+Bal & NRC+Naïve+BalCam \\
        &13398spp (3m)
 & 1spp (11.36ms)
 & 1spp (13.12ms)
 & 1spp (11.15ms)
 & 1spp (\textbf{9.72ms})
 & 1spp (11.88ms)
 & 1spp (12.12ms)
\\
\rotatebox{90}{\textsc{DiffuseBal}}\hspace{-1.5em}
&\includegraphics[width=\linewidth]{figures/py/tests/quality_comparison/refpt_3min_diffuse.png}
& \includegraphics[width=\linewidth]{figures/py/tests/quality_comparison/nrc+lt_1spp_diffuse.png}
& \includegraphics[width=\linewidth]{figures/py/tests/quality_comparison/nrc+lt+bal_1spp_diffuse.png}
& \includegraphics[width=\linewidth]{figures/py/tests/quality_comparison/nrc+lt+balcam_1spp_diffuse.png}
& \includegraphics[width=\linewidth]{figures/py/tests/quality_comparison/nrc+naive_1spp_diffuse.png}
& \includegraphics[width=\linewidth]{figures/py/tests/quality_comparison/nrc+naive+bal_1spp_diffuse.png}
& \includegraphics[width=\linewidth]{figures/py/tests/quality_comparison/nrc+naive+balcam_1spp_diffuse.png}
\\
&& \includegraphics[width=\linewidth]{figures/py/tests/quality_comparison/nrc+lt_1spp_diffuse_flip.png}
& \includegraphics[width=\linewidth]{figures/py/tests/quality_comparison/nrc+lt+bal_1spp_diffuse_flip.png}
& \includegraphics[width=\linewidth]{figures/py/tests/quality_comparison/nrc+lt+balcam_1spp_diffuse_flip.png}
& \includegraphics[width=\linewidth]{figures/py/tests/quality_comparison/nrc+naive_1spp_diffuse_flip.png}
& \includegraphics[width=\linewidth]{figures/py/tests/quality_comparison/nrc+naive+bal_1spp_diffuse_flip.png}
& \includegraphics[width=\linewidth]{figures/py/tests/quality_comparison/nrc+naive+balcam_1spp_diffuse_flip.png}
\\
&\FLIP/MSE: & \num{0.614}/\num{0.019}
 & \num{0.967}/\num{0.021}
 & \num{0.966}/\num{0.021}
 & \textbf{\num{0.534}}/\textbf{\num{0.014}}
 & \num{0.966}/\num{0.021}
 & \num{0.966}/\num{0.021}
\\
&$\mathrm{Bias}^2/\mathrm{Variance}$ & \num{5.89e-03}/\textbf{\num{8.47e-03}}
 & \num{7.61e-03}/\num{8.49e-03}
 & \num{7.56e-03}/\num{8.49e-03}
 & \textbf{\num{3.30e-03}}/\num{8.51e-03}
 & \num{7.58e-03}/\num{8.49e-03}
 & \num{7.56e-03}/\num{8.49e-03}
\\
\\
        &8608spp (3m)
 & 1spp (13.31ms)
 & 1spp (14.59ms)
 & 1spp (13.91ms)
 & 1spp (\textbf{10.57ms})
 & 1spp (13.21ms)
 & 1spp (12.13ms)
\\
\rotatebox{90}{\textsc{ThinkerBal}}\hspace{-1.5em}
&\includegraphics[width=\linewidth]{figures/py/tests/quality_comparison/refpt_3min_thinker.png}
& \includegraphics[width=\linewidth]{figures/py/tests/quality_comparison/nrc+lt_1spp_thinker.png}
& \includegraphics[width=\linewidth]{figures/py/tests/quality_comparison/nrc+lt+bal_1spp_thinker.png}
& \includegraphics[width=\linewidth]{figures/py/tests/quality_comparison/nrc+lt+balcam_1spp_thinker.png}
& \includegraphics[width=\linewidth]{figures/py/tests/quality_comparison/nrc+naive_1spp_thinker.png}
& \includegraphics[width=\linewidth]{figures/py/tests/quality_comparison/nrc+naive+bal_1spp_thinker.png}
& \includegraphics[width=\linewidth]{figures/py/tests/quality_comparison/nrc+naive+balcam_1spp_thinker.png}
\\
&& \includegraphics[width=\linewidth]{figures/py/tests/quality_comparison/nrc+lt_1spp_thinker_flip.png}
& \includegraphics[width=\linewidth]{figures/py/tests/quality_comparison/nrc+lt+bal_1spp_thinker_flip.png}
& \includegraphics[width=\linewidth]{figures/py/tests/quality_comparison/nrc+lt+balcam_1spp_thinker_flip.png}
& \includegraphics[width=\linewidth]{figures/py/tests/quality_comparison/nrc+naive_1spp_thinker_flip.png}
& \includegraphics[width=\linewidth]{figures/py/tests/quality_comparison/nrc+naive+bal_1spp_thinker_flip.png}
& \includegraphics[width=\linewidth]{figures/py/tests/quality_comparison/nrc+naive+balcam_1spp_thinker_flip.png}
\\
&\FLIP/MSE: & \num{0.915}/\num{0.559}
 & \num{0.958}/\num{0.220}
 & \num{0.957}/\num{0.220}
 & \textbf{\num{0.618}}/\num{0.203}
 & \num{0.957}/\textbf{\num{0.083}}
 & \num{0.954}/\num{0.086}
\\
&$\mathrm{Bias}^2/\mathrm{Variance}$ & \num{1.78e-01}/\num{1.10e-01}
 & \num{2.47e-04}/\num{1.14e-01}
 & \textbf{\num{2.26e-04}}/\num{1.14e-01}
 & \num{1.79e-02}/\num{6.04e-02}
 & \num{1.77e-03}/\textbf{\num{4.46e-02}}
 & \num{1.75e-03}/\num{4.59e-02}
\\
\\
        &3032spp (2m)
 & 1spp (13.38ms)
 & 1spp (14.49ms)
 & 1spp (14.11ms)
 & 1spp (\textbf{10.15ms})
 & 1spp (12.37ms)
 & 1spp (12.69ms)
\\
\rotatebox{90}{\textsc{CausticsBal}}\hspace{-1.5em}
&\includegraphics[width=\linewidth]{figures/py/tests/quality_comparison/refsppm_2min.png}
& \includegraphics[width=\linewidth]{figures/py/tests/quality_comparison/nrc+lt_1spp_caustics_small.png}
& \includegraphics[width=\linewidth]{figures/py/tests/quality_comparison/nrc+lt+bal_1spp_caustics_small.png}
& \includegraphics[width=\linewidth]{figures/py/tests/quality_comparison/nrc+lt+balcam_1spp_caustics_small.png}
& \includegraphics[width=\linewidth]{figures/py/tests/quality_comparison/nrc+naive_1spp_caustics_small.png}
& \includegraphics[width=\linewidth]{figures/py/tests/quality_comparison/nrc+naive+bal_1spp_caustics_small.png}
& \includegraphics[width=\linewidth]{figures/py/tests/quality_comparison/nrc+naive+balcam_1spp_caustics_small.png}
\\
&& \includegraphics[width=\linewidth]{figures/py/tests/quality_comparison/nrc+lt_1spp_caustics_small_flip.png}
& \includegraphics[width=\linewidth]{figures/py/tests/quality_comparison/nrc+lt+bal_1spp_caustics_small_flip.png}
& \includegraphics[width=\linewidth]{figures/py/tests/quality_comparison/nrc+lt+balcam_1spp_caustics_small_flip.png}
& \includegraphics[width=\linewidth]{figures/py/tests/quality_comparison/nrc+naive_1spp_caustics_small_flip.png}
& \includegraphics[width=\linewidth]{figures/py/tests/quality_comparison/nrc+naive+bal_1spp_caustics_small_flip.png}
& \includegraphics[width=\linewidth]{figures/py/tests/quality_comparison/nrc+naive+balcam_1spp_caustics_small_flip.png}
\\
&\FLIP/MSE: & \num{0.964}/\num{416.182}
 & \num{0.982}/\num{438.908}
 & \num{0.979}/\num{411.184}
 & \textbf{\num{0.568}}/\textbf{\num{405.410}}
 & \num{0.954}/\num{406.380}
 & \num{0.954}/\num{431.758}
\\
&$\mathrm{Bias}^2/\mathrm{Variance}$ & \num{9.75e+00}/\textbf{\num{6.15e+02}}
 & \num{9.95e-02}/\num{6.76e+02}
 & \num{8.53e-02}/\num{6.26e+02}
 & \num{1.37e-02}/\num{6.39e+02}
 & \num{7.50e-04}/\num{6.27e+02}
 & \textbf{\num{6.38e-04}}/\num{6.36e+02}
\\
\\
    \end{tabularx}
    \caption{Comparing light training and balancing strategies specifically.}
    \label{fig:light_training_comparison}
\end{figure}

\begin{figure}[htb!]
    \centering
    %% Creator: Matplotlib, PGF backend
%%
%% To include the figure in your LaTeX document, write
%%   \input{<filename>.pgf}
%%
%% Make sure the required packages are loaded in your preamble
%%   \usepackage{pgf}
%%
%% Also ensure that all the required font packages are loaded; for instance,
%% the lmodern package is sometimes necessary when using math font.
%%   \usepackage{lmodern}
%%
%% Figures using additional raster images can only be included by \input if
%% they are in the same directory as the main LaTeX file. For loading figures
%% from other directories you can use the `import` package
%%   \usepackage{import}
%%
%% and then include the figures with
%%   \import{<path to file>}{<filename>.pgf}
%%
%% Matplotlib used the following preamble
%%   \def\mathdefault#1{#1}
%%   \everymath=\expandafter{\the\everymath\displaystyle}
%%   \IfFileExists{scrextend.sty}{
%%     \usepackage[fontsize=6.000000pt]{scrextend}
%%   }{
%%     \renewcommand{\normalsize}{\fontsize{6.000000}{7.200000}\selectfont}
%%     \normalsize
%%   }
%%   
%%   \ifdefined\pdftexversion\else  % non-pdftex case.
%%     \usepackage{fontspec}
%%     \setmainfont{DejaVuSerif.ttf}[Path=\detokenize{/opt/homebrew/Cellar/python-matplotlib/3.10.5/libexec/lib/python3.13/site-packages/matplotlib/mpl-data/fonts/ttf/}]
%%     \setsansfont{DejaVuSans.ttf}[Path=\detokenize{/opt/homebrew/Cellar/python-matplotlib/3.10.5/libexec/lib/python3.13/site-packages/matplotlib/mpl-data/fonts/ttf/}]
%%     \setmonofont{DejaVuSansMono.ttf}[Path=\detokenize{/opt/homebrew/Cellar/python-matplotlib/3.10.5/libexec/lib/python3.13/site-packages/matplotlib/mpl-data/fonts/ttf/}]
%%   \fi
%%   \makeatletter\@ifpackageloaded{underscore}{}{\usepackage[strings]{underscore}}\makeatother
%%
\begingroup%
\makeatletter%
\begin{pgfpicture}%
\pgfpathrectangle{\pgfpointorigin}{\pgfqpoint{6.042836in}{1.321749in}}%
\pgfusepath{use as bounding box, clip}%
\begin{pgfscope}%
\pgfsetbuttcap%
\pgfsetmiterjoin%
\definecolor{currentfill}{rgb}{1.000000,1.000000,1.000000}%
\pgfsetfillcolor{currentfill}%
\pgfsetlinewidth{0.000000pt}%
\definecolor{currentstroke}{rgb}{1.000000,1.000000,1.000000}%
\pgfsetstrokecolor{currentstroke}%
\pgfsetdash{}{0pt}%
\pgfpathmoveto{\pgfqpoint{0.000000in}{0.000000in}}%
\pgfpathlineto{\pgfqpoint{6.042836in}{0.000000in}}%
\pgfpathlineto{\pgfqpoint{6.042836in}{1.321749in}}%
\pgfpathlineto{\pgfqpoint{0.000000in}{1.321749in}}%
\pgfpathlineto{\pgfqpoint{0.000000in}{0.000000in}}%
\pgfpathclose%
\pgfusepath{fill}%
\end{pgfscope}%
\begin{pgfscope}%
\pgfsetbuttcap%
\pgfsetmiterjoin%
\definecolor{currentfill}{rgb}{1.000000,1.000000,1.000000}%
\pgfsetfillcolor{currentfill}%
\pgfsetlinewidth{0.000000pt}%
\definecolor{currentstroke}{rgb}{0.000000,0.000000,0.000000}%
\pgfsetstrokecolor{currentstroke}%
\pgfsetstrokeopacity{0.000000}%
\pgfsetdash{}{0pt}%
\pgfpathmoveto{\pgfqpoint{0.517836in}{0.420092in}}%
\pgfpathlineto{\pgfqpoint{5.942836in}{0.420092in}}%
\pgfpathlineto{\pgfqpoint{5.942836in}{1.190092in}}%
\pgfpathlineto{\pgfqpoint{0.517836in}{1.190092in}}%
\pgfpathlineto{\pgfqpoint{0.517836in}{0.420092in}}%
\pgfpathclose%
\pgfusepath{fill}%
\end{pgfscope}%
\begin{pgfscope}%
\pgfsetbuttcap%
\pgfsetroundjoin%
\definecolor{currentfill}{rgb}{0.000000,0.000000,0.000000}%
\pgfsetfillcolor{currentfill}%
\pgfsetlinewidth{0.803000pt}%
\definecolor{currentstroke}{rgb}{0.000000,0.000000,0.000000}%
\pgfsetstrokecolor{currentstroke}%
\pgfsetdash{}{0pt}%
\pgfsys@defobject{currentmarker}{\pgfqpoint{0.000000in}{-0.048611in}}{\pgfqpoint{0.000000in}{0.000000in}}{%
\pgfpathmoveto{\pgfqpoint{0.000000in}{0.000000in}}%
\pgfpathlineto{\pgfqpoint{0.000000in}{-0.048611in}}%
\pgfusepath{stroke,fill}%
}%
\begin{pgfscope}%
\pgfsys@transformshift{0.764427in}{0.420092in}%
\pgfsys@useobject{currentmarker}{}%
\end{pgfscope}%
\end{pgfscope}%
\begin{pgfscope}%
\definecolor{textcolor}{rgb}{0.000000,0.000000,0.000000}%
\pgfsetstrokecolor{textcolor}%
\pgfsetfillcolor{textcolor}%
\pgftext[x=0.764427in,y=0.322870in,,top]{\color{textcolor}{\rmfamily\fontsize{6.000000}{7.200000}\selectfont\catcode`\^=\active\def^{\ifmmode\sp\else\^{}\fi}\catcode`\%=\active\def%{\%}$\mathdefault{10^{0}}$}}%
\end{pgfscope}%
\begin{pgfscope}%
\pgfsetbuttcap%
\pgfsetroundjoin%
\definecolor{currentfill}{rgb}{0.000000,0.000000,0.000000}%
\pgfsetfillcolor{currentfill}%
\pgfsetlinewidth{0.803000pt}%
\definecolor{currentstroke}{rgb}{0.000000,0.000000,0.000000}%
\pgfsetstrokecolor{currentstroke}%
\pgfsetdash{}{0pt}%
\pgfsys@defobject{currentmarker}{\pgfqpoint{0.000000in}{-0.048611in}}{\pgfqpoint{0.000000in}{0.000000in}}{%
\pgfpathmoveto{\pgfqpoint{0.000000in}{0.000000in}}%
\pgfpathlineto{\pgfqpoint{0.000000in}{-0.048611in}}%
\pgfusepath{stroke,fill}%
}%
\begin{pgfscope}%
\pgfsys@transformshift{2.425821in}{0.420092in}%
\pgfsys@useobject{currentmarker}{}%
\end{pgfscope}%
\end{pgfscope}%
\begin{pgfscope}%
\definecolor{textcolor}{rgb}{0.000000,0.000000,0.000000}%
\pgfsetstrokecolor{textcolor}%
\pgfsetfillcolor{textcolor}%
\pgftext[x=2.425821in,y=0.322870in,,top]{\color{textcolor}{\rmfamily\fontsize{6.000000}{7.200000}\selectfont\catcode`\^=\active\def^{\ifmmode\sp\else\^{}\fi}\catcode`\%=\active\def%{\%}$\mathdefault{10^{1}}$}}%
\end{pgfscope}%
\begin{pgfscope}%
\pgfsetbuttcap%
\pgfsetroundjoin%
\definecolor{currentfill}{rgb}{0.000000,0.000000,0.000000}%
\pgfsetfillcolor{currentfill}%
\pgfsetlinewidth{0.803000pt}%
\definecolor{currentstroke}{rgb}{0.000000,0.000000,0.000000}%
\pgfsetstrokecolor{currentstroke}%
\pgfsetdash{}{0pt}%
\pgfsys@defobject{currentmarker}{\pgfqpoint{0.000000in}{-0.048611in}}{\pgfqpoint{0.000000in}{0.000000in}}{%
\pgfpathmoveto{\pgfqpoint{0.000000in}{0.000000in}}%
\pgfpathlineto{\pgfqpoint{0.000000in}{-0.048611in}}%
\pgfusepath{stroke,fill}%
}%
\begin{pgfscope}%
\pgfsys@transformshift{4.087214in}{0.420092in}%
\pgfsys@useobject{currentmarker}{}%
\end{pgfscope}%
\end{pgfscope}%
\begin{pgfscope}%
\definecolor{textcolor}{rgb}{0.000000,0.000000,0.000000}%
\pgfsetstrokecolor{textcolor}%
\pgfsetfillcolor{textcolor}%
\pgftext[x=4.087214in,y=0.322870in,,top]{\color{textcolor}{\rmfamily\fontsize{6.000000}{7.200000}\selectfont\catcode`\^=\active\def^{\ifmmode\sp\else\^{}\fi}\catcode`\%=\active\def%{\%}$\mathdefault{10^{2}}$}}%
\end{pgfscope}%
\begin{pgfscope}%
\pgfsetbuttcap%
\pgfsetroundjoin%
\definecolor{currentfill}{rgb}{0.000000,0.000000,0.000000}%
\pgfsetfillcolor{currentfill}%
\pgfsetlinewidth{0.803000pt}%
\definecolor{currentstroke}{rgb}{0.000000,0.000000,0.000000}%
\pgfsetstrokecolor{currentstroke}%
\pgfsetdash{}{0pt}%
\pgfsys@defobject{currentmarker}{\pgfqpoint{0.000000in}{-0.048611in}}{\pgfqpoint{0.000000in}{0.000000in}}{%
\pgfpathmoveto{\pgfqpoint{0.000000in}{0.000000in}}%
\pgfpathlineto{\pgfqpoint{0.000000in}{-0.048611in}}%
\pgfusepath{stroke,fill}%
}%
\begin{pgfscope}%
\pgfsys@transformshift{5.748608in}{0.420092in}%
\pgfsys@useobject{currentmarker}{}%
\end{pgfscope}%
\end{pgfscope}%
\begin{pgfscope}%
\definecolor{textcolor}{rgb}{0.000000,0.000000,0.000000}%
\pgfsetstrokecolor{textcolor}%
\pgfsetfillcolor{textcolor}%
\pgftext[x=5.748608in,y=0.322870in,,top]{\color{textcolor}{\rmfamily\fontsize{6.000000}{7.200000}\selectfont\catcode`\^=\active\def^{\ifmmode\sp\else\^{}\fi}\catcode`\%=\active\def%{\%}$\mathdefault{10^{3}}$}}%
\end{pgfscope}%
\begin{pgfscope}%
\pgfsetbuttcap%
\pgfsetroundjoin%
\definecolor{currentfill}{rgb}{0.000000,0.000000,0.000000}%
\pgfsetfillcolor{currentfill}%
\pgfsetlinewidth{0.602250pt}%
\definecolor{currentstroke}{rgb}{0.000000,0.000000,0.000000}%
\pgfsetstrokecolor{currentstroke}%
\pgfsetdash{}{0pt}%
\pgfsys@defobject{currentmarker}{\pgfqpoint{0.000000in}{-0.027778in}}{\pgfqpoint{0.000000in}{0.000000in}}{%
\pgfpathmoveto{\pgfqpoint{0.000000in}{0.000000in}}%
\pgfpathlineto{\pgfqpoint{0.000000in}{-0.027778in}}%
\pgfusepath{stroke,fill}%
}%
\begin{pgfscope}%
\pgfsys@transformshift{0.603422in}{0.420092in}%
\pgfsys@useobject{currentmarker}{}%
\end{pgfscope}%
\end{pgfscope}%
\begin{pgfscope}%
\pgfsetbuttcap%
\pgfsetroundjoin%
\definecolor{currentfill}{rgb}{0.000000,0.000000,0.000000}%
\pgfsetfillcolor{currentfill}%
\pgfsetlinewidth{0.602250pt}%
\definecolor{currentstroke}{rgb}{0.000000,0.000000,0.000000}%
\pgfsetstrokecolor{currentstroke}%
\pgfsetdash{}{0pt}%
\pgfsys@defobject{currentmarker}{\pgfqpoint{0.000000in}{-0.027778in}}{\pgfqpoint{0.000000in}{0.000000in}}{%
\pgfpathmoveto{\pgfqpoint{0.000000in}{0.000000in}}%
\pgfpathlineto{\pgfqpoint{0.000000in}{-0.027778in}}%
\pgfusepath{stroke,fill}%
}%
\begin{pgfscope}%
\pgfsys@transformshift{0.688406in}{0.420092in}%
\pgfsys@useobject{currentmarker}{}%
\end{pgfscope}%
\end{pgfscope}%
\begin{pgfscope}%
\pgfsetbuttcap%
\pgfsetroundjoin%
\definecolor{currentfill}{rgb}{0.000000,0.000000,0.000000}%
\pgfsetfillcolor{currentfill}%
\pgfsetlinewidth{0.602250pt}%
\definecolor{currentstroke}{rgb}{0.000000,0.000000,0.000000}%
\pgfsetstrokecolor{currentstroke}%
\pgfsetdash{}{0pt}%
\pgfsys@defobject{currentmarker}{\pgfqpoint{0.000000in}{-0.027778in}}{\pgfqpoint{0.000000in}{0.000000in}}{%
\pgfpathmoveto{\pgfqpoint{0.000000in}{0.000000in}}%
\pgfpathlineto{\pgfqpoint{0.000000in}{-0.027778in}}%
\pgfusepath{stroke,fill}%
}%
\begin{pgfscope}%
\pgfsys@transformshift{1.264557in}{0.420092in}%
\pgfsys@useobject{currentmarker}{}%
\end{pgfscope}%
\end{pgfscope}%
\begin{pgfscope}%
\pgfsetbuttcap%
\pgfsetroundjoin%
\definecolor{currentfill}{rgb}{0.000000,0.000000,0.000000}%
\pgfsetfillcolor{currentfill}%
\pgfsetlinewidth{0.602250pt}%
\definecolor{currentstroke}{rgb}{0.000000,0.000000,0.000000}%
\pgfsetstrokecolor{currentstroke}%
\pgfsetdash{}{0pt}%
\pgfsys@defobject{currentmarker}{\pgfqpoint{0.000000in}{-0.027778in}}{\pgfqpoint{0.000000in}{0.000000in}}{%
\pgfpathmoveto{\pgfqpoint{0.000000in}{0.000000in}}%
\pgfpathlineto{\pgfqpoint{0.000000in}{-0.027778in}}%
\pgfusepath{stroke,fill}%
}%
\begin{pgfscope}%
\pgfsys@transformshift{1.557113in}{0.420092in}%
\pgfsys@useobject{currentmarker}{}%
\end{pgfscope}%
\end{pgfscope}%
\begin{pgfscope}%
\pgfsetbuttcap%
\pgfsetroundjoin%
\definecolor{currentfill}{rgb}{0.000000,0.000000,0.000000}%
\pgfsetfillcolor{currentfill}%
\pgfsetlinewidth{0.602250pt}%
\definecolor{currentstroke}{rgb}{0.000000,0.000000,0.000000}%
\pgfsetstrokecolor{currentstroke}%
\pgfsetdash{}{0pt}%
\pgfsys@defobject{currentmarker}{\pgfqpoint{0.000000in}{-0.027778in}}{\pgfqpoint{0.000000in}{0.000000in}}{%
\pgfpathmoveto{\pgfqpoint{0.000000in}{0.000000in}}%
\pgfpathlineto{\pgfqpoint{0.000000in}{-0.027778in}}%
\pgfusepath{stroke,fill}%
}%
\begin{pgfscope}%
\pgfsys@transformshift{1.764686in}{0.420092in}%
\pgfsys@useobject{currentmarker}{}%
\end{pgfscope}%
\end{pgfscope}%
\begin{pgfscope}%
\pgfsetbuttcap%
\pgfsetroundjoin%
\definecolor{currentfill}{rgb}{0.000000,0.000000,0.000000}%
\pgfsetfillcolor{currentfill}%
\pgfsetlinewidth{0.602250pt}%
\definecolor{currentstroke}{rgb}{0.000000,0.000000,0.000000}%
\pgfsetstrokecolor{currentstroke}%
\pgfsetdash{}{0pt}%
\pgfsys@defobject{currentmarker}{\pgfqpoint{0.000000in}{-0.027778in}}{\pgfqpoint{0.000000in}{0.000000in}}{%
\pgfpathmoveto{\pgfqpoint{0.000000in}{0.000000in}}%
\pgfpathlineto{\pgfqpoint{0.000000in}{-0.027778in}}%
\pgfusepath{stroke,fill}%
}%
\begin{pgfscope}%
\pgfsys@transformshift{1.925691in}{0.420092in}%
\pgfsys@useobject{currentmarker}{}%
\end{pgfscope}%
\end{pgfscope}%
\begin{pgfscope}%
\pgfsetbuttcap%
\pgfsetroundjoin%
\definecolor{currentfill}{rgb}{0.000000,0.000000,0.000000}%
\pgfsetfillcolor{currentfill}%
\pgfsetlinewidth{0.602250pt}%
\definecolor{currentstroke}{rgb}{0.000000,0.000000,0.000000}%
\pgfsetstrokecolor{currentstroke}%
\pgfsetdash{}{0pt}%
\pgfsys@defobject{currentmarker}{\pgfqpoint{0.000000in}{-0.027778in}}{\pgfqpoint{0.000000in}{0.000000in}}{%
\pgfpathmoveto{\pgfqpoint{0.000000in}{0.000000in}}%
\pgfpathlineto{\pgfqpoint{0.000000in}{-0.027778in}}%
\pgfusepath{stroke,fill}%
}%
\begin{pgfscope}%
\pgfsys@transformshift{2.057243in}{0.420092in}%
\pgfsys@useobject{currentmarker}{}%
\end{pgfscope}%
\end{pgfscope}%
\begin{pgfscope}%
\pgfsetbuttcap%
\pgfsetroundjoin%
\definecolor{currentfill}{rgb}{0.000000,0.000000,0.000000}%
\pgfsetfillcolor{currentfill}%
\pgfsetlinewidth{0.602250pt}%
\definecolor{currentstroke}{rgb}{0.000000,0.000000,0.000000}%
\pgfsetstrokecolor{currentstroke}%
\pgfsetdash{}{0pt}%
\pgfsys@defobject{currentmarker}{\pgfqpoint{0.000000in}{-0.027778in}}{\pgfqpoint{0.000000in}{0.000000in}}{%
\pgfpathmoveto{\pgfqpoint{0.000000in}{0.000000in}}%
\pgfpathlineto{\pgfqpoint{0.000000in}{-0.027778in}}%
\pgfusepath{stroke,fill}%
}%
\begin{pgfscope}%
\pgfsys@transformshift{2.168468in}{0.420092in}%
\pgfsys@useobject{currentmarker}{}%
\end{pgfscope}%
\end{pgfscope}%
\begin{pgfscope}%
\pgfsetbuttcap%
\pgfsetroundjoin%
\definecolor{currentfill}{rgb}{0.000000,0.000000,0.000000}%
\pgfsetfillcolor{currentfill}%
\pgfsetlinewidth{0.602250pt}%
\definecolor{currentstroke}{rgb}{0.000000,0.000000,0.000000}%
\pgfsetstrokecolor{currentstroke}%
\pgfsetdash{}{0pt}%
\pgfsys@defobject{currentmarker}{\pgfqpoint{0.000000in}{-0.027778in}}{\pgfqpoint{0.000000in}{0.000000in}}{%
\pgfpathmoveto{\pgfqpoint{0.000000in}{0.000000in}}%
\pgfpathlineto{\pgfqpoint{0.000000in}{-0.027778in}}%
\pgfusepath{stroke,fill}%
}%
\begin{pgfscope}%
\pgfsys@transformshift{2.264815in}{0.420092in}%
\pgfsys@useobject{currentmarker}{}%
\end{pgfscope}%
\end{pgfscope}%
\begin{pgfscope}%
\pgfsetbuttcap%
\pgfsetroundjoin%
\definecolor{currentfill}{rgb}{0.000000,0.000000,0.000000}%
\pgfsetfillcolor{currentfill}%
\pgfsetlinewidth{0.602250pt}%
\definecolor{currentstroke}{rgb}{0.000000,0.000000,0.000000}%
\pgfsetstrokecolor{currentstroke}%
\pgfsetdash{}{0pt}%
\pgfsys@defobject{currentmarker}{\pgfqpoint{0.000000in}{-0.027778in}}{\pgfqpoint{0.000000in}{0.000000in}}{%
\pgfpathmoveto{\pgfqpoint{0.000000in}{0.000000in}}%
\pgfpathlineto{\pgfqpoint{0.000000in}{-0.027778in}}%
\pgfusepath{stroke,fill}%
}%
\begin{pgfscope}%
\pgfsys@transformshift{2.349800in}{0.420092in}%
\pgfsys@useobject{currentmarker}{}%
\end{pgfscope}%
\end{pgfscope}%
\begin{pgfscope}%
\pgfsetbuttcap%
\pgfsetroundjoin%
\definecolor{currentfill}{rgb}{0.000000,0.000000,0.000000}%
\pgfsetfillcolor{currentfill}%
\pgfsetlinewidth{0.602250pt}%
\definecolor{currentstroke}{rgb}{0.000000,0.000000,0.000000}%
\pgfsetstrokecolor{currentstroke}%
\pgfsetdash{}{0pt}%
\pgfsys@defobject{currentmarker}{\pgfqpoint{0.000000in}{-0.027778in}}{\pgfqpoint{0.000000in}{0.000000in}}{%
\pgfpathmoveto{\pgfqpoint{0.000000in}{0.000000in}}%
\pgfpathlineto{\pgfqpoint{0.000000in}{-0.027778in}}%
\pgfusepath{stroke,fill}%
}%
\begin{pgfscope}%
\pgfsys@transformshift{2.925950in}{0.420092in}%
\pgfsys@useobject{currentmarker}{}%
\end{pgfscope}%
\end{pgfscope}%
\begin{pgfscope}%
\pgfsetbuttcap%
\pgfsetroundjoin%
\definecolor{currentfill}{rgb}{0.000000,0.000000,0.000000}%
\pgfsetfillcolor{currentfill}%
\pgfsetlinewidth{0.602250pt}%
\definecolor{currentstroke}{rgb}{0.000000,0.000000,0.000000}%
\pgfsetstrokecolor{currentstroke}%
\pgfsetdash{}{0pt}%
\pgfsys@defobject{currentmarker}{\pgfqpoint{0.000000in}{-0.027778in}}{\pgfqpoint{0.000000in}{0.000000in}}{%
\pgfpathmoveto{\pgfqpoint{0.000000in}{0.000000in}}%
\pgfpathlineto{\pgfqpoint{0.000000in}{-0.027778in}}%
\pgfusepath{stroke,fill}%
}%
\begin{pgfscope}%
\pgfsys@transformshift{3.218507in}{0.420092in}%
\pgfsys@useobject{currentmarker}{}%
\end{pgfscope}%
\end{pgfscope}%
\begin{pgfscope}%
\pgfsetbuttcap%
\pgfsetroundjoin%
\definecolor{currentfill}{rgb}{0.000000,0.000000,0.000000}%
\pgfsetfillcolor{currentfill}%
\pgfsetlinewidth{0.602250pt}%
\definecolor{currentstroke}{rgb}{0.000000,0.000000,0.000000}%
\pgfsetstrokecolor{currentstroke}%
\pgfsetdash{}{0pt}%
\pgfsys@defobject{currentmarker}{\pgfqpoint{0.000000in}{-0.027778in}}{\pgfqpoint{0.000000in}{0.000000in}}{%
\pgfpathmoveto{\pgfqpoint{0.000000in}{0.000000in}}%
\pgfpathlineto{\pgfqpoint{0.000000in}{-0.027778in}}%
\pgfusepath{stroke,fill}%
}%
\begin{pgfscope}%
\pgfsys@transformshift{3.426079in}{0.420092in}%
\pgfsys@useobject{currentmarker}{}%
\end{pgfscope}%
\end{pgfscope}%
\begin{pgfscope}%
\pgfsetbuttcap%
\pgfsetroundjoin%
\definecolor{currentfill}{rgb}{0.000000,0.000000,0.000000}%
\pgfsetfillcolor{currentfill}%
\pgfsetlinewidth{0.602250pt}%
\definecolor{currentstroke}{rgb}{0.000000,0.000000,0.000000}%
\pgfsetstrokecolor{currentstroke}%
\pgfsetdash{}{0pt}%
\pgfsys@defobject{currentmarker}{\pgfqpoint{0.000000in}{-0.027778in}}{\pgfqpoint{0.000000in}{0.000000in}}{%
\pgfpathmoveto{\pgfqpoint{0.000000in}{0.000000in}}%
\pgfpathlineto{\pgfqpoint{0.000000in}{-0.027778in}}%
\pgfusepath{stroke,fill}%
}%
\begin{pgfscope}%
\pgfsys@transformshift{3.587085in}{0.420092in}%
\pgfsys@useobject{currentmarker}{}%
\end{pgfscope}%
\end{pgfscope}%
\begin{pgfscope}%
\pgfsetbuttcap%
\pgfsetroundjoin%
\definecolor{currentfill}{rgb}{0.000000,0.000000,0.000000}%
\pgfsetfillcolor{currentfill}%
\pgfsetlinewidth{0.602250pt}%
\definecolor{currentstroke}{rgb}{0.000000,0.000000,0.000000}%
\pgfsetstrokecolor{currentstroke}%
\pgfsetdash{}{0pt}%
\pgfsys@defobject{currentmarker}{\pgfqpoint{0.000000in}{-0.027778in}}{\pgfqpoint{0.000000in}{0.000000in}}{%
\pgfpathmoveto{\pgfqpoint{0.000000in}{0.000000in}}%
\pgfpathlineto{\pgfqpoint{0.000000in}{-0.027778in}}%
\pgfusepath{stroke,fill}%
}%
\begin{pgfscope}%
\pgfsys@transformshift{3.718636in}{0.420092in}%
\pgfsys@useobject{currentmarker}{}%
\end{pgfscope}%
\end{pgfscope}%
\begin{pgfscope}%
\pgfsetbuttcap%
\pgfsetroundjoin%
\definecolor{currentfill}{rgb}{0.000000,0.000000,0.000000}%
\pgfsetfillcolor{currentfill}%
\pgfsetlinewidth{0.602250pt}%
\definecolor{currentstroke}{rgb}{0.000000,0.000000,0.000000}%
\pgfsetstrokecolor{currentstroke}%
\pgfsetdash{}{0pt}%
\pgfsys@defobject{currentmarker}{\pgfqpoint{0.000000in}{-0.027778in}}{\pgfqpoint{0.000000in}{0.000000in}}{%
\pgfpathmoveto{\pgfqpoint{0.000000in}{0.000000in}}%
\pgfpathlineto{\pgfqpoint{0.000000in}{-0.027778in}}%
\pgfusepath{stroke,fill}%
}%
\begin{pgfscope}%
\pgfsys@transformshift{3.829861in}{0.420092in}%
\pgfsys@useobject{currentmarker}{}%
\end{pgfscope}%
\end{pgfscope}%
\begin{pgfscope}%
\pgfsetbuttcap%
\pgfsetroundjoin%
\definecolor{currentfill}{rgb}{0.000000,0.000000,0.000000}%
\pgfsetfillcolor{currentfill}%
\pgfsetlinewidth{0.602250pt}%
\definecolor{currentstroke}{rgb}{0.000000,0.000000,0.000000}%
\pgfsetstrokecolor{currentstroke}%
\pgfsetdash{}{0pt}%
\pgfsys@defobject{currentmarker}{\pgfqpoint{0.000000in}{-0.027778in}}{\pgfqpoint{0.000000in}{0.000000in}}{%
\pgfpathmoveto{\pgfqpoint{0.000000in}{0.000000in}}%
\pgfpathlineto{\pgfqpoint{0.000000in}{-0.027778in}}%
\pgfusepath{stroke,fill}%
}%
\begin{pgfscope}%
\pgfsys@transformshift{3.926209in}{0.420092in}%
\pgfsys@useobject{currentmarker}{}%
\end{pgfscope}%
\end{pgfscope}%
\begin{pgfscope}%
\pgfsetbuttcap%
\pgfsetroundjoin%
\definecolor{currentfill}{rgb}{0.000000,0.000000,0.000000}%
\pgfsetfillcolor{currentfill}%
\pgfsetlinewidth{0.602250pt}%
\definecolor{currentstroke}{rgb}{0.000000,0.000000,0.000000}%
\pgfsetstrokecolor{currentstroke}%
\pgfsetdash{}{0pt}%
\pgfsys@defobject{currentmarker}{\pgfqpoint{0.000000in}{-0.027778in}}{\pgfqpoint{0.000000in}{0.000000in}}{%
\pgfpathmoveto{\pgfqpoint{0.000000in}{0.000000in}}%
\pgfpathlineto{\pgfqpoint{0.000000in}{-0.027778in}}%
\pgfusepath{stroke,fill}%
}%
\begin{pgfscope}%
\pgfsys@transformshift{4.011193in}{0.420092in}%
\pgfsys@useobject{currentmarker}{}%
\end{pgfscope}%
\end{pgfscope}%
\begin{pgfscope}%
\pgfsetbuttcap%
\pgfsetroundjoin%
\definecolor{currentfill}{rgb}{0.000000,0.000000,0.000000}%
\pgfsetfillcolor{currentfill}%
\pgfsetlinewidth{0.602250pt}%
\definecolor{currentstroke}{rgb}{0.000000,0.000000,0.000000}%
\pgfsetstrokecolor{currentstroke}%
\pgfsetdash{}{0pt}%
\pgfsys@defobject{currentmarker}{\pgfqpoint{0.000000in}{-0.027778in}}{\pgfqpoint{0.000000in}{0.000000in}}{%
\pgfpathmoveto{\pgfqpoint{0.000000in}{0.000000in}}%
\pgfpathlineto{\pgfqpoint{0.000000in}{-0.027778in}}%
\pgfusepath{stroke,fill}%
}%
\begin{pgfscope}%
\pgfsys@transformshift{4.587344in}{0.420092in}%
\pgfsys@useobject{currentmarker}{}%
\end{pgfscope}%
\end{pgfscope}%
\begin{pgfscope}%
\pgfsetbuttcap%
\pgfsetroundjoin%
\definecolor{currentfill}{rgb}{0.000000,0.000000,0.000000}%
\pgfsetfillcolor{currentfill}%
\pgfsetlinewidth{0.602250pt}%
\definecolor{currentstroke}{rgb}{0.000000,0.000000,0.000000}%
\pgfsetstrokecolor{currentstroke}%
\pgfsetdash{}{0pt}%
\pgfsys@defobject{currentmarker}{\pgfqpoint{0.000000in}{-0.027778in}}{\pgfqpoint{0.000000in}{0.000000in}}{%
\pgfpathmoveto{\pgfqpoint{0.000000in}{0.000000in}}%
\pgfpathlineto{\pgfqpoint{0.000000in}{-0.027778in}}%
\pgfusepath{stroke,fill}%
}%
\begin{pgfscope}%
\pgfsys@transformshift{4.879900in}{0.420092in}%
\pgfsys@useobject{currentmarker}{}%
\end{pgfscope}%
\end{pgfscope}%
\begin{pgfscope}%
\pgfsetbuttcap%
\pgfsetroundjoin%
\definecolor{currentfill}{rgb}{0.000000,0.000000,0.000000}%
\pgfsetfillcolor{currentfill}%
\pgfsetlinewidth{0.602250pt}%
\definecolor{currentstroke}{rgb}{0.000000,0.000000,0.000000}%
\pgfsetstrokecolor{currentstroke}%
\pgfsetdash{}{0pt}%
\pgfsys@defobject{currentmarker}{\pgfqpoint{0.000000in}{-0.027778in}}{\pgfqpoint{0.000000in}{0.000000in}}{%
\pgfpathmoveto{\pgfqpoint{0.000000in}{0.000000in}}%
\pgfpathlineto{\pgfqpoint{0.000000in}{-0.027778in}}%
\pgfusepath{stroke,fill}%
}%
\begin{pgfscope}%
\pgfsys@transformshift{5.087473in}{0.420092in}%
\pgfsys@useobject{currentmarker}{}%
\end{pgfscope}%
\end{pgfscope}%
\begin{pgfscope}%
\pgfsetbuttcap%
\pgfsetroundjoin%
\definecolor{currentfill}{rgb}{0.000000,0.000000,0.000000}%
\pgfsetfillcolor{currentfill}%
\pgfsetlinewidth{0.602250pt}%
\definecolor{currentstroke}{rgb}{0.000000,0.000000,0.000000}%
\pgfsetstrokecolor{currentstroke}%
\pgfsetdash{}{0pt}%
\pgfsys@defobject{currentmarker}{\pgfqpoint{0.000000in}{-0.027778in}}{\pgfqpoint{0.000000in}{0.000000in}}{%
\pgfpathmoveto{\pgfqpoint{0.000000in}{0.000000in}}%
\pgfpathlineto{\pgfqpoint{0.000000in}{-0.027778in}}%
\pgfusepath{stroke,fill}%
}%
\begin{pgfscope}%
\pgfsys@transformshift{5.248478in}{0.420092in}%
\pgfsys@useobject{currentmarker}{}%
\end{pgfscope}%
\end{pgfscope}%
\begin{pgfscope}%
\pgfsetbuttcap%
\pgfsetroundjoin%
\definecolor{currentfill}{rgb}{0.000000,0.000000,0.000000}%
\pgfsetfillcolor{currentfill}%
\pgfsetlinewidth{0.602250pt}%
\definecolor{currentstroke}{rgb}{0.000000,0.000000,0.000000}%
\pgfsetstrokecolor{currentstroke}%
\pgfsetdash{}{0pt}%
\pgfsys@defobject{currentmarker}{\pgfqpoint{0.000000in}{-0.027778in}}{\pgfqpoint{0.000000in}{0.000000in}}{%
\pgfpathmoveto{\pgfqpoint{0.000000in}{0.000000in}}%
\pgfpathlineto{\pgfqpoint{0.000000in}{-0.027778in}}%
\pgfusepath{stroke,fill}%
}%
\begin{pgfscope}%
\pgfsys@transformshift{5.380030in}{0.420092in}%
\pgfsys@useobject{currentmarker}{}%
\end{pgfscope}%
\end{pgfscope}%
\begin{pgfscope}%
\pgfsetbuttcap%
\pgfsetroundjoin%
\definecolor{currentfill}{rgb}{0.000000,0.000000,0.000000}%
\pgfsetfillcolor{currentfill}%
\pgfsetlinewidth{0.602250pt}%
\definecolor{currentstroke}{rgb}{0.000000,0.000000,0.000000}%
\pgfsetstrokecolor{currentstroke}%
\pgfsetdash{}{0pt}%
\pgfsys@defobject{currentmarker}{\pgfqpoint{0.000000in}{-0.027778in}}{\pgfqpoint{0.000000in}{0.000000in}}{%
\pgfpathmoveto{\pgfqpoint{0.000000in}{0.000000in}}%
\pgfpathlineto{\pgfqpoint{0.000000in}{-0.027778in}}%
\pgfusepath{stroke,fill}%
}%
\begin{pgfscope}%
\pgfsys@transformshift{5.491255in}{0.420092in}%
\pgfsys@useobject{currentmarker}{}%
\end{pgfscope}%
\end{pgfscope}%
\begin{pgfscope}%
\pgfsetbuttcap%
\pgfsetroundjoin%
\definecolor{currentfill}{rgb}{0.000000,0.000000,0.000000}%
\pgfsetfillcolor{currentfill}%
\pgfsetlinewidth{0.602250pt}%
\definecolor{currentstroke}{rgb}{0.000000,0.000000,0.000000}%
\pgfsetstrokecolor{currentstroke}%
\pgfsetdash{}{0pt}%
\pgfsys@defobject{currentmarker}{\pgfqpoint{0.000000in}{-0.027778in}}{\pgfqpoint{0.000000in}{0.000000in}}{%
\pgfpathmoveto{\pgfqpoint{0.000000in}{0.000000in}}%
\pgfpathlineto{\pgfqpoint{0.000000in}{-0.027778in}}%
\pgfusepath{stroke,fill}%
}%
\begin{pgfscope}%
\pgfsys@transformshift{5.587602in}{0.420092in}%
\pgfsys@useobject{currentmarker}{}%
\end{pgfscope}%
\end{pgfscope}%
\begin{pgfscope}%
\pgfsetbuttcap%
\pgfsetroundjoin%
\definecolor{currentfill}{rgb}{0.000000,0.000000,0.000000}%
\pgfsetfillcolor{currentfill}%
\pgfsetlinewidth{0.602250pt}%
\definecolor{currentstroke}{rgb}{0.000000,0.000000,0.000000}%
\pgfsetstrokecolor{currentstroke}%
\pgfsetdash{}{0pt}%
\pgfsys@defobject{currentmarker}{\pgfqpoint{0.000000in}{-0.027778in}}{\pgfqpoint{0.000000in}{0.000000in}}{%
\pgfpathmoveto{\pgfqpoint{0.000000in}{0.000000in}}%
\pgfpathlineto{\pgfqpoint{0.000000in}{-0.027778in}}%
\pgfusepath{stroke,fill}%
}%
\begin{pgfscope}%
\pgfsys@transformshift{5.672587in}{0.420092in}%
\pgfsys@useobject{currentmarker}{}%
\end{pgfscope}%
\end{pgfscope}%
\begin{pgfscope}%
\definecolor{textcolor}{rgb}{0.000000,0.000000,0.000000}%
\pgfsetstrokecolor{textcolor}%
\pgfsetfillcolor{textcolor}%
\pgftext[x=3.230336in,y=0.180648in,,top]{\color{textcolor}{\rmfamily\fontsize{6.000000}{7.200000}\selectfont\catcode`\^=\active\def^{\ifmmode\sp\else\^{}\fi}\catcode`\%=\active\def%{\%}Iterations}}%
\end{pgfscope}%
\begin{pgfscope}%
\pgfsetbuttcap%
\pgfsetroundjoin%
\definecolor{currentfill}{rgb}{0.000000,0.000000,0.000000}%
\pgfsetfillcolor{currentfill}%
\pgfsetlinewidth{0.803000pt}%
\definecolor{currentstroke}{rgb}{0.000000,0.000000,0.000000}%
\pgfsetstrokecolor{currentstroke}%
\pgfsetdash{}{0pt}%
\pgfsys@defobject{currentmarker}{\pgfqpoint{-0.048611in}{0.000000in}}{\pgfqpoint{-0.000000in}{0.000000in}}{%
\pgfpathmoveto{\pgfqpoint{-0.000000in}{0.000000in}}%
\pgfpathlineto{\pgfqpoint{-0.048611in}{0.000000in}}%
\pgfusepath{stroke,fill}%
}%
\begin{pgfscope}%
\pgfsys@transformshift{0.517836in}{0.420092in}%
\pgfsys@useobject{currentmarker}{}%
\end{pgfscope}%
\end{pgfscope}%
\begin{pgfscope}%
\definecolor{textcolor}{rgb}{0.000000,0.000000,0.000000}%
\pgfsetstrokecolor{textcolor}%
\pgfsetfillcolor{textcolor}%
\pgftext[x=0.287128in, y=0.388435in, left, base]{\color{textcolor}{\rmfamily\fontsize{6.000000}{7.200000}\selectfont\catcode`\^=\active\def^{\ifmmode\sp\else\^{}\fi}\catcode`\%=\active\def%{\%}$\mathdefault{0.0}$}}%
\end{pgfscope}%
\begin{pgfscope}%
\pgfsetbuttcap%
\pgfsetroundjoin%
\definecolor{currentfill}{rgb}{0.000000,0.000000,0.000000}%
\pgfsetfillcolor{currentfill}%
\pgfsetlinewidth{0.803000pt}%
\definecolor{currentstroke}{rgb}{0.000000,0.000000,0.000000}%
\pgfsetstrokecolor{currentstroke}%
\pgfsetdash{}{0pt}%
\pgfsys@defobject{currentmarker}{\pgfqpoint{-0.048611in}{0.000000in}}{\pgfqpoint{-0.000000in}{0.000000in}}{%
\pgfpathmoveto{\pgfqpoint{-0.000000in}{0.000000in}}%
\pgfpathlineto{\pgfqpoint{-0.048611in}{0.000000in}}%
\pgfusepath{stroke,fill}%
}%
\begin{pgfscope}%
\pgfsys@transformshift{0.517836in}{0.612592in}%
\pgfsys@useobject{currentmarker}{}%
\end{pgfscope}%
\end{pgfscope}%
\begin{pgfscope}%
\definecolor{textcolor}{rgb}{0.000000,0.000000,0.000000}%
\pgfsetstrokecolor{textcolor}%
\pgfsetfillcolor{textcolor}%
\pgftext[x=0.287128in, y=0.580935in, left, base]{\color{textcolor}{\rmfamily\fontsize{6.000000}{7.200000}\selectfont\catcode`\^=\active\def^{\ifmmode\sp\else\^{}\fi}\catcode`\%=\active\def%{\%}$\mathdefault{2.5}$}}%
\end{pgfscope}%
\begin{pgfscope}%
\pgfsetbuttcap%
\pgfsetroundjoin%
\definecolor{currentfill}{rgb}{0.000000,0.000000,0.000000}%
\pgfsetfillcolor{currentfill}%
\pgfsetlinewidth{0.803000pt}%
\definecolor{currentstroke}{rgb}{0.000000,0.000000,0.000000}%
\pgfsetstrokecolor{currentstroke}%
\pgfsetdash{}{0pt}%
\pgfsys@defobject{currentmarker}{\pgfqpoint{-0.048611in}{0.000000in}}{\pgfqpoint{-0.000000in}{0.000000in}}{%
\pgfpathmoveto{\pgfqpoint{-0.000000in}{0.000000in}}%
\pgfpathlineto{\pgfqpoint{-0.048611in}{0.000000in}}%
\pgfusepath{stroke,fill}%
}%
\begin{pgfscope}%
\pgfsys@transformshift{0.517836in}{0.805092in}%
\pgfsys@useobject{currentmarker}{}%
\end{pgfscope}%
\end{pgfscope}%
\begin{pgfscope}%
\definecolor{textcolor}{rgb}{0.000000,0.000000,0.000000}%
\pgfsetstrokecolor{textcolor}%
\pgfsetfillcolor{textcolor}%
\pgftext[x=0.287128in, y=0.773435in, left, base]{\color{textcolor}{\rmfamily\fontsize{6.000000}{7.200000}\selectfont\catcode`\^=\active\def^{\ifmmode\sp\else\^{}\fi}\catcode`\%=\active\def%{\%}$\mathdefault{5.0}$}}%
\end{pgfscope}%
\begin{pgfscope}%
\pgfsetbuttcap%
\pgfsetroundjoin%
\definecolor{currentfill}{rgb}{0.000000,0.000000,0.000000}%
\pgfsetfillcolor{currentfill}%
\pgfsetlinewidth{0.803000pt}%
\definecolor{currentstroke}{rgb}{0.000000,0.000000,0.000000}%
\pgfsetstrokecolor{currentstroke}%
\pgfsetdash{}{0pt}%
\pgfsys@defobject{currentmarker}{\pgfqpoint{-0.048611in}{0.000000in}}{\pgfqpoint{-0.000000in}{0.000000in}}{%
\pgfpathmoveto{\pgfqpoint{-0.000000in}{0.000000in}}%
\pgfpathlineto{\pgfqpoint{-0.048611in}{0.000000in}}%
\pgfusepath{stroke,fill}%
}%
\begin{pgfscope}%
\pgfsys@transformshift{0.517836in}{0.997592in}%
\pgfsys@useobject{currentmarker}{}%
\end{pgfscope}%
\end{pgfscope}%
\begin{pgfscope}%
\definecolor{textcolor}{rgb}{0.000000,0.000000,0.000000}%
\pgfsetstrokecolor{textcolor}%
\pgfsetfillcolor{textcolor}%
\pgftext[x=0.287128in, y=0.965935in, left, base]{\color{textcolor}{\rmfamily\fontsize{6.000000}{7.200000}\selectfont\catcode`\^=\active\def^{\ifmmode\sp\else\^{}\fi}\catcode`\%=\active\def%{\%}$\mathdefault{7.5}$}}%
\end{pgfscope}%
\begin{pgfscope}%
\pgfsetbuttcap%
\pgfsetroundjoin%
\definecolor{currentfill}{rgb}{0.000000,0.000000,0.000000}%
\pgfsetfillcolor{currentfill}%
\pgfsetlinewidth{0.803000pt}%
\definecolor{currentstroke}{rgb}{0.000000,0.000000,0.000000}%
\pgfsetstrokecolor{currentstroke}%
\pgfsetdash{}{0pt}%
\pgfsys@defobject{currentmarker}{\pgfqpoint{-0.048611in}{0.000000in}}{\pgfqpoint{-0.000000in}{0.000000in}}{%
\pgfpathmoveto{\pgfqpoint{-0.000000in}{0.000000in}}%
\pgfpathlineto{\pgfqpoint{-0.048611in}{0.000000in}}%
\pgfusepath{stroke,fill}%
}%
\begin{pgfscope}%
\pgfsys@transformshift{0.517836in}{1.190092in}%
\pgfsys@useobject{currentmarker}{}%
\end{pgfscope}%
\end{pgfscope}%
\begin{pgfscope}%
\definecolor{textcolor}{rgb}{0.000000,0.000000,0.000000}%
\pgfsetstrokecolor{textcolor}%
\pgfsetfillcolor{textcolor}%
\pgftext[x=0.236203in, y=1.158435in, left, base]{\color{textcolor}{\rmfamily\fontsize{6.000000}{7.200000}\selectfont\catcode`\^=\active\def^{\ifmmode\sp\else\^{}\fi}\catcode`\%=\active\def%{\%}$\mathdefault{10.0}$}}%
\end{pgfscope}%
\begin{pgfscope}%
\definecolor{textcolor}{rgb}{0.000000,0.000000,0.000000}%
\pgfsetstrokecolor{textcolor}%
\pgfsetfillcolor{textcolor}%
\pgftext[x=0.180648in,y=0.805092in,,bottom,rotate=90.000000]{\color{textcolor}{\rmfamily\fontsize{6.000000}{7.200000}\selectfont\catcode`\^=\active\def^{\ifmmode\sp\else\^{}\fi}\catcode`\%=\active\def%{\%}Relative L2 Loss}}%
\end{pgfscope}%
\begin{pgfscope}%
\pgfpathrectangle{\pgfqpoint{0.517836in}{0.420092in}}{\pgfqpoint{5.425000in}{0.770000in}}%
\pgfusepath{clip}%
\pgfsetrectcap%
\pgfsetroundjoin%
\pgfsetlinewidth{0.501875pt}%
\definecolor{currentstroke}{rgb}{0.121569,0.466667,0.705882}%
\pgfsetstrokecolor{currentstroke}%
\pgfsetdash{}{0pt}%
\pgfpathmoveto{\pgfqpoint{0.507836in}{0.916440in}}%
\pgfpathlineto{\pgfqpoint{0.764427in}{0.916453in}}%
\pgfpathlineto{\pgfqpoint{1.264557in}{0.635084in}}%
\pgfpathlineto{\pgfqpoint{1.557113in}{0.550211in}}%
\pgfpathlineto{\pgfqpoint{1.764686in}{0.588068in}}%
\pgfpathlineto{\pgfqpoint{1.925691in}{0.644116in}}%
\pgfpathlineto{\pgfqpoint{2.057243in}{0.649091in}}%
\pgfpathlineto{\pgfqpoint{2.168468in}{0.706955in}}%
\pgfpathlineto{\pgfqpoint{2.264815in}{0.945732in}}%
\pgfpathlineto{\pgfqpoint{2.349800in}{1.089889in}}%
\pgfpathlineto{\pgfqpoint{2.425821in}{1.015957in}}%
\pgfpathlineto{\pgfqpoint{2.494590in}{0.868994in}}%
\pgfpathlineto{\pgfqpoint{2.615126in}{0.734945in}}%
\pgfpathlineto{\pgfqpoint{2.668597in}{0.626786in}}%
\pgfpathlineto{\pgfqpoint{2.718378in}{0.691917in}}%
\pgfpathlineto{\pgfqpoint{2.764944in}{0.691682in}}%
\pgfpathlineto{\pgfqpoint{2.808687in}{0.660475in}}%
\pgfpathlineto{\pgfqpoint{2.849929in}{0.728782in}}%
\pgfpathlineto{\pgfqpoint{2.888940in}{0.745398in}}%
\pgfpathlineto{\pgfqpoint{2.925950in}{0.785893in}}%
\pgfpathlineto{\pgfqpoint{2.961154in}{0.730924in}}%
\pgfpathlineto{\pgfqpoint{2.994720in}{0.733494in}}%
\pgfpathlineto{\pgfqpoint{3.026793in}{0.794431in}}%
\pgfpathlineto{\pgfqpoint{3.057501in}{0.749952in}}%
\pgfpathlineto{\pgfqpoint{3.086956in}{0.665341in}}%
\pgfpathlineto{\pgfqpoint{3.115255in}{0.689943in}}%
\pgfpathlineto{\pgfqpoint{3.142486in}{0.628971in}}%
\pgfpathlineto{\pgfqpoint{3.168726in}{0.659513in}}%
\pgfpathlineto{\pgfqpoint{3.194046in}{0.646442in}}%
\pgfpathlineto{\pgfqpoint{3.218507in}{0.614820in}}%
\pgfpathlineto{\pgfqpoint{3.242166in}{0.648727in}}%
\pgfpathlineto{\pgfqpoint{3.265074in}{0.639412in}}%
\pgfpathlineto{\pgfqpoint{3.287276in}{0.739462in}}%
\pgfpathlineto{\pgfqpoint{3.308816in}{0.674512in}}%
\pgfpathlineto{\pgfqpoint{3.329732in}{0.601802in}}%
\pgfpathlineto{\pgfqpoint{3.350058in}{0.619491in}}%
\pgfpathlineto{\pgfqpoint{3.369827in}{0.610756in}}%
\pgfpathlineto{\pgfqpoint{3.389069in}{0.609606in}}%
\pgfpathlineto{\pgfqpoint{3.407812in}{0.595835in}}%
\pgfpathlineto{\pgfqpoint{3.426079in}{0.666185in}}%
\pgfpathlineto{\pgfqpoint{3.443896in}{0.716277in}}%
\pgfpathlineto{\pgfqpoint{3.461283in}{0.698099in}}%
\pgfpathlineto{\pgfqpoint{3.478261in}{0.695928in}}%
\pgfpathlineto{\pgfqpoint{3.494849in}{0.638028in}}%
\pgfpathlineto{\pgfqpoint{3.511064in}{0.587484in}}%
\pgfpathlineto{\pgfqpoint{3.526922in}{0.557230in}}%
\pgfpathlineto{\pgfqpoint{3.542440in}{0.613213in}}%
\pgfpathlineto{\pgfqpoint{3.557631in}{0.661321in}}%
\pgfpathlineto{\pgfqpoint{3.572508in}{0.631753in}}%
\pgfpathlineto{\pgfqpoint{3.587085in}{0.657824in}}%
\pgfpathlineto{\pgfqpoint{3.601373in}{0.600765in}}%
\pgfpathlineto{\pgfqpoint{3.615384in}{0.554748in}}%
\pgfpathlineto{\pgfqpoint{3.629128in}{0.693483in}}%
\pgfpathlineto{\pgfqpoint{3.642615in}{0.677393in}}%
\pgfpathlineto{\pgfqpoint{3.655855in}{0.555909in}}%
\pgfpathlineto{\pgfqpoint{3.668855in}{0.589761in}}%
\pgfpathlineto{\pgfqpoint{3.681626in}{0.604851in}}%
\pgfpathlineto{\pgfqpoint{3.694175in}{0.595805in}}%
\pgfpathlineto{\pgfqpoint{3.706509in}{0.636355in}}%
\pgfpathlineto{\pgfqpoint{3.718636in}{0.625124in}}%
\pgfpathlineto{\pgfqpoint{3.730563in}{0.751684in}}%
\pgfpathlineto{\pgfqpoint{3.742295in}{0.735665in}}%
\pgfpathlineto{\pgfqpoint{3.765203in}{0.635569in}}%
\pgfpathlineto{\pgfqpoint{3.776390in}{0.620810in}}%
\pgfpathlineto{\pgfqpoint{3.787406in}{0.597041in}}%
\pgfpathlineto{\pgfqpoint{3.798256in}{0.521452in}}%
\pgfpathlineto{\pgfqpoint{3.808946in}{0.576999in}}%
\pgfpathlineto{\pgfqpoint{3.819479in}{0.597279in}}%
\pgfpathlineto{\pgfqpoint{3.829861in}{0.559586in}}%
\pgfpathlineto{\pgfqpoint{3.840096in}{0.557633in}}%
\pgfpathlineto{\pgfqpoint{3.850187in}{0.568230in}}%
\pgfpathlineto{\pgfqpoint{3.860140in}{0.715320in}}%
\pgfpathlineto{\pgfqpoint{3.869957in}{0.736968in}}%
\pgfpathlineto{\pgfqpoint{3.879642in}{0.622908in}}%
\pgfpathlineto{\pgfqpoint{3.889199in}{0.594754in}}%
\pgfpathlineto{\pgfqpoint{3.898631in}{0.583516in}}%
\pgfpathlineto{\pgfqpoint{3.907941in}{0.639395in}}%
\pgfpathlineto{\pgfqpoint{3.917133in}{0.714907in}}%
\pgfpathlineto{\pgfqpoint{3.926209in}{0.690792in}}%
\pgfpathlineto{\pgfqpoint{3.935172in}{0.687160in}}%
\pgfpathlineto{\pgfqpoint{3.944025in}{0.816725in}}%
\pgfpathlineto{\pgfqpoint{3.952771in}{0.814633in}}%
\pgfpathlineto{\pgfqpoint{3.961412in}{0.661571in}}%
\pgfpathlineto{\pgfqpoint{3.969951in}{0.790567in}}%
\pgfpathlineto{\pgfqpoint{3.978390in}{0.639370in}}%
\pgfpathlineto{\pgfqpoint{3.986732in}{0.590332in}}%
\pgfpathlineto{\pgfqpoint{3.994978in}{0.596136in}}%
\pgfpathlineto{\pgfqpoint{4.003131in}{0.647975in}}%
\pgfpathlineto{\pgfqpoint{4.011193in}{0.651836in}}%
\pgfpathlineto{\pgfqpoint{4.019166in}{0.609428in}}%
\pgfpathlineto{\pgfqpoint{4.027052in}{0.644958in}}%
\pgfpathlineto{\pgfqpoint{4.034852in}{0.783373in}}%
\pgfpathlineto{\pgfqpoint{4.042569in}{0.626655in}}%
\pgfpathlineto{\pgfqpoint{4.050204in}{0.609067in}}%
\pgfpathlineto{\pgfqpoint{4.057760in}{0.665361in}}%
\pgfpathlineto{\pgfqpoint{4.065237in}{0.654487in}}%
\pgfpathlineto{\pgfqpoint{4.072637in}{0.565246in}}%
\pgfpathlineto{\pgfqpoint{4.079963in}{0.580108in}}%
\pgfpathlineto{\pgfqpoint{4.094394in}{0.688925in}}%
\pgfpathlineto{\pgfqpoint{4.101503in}{0.634460in}}%
\pgfpathlineto{\pgfqpoint{4.108542in}{0.595030in}}%
\pgfpathlineto{\pgfqpoint{4.115513in}{0.602561in}}%
\pgfpathlineto{\pgfqpoint{4.129257in}{0.568339in}}%
\pgfpathlineto{\pgfqpoint{4.136032in}{0.538047in}}%
\pgfpathlineto{\pgfqpoint{4.142744in}{0.602537in}}%
\pgfpathlineto{\pgfqpoint{4.149394in}{0.587285in}}%
\pgfpathlineto{\pgfqpoint{4.155984in}{0.593361in}}%
\pgfpathlineto{\pgfqpoint{4.162514in}{0.630293in}}%
\pgfpathlineto{\pgfqpoint{4.168985in}{0.639896in}}%
\pgfpathlineto{\pgfqpoint{4.175398in}{0.652864in}}%
\pgfpathlineto{\pgfqpoint{4.181756in}{0.602388in}}%
\pgfpathlineto{\pgfqpoint{4.188057in}{0.895325in}}%
\pgfpathlineto{\pgfqpoint{4.194304in}{0.601888in}}%
\pgfpathlineto{\pgfqpoint{4.200498in}{0.632626in}}%
\pgfpathlineto{\pgfqpoint{4.206639in}{0.648845in}}%
\pgfpathlineto{\pgfqpoint{4.218765in}{0.574480in}}%
\pgfpathlineto{\pgfqpoint{4.224753in}{0.588535in}}%
\pgfpathlineto{\pgfqpoint{4.230692in}{0.618779in}}%
\pgfpathlineto{\pgfqpoint{4.236582in}{0.596084in}}%
\pgfpathlineto{\pgfqpoint{4.242424in}{0.642570in}}%
\pgfpathlineto{\pgfqpoint{4.248220in}{0.623638in}}%
\pgfpathlineto{\pgfqpoint{4.253969in}{0.593708in}}%
\pgfpathlineto{\pgfqpoint{4.259673in}{0.551668in}}%
\pgfpathlineto{\pgfqpoint{4.265332in}{0.588546in}}%
\pgfpathlineto{\pgfqpoint{4.270947in}{0.615827in}}%
\pgfpathlineto{\pgfqpoint{4.276519in}{0.596445in}}%
\pgfpathlineto{\pgfqpoint{4.282048in}{0.557814in}}%
\pgfpathlineto{\pgfqpoint{4.287535in}{0.641650in}}%
\pgfpathlineto{\pgfqpoint{4.292981in}{0.658430in}}%
\pgfpathlineto{\pgfqpoint{4.298385in}{0.602301in}}%
\pgfpathlineto{\pgfqpoint{4.303750in}{0.611754in}}%
\pgfpathlineto{\pgfqpoint{4.309075in}{0.602508in}}%
\pgfpathlineto{\pgfqpoint{4.314361in}{0.636023in}}%
\pgfpathlineto{\pgfqpoint{4.319608in}{0.619460in}}%
\pgfpathlineto{\pgfqpoint{4.324818in}{0.575536in}}%
\pgfpathlineto{\pgfqpoint{4.329990in}{0.661154in}}%
\pgfpathlineto{\pgfqpoint{4.335126in}{0.675116in}}%
\pgfpathlineto{\pgfqpoint{4.345289in}{0.555153in}}%
\pgfpathlineto{\pgfqpoint{4.350317in}{0.588930in}}%
\pgfpathlineto{\pgfqpoint{4.355310in}{0.542617in}}%
\pgfpathlineto{\pgfqpoint{4.360269in}{0.560714in}}%
\pgfpathlineto{\pgfqpoint{4.365194in}{0.628876in}}%
\pgfpathlineto{\pgfqpoint{4.370086in}{0.618676in}}%
\pgfpathlineto{\pgfqpoint{4.374945in}{0.594555in}}%
\pgfpathlineto{\pgfqpoint{4.384565in}{0.620954in}}%
\pgfpathlineto{\pgfqpoint{4.389328in}{0.573726in}}%
\pgfpathlineto{\pgfqpoint{4.394059in}{0.576219in}}%
\pgfpathlineto{\pgfqpoint{4.398760in}{0.544691in}}%
\pgfpathlineto{\pgfqpoint{4.408070in}{0.557819in}}%
\pgfpathlineto{\pgfqpoint{4.412681in}{0.588141in}}%
\pgfpathlineto{\pgfqpoint{4.417262in}{0.584405in}}%
\pgfpathlineto{\pgfqpoint{4.421814in}{0.601205in}}%
\pgfpathlineto{\pgfqpoint{4.426338in}{0.602918in}}%
\pgfpathlineto{\pgfqpoint{4.430833in}{0.569792in}}%
\pgfpathlineto{\pgfqpoint{4.435301in}{0.552998in}}%
\pgfpathlineto{\pgfqpoint{4.439741in}{0.551538in}}%
\pgfpathlineto{\pgfqpoint{4.444154in}{0.611070in}}%
\pgfpathlineto{\pgfqpoint{4.448541in}{0.637055in}}%
\pgfpathlineto{\pgfqpoint{4.452900in}{0.640370in}}%
\pgfpathlineto{\pgfqpoint{4.457234in}{0.646916in}}%
\pgfpathlineto{\pgfqpoint{4.461542in}{0.592451in}}%
\pgfpathlineto{\pgfqpoint{4.470081in}{0.623279in}}%
\pgfpathlineto{\pgfqpoint{4.474312in}{0.622836in}}%
\pgfpathlineto{\pgfqpoint{4.478520in}{0.559140in}}%
\pgfpathlineto{\pgfqpoint{4.482702in}{0.669124in}}%
\pgfpathlineto{\pgfqpoint{4.486861in}{0.656690in}}%
\pgfpathlineto{\pgfqpoint{4.490996in}{0.612627in}}%
\pgfpathlineto{\pgfqpoint{4.495107in}{0.599400in}}%
\pgfpathlineto{\pgfqpoint{4.499195in}{0.661154in}}%
\pgfpathlineto{\pgfqpoint{4.503260in}{0.653715in}}%
\pgfpathlineto{\pgfqpoint{4.507303in}{0.602407in}}%
\pgfpathlineto{\pgfqpoint{4.511322in}{0.628608in}}%
\pgfpathlineto{\pgfqpoint{4.515320in}{0.570750in}}%
\pgfpathlineto{\pgfqpoint{4.519295in}{0.531571in}}%
\pgfpathlineto{\pgfqpoint{4.523249in}{0.622239in}}%
\pgfpathlineto{\pgfqpoint{4.527181in}{0.618837in}}%
\pgfpathlineto{\pgfqpoint{4.531092in}{0.622367in}}%
\pgfpathlineto{\pgfqpoint{4.534981in}{0.611979in}}%
\pgfpathlineto{\pgfqpoint{4.538850in}{0.595554in}}%
\pgfpathlineto{\pgfqpoint{4.542698in}{0.598389in}}%
\pgfpathlineto{\pgfqpoint{4.550334in}{0.666020in}}%
\pgfpathlineto{\pgfqpoint{4.557889in}{0.586475in}}%
\pgfpathlineto{\pgfqpoint{4.561637in}{0.605339in}}%
\pgfpathlineto{\pgfqpoint{4.565366in}{0.614943in}}%
\pgfpathlineto{\pgfqpoint{4.569076in}{0.601625in}}%
\pgfpathlineto{\pgfqpoint{4.572767in}{0.543378in}}%
\pgfpathlineto{\pgfqpoint{4.576438in}{0.532834in}}%
\pgfpathlineto{\pgfqpoint{4.580092in}{0.587258in}}%
\pgfpathlineto{\pgfqpoint{4.583727in}{0.557091in}}%
\pgfpathlineto{\pgfqpoint{4.587344in}{0.582321in}}%
\pgfpathlineto{\pgfqpoint{4.590942in}{0.565475in}}%
\pgfpathlineto{\pgfqpoint{4.594523in}{0.569933in}}%
\pgfpathlineto{\pgfqpoint{4.598086in}{0.579746in}}%
\pgfpathlineto{\pgfqpoint{4.601632in}{0.566573in}}%
\pgfpathlineto{\pgfqpoint{4.605160in}{0.630121in}}%
\pgfpathlineto{\pgfqpoint{4.608671in}{0.617794in}}%
\pgfpathlineto{\pgfqpoint{4.612165in}{0.559820in}}%
\pgfpathlineto{\pgfqpoint{4.619103in}{0.538962in}}%
\pgfpathlineto{\pgfqpoint{4.625975in}{0.619012in}}%
\pgfpathlineto{\pgfqpoint{4.629387in}{0.608572in}}%
\pgfpathlineto{\pgfqpoint{4.632782in}{0.568639in}}%
\pgfpathlineto{\pgfqpoint{4.636162in}{0.563414in}}%
\pgfpathlineto{\pgfqpoint{4.639525in}{0.615748in}}%
\pgfpathlineto{\pgfqpoint{4.646206in}{0.553059in}}%
\pgfpathlineto{\pgfqpoint{4.649524in}{0.540700in}}%
\pgfpathlineto{\pgfqpoint{4.652826in}{0.605226in}}%
\pgfpathlineto{\pgfqpoint{4.656113in}{0.605900in}}%
\pgfpathlineto{\pgfqpoint{4.659385in}{0.661530in}}%
\pgfpathlineto{\pgfqpoint{4.662643in}{0.653521in}}%
\pgfpathlineto{\pgfqpoint{4.665886in}{0.547663in}}%
\pgfpathlineto{\pgfqpoint{4.669114in}{0.552975in}}%
\pgfpathlineto{\pgfqpoint{4.672328in}{0.616293in}}%
\pgfpathlineto{\pgfqpoint{4.675528in}{0.625561in}}%
\pgfpathlineto{\pgfqpoint{4.678713in}{0.639471in}}%
\pgfpathlineto{\pgfqpoint{4.681885in}{0.615785in}}%
\pgfpathlineto{\pgfqpoint{4.685043in}{0.634535in}}%
\pgfpathlineto{\pgfqpoint{4.688186in}{0.596434in}}%
\pgfpathlineto{\pgfqpoint{4.691317in}{0.541242in}}%
\pgfpathlineto{\pgfqpoint{4.694434in}{0.596982in}}%
\pgfpathlineto{\pgfqpoint{4.700627in}{0.533886in}}%
\pgfpathlineto{\pgfqpoint{4.703704in}{0.584183in}}%
\pgfpathlineto{\pgfqpoint{4.706768in}{0.605583in}}%
\pgfpathlineto{\pgfqpoint{4.709819in}{0.577765in}}%
\pgfpathlineto{\pgfqpoint{4.712857in}{0.583672in}}%
\pgfpathlineto{\pgfqpoint{4.715882in}{0.604699in}}%
\pgfpathlineto{\pgfqpoint{4.718895in}{0.549198in}}%
\pgfpathlineto{\pgfqpoint{4.721895in}{0.524958in}}%
\pgfpathlineto{\pgfqpoint{4.724883in}{0.602632in}}%
\pgfpathlineto{\pgfqpoint{4.727858in}{0.609056in}}%
\pgfpathlineto{\pgfqpoint{4.730821in}{0.573218in}}%
\pgfpathlineto{\pgfqpoint{4.733772in}{0.557613in}}%
\pgfpathlineto{\pgfqpoint{4.736711in}{0.582234in}}%
\pgfpathlineto{\pgfqpoint{4.739638in}{0.574735in}}%
\pgfpathlineto{\pgfqpoint{4.742554in}{0.603146in}}%
\pgfpathlineto{\pgfqpoint{4.745457in}{0.578133in}}%
\pgfpathlineto{\pgfqpoint{4.748349in}{0.587615in}}%
\pgfpathlineto{\pgfqpoint{4.751230in}{0.605417in}}%
\pgfpathlineto{\pgfqpoint{4.754098in}{0.560947in}}%
\pgfpathlineto{\pgfqpoint{4.756956in}{0.553395in}}%
\pgfpathlineto{\pgfqpoint{4.759802in}{0.596859in}}%
\pgfpathlineto{\pgfqpoint{4.762637in}{0.598362in}}%
\pgfpathlineto{\pgfqpoint{4.765461in}{0.544010in}}%
\pgfpathlineto{\pgfqpoint{4.768274in}{0.538046in}}%
\pgfpathlineto{\pgfqpoint{4.771077in}{0.571518in}}%
\pgfpathlineto{\pgfqpoint{4.773868in}{0.581186in}}%
\pgfpathlineto{\pgfqpoint{4.776648in}{0.579796in}}%
\pgfpathlineto{\pgfqpoint{4.779418in}{0.593268in}}%
\pgfpathlineto{\pgfqpoint{4.782177in}{0.652433in}}%
\pgfpathlineto{\pgfqpoint{4.784926in}{0.660691in}}%
\pgfpathlineto{\pgfqpoint{4.787664in}{0.608159in}}%
\pgfpathlineto{\pgfqpoint{4.790392in}{0.586978in}}%
\pgfpathlineto{\pgfqpoint{4.793110in}{0.675057in}}%
\pgfpathlineto{\pgfqpoint{4.798515in}{0.520084in}}%
\pgfpathlineto{\pgfqpoint{4.801202in}{0.571723in}}%
\pgfpathlineto{\pgfqpoint{4.803879in}{0.584306in}}%
\pgfpathlineto{\pgfqpoint{4.806547in}{0.626916in}}%
\pgfpathlineto{\pgfqpoint{4.809204in}{0.618473in}}%
\pgfpathlineto{\pgfqpoint{4.814490in}{0.538569in}}%
\pgfpathlineto{\pgfqpoint{4.817119in}{0.532389in}}%
\pgfpathlineto{\pgfqpoint{4.824947in}{0.627876in}}%
\pgfpathlineto{\pgfqpoint{4.827538in}{0.558054in}}%
\pgfpathlineto{\pgfqpoint{4.830120in}{0.549129in}}%
\pgfpathlineto{\pgfqpoint{4.832692in}{0.628709in}}%
\pgfpathlineto{\pgfqpoint{4.835255in}{0.578979in}}%
\pgfpathlineto{\pgfqpoint{4.837809in}{0.574448in}}%
\pgfpathlineto{\pgfqpoint{4.840354in}{0.595193in}}%
\pgfpathlineto{\pgfqpoint{4.842891in}{0.632224in}}%
\pgfpathlineto{\pgfqpoint{4.845418in}{0.612502in}}%
\pgfpathlineto{\pgfqpoint{4.847936in}{0.527856in}}%
\pgfpathlineto{\pgfqpoint{4.852947in}{0.611490in}}%
\pgfpathlineto{\pgfqpoint{4.855439in}{0.539761in}}%
\pgfpathlineto{\pgfqpoint{4.857923in}{0.544413in}}%
\pgfpathlineto{\pgfqpoint{4.860398in}{0.643794in}}%
\pgfpathlineto{\pgfqpoint{4.862865in}{0.619938in}}%
\pgfpathlineto{\pgfqpoint{4.865323in}{0.562194in}}%
\pgfpathlineto{\pgfqpoint{4.867773in}{0.585763in}}%
\pgfpathlineto{\pgfqpoint{4.870215in}{0.584038in}}%
\pgfpathlineto{\pgfqpoint{4.872649in}{0.561486in}}%
\pgfpathlineto{\pgfqpoint{4.877491in}{0.636303in}}%
\pgfpathlineto{\pgfqpoint{4.879900in}{0.627323in}}%
\pgfpathlineto{\pgfqpoint{4.884695in}{0.560162in}}%
\pgfpathlineto{\pgfqpoint{4.887080in}{0.572686in}}%
\pgfpathlineto{\pgfqpoint{4.889457in}{0.641295in}}%
\pgfpathlineto{\pgfqpoint{4.891827in}{0.566289in}}%
\pgfpathlineto{\pgfqpoint{4.894189in}{0.550064in}}%
\pgfpathlineto{\pgfqpoint{4.898889in}{0.588193in}}%
\pgfpathlineto{\pgfqpoint{4.901228in}{0.573525in}}%
\pgfpathlineto{\pgfqpoint{4.903559in}{0.629200in}}%
\pgfpathlineto{\pgfqpoint{4.905883in}{0.608858in}}%
\pgfpathlineto{\pgfqpoint{4.908199in}{0.563017in}}%
\pgfpathlineto{\pgfqpoint{4.910508in}{0.612041in}}%
\pgfpathlineto{\pgfqpoint{4.912810in}{0.552705in}}%
\pgfpathlineto{\pgfqpoint{4.917391in}{0.528781in}}%
\pgfpathlineto{\pgfqpoint{4.921943in}{0.576089in}}%
\pgfpathlineto{\pgfqpoint{4.924209in}{0.995801in}}%
\pgfpathlineto{\pgfqpoint{4.926467in}{0.565523in}}%
\pgfpathlineto{\pgfqpoint{4.928718in}{0.524116in}}%
\pgfpathlineto{\pgfqpoint{4.930963in}{0.601760in}}%
\pgfpathlineto{\pgfqpoint{4.933200in}{0.589145in}}%
\pgfpathlineto{\pgfqpoint{4.935430in}{0.526600in}}%
\pgfpathlineto{\pgfqpoint{4.937654in}{0.616480in}}%
\pgfpathlineto{\pgfqpoint{4.939871in}{0.570462in}}%
\pgfpathlineto{\pgfqpoint{4.942081in}{0.567730in}}%
\pgfpathlineto{\pgfqpoint{4.944284in}{0.547364in}}%
\pgfpathlineto{\pgfqpoint{4.946480in}{0.603387in}}%
\pgfpathlineto{\pgfqpoint{4.950853in}{0.524318in}}%
\pgfpathlineto{\pgfqpoint{4.953030in}{0.528460in}}%
\pgfpathlineto{\pgfqpoint{4.955200in}{0.525228in}}%
\pgfpathlineto{\pgfqpoint{4.957363in}{0.554937in}}%
\pgfpathlineto{\pgfqpoint{4.959520in}{0.551637in}}%
\pgfpathlineto{\pgfqpoint{4.961671in}{0.557991in}}%
\pgfpathlineto{\pgfqpoint{4.963760in}{1.200092in}}%
\pgfpathmoveto{\pgfqpoint{4.963871in}{1.200092in}}%
\pgfpathlineto{\pgfqpoint{4.965953in}{0.565899in}}%
\pgfpathlineto{\pgfqpoint{4.968085in}{0.586758in}}%
\pgfpathlineto{\pgfqpoint{4.970210in}{0.559408in}}%
\pgfpathlineto{\pgfqpoint{4.972329in}{0.571845in}}%
\pgfpathlineto{\pgfqpoint{4.974442in}{0.570312in}}%
\pgfpathlineto{\pgfqpoint{4.976548in}{0.586466in}}%
\pgfpathlineto{\pgfqpoint{4.978649in}{0.588321in}}%
\pgfpathlineto{\pgfqpoint{4.982832in}{0.549821in}}%
\pgfpathlineto{\pgfqpoint{4.989061in}{0.590555in}}%
\pgfpathlineto{\pgfqpoint{4.991125in}{0.539428in}}%
\pgfpathlineto{\pgfqpoint{4.993184in}{0.563976in}}%
\pgfpathlineto{\pgfqpoint{4.995237in}{0.513355in}}%
\pgfpathlineto{\pgfqpoint{4.997284in}{0.550873in}}%
\pgfpathlineto{\pgfqpoint{4.999325in}{0.561384in}}%
\pgfpathlineto{\pgfqpoint{5.001360in}{0.556197in}}%
\pgfpathlineto{\pgfqpoint{5.003390in}{0.535268in}}%
\pgfpathlineto{\pgfqpoint{5.007432in}{0.602896in}}%
\pgfpathlineto{\pgfqpoint{5.009445in}{0.577278in}}%
\pgfpathlineto{\pgfqpoint{5.011452in}{0.607692in}}%
\pgfpathlineto{\pgfqpoint{5.013453in}{0.595700in}}%
\pgfpathlineto{\pgfqpoint{5.015449in}{0.605803in}}%
\pgfpathlineto{\pgfqpoint{5.017439in}{0.572329in}}%
\pgfpathlineto{\pgfqpoint{5.019074in}{1.200092in}}%
\pgfpathmoveto{\pgfqpoint{5.021530in}{1.200092in}}%
\pgfpathlineto{\pgfqpoint{5.023378in}{0.605175in}}%
\pgfpathlineto{\pgfqpoint{5.025347in}{0.600784in}}%
\pgfpathlineto{\pgfqpoint{5.029268in}{0.562160in}}%
\pgfpathlineto{\pgfqpoint{5.031221in}{0.581554in}}%
\pgfpathlineto{\pgfqpoint{5.033168in}{0.641826in}}%
\pgfpathlineto{\pgfqpoint{5.037048in}{0.579310in}}%
\pgfpathlineto{\pgfqpoint{5.038979in}{0.555151in}}%
\pgfpathlineto{\pgfqpoint{5.042828in}{0.597485in}}%
\pgfpathlineto{\pgfqpoint{5.044744in}{0.580798in}}%
\pgfpathlineto{\pgfqpoint{5.046655in}{0.543425in}}%
\pgfpathlineto{\pgfqpoint{5.048562in}{0.532397in}}%
\pgfpathlineto{\pgfqpoint{5.050463in}{0.535352in}}%
\pgfpathlineto{\pgfqpoint{5.052359in}{0.532762in}}%
\pgfpathlineto{\pgfqpoint{5.054251in}{0.595142in}}%
\pgfpathlineto{\pgfqpoint{5.058018in}{0.611353in}}%
\pgfpathlineto{\pgfqpoint{5.059895in}{0.685033in}}%
\pgfpathlineto{\pgfqpoint{5.061767in}{0.638770in}}%
\pgfpathlineto{\pgfqpoint{5.065495in}{0.523963in}}%
\pgfpathlineto{\pgfqpoint{5.069205in}{0.579558in}}%
\pgfpathlineto{\pgfqpoint{5.071053in}{0.596701in}}%
\pgfpathlineto{\pgfqpoint{5.072896in}{0.565575in}}%
\pgfpathlineto{\pgfqpoint{5.074734in}{0.560562in}}%
\pgfpathlineto{\pgfqpoint{5.076568in}{0.867678in}}%
\pgfpathlineto{\pgfqpoint{5.078397in}{0.843032in}}%
\pgfpathlineto{\pgfqpoint{5.080221in}{0.567367in}}%
\pgfpathlineto{\pgfqpoint{5.082041in}{0.564663in}}%
\pgfpathlineto{\pgfqpoint{5.083856in}{0.564164in}}%
\pgfpathlineto{\pgfqpoint{5.085667in}{0.547979in}}%
\pgfpathlineto{\pgfqpoint{5.087473in}{0.595865in}}%
\pgfpathlineto{\pgfqpoint{5.091071in}{0.539182in}}%
\pgfpathlineto{\pgfqpoint{5.094652in}{0.602526in}}%
\pgfpathlineto{\pgfqpoint{5.096436in}{0.547633in}}%
\pgfpathlineto{\pgfqpoint{5.098215in}{0.556200in}}%
\pgfpathlineto{\pgfqpoint{5.099990in}{0.633922in}}%
\pgfpathlineto{\pgfqpoint{5.103527in}{0.531002in}}%
\pgfpathlineto{\pgfqpoint{5.107047in}{0.682431in}}%
\pgfpathlineto{\pgfqpoint{5.110550in}{0.578652in}}%
\pgfpathlineto{\pgfqpoint{5.112295in}{0.607542in}}%
\pgfpathlineto{\pgfqpoint{5.114035in}{0.562247in}}%
\pgfpathlineto{\pgfqpoint{5.115772in}{0.552842in}}%
\pgfpathlineto{\pgfqpoint{5.119232in}{0.625143in}}%
\pgfpathlineto{\pgfqpoint{5.122677in}{0.568567in}}%
\pgfpathlineto{\pgfqpoint{5.124392in}{0.570577in}}%
\pgfpathlineto{\pgfqpoint{5.127812in}{0.581457in}}%
\pgfpathlineto{\pgfqpoint{5.129516in}{0.555814in}}%
\pgfpathlineto{\pgfqpoint{5.131215in}{0.549769in}}%
\pgfpathlineto{\pgfqpoint{5.132911in}{0.631700in}}%
\pgfpathlineto{\pgfqpoint{5.134603in}{0.546601in}}%
\pgfpathlineto{\pgfqpoint{5.136291in}{0.555528in}}%
\pgfpathlineto{\pgfqpoint{5.137975in}{0.707744in}}%
\pgfpathlineto{\pgfqpoint{5.139655in}{0.676183in}}%
\pgfpathlineto{\pgfqpoint{5.141331in}{0.562715in}}%
\pgfpathlineto{\pgfqpoint{5.143003in}{0.560126in}}%
\pgfpathlineto{\pgfqpoint{5.144671in}{0.549850in}}%
\pgfpathlineto{\pgfqpoint{5.146336in}{0.559099in}}%
\pgfpathlineto{\pgfqpoint{5.147996in}{0.597543in}}%
\pgfpathlineto{\pgfqpoint{5.149653in}{0.604378in}}%
\pgfpathlineto{\pgfqpoint{5.151306in}{0.563768in}}%
\pgfpathlineto{\pgfqpoint{5.152955in}{0.581791in}}%
\pgfpathlineto{\pgfqpoint{5.154601in}{0.629540in}}%
\pgfpathlineto{\pgfqpoint{5.156242in}{0.600495in}}%
\pgfpathlineto{\pgfqpoint{5.157880in}{0.538953in}}%
\pgfpathlineto{\pgfqpoint{5.161145in}{0.569902in}}%
\pgfpathlineto{\pgfqpoint{5.162772in}{0.565891in}}%
\pgfpathlineto{\pgfqpoint{5.164395in}{0.602047in}}%
\pgfpathlineto{\pgfqpoint{5.167631in}{0.546830in}}%
\pgfpathlineto{\pgfqpoint{5.169243in}{0.640689in}}%
\pgfpathlineto{\pgfqpoint{5.170852in}{0.603631in}}%
\pgfpathlineto{\pgfqpoint{5.172457in}{0.625679in}}%
\pgfpathlineto{\pgfqpoint{5.174059in}{0.619872in}}%
\pgfpathlineto{\pgfqpoint{5.175657in}{0.558142in}}%
\pgfpathlineto{\pgfqpoint{5.177252in}{0.551848in}}%
\pgfpathlineto{\pgfqpoint{5.178843in}{0.570904in}}%
\pgfpathlineto{\pgfqpoint{5.182014in}{0.654385in}}%
\pgfpathlineto{\pgfqpoint{5.183595in}{0.676602in}}%
\pgfpathlineto{\pgfqpoint{5.185172in}{0.617101in}}%
\pgfpathlineto{\pgfqpoint{5.186746in}{0.635198in}}%
\pgfpathlineto{\pgfqpoint{5.188316in}{0.567018in}}%
\pgfpathlineto{\pgfqpoint{5.189883in}{0.556559in}}%
\pgfpathlineto{\pgfqpoint{5.191446in}{0.553829in}}%
\pgfpathlineto{\pgfqpoint{5.193006in}{0.545209in}}%
\pgfpathlineto{\pgfqpoint{5.194563in}{0.565933in}}%
\pgfpathlineto{\pgfqpoint{5.196116in}{0.560050in}}%
\pgfpathlineto{\pgfqpoint{5.197666in}{0.690605in}}%
\pgfpathlineto{\pgfqpoint{5.199213in}{0.673675in}}%
\pgfpathlineto{\pgfqpoint{5.200756in}{0.595376in}}%
\pgfpathlineto{\pgfqpoint{5.202296in}{0.595248in}}%
\pgfpathlineto{\pgfqpoint{5.203833in}{0.598685in}}%
\pgfpathlineto{\pgfqpoint{5.205367in}{0.523341in}}%
\pgfpathlineto{\pgfqpoint{5.206897in}{0.576703in}}%
\pgfpathlineto{\pgfqpoint{5.208424in}{0.583222in}}%
\pgfpathlineto{\pgfqpoint{5.209948in}{0.565406in}}%
\pgfpathlineto{\pgfqpoint{5.211469in}{0.569901in}}%
\pgfpathlineto{\pgfqpoint{5.212986in}{0.560370in}}%
\pgfpathlineto{\pgfqpoint{5.214500in}{0.566341in}}%
\pgfpathlineto{\pgfqpoint{5.216011in}{0.604145in}}%
\pgfpathlineto{\pgfqpoint{5.217519in}{0.598577in}}%
\pgfpathlineto{\pgfqpoint{5.219024in}{0.541153in}}%
\pgfpathlineto{\pgfqpoint{5.220526in}{0.586296in}}%
\pgfpathlineto{\pgfqpoint{5.222024in}{0.586692in}}%
\pgfpathlineto{\pgfqpoint{5.223520in}{0.564483in}}%
\pgfpathlineto{\pgfqpoint{5.225012in}{0.589297in}}%
\pgfpathlineto{\pgfqpoint{5.226501in}{0.572977in}}%
\pgfpathlineto{\pgfqpoint{5.227987in}{0.600819in}}%
\pgfpathlineto{\pgfqpoint{5.229470in}{0.547612in}}%
\pgfpathlineto{\pgfqpoint{5.232427in}{0.607590in}}%
\pgfpathlineto{\pgfqpoint{5.233901in}{0.544314in}}%
\pgfpathlineto{\pgfqpoint{5.236841in}{0.593523in}}%
\pgfpathlineto{\pgfqpoint{5.238306in}{0.551816in}}%
\pgfpathlineto{\pgfqpoint{5.239768in}{0.551153in}}%
\pgfpathlineto{\pgfqpoint{5.241227in}{0.547797in}}%
\pgfpathlineto{\pgfqpoint{5.242683in}{0.553677in}}%
\pgfpathlineto{\pgfqpoint{5.245587in}{0.606604in}}%
\pgfpathlineto{\pgfqpoint{5.248478in}{0.561985in}}%
\pgfpathlineto{\pgfqpoint{5.249920in}{0.601972in}}%
\pgfpathlineto{\pgfqpoint{5.251359in}{0.607171in}}%
\pgfpathlineto{\pgfqpoint{5.252795in}{0.581792in}}%
\pgfpathlineto{\pgfqpoint{5.254228in}{0.609187in}}%
\pgfpathlineto{\pgfqpoint{5.257085in}{0.544488in}}%
\pgfpathlineto{\pgfqpoint{5.258510in}{0.623788in}}%
\pgfpathlineto{\pgfqpoint{5.259932in}{0.555225in}}%
\pgfpathlineto{\pgfqpoint{5.261351in}{0.555745in}}%
\pgfpathlineto{\pgfqpoint{5.262767in}{0.568773in}}%
\pgfpathlineto{\pgfqpoint{5.264180in}{0.517257in}}%
\pgfpathlineto{\pgfqpoint{5.265591in}{0.568767in}}%
\pgfpathlineto{\pgfqpoint{5.266999in}{0.688165in}}%
\pgfpathlineto{\pgfqpoint{5.268404in}{0.557238in}}%
\pgfpathlineto{\pgfqpoint{5.272603in}{0.637586in}}%
\pgfpathlineto{\pgfqpoint{5.273997in}{0.622921in}}%
\pgfpathlineto{\pgfqpoint{5.275389in}{0.650121in}}%
\pgfpathlineto{\pgfqpoint{5.278164in}{0.583861in}}%
\pgfpathlineto{\pgfqpoint{5.279547in}{0.599311in}}%
\pgfpathlineto{\pgfqpoint{5.280928in}{0.538140in}}%
\pgfpathlineto{\pgfqpoint{5.282307in}{0.549787in}}%
\pgfpathlineto{\pgfqpoint{5.283682in}{0.617004in}}%
\pgfpathlineto{\pgfqpoint{5.285055in}{0.538217in}}%
\pgfpathlineto{\pgfqpoint{5.286426in}{0.604794in}}%
\pgfpathlineto{\pgfqpoint{5.287794in}{0.615101in}}%
\pgfpathlineto{\pgfqpoint{5.289159in}{0.563944in}}%
\pgfpathlineto{\pgfqpoint{5.290521in}{0.584548in}}%
\pgfpathlineto{\pgfqpoint{5.294594in}{0.507054in}}%
\pgfpathlineto{\pgfqpoint{5.298644in}{0.649921in}}%
\pgfpathlineto{\pgfqpoint{5.299989in}{0.658818in}}%
\pgfpathlineto{\pgfqpoint{5.301331in}{0.556643in}}%
\pgfpathlineto{\pgfqpoint{5.302671in}{0.561273in}}%
\pgfpathlineto{\pgfqpoint{5.304008in}{0.529955in}}%
\pgfpathlineto{\pgfqpoint{5.306676in}{0.588615in}}%
\pgfpathlineto{\pgfqpoint{5.308006in}{0.567862in}}%
\pgfpathlineto{\pgfqpoint{5.309333in}{0.574474in}}%
\pgfpathlineto{\pgfqpoint{5.310659in}{0.606051in}}%
\pgfpathlineto{\pgfqpoint{5.313302in}{0.554873in}}%
\pgfpathlineto{\pgfqpoint{5.314619in}{0.562130in}}%
\pgfpathlineto{\pgfqpoint{5.315935in}{0.587779in}}%
\pgfpathlineto{\pgfqpoint{5.318559in}{0.525021in}}%
\pgfpathlineto{\pgfqpoint{5.321173in}{0.598872in}}%
\pgfpathlineto{\pgfqpoint{5.322477in}{0.577449in}}%
\pgfpathlineto{\pgfqpoint{5.325077in}{0.579076in}}%
\pgfpathlineto{\pgfqpoint{5.326373in}{0.530785in}}%
\pgfpathlineto{\pgfqpoint{5.327667in}{0.554393in}}%
\pgfpathlineto{\pgfqpoint{5.328959in}{0.556705in}}%
\pgfpathlineto{\pgfqpoint{5.330249in}{0.537181in}}%
\pgfpathlineto{\pgfqpoint{5.331536in}{0.605723in}}%
\pgfpathlineto{\pgfqpoint{5.332821in}{0.590980in}}%
\pgfpathlineto{\pgfqpoint{5.335384in}{0.620564in}}%
\pgfpathlineto{\pgfqpoint{5.337939in}{0.552908in}}%
\pgfpathlineto{\pgfqpoint{5.339212in}{0.533806in}}%
\pgfpathlineto{\pgfqpoint{5.340484in}{0.545784in}}%
\pgfpathlineto{\pgfqpoint{5.341753in}{0.581317in}}%
\pgfpathlineto{\pgfqpoint{5.344285in}{0.578883in}}%
\pgfpathlineto{\pgfqpoint{5.345547in}{0.563157in}}%
\pgfpathlineto{\pgfqpoint{5.346807in}{0.594545in}}%
\pgfpathlineto{\pgfqpoint{5.349321in}{0.528215in}}%
\pgfpathlineto{\pgfqpoint{5.350575in}{0.595448in}}%
\pgfpathlineto{\pgfqpoint{5.351827in}{0.586999in}}%
\pgfpathlineto{\pgfqpoint{5.353076in}{0.547546in}}%
\pgfpathlineto{\pgfqpoint{5.354323in}{0.573232in}}%
\pgfpathlineto{\pgfqpoint{5.355569in}{0.655550in}}%
\pgfpathlineto{\pgfqpoint{5.356811in}{0.640695in}}%
\pgfpathlineto{\pgfqpoint{5.359291in}{0.545042in}}%
\pgfpathlineto{\pgfqpoint{5.360528in}{0.533079in}}%
\pgfpathlineto{\pgfqpoint{5.361762in}{0.621109in}}%
\pgfpathlineto{\pgfqpoint{5.362994in}{0.603422in}}%
\pgfpathlineto{\pgfqpoint{5.365453in}{0.547203in}}%
\pgfpathlineto{\pgfqpoint{5.367903in}{0.593154in}}%
\pgfpathlineto{\pgfqpoint{5.369125in}{0.582997in}}%
\pgfpathlineto{\pgfqpoint{5.370344in}{0.562447in}}%
\pgfpathlineto{\pgfqpoint{5.371562in}{0.595194in}}%
\pgfpathlineto{\pgfqpoint{5.375203in}{0.562008in}}%
\pgfpathlineto{\pgfqpoint{5.376413in}{0.604480in}}%
\pgfpathlineto{\pgfqpoint{5.377621in}{0.583464in}}%
\pgfpathlineto{\pgfqpoint{5.378826in}{0.597177in}}%
\pgfpathlineto{\pgfqpoint{5.381231in}{0.547596in}}%
\pgfpathlineto{\pgfqpoint{5.382431in}{0.551486in}}%
\pgfpathlineto{\pgfqpoint{5.386018in}{0.595505in}}%
\pgfpathlineto{\pgfqpoint{5.387209in}{0.581560in}}%
\pgfpathlineto{\pgfqpoint{5.388399in}{0.629935in}}%
\pgfpathlineto{\pgfqpoint{5.390772in}{0.530473in}}%
\pgfpathlineto{\pgfqpoint{5.391956in}{0.595891in}}%
\pgfpathlineto{\pgfqpoint{5.393138in}{0.574716in}}%
\pgfpathlineto{\pgfqpoint{5.394318in}{0.606944in}}%
\pgfpathlineto{\pgfqpoint{5.396672in}{0.509469in}}%
\pgfpathlineto{\pgfqpoint{5.397846in}{0.531829in}}%
\pgfpathlineto{\pgfqpoint{5.400189in}{0.522917in}}%
\pgfpathlineto{\pgfqpoint{5.403689in}{0.614253in}}%
\pgfpathlineto{\pgfqpoint{5.404851in}{0.568829in}}%
\pgfpathlineto{\pgfqpoint{5.406012in}{0.570319in}}%
\pgfpathlineto{\pgfqpoint{5.408329in}{0.536749in}}%
\pgfpathlineto{\pgfqpoint{5.409484in}{0.792386in}}%
\pgfpathlineto{\pgfqpoint{5.411789in}{0.575212in}}%
\pgfpathlineto{\pgfqpoint{5.414087in}{0.516330in}}%
\pgfpathlineto{\pgfqpoint{5.416378in}{0.560369in}}%
\pgfpathlineto{\pgfqpoint{5.417520in}{0.893385in}}%
\pgfpathlineto{\pgfqpoint{5.418661in}{0.540086in}}%
\pgfpathlineto{\pgfqpoint{5.420937in}{0.576965in}}%
\pgfpathlineto{\pgfqpoint{5.422073in}{0.552282in}}%
\pgfpathlineto{\pgfqpoint{5.423206in}{0.588721in}}%
\pgfpathlineto{\pgfqpoint{5.424338in}{0.547519in}}%
\pgfpathlineto{\pgfqpoint{5.426596in}{0.627097in}}%
\pgfpathlineto{\pgfqpoint{5.427723in}{0.617862in}}%
\pgfpathlineto{\pgfqpoint{5.428848in}{0.589875in}}%
\pgfpathlineto{\pgfqpoint{5.429971in}{0.598616in}}%
\pgfpathlineto{\pgfqpoint{5.431092in}{0.618635in}}%
\pgfpathlineto{\pgfqpoint{5.432211in}{0.564098in}}%
\pgfpathlineto{\pgfqpoint{5.433329in}{0.588857in}}%
\pgfpathlineto{\pgfqpoint{5.434445in}{0.561911in}}%
\pgfpathlineto{\pgfqpoint{5.435560in}{0.691443in}}%
\pgfpathlineto{\pgfqpoint{5.436672in}{0.575905in}}%
\pgfpathlineto{\pgfqpoint{5.438892in}{0.604036in}}%
\pgfpathlineto{\pgfqpoint{5.440000in}{0.599971in}}%
\pgfpathlineto{\pgfqpoint{5.441106in}{0.607023in}}%
\pgfpathlineto{\pgfqpoint{5.443312in}{0.517803in}}%
\pgfpathlineto{\pgfqpoint{5.445512in}{0.587938in}}%
\pgfpathlineto{\pgfqpoint{5.447705in}{0.551241in}}%
\pgfpathlineto{\pgfqpoint{5.448799in}{0.623939in}}%
\pgfpathlineto{\pgfqpoint{5.449892in}{0.551056in}}%
\pgfpathlineto{\pgfqpoint{5.450982in}{0.574581in}}%
\pgfpathlineto{\pgfqpoint{5.452071in}{0.622255in}}%
\pgfpathlineto{\pgfqpoint{5.454245in}{0.549446in}}%
\pgfpathlineto{\pgfqpoint{5.455329in}{0.575456in}}%
\pgfpathlineto{\pgfqpoint{5.456412in}{0.552583in}}%
\pgfpathlineto{\pgfqpoint{5.457492in}{0.555474in}}%
\pgfpathlineto{\pgfqpoint{5.458572in}{0.537929in}}%
\pgfpathlineto{\pgfqpoint{5.459650in}{0.545290in}}%
\pgfpathlineto{\pgfqpoint{5.460726in}{0.490900in}}%
\pgfpathlineto{\pgfqpoint{5.462873in}{0.590018in}}%
\pgfpathlineto{\pgfqpoint{5.463944in}{0.548813in}}%
\pgfpathlineto{\pgfqpoint{5.465014in}{0.580572in}}%
\pgfpathlineto{\pgfqpoint{5.466082in}{0.559775in}}%
\pgfpathlineto{\pgfqpoint{5.467149in}{0.575635in}}%
\pgfpathlineto{\pgfqpoint{5.468214in}{0.622216in}}%
\pgfpathlineto{\pgfqpoint{5.470339in}{0.573958in}}%
\pgfpathlineto{\pgfqpoint{5.471399in}{0.589284in}}%
\pgfpathlineto{\pgfqpoint{5.473515in}{0.531057in}}%
\pgfpathlineto{\pgfqpoint{5.475625in}{0.593589in}}%
\pgfpathlineto{\pgfqpoint{5.476678in}{0.581068in}}%
\pgfpathlineto{\pgfqpoint{5.477729in}{0.683403in}}%
\pgfpathlineto{\pgfqpoint{5.478778in}{0.681630in}}%
\pgfpathlineto{\pgfqpoint{5.481918in}{0.555578in}}%
\pgfpathlineto{\pgfqpoint{5.482961in}{0.572910in}}%
\pgfpathlineto{\pgfqpoint{5.485043in}{0.528384in}}%
\pgfpathlineto{\pgfqpoint{5.487120in}{0.588042in}}%
\pgfpathlineto{\pgfqpoint{5.488156in}{0.537351in}}%
\pgfpathlineto{\pgfqpoint{5.490223in}{0.544494in}}%
\pgfpathlineto{\pgfqpoint{5.491255in}{0.539258in}}%
\pgfpathlineto{\pgfqpoint{5.493313in}{0.584093in}}%
\pgfpathlineto{\pgfqpoint{5.494340in}{0.561608in}}%
\pgfpathlineto{\pgfqpoint{5.496390in}{0.583850in}}%
\pgfpathlineto{\pgfqpoint{5.497413in}{0.608522in}}%
\pgfpathlineto{\pgfqpoint{5.498434in}{0.568647in}}%
\pgfpathlineto{\pgfqpoint{5.499454in}{0.617128in}}%
\pgfpathlineto{\pgfqpoint{5.500472in}{0.615541in}}%
\pgfpathlineto{\pgfqpoint{5.501489in}{0.588102in}}%
\pgfpathlineto{\pgfqpoint{5.502505in}{0.604216in}}%
\pgfpathlineto{\pgfqpoint{5.503519in}{0.591985in}}%
\pgfpathlineto{\pgfqpoint{5.505543in}{0.523894in}}%
\pgfpathlineto{\pgfqpoint{5.506553in}{0.528883in}}%
\pgfpathlineto{\pgfqpoint{5.508568in}{0.558290in}}%
\pgfpathlineto{\pgfqpoint{5.510578in}{0.605634in}}%
\pgfpathlineto{\pgfqpoint{5.512582in}{0.554146in}}%
\pgfpathlineto{\pgfqpoint{5.513582in}{0.874180in}}%
\pgfpathlineto{\pgfqpoint{5.514581in}{0.870404in}}%
\pgfpathlineto{\pgfqpoint{5.516574in}{0.530637in}}%
\pgfpathlineto{\pgfqpoint{5.517569in}{0.537089in}}%
\pgfpathlineto{\pgfqpoint{5.518562in}{0.986836in}}%
\pgfpathlineto{\pgfqpoint{5.520544in}{0.603835in}}%
\pgfpathlineto{\pgfqpoint{5.521533in}{0.615159in}}%
\pgfpathlineto{\pgfqpoint{5.523507in}{0.554709in}}%
\pgfpathlineto{\pgfqpoint{5.525476in}{0.599654in}}%
\pgfpathlineto{\pgfqpoint{5.527439in}{0.555197in}}%
\pgfpathlineto{\pgfqpoint{5.528419in}{0.628924in}}%
\pgfpathlineto{\pgfqpoint{5.529397in}{0.560087in}}%
\pgfpathlineto{\pgfqpoint{5.531350in}{0.611706in}}%
\pgfpathlineto{\pgfqpoint{5.535240in}{0.545036in}}%
\pgfpathlineto{\pgfqpoint{5.536209in}{0.589080in}}%
\pgfpathlineto{\pgfqpoint{5.537177in}{0.530749in}}%
\pgfpathlineto{\pgfqpoint{5.539109in}{0.583156in}}%
\pgfpathlineto{\pgfqpoint{5.540073in}{0.555001in}}%
\pgfpathlineto{\pgfqpoint{5.541035in}{0.615185in}}%
\pgfpathlineto{\pgfqpoint{5.542957in}{0.539373in}}%
\pgfpathlineto{\pgfqpoint{5.543916in}{0.519904in}}%
\pgfpathlineto{\pgfqpoint{5.545830in}{0.588926in}}%
\pgfpathlineto{\pgfqpoint{5.546785in}{0.533221in}}%
\pgfpathlineto{\pgfqpoint{5.547738in}{0.534569in}}%
\pgfpathlineto{\pgfqpoint{5.548691in}{0.584744in}}%
\pgfpathlineto{\pgfqpoint{5.549642in}{0.564168in}}%
\pgfpathlineto{\pgfqpoint{5.550592in}{0.620298in}}%
\pgfpathlineto{\pgfqpoint{5.552488in}{0.537541in}}%
\pgfpathlineto{\pgfqpoint{5.553435in}{0.599546in}}%
\pgfpathlineto{\pgfqpoint{5.556266in}{0.522590in}}%
\pgfpathlineto{\pgfqpoint{5.558148in}{0.622682in}}%
\pgfpathlineto{\pgfqpoint{5.559086in}{0.542613in}}%
\pgfpathlineto{\pgfqpoint{5.560961in}{0.558536in}}%
\pgfpathlineto{\pgfqpoint{5.561896in}{0.553043in}}%
\pgfpathlineto{\pgfqpoint{5.562830in}{0.576264in}}%
\pgfpathlineto{\pgfqpoint{5.563763in}{0.526444in}}%
\pgfpathlineto{\pgfqpoint{5.565625in}{0.586375in}}%
\pgfpathlineto{\pgfqpoint{5.566554in}{0.556795in}}%
\pgfpathlineto{\pgfqpoint{5.568409in}{0.586215in}}%
\pgfpathlineto{\pgfqpoint{5.569334in}{0.539902in}}%
\pgfpathlineto{\pgfqpoint{5.571182in}{0.585068in}}%
\pgfpathlineto{\pgfqpoint{5.572104in}{0.516553in}}%
\pgfpathlineto{\pgfqpoint{5.573025in}{0.552968in}}%
\pgfpathlineto{\pgfqpoint{5.573945in}{0.551268in}}%
\pgfpathlineto{\pgfqpoint{5.574863in}{0.520908in}}%
\pgfpathlineto{\pgfqpoint{5.575781in}{0.574975in}}%
\pgfpathlineto{\pgfqpoint{5.576697in}{0.548420in}}%
\pgfpathlineto{\pgfqpoint{5.577612in}{0.566474in}}%
\pgfpathlineto{\pgfqpoint{5.578526in}{0.651602in}}%
\pgfpathlineto{\pgfqpoint{5.579439in}{0.651367in}}%
\pgfpathlineto{\pgfqpoint{5.581261in}{0.556166in}}%
\pgfpathlineto{\pgfqpoint{5.582170in}{0.556892in}}%
\pgfpathlineto{\pgfqpoint{5.583985in}{0.550666in}}%
\pgfpathlineto{\pgfqpoint{5.584891in}{0.595032in}}%
\pgfpathlineto{\pgfqpoint{5.587602in}{0.552028in}}%
\pgfpathlineto{\pgfqpoint{5.589404in}{0.573731in}}%
\pgfpathlineto{\pgfqpoint{5.590303in}{0.523374in}}%
\pgfpathlineto{\pgfqpoint{5.591201in}{0.536767in}}%
\pgfpathlineto{\pgfqpoint{5.592098in}{0.572386in}}%
\pgfpathlineto{\pgfqpoint{5.593888in}{0.524295in}}%
\pgfpathlineto{\pgfqpoint{5.595674in}{0.602996in}}%
\pgfpathlineto{\pgfqpoint{5.596565in}{0.555314in}}%
\pgfpathlineto{\pgfqpoint{5.597456in}{0.565947in}}%
\pgfpathlineto{\pgfqpoint{5.598345in}{0.615440in}}%
\pgfpathlineto{\pgfqpoint{5.599233in}{0.579815in}}%
\pgfpathlineto{\pgfqpoint{5.600120in}{0.610644in}}%
\pgfpathlineto{\pgfqpoint{5.601006in}{0.606546in}}%
\pgfpathlineto{\pgfqpoint{5.601890in}{0.591471in}}%
\pgfpathlineto{\pgfqpoint{5.603657in}{0.652929in}}%
\pgfpathlineto{\pgfqpoint{5.604538in}{0.647763in}}%
\pgfpathlineto{\pgfqpoint{5.606298in}{0.542231in}}%
\pgfpathlineto{\pgfqpoint{5.607176in}{0.551250in}}%
\pgfpathlineto{\pgfqpoint{5.608054in}{0.549435in}}%
\pgfpathlineto{\pgfqpoint{5.608930in}{0.544722in}}%
\pgfpathlineto{\pgfqpoint{5.609805in}{0.592133in}}%
\pgfpathlineto{\pgfqpoint{5.610679in}{0.591534in}}%
\pgfpathlineto{\pgfqpoint{5.612424in}{0.543190in}}%
\pgfpathlineto{\pgfqpoint{5.613295in}{0.642375in}}%
\pgfpathlineto{\pgfqpoint{5.614165in}{0.572597in}}%
\pgfpathlineto{\pgfqpoint{5.615901in}{0.631246in}}%
\pgfpathlineto{\pgfqpoint{5.616768in}{0.563539in}}%
\pgfpathlineto{\pgfqpoint{5.617633in}{0.568573in}}%
\pgfpathlineto{\pgfqpoint{5.618498in}{0.582283in}}%
\pgfpathlineto{\pgfqpoint{5.619362in}{0.933002in}}%
\pgfpathlineto{\pgfqpoint{5.621086in}{0.529533in}}%
\pgfpathlineto{\pgfqpoint{5.621946in}{0.590467in}}%
\pgfpathlineto{\pgfqpoint{5.622806in}{0.574482in}}%
\pgfpathlineto{\pgfqpoint{5.623664in}{0.524414in}}%
\pgfpathlineto{\pgfqpoint{5.624522in}{0.531474in}}%
\pgfpathlineto{\pgfqpoint{5.625378in}{0.577214in}}%
\pgfpathlineto{\pgfqpoint{5.626234in}{0.565763in}}%
\pgfpathlineto{\pgfqpoint{5.627941in}{0.548809in}}%
\pgfpathlineto{\pgfqpoint{5.628794in}{0.587389in}}%
\pgfpathlineto{\pgfqpoint{5.629645in}{0.531137in}}%
\pgfpathlineto{\pgfqpoint{5.631345in}{0.577346in}}%
\pgfpathlineto{\pgfqpoint{5.633887in}{0.506619in}}%
\pgfpathlineto{\pgfqpoint{5.634732in}{0.785884in}}%
\pgfpathlineto{\pgfqpoint{5.636420in}{0.547183in}}%
\pgfpathlineto{\pgfqpoint{5.637262in}{0.569146in}}%
\pgfpathlineto{\pgfqpoint{5.638104in}{0.560812in}}%
\pgfpathlineto{\pgfqpoint{5.638944in}{0.566415in}}%
\pgfpathlineto{\pgfqpoint{5.639784in}{0.515013in}}%
\pgfpathlineto{\pgfqpoint{5.643132in}{0.577791in}}%
\pgfpathlineto{\pgfqpoint{5.643967in}{0.534813in}}%
\pgfpathlineto{\pgfqpoint{5.646465in}{0.580332in}}%
\pgfpathlineto{\pgfqpoint{5.647296in}{0.569382in}}%
\pgfpathlineto{\pgfqpoint{5.648125in}{0.541360in}}%
\pgfpathlineto{\pgfqpoint{5.649782in}{0.572891in}}%
\pgfpathlineto{\pgfqpoint{5.650609in}{0.554085in}}%
\pgfpathlineto{\pgfqpoint{5.652260in}{0.660841in}}%
\pgfpathlineto{\pgfqpoint{5.653084in}{0.509425in}}%
\pgfpathlineto{\pgfqpoint{5.653908in}{0.540513in}}%
\pgfpathlineto{\pgfqpoint{5.654730in}{0.538439in}}%
\pgfpathlineto{\pgfqpoint{5.656372in}{0.576822in}}%
\pgfpathlineto{\pgfqpoint{5.658010in}{0.510988in}}%
\pgfpathlineto{\pgfqpoint{5.659644in}{0.575701in}}%
\pgfpathlineto{\pgfqpoint{5.660460in}{0.531544in}}%
\pgfpathlineto{\pgfqpoint{5.662088in}{0.717820in}}%
\pgfpathlineto{\pgfqpoint{5.663713in}{0.557035in}}%
\pgfpathlineto{\pgfqpoint{5.664525in}{0.580407in}}%
\pgfpathlineto{\pgfqpoint{5.665335in}{0.571165in}}%
\pgfpathlineto{\pgfqpoint{5.666144in}{0.495570in}}%
\pgfpathlineto{\pgfqpoint{5.666953in}{0.596866in}}%
\pgfpathlineto{\pgfqpoint{5.667760in}{0.590844in}}%
\pgfpathlineto{\pgfqpoint{5.669373in}{0.547147in}}%
\pgfpathlineto{\pgfqpoint{5.670981in}{0.600565in}}%
\pgfpathlineto{\pgfqpoint{5.672587in}{0.516568in}}%
\pgfpathlineto{\pgfqpoint{5.674188in}{0.599101in}}%
\pgfpathlineto{\pgfqpoint{5.675786in}{0.552959in}}%
\pgfpathlineto{\pgfqpoint{5.676584in}{0.565696in}}%
\pgfpathlineto{\pgfqpoint{5.677381in}{0.533831in}}%
\pgfpathlineto{\pgfqpoint{5.679766in}{0.625863in}}%
\pgfpathlineto{\pgfqpoint{5.680559in}{0.693265in}}%
\pgfpathlineto{\pgfqpoint{5.682143in}{0.532565in}}%
\pgfpathlineto{\pgfqpoint{5.682934in}{0.561790in}}%
\pgfpathlineto{\pgfqpoint{5.683724in}{0.556408in}}%
\pgfpathlineto{\pgfqpoint{5.686088in}{0.628494in}}%
\pgfpathlineto{\pgfqpoint{5.687660in}{0.519574in}}%
\pgfpathlineto{\pgfqpoint{5.688445in}{0.559192in}}%
\pgfpathlineto{\pgfqpoint{5.688445in}{0.559192in}}%
\pgfusepath{stroke}%
\end{pgfscope}%
\begin{pgfscope}%
\pgfpathrectangle{\pgfqpoint{0.517836in}{0.420092in}}{\pgfqpoint{5.425000in}{0.770000in}}%
\pgfusepath{clip}%
\pgfsetrectcap%
\pgfsetroundjoin%
\pgfsetlinewidth{0.501875pt}%
\definecolor{currentstroke}{rgb}{1.000000,0.498039,0.054902}%
\pgfsetstrokecolor{currentstroke}%
\pgfsetdash{}{0pt}%
\pgfpathmoveto{\pgfqpoint{0.507836in}{0.970875in}}%
\pgfpathlineto{\pgfqpoint{0.764427in}{0.970901in}}%
\pgfpathlineto{\pgfqpoint{1.264557in}{0.610488in}}%
\pgfpathlineto{\pgfqpoint{1.557113in}{0.513235in}}%
\pgfpathlineto{\pgfqpoint{1.764686in}{0.510417in}}%
\pgfpathlineto{\pgfqpoint{1.925691in}{0.516765in}}%
\pgfpathlineto{\pgfqpoint{2.057243in}{0.503986in}}%
\pgfpathlineto{\pgfqpoint{2.168468in}{0.503099in}}%
\pgfpathlineto{\pgfqpoint{2.264815in}{0.568588in}}%
\pgfpathlineto{\pgfqpoint{2.349800in}{0.619826in}}%
\pgfpathlineto{\pgfqpoint{2.425821in}{0.624061in}}%
\pgfpathlineto{\pgfqpoint{2.494590in}{0.607639in}}%
\pgfpathlineto{\pgfqpoint{2.615126in}{0.612206in}}%
\pgfpathlineto{\pgfqpoint{2.668597in}{0.563752in}}%
\pgfpathlineto{\pgfqpoint{2.718378in}{0.609281in}}%
\pgfpathlineto{\pgfqpoint{2.764944in}{0.588707in}}%
\pgfpathlineto{\pgfqpoint{2.808687in}{0.544963in}}%
\pgfpathlineto{\pgfqpoint{2.849929in}{0.554990in}}%
\pgfpathlineto{\pgfqpoint{2.888940in}{0.545831in}}%
\pgfpathlineto{\pgfqpoint{2.925950in}{0.550802in}}%
\pgfpathlineto{\pgfqpoint{2.961154in}{0.535674in}}%
\pgfpathlineto{\pgfqpoint{2.994720in}{0.535701in}}%
\pgfpathlineto{\pgfqpoint{3.026793in}{0.555843in}}%
\pgfpathlineto{\pgfqpoint{3.057501in}{0.540826in}}%
\pgfpathlineto{\pgfqpoint{3.086956in}{0.521321in}}%
\pgfpathlineto{\pgfqpoint{3.115255in}{0.528610in}}%
\pgfpathlineto{\pgfqpoint{3.142486in}{0.502931in}}%
\pgfpathlineto{\pgfqpoint{3.168726in}{0.525484in}}%
\pgfpathlineto{\pgfqpoint{3.218507in}{0.505514in}}%
\pgfpathlineto{\pgfqpoint{3.242166in}{0.517787in}}%
\pgfpathlineto{\pgfqpoint{3.265074in}{0.514136in}}%
\pgfpathlineto{\pgfqpoint{3.287276in}{0.545838in}}%
\pgfpathlineto{\pgfqpoint{3.308816in}{0.524516in}}%
\pgfpathlineto{\pgfqpoint{3.329732in}{0.506414in}}%
\pgfpathlineto{\pgfqpoint{3.350058in}{0.509365in}}%
\pgfpathlineto{\pgfqpoint{3.369827in}{0.503964in}}%
\pgfpathlineto{\pgfqpoint{3.389069in}{0.505100in}}%
\pgfpathlineto{\pgfqpoint{3.407812in}{0.502982in}}%
\pgfpathlineto{\pgfqpoint{3.426079in}{0.527468in}}%
\pgfpathlineto{\pgfqpoint{3.443896in}{0.533738in}}%
\pgfpathlineto{\pgfqpoint{3.461283in}{0.538167in}}%
\pgfpathlineto{\pgfqpoint{3.478261in}{0.540748in}}%
\pgfpathlineto{\pgfqpoint{3.494849in}{0.513384in}}%
\pgfpathlineto{\pgfqpoint{3.511064in}{0.503098in}}%
\pgfpathlineto{\pgfqpoint{3.526922in}{0.484433in}}%
\pgfpathlineto{\pgfqpoint{3.542440in}{0.508972in}}%
\pgfpathlineto{\pgfqpoint{3.557631in}{0.524935in}}%
\pgfpathlineto{\pgfqpoint{3.572508in}{0.513163in}}%
\pgfpathlineto{\pgfqpoint{3.587085in}{0.520189in}}%
\pgfpathlineto{\pgfqpoint{3.601373in}{0.498065in}}%
\pgfpathlineto{\pgfqpoint{3.615384in}{0.485147in}}%
\pgfpathlineto{\pgfqpoint{3.629128in}{0.549126in}}%
\pgfpathlineto{\pgfqpoint{3.642615in}{0.537956in}}%
\pgfpathlineto{\pgfqpoint{3.655855in}{0.486077in}}%
\pgfpathlineto{\pgfqpoint{3.668855in}{0.502621in}}%
\pgfpathlineto{\pgfqpoint{3.681626in}{0.513201in}}%
\pgfpathlineto{\pgfqpoint{3.694175in}{0.504807in}}%
\pgfpathlineto{\pgfqpoint{3.706509in}{0.525425in}}%
\pgfpathlineto{\pgfqpoint{3.718636in}{0.514959in}}%
\pgfpathlineto{\pgfqpoint{3.730563in}{0.556508in}}%
\pgfpathlineto{\pgfqpoint{3.742295in}{0.551193in}}%
\pgfpathlineto{\pgfqpoint{3.765203in}{0.519612in}}%
\pgfpathlineto{\pgfqpoint{3.776390in}{0.517749in}}%
\pgfpathlineto{\pgfqpoint{3.787406in}{0.500855in}}%
\pgfpathlineto{\pgfqpoint{3.798256in}{0.472795in}}%
\pgfpathlineto{\pgfqpoint{3.808946in}{0.508434in}}%
\pgfpathlineto{\pgfqpoint{3.819479in}{0.511075in}}%
\pgfpathlineto{\pgfqpoint{3.829861in}{0.496421in}}%
\pgfpathlineto{\pgfqpoint{3.840096in}{0.491489in}}%
\pgfpathlineto{\pgfqpoint{3.850187in}{0.491234in}}%
\pgfpathlineto{\pgfqpoint{3.860140in}{0.540894in}}%
\pgfpathlineto{\pgfqpoint{3.869957in}{0.550659in}}%
\pgfpathlineto{\pgfqpoint{3.879642in}{0.514810in}}%
\pgfpathlineto{\pgfqpoint{3.898631in}{0.499929in}}%
\pgfpathlineto{\pgfqpoint{3.907941in}{0.513898in}}%
\pgfpathlineto{\pgfqpoint{3.917133in}{0.541402in}}%
\pgfpathlineto{\pgfqpoint{3.935172in}{0.535948in}}%
\pgfpathlineto{\pgfqpoint{3.944025in}{0.575376in}}%
\pgfpathlineto{\pgfqpoint{3.952771in}{0.577396in}}%
\pgfpathlineto{\pgfqpoint{3.961412in}{0.535561in}}%
\pgfpathlineto{\pgfqpoint{3.969951in}{0.547574in}}%
\pgfpathlineto{\pgfqpoint{3.978390in}{0.517928in}}%
\pgfpathlineto{\pgfqpoint{3.986732in}{0.495772in}}%
\pgfpathlineto{\pgfqpoint{3.994978in}{0.500667in}}%
\pgfpathlineto{\pgfqpoint{4.003131in}{0.515506in}}%
\pgfpathlineto{\pgfqpoint{4.011193in}{0.518439in}}%
\pgfpathlineto{\pgfqpoint{4.019166in}{0.505579in}}%
\pgfpathlineto{\pgfqpoint{4.027052in}{0.513708in}}%
\pgfpathlineto{\pgfqpoint{4.034852in}{0.554360in}}%
\pgfpathlineto{\pgfqpoint{4.042569in}{0.504230in}}%
\pgfpathlineto{\pgfqpoint{4.050204in}{0.507710in}}%
\pgfpathlineto{\pgfqpoint{4.057760in}{0.523958in}}%
\pgfpathlineto{\pgfqpoint{4.065237in}{0.516864in}}%
\pgfpathlineto{\pgfqpoint{4.072637in}{0.491950in}}%
\pgfpathlineto{\pgfqpoint{4.079963in}{0.500420in}}%
\pgfpathlineto{\pgfqpoint{4.087214in}{0.520386in}}%
\pgfpathlineto{\pgfqpoint{4.094394in}{0.532739in}}%
\pgfpathlineto{\pgfqpoint{4.101503in}{0.510944in}}%
\pgfpathlineto{\pgfqpoint{4.108542in}{0.502867in}}%
\pgfpathlineto{\pgfqpoint{4.115513in}{0.507333in}}%
\pgfpathlineto{\pgfqpoint{4.122418in}{0.493882in}}%
\pgfpathlineto{\pgfqpoint{4.129257in}{0.494192in}}%
\pgfpathlineto{\pgfqpoint{4.136032in}{0.480844in}}%
\pgfpathlineto{\pgfqpoint{4.142744in}{0.503320in}}%
\pgfpathlineto{\pgfqpoint{4.149394in}{0.505350in}}%
\pgfpathlineto{\pgfqpoint{4.155984in}{0.509154in}}%
\pgfpathlineto{\pgfqpoint{4.162514in}{0.516628in}}%
\pgfpathlineto{\pgfqpoint{4.168985in}{0.517031in}}%
\pgfpathlineto{\pgfqpoint{4.175398in}{0.519166in}}%
\pgfpathlineto{\pgfqpoint{4.181756in}{0.500770in}}%
\pgfpathlineto{\pgfqpoint{4.188057in}{0.599237in}}%
\pgfpathlineto{\pgfqpoint{4.194304in}{0.501782in}}%
\pgfpathlineto{\pgfqpoint{4.200498in}{0.515042in}}%
\pgfpathlineto{\pgfqpoint{4.206639in}{0.523339in}}%
\pgfpathlineto{\pgfqpoint{4.212727in}{0.507476in}}%
\pgfpathlineto{\pgfqpoint{4.218765in}{0.495556in}}%
\pgfpathlineto{\pgfqpoint{4.224753in}{0.496799in}}%
\pgfpathlineto{\pgfqpoint{4.230692in}{0.514345in}}%
\pgfpathlineto{\pgfqpoint{4.236582in}{0.507930in}}%
\pgfpathlineto{\pgfqpoint{4.242424in}{0.519677in}}%
\pgfpathlineto{\pgfqpoint{4.248220in}{0.507170in}}%
\pgfpathlineto{\pgfqpoint{4.253969in}{0.499174in}}%
\pgfpathlineto{\pgfqpoint{4.259673in}{0.485743in}}%
\pgfpathlineto{\pgfqpoint{4.265332in}{0.500244in}}%
\pgfpathlineto{\pgfqpoint{4.270947in}{0.508907in}}%
\pgfpathlineto{\pgfqpoint{4.276519in}{0.504225in}}%
\pgfpathlineto{\pgfqpoint{4.282048in}{0.492062in}}%
\pgfpathlineto{\pgfqpoint{4.287535in}{0.511275in}}%
\pgfpathlineto{\pgfqpoint{4.292981in}{0.516805in}}%
\pgfpathlineto{\pgfqpoint{4.298385in}{0.505055in}}%
\pgfpathlineto{\pgfqpoint{4.303750in}{0.505516in}}%
\pgfpathlineto{\pgfqpoint{4.309075in}{0.497594in}}%
\pgfpathlineto{\pgfqpoint{4.314361in}{0.517055in}}%
\pgfpathlineto{\pgfqpoint{4.324818in}{0.497721in}}%
\pgfpathlineto{\pgfqpoint{4.329990in}{0.527053in}}%
\pgfpathlineto{\pgfqpoint{4.335126in}{0.532413in}}%
\pgfpathlineto{\pgfqpoint{4.340225in}{0.518922in}}%
\pgfpathlineto{\pgfqpoint{4.345289in}{0.484591in}}%
\pgfpathlineto{\pgfqpoint{4.350317in}{0.495094in}}%
\pgfpathlineto{\pgfqpoint{4.355310in}{0.478572in}}%
\pgfpathlineto{\pgfqpoint{4.360269in}{0.497683in}}%
\pgfpathlineto{\pgfqpoint{4.365194in}{0.510831in}}%
\pgfpathlineto{\pgfqpoint{4.370086in}{0.510188in}}%
\pgfpathlineto{\pgfqpoint{4.374945in}{0.506469in}}%
\pgfpathlineto{\pgfqpoint{4.379771in}{0.505630in}}%
\pgfpathlineto{\pgfqpoint{4.384565in}{0.514968in}}%
\pgfpathlineto{\pgfqpoint{4.389328in}{0.495652in}}%
\pgfpathlineto{\pgfqpoint{4.394059in}{0.497266in}}%
\pgfpathlineto{\pgfqpoint{4.398760in}{0.487682in}}%
\pgfpathlineto{\pgfqpoint{4.403430in}{0.493004in}}%
\pgfpathlineto{\pgfqpoint{4.408070in}{0.493029in}}%
\pgfpathlineto{\pgfqpoint{4.412681in}{0.500386in}}%
\pgfpathlineto{\pgfqpoint{4.417262in}{0.498055in}}%
\pgfpathlineto{\pgfqpoint{4.421814in}{0.508617in}}%
\pgfpathlineto{\pgfqpoint{4.430833in}{0.493090in}}%
\pgfpathlineto{\pgfqpoint{4.435301in}{0.488791in}}%
\pgfpathlineto{\pgfqpoint{4.439741in}{0.489212in}}%
\pgfpathlineto{\pgfqpoint{4.444154in}{0.517372in}}%
\pgfpathlineto{\pgfqpoint{4.448541in}{0.518599in}}%
\pgfpathlineto{\pgfqpoint{4.452900in}{0.513464in}}%
\pgfpathlineto{\pgfqpoint{4.457234in}{0.517626in}}%
\pgfpathlineto{\pgfqpoint{4.461542in}{0.502290in}}%
\pgfpathlineto{\pgfqpoint{4.470081in}{0.508364in}}%
\pgfpathlineto{\pgfqpoint{4.474312in}{0.516318in}}%
\pgfpathlineto{\pgfqpoint{4.478520in}{0.487844in}}%
\pgfpathlineto{\pgfqpoint{4.482702in}{0.528467in}}%
\pgfpathlineto{\pgfqpoint{4.486861in}{0.522094in}}%
\pgfpathlineto{\pgfqpoint{4.490996in}{0.501987in}}%
\pgfpathlineto{\pgfqpoint{4.495107in}{0.496444in}}%
\pgfpathlineto{\pgfqpoint{4.499195in}{0.530887in}}%
\pgfpathlineto{\pgfqpoint{4.503260in}{0.522620in}}%
\pgfpathlineto{\pgfqpoint{4.507303in}{0.502283in}}%
\pgfpathlineto{\pgfqpoint{4.511322in}{0.517370in}}%
\pgfpathlineto{\pgfqpoint{4.515320in}{0.493177in}}%
\pgfpathlineto{\pgfqpoint{4.519295in}{0.482799in}}%
\pgfpathlineto{\pgfqpoint{4.523249in}{0.506362in}}%
\pgfpathlineto{\pgfqpoint{4.527181in}{0.497337in}}%
\pgfpathlineto{\pgfqpoint{4.531092in}{0.507433in}}%
\pgfpathlineto{\pgfqpoint{4.538850in}{0.497163in}}%
\pgfpathlineto{\pgfqpoint{4.542698in}{0.499718in}}%
\pgfpathlineto{\pgfqpoint{4.546526in}{0.525117in}}%
\pgfpathlineto{\pgfqpoint{4.550334in}{0.527991in}}%
\pgfpathlineto{\pgfqpoint{4.554121in}{0.509198in}}%
\pgfpathlineto{\pgfqpoint{4.557889in}{0.496904in}}%
\pgfpathlineto{\pgfqpoint{4.561637in}{0.515627in}}%
\pgfpathlineto{\pgfqpoint{4.565366in}{0.513474in}}%
\pgfpathlineto{\pgfqpoint{4.569076in}{0.507578in}}%
\pgfpathlineto{\pgfqpoint{4.572767in}{0.487217in}}%
\pgfpathlineto{\pgfqpoint{4.576438in}{0.479562in}}%
\pgfpathlineto{\pgfqpoint{4.580092in}{0.496949in}}%
\pgfpathlineto{\pgfqpoint{4.583727in}{0.488787in}}%
\pgfpathlineto{\pgfqpoint{4.587344in}{0.504749in}}%
\pgfpathlineto{\pgfqpoint{4.590942in}{0.486837in}}%
\pgfpathlineto{\pgfqpoint{4.594523in}{0.494257in}}%
\pgfpathlineto{\pgfqpoint{4.598086in}{0.497426in}}%
\pgfpathlineto{\pgfqpoint{4.601632in}{0.493255in}}%
\pgfpathlineto{\pgfqpoint{4.605160in}{0.522395in}}%
\pgfpathlineto{\pgfqpoint{4.608671in}{0.515358in}}%
\pgfpathlineto{\pgfqpoint{4.612165in}{0.490635in}}%
\pgfpathlineto{\pgfqpoint{4.619103in}{0.482241in}}%
\pgfpathlineto{\pgfqpoint{4.629387in}{0.518318in}}%
\pgfpathlineto{\pgfqpoint{4.632782in}{0.492666in}}%
\pgfpathlineto{\pgfqpoint{4.636162in}{0.493034in}}%
\pgfpathlineto{\pgfqpoint{4.639525in}{0.516734in}}%
\pgfpathlineto{\pgfqpoint{4.649524in}{0.479755in}}%
\pgfpathlineto{\pgfqpoint{4.652826in}{0.498226in}}%
\pgfpathlineto{\pgfqpoint{4.656113in}{0.499527in}}%
\pgfpathlineto{\pgfqpoint{4.659385in}{0.516764in}}%
\pgfpathlineto{\pgfqpoint{4.662643in}{0.514008in}}%
\pgfpathlineto{\pgfqpoint{4.665886in}{0.488436in}}%
\pgfpathlineto{\pgfqpoint{4.669114in}{0.490046in}}%
\pgfpathlineto{\pgfqpoint{4.672328in}{0.513977in}}%
\pgfpathlineto{\pgfqpoint{4.675528in}{0.515850in}}%
\pgfpathlineto{\pgfqpoint{4.678713in}{0.521168in}}%
\pgfpathlineto{\pgfqpoint{4.681885in}{0.505688in}}%
\pgfpathlineto{\pgfqpoint{4.685043in}{0.523063in}}%
\pgfpathlineto{\pgfqpoint{4.688186in}{0.507865in}}%
\pgfpathlineto{\pgfqpoint{4.691317in}{0.482994in}}%
\pgfpathlineto{\pgfqpoint{4.694434in}{0.507143in}}%
\pgfpathlineto{\pgfqpoint{4.697537in}{0.486604in}}%
\pgfpathlineto{\pgfqpoint{4.700627in}{0.483672in}}%
\pgfpathlineto{\pgfqpoint{4.706768in}{0.505168in}}%
\pgfpathlineto{\pgfqpoint{4.709819in}{0.495797in}}%
\pgfpathlineto{\pgfqpoint{4.715882in}{0.504161in}}%
\pgfpathlineto{\pgfqpoint{4.721895in}{0.477901in}}%
\pgfpathlineto{\pgfqpoint{4.724883in}{0.502219in}}%
\pgfpathlineto{\pgfqpoint{4.727858in}{0.509360in}}%
\pgfpathlineto{\pgfqpoint{4.733772in}{0.487140in}}%
\pgfpathlineto{\pgfqpoint{4.736711in}{0.491675in}}%
\pgfpathlineto{\pgfqpoint{4.739638in}{0.493161in}}%
\pgfpathlineto{\pgfqpoint{4.742554in}{0.509402in}}%
\pgfpathlineto{\pgfqpoint{4.745457in}{0.499080in}}%
\pgfpathlineto{\pgfqpoint{4.748349in}{0.500555in}}%
\pgfpathlineto{\pgfqpoint{4.751230in}{0.506758in}}%
\pgfpathlineto{\pgfqpoint{4.754098in}{0.492277in}}%
\pgfpathlineto{\pgfqpoint{4.756956in}{0.490105in}}%
\pgfpathlineto{\pgfqpoint{4.762637in}{0.503629in}}%
\pgfpathlineto{\pgfqpoint{4.765461in}{0.483229in}}%
\pgfpathlineto{\pgfqpoint{4.771077in}{0.492374in}}%
\pgfpathlineto{\pgfqpoint{4.773868in}{0.498185in}}%
\pgfpathlineto{\pgfqpoint{4.776648in}{0.494495in}}%
\pgfpathlineto{\pgfqpoint{4.779418in}{0.498614in}}%
\pgfpathlineto{\pgfqpoint{4.782177in}{0.522846in}}%
\pgfpathlineto{\pgfqpoint{4.784926in}{0.525154in}}%
\pgfpathlineto{\pgfqpoint{4.787664in}{0.505307in}}%
\pgfpathlineto{\pgfqpoint{4.790392in}{0.502480in}}%
\pgfpathlineto{\pgfqpoint{4.793110in}{0.519340in}}%
\pgfpathlineto{\pgfqpoint{4.798515in}{0.477713in}}%
\pgfpathlineto{\pgfqpoint{4.801202in}{0.497282in}}%
\pgfpathlineto{\pgfqpoint{4.806547in}{0.509688in}}%
\pgfpathlineto{\pgfqpoint{4.814490in}{0.481092in}}%
\pgfpathlineto{\pgfqpoint{4.817119in}{0.478559in}}%
\pgfpathlineto{\pgfqpoint{4.819738in}{0.489160in}}%
\pgfpathlineto{\pgfqpoint{4.822347in}{0.490339in}}%
\pgfpathlineto{\pgfqpoint{4.824947in}{0.519869in}}%
\pgfpathlineto{\pgfqpoint{4.827538in}{0.487489in}}%
\pgfpathlineto{\pgfqpoint{4.830120in}{0.492295in}}%
\pgfpathlineto{\pgfqpoint{4.832692in}{0.513179in}}%
\pgfpathlineto{\pgfqpoint{4.835255in}{0.494679in}}%
\pgfpathlineto{\pgfqpoint{4.837809in}{0.497822in}}%
\pgfpathlineto{\pgfqpoint{4.840354in}{0.497171in}}%
\pgfpathlineto{\pgfqpoint{4.842891in}{0.507205in}}%
\pgfpathlineto{\pgfqpoint{4.845418in}{0.501859in}}%
\pgfpathlineto{\pgfqpoint{4.847936in}{0.484285in}}%
\pgfpathlineto{\pgfqpoint{4.850446in}{0.501186in}}%
\pgfpathlineto{\pgfqpoint{4.852947in}{0.506491in}}%
\pgfpathlineto{\pgfqpoint{4.855439in}{0.486962in}}%
\pgfpathlineto{\pgfqpoint{4.857923in}{0.486803in}}%
\pgfpathlineto{\pgfqpoint{4.860398in}{0.513275in}}%
\pgfpathlineto{\pgfqpoint{4.865323in}{0.487007in}}%
\pgfpathlineto{\pgfqpoint{4.867773in}{0.499402in}}%
\pgfpathlineto{\pgfqpoint{4.870215in}{0.493299in}}%
\pgfpathlineto{\pgfqpoint{4.872649in}{0.492776in}}%
\pgfpathlineto{\pgfqpoint{4.877491in}{0.518656in}}%
\pgfpathlineto{\pgfqpoint{4.882301in}{0.497699in}}%
\pgfpathlineto{\pgfqpoint{4.884695in}{0.495274in}}%
\pgfpathlineto{\pgfqpoint{4.887080in}{0.499241in}}%
\pgfpathlineto{\pgfqpoint{4.889457in}{0.526138in}}%
\pgfpathlineto{\pgfqpoint{4.891827in}{0.497062in}}%
\pgfpathlineto{\pgfqpoint{4.894189in}{0.489542in}}%
\pgfpathlineto{\pgfqpoint{4.896543in}{0.490333in}}%
\pgfpathlineto{\pgfqpoint{4.898889in}{0.497476in}}%
\pgfpathlineto{\pgfqpoint{4.901228in}{0.494712in}}%
\pgfpathlineto{\pgfqpoint{4.903559in}{0.517325in}}%
\pgfpathlineto{\pgfqpoint{4.905883in}{0.500351in}}%
\pgfpathlineto{\pgfqpoint{4.908199in}{0.494643in}}%
\pgfpathlineto{\pgfqpoint{4.910508in}{0.506047in}}%
\pgfpathlineto{\pgfqpoint{4.912810in}{0.485001in}}%
\pgfpathlineto{\pgfqpoint{4.917391in}{0.478713in}}%
\pgfpathlineto{\pgfqpoint{4.921943in}{0.498976in}}%
\pgfpathlineto{\pgfqpoint{4.924209in}{0.681589in}}%
\pgfpathlineto{\pgfqpoint{4.926467in}{0.493515in}}%
\pgfpathlineto{\pgfqpoint{4.928718in}{0.486443in}}%
\pgfpathlineto{\pgfqpoint{4.930963in}{0.510516in}}%
\pgfpathlineto{\pgfqpoint{4.933200in}{0.503022in}}%
\pgfpathlineto{\pgfqpoint{4.935430in}{0.484102in}}%
\pgfpathlineto{\pgfqpoint{4.937654in}{0.522315in}}%
\pgfpathlineto{\pgfqpoint{4.939871in}{0.495690in}}%
\pgfpathlineto{\pgfqpoint{4.942081in}{0.501524in}}%
\pgfpathlineto{\pgfqpoint{4.944284in}{0.486969in}}%
\pgfpathlineto{\pgfqpoint{4.946480in}{0.520376in}}%
\pgfpathlineto{\pgfqpoint{4.948670in}{0.489813in}}%
\pgfpathlineto{\pgfqpoint{4.950853in}{0.482525in}}%
\pgfpathlineto{\pgfqpoint{4.953030in}{0.479629in}}%
\pgfpathlineto{\pgfqpoint{4.955200in}{0.481038in}}%
\pgfpathlineto{\pgfqpoint{4.957363in}{0.492108in}}%
\pgfpathlineto{\pgfqpoint{4.959520in}{0.485635in}}%
\pgfpathlineto{\pgfqpoint{4.961671in}{0.491026in}}%
\pgfpathlineto{\pgfqpoint{4.963815in}{0.759713in}}%
\pgfpathlineto{\pgfqpoint{4.965953in}{0.487642in}}%
\pgfpathlineto{\pgfqpoint{4.968085in}{0.495268in}}%
\pgfpathlineto{\pgfqpoint{4.970210in}{0.484246in}}%
\pgfpathlineto{\pgfqpoint{4.972329in}{0.504697in}}%
\pgfpathlineto{\pgfqpoint{4.974442in}{0.502939in}}%
\pgfpathlineto{\pgfqpoint{4.978649in}{0.504498in}}%
\pgfpathlineto{\pgfqpoint{4.980743in}{0.497274in}}%
\pgfpathlineto{\pgfqpoint{4.982832in}{0.483055in}}%
\pgfpathlineto{\pgfqpoint{4.984914in}{0.489357in}}%
\pgfpathlineto{\pgfqpoint{4.986990in}{0.487609in}}%
\pgfpathlineto{\pgfqpoint{4.989061in}{0.496390in}}%
\pgfpathlineto{\pgfqpoint{4.991125in}{0.480753in}}%
\pgfpathlineto{\pgfqpoint{4.993184in}{0.492133in}}%
\pgfpathlineto{\pgfqpoint{4.995237in}{0.473586in}}%
\pgfpathlineto{\pgfqpoint{4.997284in}{0.491433in}}%
\pgfpathlineto{\pgfqpoint{4.999325in}{0.486469in}}%
\pgfpathlineto{\pgfqpoint{5.001360in}{0.491807in}}%
\pgfpathlineto{\pgfqpoint{5.003390in}{0.486353in}}%
\pgfpathlineto{\pgfqpoint{5.005414in}{0.492688in}}%
\pgfpathlineto{\pgfqpoint{5.007432in}{0.512531in}}%
\pgfpathlineto{\pgfqpoint{5.009445in}{0.490297in}}%
\pgfpathlineto{\pgfqpoint{5.011452in}{0.522212in}}%
\pgfpathlineto{\pgfqpoint{5.013453in}{0.508202in}}%
\pgfpathlineto{\pgfqpoint{5.015449in}{0.505843in}}%
\pgfpathlineto{\pgfqpoint{5.017439in}{0.494171in}}%
\pgfpathlineto{\pgfqpoint{5.019424in}{0.753484in}}%
\pgfpathlineto{\pgfqpoint{5.021404in}{0.731276in}}%
\pgfpathlineto{\pgfqpoint{5.023378in}{0.511625in}}%
\pgfpathlineto{\pgfqpoint{5.025347in}{0.512273in}}%
\pgfpathlineto{\pgfqpoint{5.027310in}{0.493443in}}%
\pgfpathlineto{\pgfqpoint{5.029268in}{0.493821in}}%
\pgfpathlineto{\pgfqpoint{5.031221in}{0.502850in}}%
\pgfpathlineto{\pgfqpoint{5.033168in}{0.521212in}}%
\pgfpathlineto{\pgfqpoint{5.035111in}{0.514490in}}%
\pgfpathlineto{\pgfqpoint{5.038979in}{0.483706in}}%
\pgfpathlineto{\pgfqpoint{5.040906in}{0.505310in}}%
\pgfpathlineto{\pgfqpoint{5.042828in}{0.512820in}}%
\pgfpathlineto{\pgfqpoint{5.046655in}{0.480418in}}%
\pgfpathlineto{\pgfqpoint{5.050463in}{0.479414in}}%
\pgfpathlineto{\pgfqpoint{5.052359in}{0.483392in}}%
\pgfpathlineto{\pgfqpoint{5.054251in}{0.508436in}}%
\pgfpathlineto{\pgfqpoint{5.058018in}{0.511118in}}%
\pgfpathlineto{\pgfqpoint{5.059895in}{0.530489in}}%
\pgfpathlineto{\pgfqpoint{5.061767in}{0.517854in}}%
\pgfpathlineto{\pgfqpoint{5.065495in}{0.479684in}}%
\pgfpathlineto{\pgfqpoint{5.071053in}{0.507334in}}%
\pgfpathlineto{\pgfqpoint{5.072896in}{0.492354in}}%
\pgfpathlineto{\pgfqpoint{5.074734in}{0.492493in}}%
\pgfpathlineto{\pgfqpoint{5.076568in}{0.591745in}}%
\pgfpathlineto{\pgfqpoint{5.078397in}{0.590999in}}%
\pgfpathlineto{\pgfqpoint{5.080221in}{0.491177in}}%
\pgfpathlineto{\pgfqpoint{5.083856in}{0.503218in}}%
\pgfpathlineto{\pgfqpoint{5.085667in}{0.496718in}}%
\pgfpathlineto{\pgfqpoint{5.087473in}{0.511175in}}%
\pgfpathlineto{\pgfqpoint{5.089274in}{0.489924in}}%
\pgfpathlineto{\pgfqpoint{5.091071in}{0.484244in}}%
\pgfpathlineto{\pgfqpoint{5.092864in}{0.493373in}}%
\pgfpathlineto{\pgfqpoint{5.094652in}{0.495750in}}%
\pgfpathlineto{\pgfqpoint{5.096436in}{0.487288in}}%
\pgfpathlineto{\pgfqpoint{5.098215in}{0.489053in}}%
\pgfpathlineto{\pgfqpoint{5.099990in}{0.520332in}}%
\pgfpathlineto{\pgfqpoint{5.103527in}{0.484893in}}%
\pgfpathlineto{\pgfqpoint{5.105289in}{0.500432in}}%
\pgfpathlineto{\pgfqpoint{5.107047in}{0.531721in}}%
\pgfpathlineto{\pgfqpoint{5.108800in}{0.503499in}}%
\pgfpathlineto{\pgfqpoint{5.110550in}{0.506483in}}%
\pgfpathlineto{\pgfqpoint{5.114035in}{0.496146in}}%
\pgfpathlineto{\pgfqpoint{5.115772in}{0.495919in}}%
\pgfpathlineto{\pgfqpoint{5.119232in}{0.506506in}}%
\pgfpathlineto{\pgfqpoint{5.120957in}{0.504119in}}%
\pgfpathlineto{\pgfqpoint{5.122677in}{0.497344in}}%
\pgfpathlineto{\pgfqpoint{5.124392in}{0.499115in}}%
\pgfpathlineto{\pgfqpoint{5.126104in}{0.495800in}}%
\pgfpathlineto{\pgfqpoint{5.127812in}{0.497400in}}%
\pgfpathlineto{\pgfqpoint{5.131215in}{0.482692in}}%
\pgfpathlineto{\pgfqpoint{5.132911in}{0.506907in}}%
\pgfpathlineto{\pgfqpoint{5.134603in}{0.491539in}}%
\pgfpathlineto{\pgfqpoint{5.136291in}{0.489041in}}%
\pgfpathlineto{\pgfqpoint{5.137975in}{0.541668in}}%
\pgfpathlineto{\pgfqpoint{5.139655in}{0.530036in}}%
\pgfpathlineto{\pgfqpoint{5.141331in}{0.498585in}}%
\pgfpathlineto{\pgfqpoint{5.143003in}{0.496460in}}%
\pgfpathlineto{\pgfqpoint{5.144671in}{0.484404in}}%
\pgfpathlineto{\pgfqpoint{5.147996in}{0.508581in}}%
\pgfpathlineto{\pgfqpoint{5.149653in}{0.505189in}}%
\pgfpathlineto{\pgfqpoint{5.151306in}{0.483468in}}%
\pgfpathlineto{\pgfqpoint{5.154601in}{0.511264in}}%
\pgfpathlineto{\pgfqpoint{5.156242in}{0.500897in}}%
\pgfpathlineto{\pgfqpoint{5.157880in}{0.475210in}}%
\pgfpathlineto{\pgfqpoint{5.159515in}{0.496753in}}%
\pgfpathlineto{\pgfqpoint{5.161145in}{0.499357in}}%
\pgfpathlineto{\pgfqpoint{5.162772in}{0.504764in}}%
\pgfpathlineto{\pgfqpoint{5.166015in}{0.497889in}}%
\pgfpathlineto{\pgfqpoint{5.167631in}{0.485145in}}%
\pgfpathlineto{\pgfqpoint{5.169243in}{0.522039in}}%
\pgfpathlineto{\pgfqpoint{5.170852in}{0.503540in}}%
\pgfpathlineto{\pgfqpoint{5.172457in}{0.520783in}}%
\pgfpathlineto{\pgfqpoint{5.174059in}{0.516063in}}%
\pgfpathlineto{\pgfqpoint{5.175657in}{0.497159in}}%
\pgfpathlineto{\pgfqpoint{5.177252in}{0.492994in}}%
\pgfpathlineto{\pgfqpoint{5.178843in}{0.493559in}}%
\pgfpathlineto{\pgfqpoint{5.183595in}{0.531780in}}%
\pgfpathlineto{\pgfqpoint{5.185172in}{0.501725in}}%
\pgfpathlineto{\pgfqpoint{5.186746in}{0.523350in}}%
\pgfpathlineto{\pgfqpoint{5.189883in}{0.485785in}}%
\pgfpathlineto{\pgfqpoint{5.191446in}{0.481103in}}%
\pgfpathlineto{\pgfqpoint{5.193006in}{0.480323in}}%
\pgfpathlineto{\pgfqpoint{5.194563in}{0.500303in}}%
\pgfpathlineto{\pgfqpoint{5.196116in}{0.492912in}}%
\pgfpathlineto{\pgfqpoint{5.197666in}{0.558253in}}%
\pgfpathlineto{\pgfqpoint{5.199213in}{0.551869in}}%
\pgfpathlineto{\pgfqpoint{5.200756in}{0.506617in}}%
\pgfpathlineto{\pgfqpoint{5.202296in}{0.506486in}}%
\pgfpathlineto{\pgfqpoint{5.203833in}{0.504046in}}%
\pgfpathlineto{\pgfqpoint{5.205367in}{0.477879in}}%
\pgfpathlineto{\pgfqpoint{5.206897in}{0.504339in}}%
\pgfpathlineto{\pgfqpoint{5.208424in}{0.509752in}}%
\pgfpathlineto{\pgfqpoint{5.209948in}{0.510369in}}%
\pgfpathlineto{\pgfqpoint{5.212986in}{0.486538in}}%
\pgfpathlineto{\pgfqpoint{5.216011in}{0.510565in}}%
\pgfpathlineto{\pgfqpoint{5.217519in}{0.503539in}}%
\pgfpathlineto{\pgfqpoint{5.219024in}{0.484065in}}%
\pgfpathlineto{\pgfqpoint{5.220526in}{0.500077in}}%
\pgfpathlineto{\pgfqpoint{5.223520in}{0.497483in}}%
\pgfpathlineto{\pgfqpoint{5.225012in}{0.504590in}}%
\pgfpathlineto{\pgfqpoint{5.226501in}{0.484887in}}%
\pgfpathlineto{\pgfqpoint{5.227987in}{0.511074in}}%
\pgfpathlineto{\pgfqpoint{5.229470in}{0.489202in}}%
\pgfpathlineto{\pgfqpoint{5.230950in}{0.498600in}}%
\pgfpathlineto{\pgfqpoint{5.232427in}{0.499011in}}%
\pgfpathlineto{\pgfqpoint{5.233901in}{0.485642in}}%
\pgfpathlineto{\pgfqpoint{5.236841in}{0.510778in}}%
\pgfpathlineto{\pgfqpoint{5.238306in}{0.489780in}}%
\pgfpathlineto{\pgfqpoint{5.239768in}{0.489193in}}%
\pgfpathlineto{\pgfqpoint{5.241227in}{0.483437in}}%
\pgfpathlineto{\pgfqpoint{5.242683in}{0.490251in}}%
\pgfpathlineto{\pgfqpoint{5.244136in}{0.508191in}}%
\pgfpathlineto{\pgfqpoint{5.247034in}{0.492827in}}%
\pgfpathlineto{\pgfqpoint{5.248478in}{0.496152in}}%
\pgfpathlineto{\pgfqpoint{5.251359in}{0.517873in}}%
\pgfpathlineto{\pgfqpoint{5.252795in}{0.499119in}}%
\pgfpathlineto{\pgfqpoint{5.254228in}{0.510253in}}%
\pgfpathlineto{\pgfqpoint{5.255658in}{0.487477in}}%
\pgfpathlineto{\pgfqpoint{5.257085in}{0.487613in}}%
\pgfpathlineto{\pgfqpoint{5.258510in}{0.519408in}}%
\pgfpathlineto{\pgfqpoint{5.259932in}{0.485567in}}%
\pgfpathlineto{\pgfqpoint{5.261351in}{0.487373in}}%
\pgfpathlineto{\pgfqpoint{5.262767in}{0.486947in}}%
\pgfpathlineto{\pgfqpoint{5.264180in}{0.475477in}}%
\pgfpathlineto{\pgfqpoint{5.265591in}{0.498741in}}%
\pgfpathlineto{\pgfqpoint{5.266999in}{0.541884in}}%
\pgfpathlineto{\pgfqpoint{5.268404in}{0.490703in}}%
\pgfpathlineto{\pgfqpoint{5.269806in}{0.505094in}}%
\pgfpathlineto{\pgfqpoint{5.271206in}{0.497406in}}%
\pgfpathlineto{\pgfqpoint{5.272603in}{0.523458in}}%
\pgfpathlineto{\pgfqpoint{5.273997in}{0.511494in}}%
\pgfpathlineto{\pgfqpoint{5.275389in}{0.510724in}}%
\pgfpathlineto{\pgfqpoint{5.276778in}{0.505915in}}%
\pgfpathlineto{\pgfqpoint{5.278164in}{0.495099in}}%
\pgfpathlineto{\pgfqpoint{5.279547in}{0.503568in}}%
\pgfpathlineto{\pgfqpoint{5.280928in}{0.484278in}}%
\pgfpathlineto{\pgfqpoint{5.282307in}{0.491942in}}%
\pgfpathlineto{\pgfqpoint{5.283682in}{0.517572in}}%
\pgfpathlineto{\pgfqpoint{5.285055in}{0.487487in}}%
\pgfpathlineto{\pgfqpoint{5.287794in}{0.502820in}}%
\pgfpathlineto{\pgfqpoint{5.289159in}{0.495642in}}%
\pgfpathlineto{\pgfqpoint{5.290521in}{0.496556in}}%
\pgfpathlineto{\pgfqpoint{5.294594in}{0.475890in}}%
\pgfpathlineto{\pgfqpoint{5.295947in}{0.503522in}}%
\pgfpathlineto{\pgfqpoint{5.297296in}{0.502411in}}%
\pgfpathlineto{\pgfqpoint{5.299989in}{0.524428in}}%
\pgfpathlineto{\pgfqpoint{5.301331in}{0.489047in}}%
\pgfpathlineto{\pgfqpoint{5.302671in}{0.493650in}}%
\pgfpathlineto{\pgfqpoint{5.304008in}{0.481483in}}%
\pgfpathlineto{\pgfqpoint{5.306676in}{0.503306in}}%
\pgfpathlineto{\pgfqpoint{5.308006in}{0.498017in}}%
\pgfpathlineto{\pgfqpoint{5.309333in}{0.500073in}}%
\pgfpathlineto{\pgfqpoint{5.310659in}{0.499717in}}%
\pgfpathlineto{\pgfqpoint{5.311981in}{0.501126in}}%
\pgfpathlineto{\pgfqpoint{5.313302in}{0.493155in}}%
\pgfpathlineto{\pgfqpoint{5.315935in}{0.502068in}}%
\pgfpathlineto{\pgfqpoint{5.318559in}{0.479210in}}%
\pgfpathlineto{\pgfqpoint{5.319867in}{0.501593in}}%
\pgfpathlineto{\pgfqpoint{5.321173in}{0.503309in}}%
\pgfpathlineto{\pgfqpoint{5.322477in}{0.496825in}}%
\pgfpathlineto{\pgfqpoint{5.323778in}{0.500809in}}%
\pgfpathlineto{\pgfqpoint{5.325077in}{0.497697in}}%
\pgfpathlineto{\pgfqpoint{5.326373in}{0.489138in}}%
\pgfpathlineto{\pgfqpoint{5.327667in}{0.502729in}}%
\pgfpathlineto{\pgfqpoint{5.330249in}{0.485884in}}%
\pgfpathlineto{\pgfqpoint{5.331536in}{0.506222in}}%
\pgfpathlineto{\pgfqpoint{5.332821in}{0.502215in}}%
\pgfpathlineto{\pgfqpoint{5.334104in}{0.515316in}}%
\pgfpathlineto{\pgfqpoint{5.335384in}{0.512830in}}%
\pgfpathlineto{\pgfqpoint{5.336663in}{0.490688in}}%
\pgfpathlineto{\pgfqpoint{5.337939in}{0.492129in}}%
\pgfpathlineto{\pgfqpoint{5.339212in}{0.481777in}}%
\pgfpathlineto{\pgfqpoint{5.343020in}{0.509376in}}%
\pgfpathlineto{\pgfqpoint{5.344285in}{0.494648in}}%
\pgfpathlineto{\pgfqpoint{5.345547in}{0.494217in}}%
\pgfpathlineto{\pgfqpoint{5.346807in}{0.501251in}}%
\pgfpathlineto{\pgfqpoint{5.348065in}{0.482620in}}%
\pgfpathlineto{\pgfqpoint{5.349321in}{0.480525in}}%
\pgfpathlineto{\pgfqpoint{5.350575in}{0.506581in}}%
\pgfpathlineto{\pgfqpoint{5.351827in}{0.505208in}}%
\pgfpathlineto{\pgfqpoint{5.353076in}{0.494488in}}%
\pgfpathlineto{\pgfqpoint{5.354323in}{0.498169in}}%
\pgfpathlineto{\pgfqpoint{5.355569in}{0.528591in}}%
\pgfpathlineto{\pgfqpoint{5.356811in}{0.522454in}}%
\pgfpathlineto{\pgfqpoint{5.359291in}{0.492828in}}%
\pgfpathlineto{\pgfqpoint{5.360528in}{0.487101in}}%
\pgfpathlineto{\pgfqpoint{5.361762in}{0.522068in}}%
\pgfpathlineto{\pgfqpoint{5.364225in}{0.497401in}}%
\pgfpathlineto{\pgfqpoint{5.365453in}{0.485788in}}%
\pgfpathlineto{\pgfqpoint{5.367903in}{0.507942in}}%
\pgfpathlineto{\pgfqpoint{5.369125in}{0.496019in}}%
\pgfpathlineto{\pgfqpoint{5.370344in}{0.496176in}}%
\pgfpathlineto{\pgfqpoint{5.371562in}{0.511363in}}%
\pgfpathlineto{\pgfqpoint{5.373992in}{0.493751in}}%
\pgfpathlineto{\pgfqpoint{5.375203in}{0.488699in}}%
\pgfpathlineto{\pgfqpoint{5.376413in}{0.524373in}}%
\pgfpathlineto{\pgfqpoint{5.377621in}{0.496258in}}%
\pgfpathlineto{\pgfqpoint{5.378826in}{0.504610in}}%
\pgfpathlineto{\pgfqpoint{5.380030in}{0.487375in}}%
\pgfpathlineto{\pgfqpoint{5.381231in}{0.487860in}}%
\pgfpathlineto{\pgfqpoint{5.382431in}{0.492466in}}%
\pgfpathlineto{\pgfqpoint{5.384824in}{0.508188in}}%
\pgfpathlineto{\pgfqpoint{5.387209in}{0.494378in}}%
\pgfpathlineto{\pgfqpoint{5.388399in}{0.512372in}}%
\pgfpathlineto{\pgfqpoint{5.390772in}{0.474076in}}%
\pgfpathlineto{\pgfqpoint{5.391956in}{0.507652in}}%
\pgfpathlineto{\pgfqpoint{5.393138in}{0.503490in}}%
\pgfpathlineto{\pgfqpoint{5.394318in}{0.515938in}}%
\pgfpathlineto{\pgfqpoint{5.396672in}{0.479887in}}%
\pgfpathlineto{\pgfqpoint{5.397846in}{0.483312in}}%
\pgfpathlineto{\pgfqpoint{5.400189in}{0.476690in}}%
\pgfpathlineto{\pgfqpoint{5.402524in}{0.505430in}}%
\pgfpathlineto{\pgfqpoint{5.403689in}{0.502879in}}%
\pgfpathlineto{\pgfqpoint{5.404851in}{0.483125in}}%
\pgfpathlineto{\pgfqpoint{5.406012in}{0.497801in}}%
\pgfpathlineto{\pgfqpoint{5.407171in}{0.483829in}}%
\pgfpathlineto{\pgfqpoint{5.408329in}{0.483398in}}%
\pgfpathlineto{\pgfqpoint{5.409484in}{0.590013in}}%
\pgfpathlineto{\pgfqpoint{5.411789in}{0.503670in}}%
\pgfpathlineto{\pgfqpoint{5.414087in}{0.474166in}}%
\pgfpathlineto{\pgfqpoint{5.416378in}{0.498163in}}%
\pgfpathlineto{\pgfqpoint{5.417520in}{0.655995in}}%
\pgfpathlineto{\pgfqpoint{5.418661in}{0.490006in}}%
\pgfpathlineto{\pgfqpoint{5.419800in}{0.501832in}}%
\pgfpathlineto{\pgfqpoint{5.420937in}{0.501427in}}%
\pgfpathlineto{\pgfqpoint{5.422073in}{0.492702in}}%
\pgfpathlineto{\pgfqpoint{5.423206in}{0.498027in}}%
\pgfpathlineto{\pgfqpoint{5.424338in}{0.487359in}}%
\pgfpathlineto{\pgfqpoint{5.427723in}{0.518349in}}%
\pgfpathlineto{\pgfqpoint{5.429971in}{0.499333in}}%
\pgfpathlineto{\pgfqpoint{5.431092in}{0.516454in}}%
\pgfpathlineto{\pgfqpoint{5.432211in}{0.500055in}}%
\pgfpathlineto{\pgfqpoint{5.433329in}{0.504522in}}%
\pgfpathlineto{\pgfqpoint{5.434445in}{0.490287in}}%
\pgfpathlineto{\pgfqpoint{5.435560in}{0.531953in}}%
\pgfpathlineto{\pgfqpoint{5.436672in}{0.491404in}}%
\pgfpathlineto{\pgfqpoint{5.437783in}{0.498148in}}%
\pgfpathlineto{\pgfqpoint{5.438892in}{0.514981in}}%
\pgfpathlineto{\pgfqpoint{5.440000in}{0.507463in}}%
\pgfpathlineto{\pgfqpoint{5.441106in}{0.517174in}}%
\pgfpathlineto{\pgfqpoint{5.443312in}{0.478925in}}%
\pgfpathlineto{\pgfqpoint{5.446609in}{0.494150in}}%
\pgfpathlineto{\pgfqpoint{5.447705in}{0.483205in}}%
\pgfpathlineto{\pgfqpoint{5.448799in}{0.513430in}}%
\pgfpathlineto{\pgfqpoint{5.449892in}{0.489467in}}%
\pgfpathlineto{\pgfqpoint{5.452071in}{0.507396in}}%
\pgfpathlineto{\pgfqpoint{5.454245in}{0.489053in}}%
\pgfpathlineto{\pgfqpoint{5.456412in}{0.497160in}}%
\pgfpathlineto{\pgfqpoint{5.457492in}{0.499218in}}%
\pgfpathlineto{\pgfqpoint{5.458572in}{0.482706in}}%
\pgfpathlineto{\pgfqpoint{5.459650in}{0.486922in}}%
\pgfpathlineto{\pgfqpoint{5.460726in}{0.465585in}}%
\pgfpathlineto{\pgfqpoint{5.461800in}{0.488100in}}%
\pgfpathlineto{\pgfqpoint{5.462873in}{0.486022in}}%
\pgfpathlineto{\pgfqpoint{5.465014in}{0.499247in}}%
\pgfpathlineto{\pgfqpoint{5.466082in}{0.491396in}}%
\pgfpathlineto{\pgfqpoint{5.467149in}{0.500457in}}%
\pgfpathlineto{\pgfqpoint{5.468214in}{0.523104in}}%
\pgfpathlineto{\pgfqpoint{5.470339in}{0.494616in}}%
\pgfpathlineto{\pgfqpoint{5.471399in}{0.497900in}}%
\pgfpathlineto{\pgfqpoint{5.473515in}{0.486764in}}%
\pgfpathlineto{\pgfqpoint{5.474571in}{0.489565in}}%
\pgfpathlineto{\pgfqpoint{5.475625in}{0.505011in}}%
\pgfpathlineto{\pgfqpoint{5.476678in}{0.502648in}}%
\pgfpathlineto{\pgfqpoint{5.477729in}{0.557348in}}%
\pgfpathlineto{\pgfqpoint{5.480873in}{0.502994in}}%
\pgfpathlineto{\pgfqpoint{5.481918in}{0.490906in}}%
\pgfpathlineto{\pgfqpoint{5.482961in}{0.492359in}}%
\pgfpathlineto{\pgfqpoint{5.484003in}{0.482485in}}%
\pgfpathlineto{\pgfqpoint{5.485043in}{0.484653in}}%
\pgfpathlineto{\pgfqpoint{5.487120in}{0.506039in}}%
\pgfpathlineto{\pgfqpoint{5.488156in}{0.486448in}}%
\pgfpathlineto{\pgfqpoint{5.489190in}{0.489738in}}%
\pgfpathlineto{\pgfqpoint{5.491255in}{0.485127in}}%
\pgfpathlineto{\pgfqpoint{5.493313in}{0.500222in}}%
\pgfpathlineto{\pgfqpoint{5.494340in}{0.495706in}}%
\pgfpathlineto{\pgfqpoint{5.497413in}{0.506409in}}%
\pgfpathlineto{\pgfqpoint{5.498434in}{0.492104in}}%
\pgfpathlineto{\pgfqpoint{5.499454in}{0.517571in}}%
\pgfpathlineto{\pgfqpoint{5.501489in}{0.499945in}}%
\pgfpathlineto{\pgfqpoint{5.503519in}{0.507913in}}%
\pgfpathlineto{\pgfqpoint{5.505543in}{0.480709in}}%
\pgfpathlineto{\pgfqpoint{5.506553in}{0.479948in}}%
\pgfpathlineto{\pgfqpoint{5.508568in}{0.488951in}}%
\pgfpathlineto{\pgfqpoint{5.510578in}{0.507418in}}%
\pgfpathlineto{\pgfqpoint{5.512582in}{0.492329in}}%
\pgfpathlineto{\pgfqpoint{5.514581in}{0.572095in}}%
\pgfpathlineto{\pgfqpoint{5.516574in}{0.480681in}}%
\pgfpathlineto{\pgfqpoint{5.517569in}{0.480466in}}%
\pgfpathlineto{\pgfqpoint{5.518562in}{0.679147in}}%
\pgfpathlineto{\pgfqpoint{5.522521in}{0.495697in}}%
\pgfpathlineto{\pgfqpoint{5.523507in}{0.487920in}}%
\pgfpathlineto{\pgfqpoint{5.524492in}{0.494183in}}%
\pgfpathlineto{\pgfqpoint{5.524492in}{0.494183in}}%
\pgfusepath{stroke}%
\end{pgfscope}%
\begin{pgfscope}%
\pgfpathrectangle{\pgfqpoint{0.517836in}{0.420092in}}{\pgfqpoint{5.425000in}{0.770000in}}%
\pgfusepath{clip}%
\pgfsetrectcap%
\pgfsetroundjoin%
\pgfsetlinewidth{0.501875pt}%
\definecolor{currentstroke}{rgb}{0.172549,0.627451,0.172549}%
\pgfsetstrokecolor{currentstroke}%
\pgfsetdash{}{0pt}%
\pgfpathmoveto{\pgfqpoint{0.507836in}{0.974480in}}%
\pgfpathlineto{\pgfqpoint{0.764427in}{0.974531in}}%
\pgfpathlineto{\pgfqpoint{1.264557in}{0.611157in}}%
\pgfpathlineto{\pgfqpoint{1.557113in}{0.590078in}}%
\pgfpathlineto{\pgfqpoint{1.764686in}{0.684885in}}%
\pgfpathlineto{\pgfqpoint{1.925691in}{0.891600in}}%
\pgfpathlineto{\pgfqpoint{2.057243in}{1.091011in}}%
\pgfpathlineto{\pgfqpoint{2.138075in}{1.200092in}}%
\pgfpathmoveto{\pgfqpoint{2.287765in}{1.200092in}}%
\pgfpathlineto{\pgfqpoint{2.349800in}{1.057962in}}%
\pgfpathlineto{\pgfqpoint{2.425821in}{0.895559in}}%
\pgfpathlineto{\pgfqpoint{2.494590in}{0.777263in}}%
\pgfpathlineto{\pgfqpoint{2.557372in}{0.735939in}}%
\pgfpathlineto{\pgfqpoint{2.615126in}{0.761023in}}%
\pgfpathlineto{\pgfqpoint{2.668597in}{0.799661in}}%
\pgfpathlineto{\pgfqpoint{2.718378in}{0.868047in}}%
\pgfpathlineto{\pgfqpoint{2.764944in}{0.898057in}}%
\pgfpathlineto{\pgfqpoint{2.808687in}{0.928913in}}%
\pgfpathlineto{\pgfqpoint{2.849929in}{0.918784in}}%
\pgfpathlineto{\pgfqpoint{2.888940in}{0.893599in}}%
\pgfpathlineto{\pgfqpoint{2.961154in}{0.899147in}}%
\pgfpathlineto{\pgfqpoint{2.994720in}{0.854863in}}%
\pgfpathlineto{\pgfqpoint{3.026793in}{0.788267in}}%
\pgfpathlineto{\pgfqpoint{3.057501in}{0.793720in}}%
\pgfpathlineto{\pgfqpoint{3.086956in}{0.807103in}}%
\pgfpathlineto{\pgfqpoint{3.115255in}{0.867212in}}%
\pgfpathlineto{\pgfqpoint{3.142486in}{0.883807in}}%
\pgfpathlineto{\pgfqpoint{3.168726in}{0.855690in}}%
\pgfpathlineto{\pgfqpoint{3.194046in}{0.836229in}}%
\pgfpathlineto{\pgfqpoint{3.218507in}{0.791309in}}%
\pgfpathlineto{\pgfqpoint{3.242166in}{0.786372in}}%
\pgfpathlineto{\pgfqpoint{3.265074in}{0.777778in}}%
\pgfpathlineto{\pgfqpoint{3.287276in}{0.776921in}}%
\pgfpathlineto{\pgfqpoint{3.308816in}{0.795108in}}%
\pgfpathlineto{\pgfqpoint{3.329732in}{0.805219in}}%
\pgfpathlineto{\pgfqpoint{3.350058in}{0.813213in}}%
\pgfpathlineto{\pgfqpoint{3.369827in}{0.785276in}}%
\pgfpathlineto{\pgfqpoint{3.407812in}{0.788496in}}%
\pgfpathlineto{\pgfqpoint{3.426079in}{0.772297in}}%
\pgfpathlineto{\pgfqpoint{3.443896in}{0.790053in}}%
\pgfpathlineto{\pgfqpoint{3.461283in}{0.778322in}}%
\pgfpathlineto{\pgfqpoint{3.478261in}{0.790973in}}%
\pgfpathlineto{\pgfqpoint{3.494849in}{0.797962in}}%
\pgfpathlineto{\pgfqpoint{3.511064in}{0.753007in}}%
\pgfpathlineto{\pgfqpoint{3.526922in}{0.775238in}}%
\pgfpathlineto{\pgfqpoint{3.542440in}{0.751650in}}%
\pgfpathlineto{\pgfqpoint{3.557631in}{0.731997in}}%
\pgfpathlineto{\pgfqpoint{3.572508in}{0.747371in}}%
\pgfpathlineto{\pgfqpoint{3.587085in}{0.729985in}}%
\pgfpathlineto{\pgfqpoint{3.601373in}{0.730171in}}%
\pgfpathlineto{\pgfqpoint{3.615384in}{0.720783in}}%
\pgfpathlineto{\pgfqpoint{3.629128in}{0.743130in}}%
\pgfpathlineto{\pgfqpoint{3.642615in}{0.754204in}}%
\pgfpathlineto{\pgfqpoint{3.655855in}{0.787093in}}%
\pgfpathlineto{\pgfqpoint{3.668855in}{0.793109in}}%
\pgfpathlineto{\pgfqpoint{3.681626in}{0.797439in}}%
\pgfpathlineto{\pgfqpoint{3.694175in}{0.809596in}}%
\pgfpathlineto{\pgfqpoint{3.706509in}{0.840875in}}%
\pgfpathlineto{\pgfqpoint{3.718636in}{0.837220in}}%
\pgfpathlineto{\pgfqpoint{3.730563in}{0.823027in}}%
\pgfpathlineto{\pgfqpoint{3.742295in}{0.820055in}}%
\pgfpathlineto{\pgfqpoint{3.753840in}{0.803113in}}%
\pgfpathlineto{\pgfqpoint{3.765203in}{0.783181in}}%
\pgfpathlineto{\pgfqpoint{3.776390in}{0.769179in}}%
\pgfpathlineto{\pgfqpoint{3.787406in}{0.789309in}}%
\pgfpathlineto{\pgfqpoint{3.798256in}{0.794433in}}%
\pgfpathlineto{\pgfqpoint{3.808946in}{0.790871in}}%
\pgfpathlineto{\pgfqpoint{3.819479in}{0.820562in}}%
\pgfpathlineto{\pgfqpoint{3.829861in}{0.846417in}}%
\pgfpathlineto{\pgfqpoint{3.840096in}{0.841911in}}%
\pgfpathlineto{\pgfqpoint{3.850187in}{0.823351in}}%
\pgfpathlineto{\pgfqpoint{3.860140in}{0.840512in}}%
\pgfpathlineto{\pgfqpoint{3.869957in}{0.852246in}}%
\pgfpathlineto{\pgfqpoint{3.879642in}{0.816704in}}%
\pgfpathlineto{\pgfqpoint{3.889199in}{0.814196in}}%
\pgfpathlineto{\pgfqpoint{3.898631in}{0.827677in}}%
\pgfpathlineto{\pgfqpoint{3.907941in}{0.847003in}}%
\pgfpathlineto{\pgfqpoint{3.917133in}{0.826430in}}%
\pgfpathlineto{\pgfqpoint{3.926209in}{0.820594in}}%
\pgfpathlineto{\pgfqpoint{3.935172in}{0.848100in}}%
\pgfpathlineto{\pgfqpoint{3.944025in}{0.837880in}}%
\pgfpathlineto{\pgfqpoint{3.952771in}{0.823524in}}%
\pgfpathlineto{\pgfqpoint{3.961412in}{0.811769in}}%
\pgfpathlineto{\pgfqpoint{3.969951in}{0.821771in}}%
\pgfpathlineto{\pgfqpoint{3.978390in}{0.811980in}}%
\pgfpathlineto{\pgfqpoint{3.986732in}{0.811473in}}%
\pgfpathlineto{\pgfqpoint{3.994978in}{0.817757in}}%
\pgfpathlineto{\pgfqpoint{4.003131in}{0.834580in}}%
\pgfpathlineto{\pgfqpoint{4.019166in}{0.843193in}}%
\pgfpathlineto{\pgfqpoint{4.027052in}{0.865218in}}%
\pgfpathlineto{\pgfqpoint{4.034852in}{0.825497in}}%
\pgfpathlineto{\pgfqpoint{4.042569in}{0.865180in}}%
\pgfpathlineto{\pgfqpoint{4.050204in}{0.852231in}}%
\pgfpathlineto{\pgfqpoint{4.057760in}{0.810200in}}%
\pgfpathlineto{\pgfqpoint{4.065237in}{0.824280in}}%
\pgfpathlineto{\pgfqpoint{4.072637in}{0.828816in}}%
\pgfpathlineto{\pgfqpoint{4.087214in}{0.805890in}}%
\pgfpathlineto{\pgfqpoint{4.094394in}{0.845309in}}%
\pgfpathlineto{\pgfqpoint{4.101503in}{0.847713in}}%
\pgfpathlineto{\pgfqpoint{4.108542in}{0.831987in}}%
\pgfpathlineto{\pgfqpoint{4.115513in}{0.854466in}}%
\pgfpathlineto{\pgfqpoint{4.122418in}{0.841450in}}%
\pgfpathlineto{\pgfqpoint{4.129257in}{0.822151in}}%
\pgfpathlineto{\pgfqpoint{4.136032in}{0.829786in}}%
\pgfpathlineto{\pgfqpoint{4.142744in}{0.800938in}}%
\pgfpathlineto{\pgfqpoint{4.149394in}{0.777780in}}%
\pgfpathlineto{\pgfqpoint{4.155984in}{0.787025in}}%
\pgfpathlineto{\pgfqpoint{4.162514in}{0.769332in}}%
\pgfpathlineto{\pgfqpoint{4.168985in}{0.787140in}}%
\pgfpathlineto{\pgfqpoint{4.175398in}{0.785664in}}%
\pgfpathlineto{\pgfqpoint{4.181756in}{0.776229in}}%
\pgfpathlineto{\pgfqpoint{4.188057in}{0.814433in}}%
\pgfpathlineto{\pgfqpoint{4.194304in}{0.741865in}}%
\pgfpathlineto{\pgfqpoint{4.200498in}{0.863555in}}%
\pgfpathlineto{\pgfqpoint{4.206639in}{0.719875in}}%
\pgfpathlineto{\pgfqpoint{4.212727in}{0.758504in}}%
\pgfpathlineto{\pgfqpoint{4.218765in}{0.852319in}}%
\pgfpathlineto{\pgfqpoint{4.230692in}{0.720315in}}%
\pgfpathlineto{\pgfqpoint{4.242424in}{0.816718in}}%
\pgfpathlineto{\pgfqpoint{4.248220in}{0.785066in}}%
\pgfpathlineto{\pgfqpoint{4.253969in}{0.745709in}}%
\pgfpathlineto{\pgfqpoint{4.259673in}{0.770866in}}%
\pgfpathlineto{\pgfqpoint{4.265332in}{0.830279in}}%
\pgfpathlineto{\pgfqpoint{4.276519in}{0.729590in}}%
\pgfpathlineto{\pgfqpoint{4.282048in}{0.750609in}}%
\pgfpathlineto{\pgfqpoint{4.287535in}{0.784739in}}%
\pgfpathlineto{\pgfqpoint{4.292981in}{0.763754in}}%
\pgfpathlineto{\pgfqpoint{4.298385in}{0.726770in}}%
\pgfpathlineto{\pgfqpoint{4.309075in}{0.762522in}}%
\pgfpathlineto{\pgfqpoint{4.314361in}{0.770358in}}%
\pgfpathlineto{\pgfqpoint{4.319608in}{0.741438in}}%
\pgfpathlineto{\pgfqpoint{4.324818in}{0.738612in}}%
\pgfpathlineto{\pgfqpoint{4.329990in}{0.791335in}}%
\pgfpathlineto{\pgfqpoint{4.335126in}{0.796739in}}%
\pgfpathlineto{\pgfqpoint{4.340225in}{0.794724in}}%
\pgfpathlineto{\pgfqpoint{4.345289in}{0.784417in}}%
\pgfpathlineto{\pgfqpoint{4.350317in}{0.795405in}}%
\pgfpathlineto{\pgfqpoint{4.355310in}{0.784901in}}%
\pgfpathlineto{\pgfqpoint{4.360269in}{0.754898in}}%
\pgfpathlineto{\pgfqpoint{4.365194in}{0.770997in}}%
\pgfpathlineto{\pgfqpoint{4.370086in}{0.759087in}}%
\pgfpathlineto{\pgfqpoint{4.379771in}{0.750364in}}%
\pgfpathlineto{\pgfqpoint{4.384565in}{0.772766in}}%
\pgfpathlineto{\pgfqpoint{4.389328in}{0.788155in}}%
\pgfpathlineto{\pgfqpoint{4.394059in}{0.780099in}}%
\pgfpathlineto{\pgfqpoint{4.398760in}{0.806578in}}%
\pgfpathlineto{\pgfqpoint{4.403430in}{0.808361in}}%
\pgfpathlineto{\pgfqpoint{4.412681in}{0.823886in}}%
\pgfpathlineto{\pgfqpoint{4.417262in}{0.815032in}}%
\pgfpathlineto{\pgfqpoint{4.421814in}{0.828436in}}%
\pgfpathlineto{\pgfqpoint{4.426338in}{0.807881in}}%
\pgfpathlineto{\pgfqpoint{4.430833in}{0.811128in}}%
\pgfpathlineto{\pgfqpoint{4.435301in}{0.819710in}}%
\pgfpathlineto{\pgfqpoint{4.439741in}{0.817228in}}%
\pgfpathlineto{\pgfqpoint{4.444154in}{0.811713in}}%
\pgfpathlineto{\pgfqpoint{4.448541in}{0.812300in}}%
\pgfpathlineto{\pgfqpoint{4.452900in}{0.816015in}}%
\pgfpathlineto{\pgfqpoint{4.457234in}{0.831922in}}%
\pgfpathlineto{\pgfqpoint{4.461542in}{0.854070in}}%
\pgfpathlineto{\pgfqpoint{4.465824in}{0.841943in}}%
\pgfpathlineto{\pgfqpoint{4.470081in}{0.835691in}}%
\pgfpathlineto{\pgfqpoint{4.474312in}{0.839977in}}%
\pgfpathlineto{\pgfqpoint{4.478520in}{0.858421in}}%
\pgfpathlineto{\pgfqpoint{4.482702in}{0.864726in}}%
\pgfpathlineto{\pgfqpoint{4.490996in}{0.840289in}}%
\pgfpathlineto{\pgfqpoint{4.495107in}{0.851350in}}%
\pgfpathlineto{\pgfqpoint{4.499195in}{0.842226in}}%
\pgfpathlineto{\pgfqpoint{4.503260in}{0.841883in}}%
\pgfpathlineto{\pgfqpoint{4.507303in}{0.819648in}}%
\pgfpathlineto{\pgfqpoint{4.511322in}{0.820917in}}%
\pgfpathlineto{\pgfqpoint{4.515320in}{0.839323in}}%
\pgfpathlineto{\pgfqpoint{4.519295in}{0.839137in}}%
\pgfpathlineto{\pgfqpoint{4.523249in}{0.840671in}}%
\pgfpathlineto{\pgfqpoint{4.527181in}{0.815580in}}%
\pgfpathlineto{\pgfqpoint{4.531092in}{0.832105in}}%
\pgfpathlineto{\pgfqpoint{4.534981in}{0.815789in}}%
\pgfpathlineto{\pgfqpoint{4.538850in}{0.814881in}}%
\pgfpathlineto{\pgfqpoint{4.546526in}{0.782603in}}%
\pgfpathlineto{\pgfqpoint{4.550334in}{0.794080in}}%
\pgfpathlineto{\pgfqpoint{4.554121in}{0.799848in}}%
\pgfpathlineto{\pgfqpoint{4.557889in}{0.832314in}}%
\pgfpathlineto{\pgfqpoint{4.561637in}{0.812600in}}%
\pgfpathlineto{\pgfqpoint{4.565366in}{0.803444in}}%
\pgfpathlineto{\pgfqpoint{4.569076in}{0.831551in}}%
\pgfpathlineto{\pgfqpoint{4.572767in}{0.838599in}}%
\pgfpathlineto{\pgfqpoint{4.576438in}{0.823233in}}%
\pgfpathlineto{\pgfqpoint{4.580092in}{0.812741in}}%
\pgfpathlineto{\pgfqpoint{4.583727in}{0.820458in}}%
\pgfpathlineto{\pgfqpoint{4.587344in}{0.824325in}}%
\pgfpathlineto{\pgfqpoint{4.590942in}{0.807686in}}%
\pgfpathlineto{\pgfqpoint{4.594523in}{0.807554in}}%
\pgfpathlineto{\pgfqpoint{4.598086in}{0.805237in}}%
\pgfpathlineto{\pgfqpoint{4.601632in}{0.783180in}}%
\pgfpathlineto{\pgfqpoint{4.605160in}{0.808376in}}%
\pgfpathlineto{\pgfqpoint{4.608671in}{0.798840in}}%
\pgfpathlineto{\pgfqpoint{4.612165in}{0.777606in}}%
\pgfpathlineto{\pgfqpoint{4.615643in}{0.787755in}}%
\pgfpathlineto{\pgfqpoint{4.622547in}{0.836397in}}%
\pgfpathlineto{\pgfqpoint{4.625975in}{0.827328in}}%
\pgfpathlineto{\pgfqpoint{4.629387in}{0.824280in}}%
\pgfpathlineto{\pgfqpoint{4.632782in}{0.844783in}}%
\pgfpathlineto{\pgfqpoint{4.636162in}{0.836369in}}%
\pgfpathlineto{\pgfqpoint{4.639525in}{0.813965in}}%
\pgfpathlineto{\pgfqpoint{4.642874in}{0.799584in}}%
\pgfpathlineto{\pgfqpoint{4.646206in}{0.797893in}}%
\pgfpathlineto{\pgfqpoint{4.649524in}{0.813493in}}%
\pgfpathlineto{\pgfqpoint{4.652826in}{0.811235in}}%
\pgfpathlineto{\pgfqpoint{4.656113in}{0.801087in}}%
\pgfpathlineto{\pgfqpoint{4.659385in}{0.812469in}}%
\pgfpathlineto{\pgfqpoint{4.662643in}{0.841565in}}%
\pgfpathlineto{\pgfqpoint{4.669114in}{0.830180in}}%
\pgfpathlineto{\pgfqpoint{4.678713in}{0.869887in}}%
\pgfpathlineto{\pgfqpoint{4.681885in}{0.851088in}}%
\pgfpathlineto{\pgfqpoint{4.685043in}{0.848285in}}%
\pgfpathlineto{\pgfqpoint{4.691317in}{0.867384in}}%
\pgfpathlineto{\pgfqpoint{4.694434in}{0.844370in}}%
\pgfpathlineto{\pgfqpoint{4.697537in}{0.846305in}}%
\pgfpathlineto{\pgfqpoint{4.700627in}{0.841324in}}%
\pgfpathlineto{\pgfqpoint{4.703704in}{0.827314in}}%
\pgfpathlineto{\pgfqpoint{4.706768in}{0.826477in}}%
\pgfpathlineto{\pgfqpoint{4.709819in}{0.805081in}}%
\pgfpathlineto{\pgfqpoint{4.712857in}{0.811628in}}%
\pgfpathlineto{\pgfqpoint{4.715882in}{0.814093in}}%
\pgfpathlineto{\pgfqpoint{4.718895in}{0.818326in}}%
\pgfpathlineto{\pgfqpoint{4.721895in}{0.828470in}}%
\pgfpathlineto{\pgfqpoint{4.724883in}{0.832564in}}%
\pgfpathlineto{\pgfqpoint{4.727858in}{0.823938in}}%
\pgfpathlineto{\pgfqpoint{4.730821in}{0.824478in}}%
\pgfpathlineto{\pgfqpoint{4.733772in}{0.811672in}}%
\pgfpathlineto{\pgfqpoint{4.736711in}{0.824064in}}%
\pgfpathlineto{\pgfqpoint{4.739638in}{0.822904in}}%
\pgfpathlineto{\pgfqpoint{4.742554in}{0.829801in}}%
\pgfpathlineto{\pgfqpoint{4.745457in}{0.847753in}}%
\pgfpathlineto{\pgfqpoint{4.748349in}{0.844650in}}%
\pgfpathlineto{\pgfqpoint{4.751230in}{0.865265in}}%
\pgfpathlineto{\pgfqpoint{4.754098in}{0.841317in}}%
\pgfpathlineto{\pgfqpoint{4.756956in}{0.847828in}}%
\pgfpathlineto{\pgfqpoint{4.759802in}{0.859204in}}%
\pgfpathlineto{\pgfqpoint{4.762637in}{0.834483in}}%
\pgfpathlineto{\pgfqpoint{4.765461in}{0.840877in}}%
\pgfpathlineto{\pgfqpoint{4.771077in}{0.825846in}}%
\pgfpathlineto{\pgfqpoint{4.773868in}{0.844910in}}%
\pgfpathlineto{\pgfqpoint{4.776648in}{0.817441in}}%
\pgfpathlineto{\pgfqpoint{4.779418in}{0.825495in}}%
\pgfpathlineto{\pgfqpoint{4.782177in}{0.843417in}}%
\pgfpathlineto{\pgfqpoint{4.784926in}{0.817104in}}%
\pgfpathlineto{\pgfqpoint{4.787664in}{0.836089in}}%
\pgfpathlineto{\pgfqpoint{4.790392in}{0.825360in}}%
\pgfpathlineto{\pgfqpoint{4.793110in}{0.825883in}}%
\pgfpathlineto{\pgfqpoint{4.795817in}{0.855593in}}%
\pgfpathlineto{\pgfqpoint{4.798515in}{0.818897in}}%
\pgfpathlineto{\pgfqpoint{4.801202in}{0.820753in}}%
\pgfpathlineto{\pgfqpoint{4.803879in}{0.812406in}}%
\pgfpathlineto{\pgfqpoint{4.806547in}{0.831218in}}%
\pgfpathlineto{\pgfqpoint{4.809204in}{0.832336in}}%
\pgfpathlineto{\pgfqpoint{4.811852in}{0.825251in}}%
\pgfpathlineto{\pgfqpoint{4.814490in}{0.836028in}}%
\pgfpathlineto{\pgfqpoint{4.817119in}{0.854743in}}%
\pgfpathlineto{\pgfqpoint{4.819738in}{0.837973in}}%
\pgfpathlineto{\pgfqpoint{4.822347in}{0.866426in}}%
\pgfpathlineto{\pgfqpoint{4.824947in}{0.836709in}}%
\pgfpathlineto{\pgfqpoint{4.827538in}{0.835714in}}%
\pgfpathlineto{\pgfqpoint{4.830120in}{0.817494in}}%
\pgfpathlineto{\pgfqpoint{4.835255in}{0.846849in}}%
\pgfpathlineto{\pgfqpoint{4.837809in}{0.803708in}}%
\pgfpathlineto{\pgfqpoint{4.840354in}{0.840348in}}%
\pgfpathlineto{\pgfqpoint{4.842891in}{0.811415in}}%
\pgfpathlineto{\pgfqpoint{4.845418in}{0.840216in}}%
\pgfpathlineto{\pgfqpoint{4.847936in}{0.828523in}}%
\pgfpathlineto{\pgfqpoint{4.850446in}{0.792104in}}%
\pgfpathlineto{\pgfqpoint{4.852947in}{0.847441in}}%
\pgfpathlineto{\pgfqpoint{4.855439in}{0.844152in}}%
\pgfpathlineto{\pgfqpoint{4.857923in}{0.851250in}}%
\pgfpathlineto{\pgfqpoint{4.860398in}{0.891481in}}%
\pgfpathlineto{\pgfqpoint{4.862865in}{0.831426in}}%
\pgfpathlineto{\pgfqpoint{4.865323in}{0.869446in}}%
\pgfpathlineto{\pgfqpoint{4.870215in}{0.831784in}}%
\pgfpathlineto{\pgfqpoint{4.872649in}{0.854416in}}%
\pgfpathlineto{\pgfqpoint{4.875074in}{0.849742in}}%
\pgfpathlineto{\pgfqpoint{4.877491in}{0.863868in}}%
\pgfpathlineto{\pgfqpoint{4.879900in}{0.861032in}}%
\pgfpathlineto{\pgfqpoint{4.882301in}{0.900939in}}%
\pgfpathlineto{\pgfqpoint{4.884695in}{0.825568in}}%
\pgfpathlineto{\pgfqpoint{4.889457in}{0.855908in}}%
\pgfpathlineto{\pgfqpoint{4.891827in}{0.802372in}}%
\pgfpathlineto{\pgfqpoint{4.894189in}{0.837804in}}%
\pgfpathlineto{\pgfqpoint{4.896543in}{0.824300in}}%
\pgfpathlineto{\pgfqpoint{4.898889in}{0.842223in}}%
\pgfpathlineto{\pgfqpoint{4.901228in}{0.842814in}}%
\pgfpathlineto{\pgfqpoint{4.903559in}{0.840062in}}%
\pgfpathlineto{\pgfqpoint{4.905883in}{0.833516in}}%
\pgfpathlineto{\pgfqpoint{4.908199in}{0.854425in}}%
\pgfpathlineto{\pgfqpoint{4.910508in}{0.859542in}}%
\pgfpathlineto{\pgfqpoint{4.912810in}{0.849863in}}%
\pgfpathlineto{\pgfqpoint{4.915104in}{0.890724in}}%
\pgfpathlineto{\pgfqpoint{4.917391in}{0.873666in}}%
\pgfpathlineto{\pgfqpoint{4.919671in}{0.866441in}}%
\pgfpathlineto{\pgfqpoint{4.921943in}{0.868382in}}%
\pgfpathlineto{\pgfqpoint{4.924209in}{0.846323in}}%
\pgfpathlineto{\pgfqpoint{4.926467in}{0.856490in}}%
\pgfpathlineto{\pgfqpoint{4.928718in}{0.846222in}}%
\pgfpathlineto{\pgfqpoint{4.930963in}{0.842565in}}%
\pgfpathlineto{\pgfqpoint{4.935430in}{0.878037in}}%
\pgfpathlineto{\pgfqpoint{4.937654in}{0.863801in}}%
\pgfpathlineto{\pgfqpoint{4.939871in}{0.877645in}}%
\pgfpathlineto{\pgfqpoint{4.942081in}{0.842129in}}%
\pgfpathlineto{\pgfqpoint{4.944284in}{0.831243in}}%
\pgfpathlineto{\pgfqpoint{4.946480in}{0.840609in}}%
\pgfpathlineto{\pgfqpoint{4.948670in}{0.803676in}}%
\pgfpathlineto{\pgfqpoint{4.950853in}{0.824192in}}%
\pgfpathlineto{\pgfqpoint{4.953030in}{0.790664in}}%
\pgfpathlineto{\pgfqpoint{4.957363in}{0.855914in}}%
\pgfpathlineto{\pgfqpoint{4.959520in}{0.845317in}}%
\pgfpathlineto{\pgfqpoint{4.961671in}{0.863095in}}%
\pgfpathlineto{\pgfqpoint{4.963815in}{0.839918in}}%
\pgfpathlineto{\pgfqpoint{4.965953in}{0.860755in}}%
\pgfpathlineto{\pgfqpoint{4.968085in}{0.864710in}}%
\pgfpathlineto{\pgfqpoint{4.970210in}{0.834701in}}%
\pgfpathlineto{\pgfqpoint{4.972329in}{0.851720in}}%
\pgfpathlineto{\pgfqpoint{4.974442in}{0.843979in}}%
\pgfpathlineto{\pgfqpoint{4.976548in}{0.845383in}}%
\pgfpathlineto{\pgfqpoint{4.978649in}{0.848862in}}%
\pgfpathlineto{\pgfqpoint{4.980743in}{0.838641in}}%
\pgfpathlineto{\pgfqpoint{4.982832in}{0.868577in}}%
\pgfpathlineto{\pgfqpoint{4.984914in}{0.837905in}}%
\pgfpathlineto{\pgfqpoint{4.986990in}{0.868820in}}%
\pgfpathlineto{\pgfqpoint{4.989061in}{0.868446in}}%
\pgfpathlineto{\pgfqpoint{4.991125in}{0.832489in}}%
\pgfpathlineto{\pgfqpoint{4.993184in}{0.853904in}}%
\pgfpathlineto{\pgfqpoint{4.995237in}{0.812188in}}%
\pgfpathlineto{\pgfqpoint{4.997284in}{0.855714in}}%
\pgfpathlineto{\pgfqpoint{4.999325in}{0.812541in}}%
\pgfpathlineto{\pgfqpoint{5.001360in}{0.829859in}}%
\pgfpathlineto{\pgfqpoint{5.003390in}{0.823044in}}%
\pgfpathlineto{\pgfqpoint{5.005414in}{0.820927in}}%
\pgfpathlineto{\pgfqpoint{5.007432in}{0.813530in}}%
\pgfpathlineto{\pgfqpoint{5.009445in}{0.826035in}}%
\pgfpathlineto{\pgfqpoint{5.011452in}{0.824302in}}%
\pgfpathlineto{\pgfqpoint{5.013453in}{0.808298in}}%
\pgfpathlineto{\pgfqpoint{5.015449in}{0.839786in}}%
\pgfpathlineto{\pgfqpoint{5.017439in}{0.810183in}}%
\pgfpathlineto{\pgfqpoint{5.019424in}{0.832038in}}%
\pgfpathlineto{\pgfqpoint{5.021404in}{0.797479in}}%
\pgfpathlineto{\pgfqpoint{5.023378in}{0.844002in}}%
\pgfpathlineto{\pgfqpoint{5.025347in}{0.806331in}}%
\pgfpathlineto{\pgfqpoint{5.027310in}{0.817019in}}%
\pgfpathlineto{\pgfqpoint{5.029268in}{0.812881in}}%
\pgfpathlineto{\pgfqpoint{5.031221in}{0.805614in}}%
\pgfpathlineto{\pgfqpoint{5.033168in}{0.806062in}}%
\pgfpathlineto{\pgfqpoint{5.035111in}{0.791191in}}%
\pgfpathlineto{\pgfqpoint{5.037048in}{0.802837in}}%
\pgfpathlineto{\pgfqpoint{5.038979in}{0.789292in}}%
\pgfpathlineto{\pgfqpoint{5.040906in}{0.819736in}}%
\pgfpathlineto{\pgfqpoint{5.042828in}{0.816804in}}%
\pgfpathlineto{\pgfqpoint{5.044744in}{0.846732in}}%
\pgfpathlineto{\pgfqpoint{5.046655in}{0.824136in}}%
\pgfpathlineto{\pgfqpoint{5.048562in}{0.852353in}}%
\pgfpathlineto{\pgfqpoint{5.050463in}{0.816394in}}%
\pgfpathlineto{\pgfqpoint{5.052359in}{0.831777in}}%
\pgfpathlineto{\pgfqpoint{5.054251in}{0.782348in}}%
\pgfpathlineto{\pgfqpoint{5.056137in}{0.803053in}}%
\pgfpathlineto{\pgfqpoint{5.058018in}{0.780203in}}%
\pgfpathlineto{\pgfqpoint{5.059895in}{0.783610in}}%
\pgfpathlineto{\pgfqpoint{5.061767in}{0.809546in}}%
\pgfpathlineto{\pgfqpoint{5.063633in}{0.781398in}}%
\pgfpathlineto{\pgfqpoint{5.065495in}{0.817575in}}%
\pgfpathlineto{\pgfqpoint{5.069205in}{0.801425in}}%
\pgfpathlineto{\pgfqpoint{5.071053in}{0.835607in}}%
\pgfpathlineto{\pgfqpoint{5.072896in}{0.788641in}}%
\pgfpathlineto{\pgfqpoint{5.074734in}{0.834644in}}%
\pgfpathlineto{\pgfqpoint{5.078397in}{0.804712in}}%
\pgfpathlineto{\pgfqpoint{5.080221in}{0.851024in}}%
\pgfpathlineto{\pgfqpoint{5.082041in}{0.806295in}}%
\pgfpathlineto{\pgfqpoint{5.085667in}{0.836846in}}%
\pgfpathlineto{\pgfqpoint{5.087473in}{0.790098in}}%
\pgfpathlineto{\pgfqpoint{5.089274in}{0.826129in}}%
\pgfpathlineto{\pgfqpoint{5.091071in}{0.830729in}}%
\pgfpathlineto{\pgfqpoint{5.092864in}{0.805447in}}%
\pgfpathlineto{\pgfqpoint{5.094652in}{0.846464in}}%
\pgfpathlineto{\pgfqpoint{5.096436in}{0.821447in}}%
\pgfpathlineto{\pgfqpoint{5.098215in}{0.829124in}}%
\pgfpathlineto{\pgfqpoint{5.099990in}{0.844577in}}%
\pgfpathlineto{\pgfqpoint{5.101761in}{0.813869in}}%
\pgfpathlineto{\pgfqpoint{5.103527in}{0.829316in}}%
\pgfpathlineto{\pgfqpoint{5.107047in}{0.807470in}}%
\pgfpathlineto{\pgfqpoint{5.108800in}{0.814212in}}%
\pgfpathlineto{\pgfqpoint{5.110550in}{0.783478in}}%
\pgfpathlineto{\pgfqpoint{5.112295in}{0.795842in}}%
\pgfpathlineto{\pgfqpoint{5.114035in}{0.798304in}}%
\pgfpathlineto{\pgfqpoint{5.115772in}{0.802743in}}%
\pgfpathlineto{\pgfqpoint{5.117504in}{0.817805in}}%
\pgfpathlineto{\pgfqpoint{5.119232in}{0.821629in}}%
\pgfpathlineto{\pgfqpoint{5.120957in}{0.840456in}}%
\pgfpathlineto{\pgfqpoint{5.122677in}{0.805513in}}%
\pgfpathlineto{\pgfqpoint{5.124392in}{0.836777in}}%
\pgfpathlineto{\pgfqpoint{5.126104in}{0.807885in}}%
\pgfpathlineto{\pgfqpoint{5.129516in}{0.810908in}}%
\pgfpathlineto{\pgfqpoint{5.131215in}{0.779956in}}%
\pgfpathlineto{\pgfqpoint{5.132911in}{0.849535in}}%
\pgfpathlineto{\pgfqpoint{5.134603in}{0.808153in}}%
\pgfpathlineto{\pgfqpoint{5.136291in}{0.812595in}}%
\pgfpathlineto{\pgfqpoint{5.137975in}{0.844944in}}%
\pgfpathlineto{\pgfqpoint{5.139655in}{0.797534in}}%
\pgfpathlineto{\pgfqpoint{5.141331in}{0.831578in}}%
\pgfpathlineto{\pgfqpoint{5.144671in}{0.788682in}}%
\pgfpathlineto{\pgfqpoint{5.146336in}{0.831599in}}%
\pgfpathlineto{\pgfqpoint{5.147996in}{0.815696in}}%
\pgfpathlineto{\pgfqpoint{5.151306in}{0.842896in}}%
\pgfpathlineto{\pgfqpoint{5.152955in}{0.843135in}}%
\pgfpathlineto{\pgfqpoint{5.154601in}{0.839447in}}%
\pgfpathlineto{\pgfqpoint{5.156242in}{0.839965in}}%
\pgfpathlineto{\pgfqpoint{5.157880in}{0.822684in}}%
\pgfpathlineto{\pgfqpoint{5.159515in}{0.841717in}}%
\pgfpathlineto{\pgfqpoint{5.161145in}{0.804330in}}%
\pgfpathlineto{\pgfqpoint{5.162772in}{0.832700in}}%
\pgfpathlineto{\pgfqpoint{5.164395in}{0.820136in}}%
\pgfpathlineto{\pgfqpoint{5.166015in}{0.829308in}}%
\pgfpathlineto{\pgfqpoint{5.167631in}{0.844886in}}%
\pgfpathlineto{\pgfqpoint{5.169243in}{0.821705in}}%
\pgfpathlineto{\pgfqpoint{5.170852in}{0.866364in}}%
\pgfpathlineto{\pgfqpoint{5.172457in}{0.838637in}}%
\pgfpathlineto{\pgfqpoint{5.174059in}{0.857302in}}%
\pgfpathlineto{\pgfqpoint{5.175657in}{0.850303in}}%
\pgfpathlineto{\pgfqpoint{5.177252in}{0.856812in}}%
\pgfpathlineto{\pgfqpoint{5.178843in}{0.856824in}}%
\pgfpathlineto{\pgfqpoint{5.182014in}{0.847961in}}%
\pgfpathlineto{\pgfqpoint{5.183595in}{0.882607in}}%
\pgfpathlineto{\pgfqpoint{5.185172in}{0.838863in}}%
\pgfpathlineto{\pgfqpoint{5.186746in}{0.876267in}}%
\pgfpathlineto{\pgfqpoint{5.188316in}{0.807609in}}%
\pgfpathlineto{\pgfqpoint{5.189883in}{0.876374in}}%
\pgfpathlineto{\pgfqpoint{5.191446in}{0.820272in}}%
\pgfpathlineto{\pgfqpoint{5.194563in}{0.823701in}}%
\pgfpathlineto{\pgfqpoint{5.196116in}{0.834136in}}%
\pgfpathlineto{\pgfqpoint{5.197666in}{0.834868in}}%
\pgfpathlineto{\pgfqpoint{5.199213in}{0.852150in}}%
\pgfpathlineto{\pgfqpoint{5.200756in}{0.839768in}}%
\pgfpathlineto{\pgfqpoint{5.202296in}{0.845018in}}%
\pgfpathlineto{\pgfqpoint{5.203833in}{0.839381in}}%
\pgfpathlineto{\pgfqpoint{5.205367in}{0.807729in}}%
\pgfpathlineto{\pgfqpoint{5.206897in}{0.839047in}}%
\pgfpathlineto{\pgfqpoint{5.208424in}{0.812039in}}%
\pgfpathlineto{\pgfqpoint{5.209948in}{0.812579in}}%
\pgfpathlineto{\pgfqpoint{5.211469in}{0.802168in}}%
\pgfpathlineto{\pgfqpoint{5.212986in}{0.829389in}}%
\pgfpathlineto{\pgfqpoint{5.214500in}{0.827336in}}%
\pgfpathlineto{\pgfqpoint{5.216011in}{0.835424in}}%
\pgfpathlineto{\pgfqpoint{5.217519in}{0.829072in}}%
\pgfpathlineto{\pgfqpoint{5.220526in}{0.842655in}}%
\pgfpathlineto{\pgfqpoint{5.222024in}{0.817717in}}%
\pgfpathlineto{\pgfqpoint{5.223520in}{0.815169in}}%
\pgfpathlineto{\pgfqpoint{5.225012in}{0.791253in}}%
\pgfpathlineto{\pgfqpoint{5.226501in}{0.815552in}}%
\pgfpathlineto{\pgfqpoint{5.227987in}{0.792542in}}%
\pgfpathlineto{\pgfqpoint{5.229470in}{0.831791in}}%
\pgfpathlineto{\pgfqpoint{5.230950in}{0.791885in}}%
\pgfpathlineto{\pgfqpoint{5.232427in}{0.916061in}}%
\pgfpathlineto{\pgfqpoint{5.233901in}{0.835834in}}%
\pgfpathlineto{\pgfqpoint{5.236841in}{0.903617in}}%
\pgfpathlineto{\pgfqpoint{5.238306in}{0.851376in}}%
\pgfpathlineto{\pgfqpoint{5.239768in}{0.892279in}}%
\pgfpathlineto{\pgfqpoint{5.241227in}{0.880107in}}%
\pgfpathlineto{\pgfqpoint{5.242683in}{0.835914in}}%
\pgfpathlineto{\pgfqpoint{5.244136in}{0.860370in}}%
\pgfpathlineto{\pgfqpoint{5.245587in}{0.863151in}}%
\pgfpathlineto{\pgfqpoint{5.247034in}{0.815007in}}%
\pgfpathlineto{\pgfqpoint{5.248478in}{0.849053in}}%
\pgfpathlineto{\pgfqpoint{5.251359in}{0.817261in}}%
\pgfpathlineto{\pgfqpoint{5.252795in}{0.856453in}}%
\pgfpathlineto{\pgfqpoint{5.255658in}{0.820032in}}%
\pgfpathlineto{\pgfqpoint{5.257085in}{0.839331in}}%
\pgfpathlineto{\pgfqpoint{5.258510in}{0.831662in}}%
\pgfpathlineto{\pgfqpoint{5.259932in}{0.839284in}}%
\pgfpathlineto{\pgfqpoint{5.261351in}{0.824192in}}%
\pgfpathlineto{\pgfqpoint{5.262767in}{0.832637in}}%
\pgfpathlineto{\pgfqpoint{5.264180in}{0.831580in}}%
\pgfpathlineto{\pgfqpoint{5.265591in}{0.837696in}}%
\pgfpathlineto{\pgfqpoint{5.268404in}{0.794810in}}%
\pgfpathlineto{\pgfqpoint{5.269806in}{0.817488in}}%
\pgfpathlineto{\pgfqpoint{5.271206in}{0.771589in}}%
\pgfpathlineto{\pgfqpoint{5.272603in}{0.806692in}}%
\pgfpathlineto{\pgfqpoint{5.273997in}{0.796742in}}%
\pgfpathlineto{\pgfqpoint{5.276778in}{0.842024in}}%
\pgfpathlineto{\pgfqpoint{5.278164in}{0.813027in}}%
\pgfpathlineto{\pgfqpoint{5.279547in}{0.865600in}}%
\pgfpathlineto{\pgfqpoint{5.280928in}{0.845577in}}%
\pgfpathlineto{\pgfqpoint{5.282307in}{0.862651in}}%
\pgfpathlineto{\pgfqpoint{5.283682in}{0.853048in}}%
\pgfpathlineto{\pgfqpoint{5.285055in}{0.824086in}}%
\pgfpathlineto{\pgfqpoint{5.286426in}{0.846496in}}%
\pgfpathlineto{\pgfqpoint{5.289159in}{0.807549in}}%
\pgfpathlineto{\pgfqpoint{5.290521in}{0.816313in}}%
\pgfpathlineto{\pgfqpoint{5.291882in}{0.806296in}}%
\pgfpathlineto{\pgfqpoint{5.294594in}{0.843900in}}%
\pgfpathlineto{\pgfqpoint{5.295947in}{0.818060in}}%
\pgfpathlineto{\pgfqpoint{5.297296in}{0.862884in}}%
\pgfpathlineto{\pgfqpoint{5.298644in}{0.823184in}}%
\pgfpathlineto{\pgfqpoint{5.299989in}{0.844733in}}%
\pgfpathlineto{\pgfqpoint{5.301331in}{0.846056in}}%
\pgfpathlineto{\pgfqpoint{5.302671in}{0.801737in}}%
\pgfpathlineto{\pgfqpoint{5.304008in}{0.849702in}}%
\pgfpathlineto{\pgfqpoint{5.305343in}{0.814298in}}%
\pgfpathlineto{\pgfqpoint{5.306676in}{0.810080in}}%
\pgfpathlineto{\pgfqpoint{5.308006in}{0.853526in}}%
\pgfpathlineto{\pgfqpoint{5.309333in}{0.813340in}}%
\pgfpathlineto{\pgfqpoint{5.310659in}{0.836824in}}%
\pgfpathlineto{\pgfqpoint{5.311981in}{0.839675in}}%
\pgfpathlineto{\pgfqpoint{5.313302in}{0.797419in}}%
\pgfpathlineto{\pgfqpoint{5.314619in}{0.841212in}}%
\pgfpathlineto{\pgfqpoint{5.315935in}{0.790409in}}%
\pgfpathlineto{\pgfqpoint{5.318559in}{0.834093in}}%
\pgfpathlineto{\pgfqpoint{5.319867in}{0.818583in}}%
\pgfpathlineto{\pgfqpoint{5.321173in}{0.843639in}}%
\pgfpathlineto{\pgfqpoint{5.322477in}{0.847267in}}%
\pgfpathlineto{\pgfqpoint{5.323778in}{0.834699in}}%
\pgfpathlineto{\pgfqpoint{5.325077in}{0.878605in}}%
\pgfpathlineto{\pgfqpoint{5.327667in}{0.839568in}}%
\pgfpathlineto{\pgfqpoint{5.328959in}{0.864033in}}%
\pgfpathlineto{\pgfqpoint{5.330249in}{0.816800in}}%
\pgfpathlineto{\pgfqpoint{5.331536in}{0.853155in}}%
\pgfpathlineto{\pgfqpoint{5.332821in}{0.817558in}}%
\pgfpathlineto{\pgfqpoint{5.339212in}{0.859902in}}%
\pgfpathlineto{\pgfqpoint{5.340484in}{0.847554in}}%
\pgfpathlineto{\pgfqpoint{5.343020in}{0.860289in}}%
\pgfpathlineto{\pgfqpoint{5.344285in}{0.824612in}}%
\pgfpathlineto{\pgfqpoint{5.345547in}{0.852700in}}%
\pgfpathlineto{\pgfqpoint{5.346807in}{0.794796in}}%
\pgfpathlineto{\pgfqpoint{5.348065in}{0.821488in}}%
\pgfpathlineto{\pgfqpoint{5.350575in}{0.786899in}}%
\pgfpathlineto{\pgfqpoint{5.351827in}{0.825437in}}%
\pgfpathlineto{\pgfqpoint{5.353076in}{0.781989in}}%
\pgfpathlineto{\pgfqpoint{5.354323in}{0.846816in}}%
\pgfpathlineto{\pgfqpoint{5.355569in}{0.835954in}}%
\pgfpathlineto{\pgfqpoint{5.356811in}{0.810484in}}%
\pgfpathlineto{\pgfqpoint{5.358052in}{0.847586in}}%
\pgfpathlineto{\pgfqpoint{5.359291in}{0.796121in}}%
\pgfpathlineto{\pgfqpoint{5.360528in}{0.826292in}}%
\pgfpathlineto{\pgfqpoint{5.362994in}{0.773826in}}%
\pgfpathlineto{\pgfqpoint{5.364225in}{0.828351in}}%
\pgfpathlineto{\pgfqpoint{5.365453in}{0.811230in}}%
\pgfpathlineto{\pgfqpoint{5.367903in}{0.844199in}}%
\pgfpathlineto{\pgfqpoint{5.369125in}{0.844541in}}%
\pgfpathlineto{\pgfqpoint{5.370344in}{0.858442in}}%
\pgfpathlineto{\pgfqpoint{5.376413in}{0.808411in}}%
\pgfpathlineto{\pgfqpoint{5.377621in}{0.815802in}}%
\pgfpathlineto{\pgfqpoint{5.378826in}{0.802057in}}%
\pgfpathlineto{\pgfqpoint{5.380030in}{0.810671in}}%
\pgfpathlineto{\pgfqpoint{5.381231in}{0.832392in}}%
\pgfpathlineto{\pgfqpoint{5.382431in}{0.792923in}}%
\pgfpathlineto{\pgfqpoint{5.383628in}{0.832010in}}%
\pgfpathlineto{\pgfqpoint{5.386018in}{0.791183in}}%
\pgfpathlineto{\pgfqpoint{5.387209in}{0.847352in}}%
\pgfpathlineto{\pgfqpoint{5.388399in}{0.782111in}}%
\pgfpathlineto{\pgfqpoint{5.390772in}{0.848055in}}%
\pgfpathlineto{\pgfqpoint{5.391956in}{0.802509in}}%
\pgfpathlineto{\pgfqpoint{5.393138in}{0.846061in}}%
\pgfpathlineto{\pgfqpoint{5.395496in}{0.806689in}}%
\pgfpathlineto{\pgfqpoint{5.396672in}{0.831583in}}%
\pgfpathlineto{\pgfqpoint{5.397846in}{0.824508in}}%
\pgfpathlineto{\pgfqpoint{5.399018in}{0.807127in}}%
\pgfpathlineto{\pgfqpoint{5.400189in}{0.859505in}}%
\pgfpathlineto{\pgfqpoint{5.401357in}{0.820049in}}%
\pgfpathlineto{\pgfqpoint{5.402524in}{0.851400in}}%
\pgfpathlineto{\pgfqpoint{5.403689in}{0.851247in}}%
\pgfpathlineto{\pgfqpoint{5.404851in}{0.812575in}}%
\pgfpathlineto{\pgfqpoint{5.406012in}{0.844045in}}%
\pgfpathlineto{\pgfqpoint{5.407171in}{0.816352in}}%
\pgfpathlineto{\pgfqpoint{5.411789in}{0.861662in}}%
\pgfpathlineto{\pgfqpoint{5.412939in}{0.870803in}}%
\pgfpathlineto{\pgfqpoint{5.414087in}{0.849094in}}%
\pgfpathlineto{\pgfqpoint{5.415233in}{0.872145in}}%
\pgfpathlineto{\pgfqpoint{5.417520in}{0.827929in}}%
\pgfpathlineto{\pgfqpoint{5.418661in}{0.840054in}}%
\pgfpathlineto{\pgfqpoint{5.419800in}{0.810489in}}%
\pgfpathlineto{\pgfqpoint{5.420937in}{0.841603in}}%
\pgfpathlineto{\pgfqpoint{5.423206in}{0.808575in}}%
\pgfpathlineto{\pgfqpoint{5.424338in}{0.845635in}}%
\pgfpathlineto{\pgfqpoint{5.425468in}{0.797689in}}%
\pgfpathlineto{\pgfqpoint{5.426596in}{0.865585in}}%
\pgfpathlineto{\pgfqpoint{5.428848in}{0.826503in}}%
\pgfpathlineto{\pgfqpoint{5.429971in}{0.883709in}}%
\pgfpathlineto{\pgfqpoint{5.431092in}{0.809927in}}%
\pgfpathlineto{\pgfqpoint{5.432211in}{0.867180in}}%
\pgfpathlineto{\pgfqpoint{5.433329in}{0.844507in}}%
\pgfpathlineto{\pgfqpoint{5.434445in}{0.782645in}}%
\pgfpathlineto{\pgfqpoint{5.435560in}{0.855126in}}%
\pgfpathlineto{\pgfqpoint{5.436672in}{0.804461in}}%
\pgfpathlineto{\pgfqpoint{5.438892in}{0.879745in}}%
\pgfpathlineto{\pgfqpoint{5.440000in}{0.824158in}}%
\pgfpathlineto{\pgfqpoint{5.441106in}{0.875889in}}%
\pgfpathlineto{\pgfqpoint{5.443312in}{0.843280in}}%
\pgfpathlineto{\pgfqpoint{5.444413in}{0.875016in}}%
\pgfpathlineto{\pgfqpoint{5.446609in}{0.854588in}}%
\pgfpathlineto{\pgfqpoint{5.448799in}{0.885605in}}%
\pgfpathlineto{\pgfqpoint{5.449892in}{0.883457in}}%
\pgfpathlineto{\pgfqpoint{5.450982in}{0.883938in}}%
\pgfpathlineto{\pgfqpoint{5.453159in}{0.857131in}}%
\pgfpathlineto{\pgfqpoint{5.454245in}{0.879954in}}%
\pgfpathlineto{\pgfqpoint{5.455329in}{0.835401in}}%
\pgfpathlineto{\pgfqpoint{5.457492in}{0.867411in}}%
\pgfpathlineto{\pgfqpoint{5.458572in}{0.823953in}}%
\pgfpathlineto{\pgfqpoint{5.459650in}{0.881047in}}%
\pgfpathlineto{\pgfqpoint{5.460726in}{0.834739in}}%
\pgfpathlineto{\pgfqpoint{5.462873in}{0.857788in}}%
\pgfpathlineto{\pgfqpoint{5.463944in}{0.812966in}}%
\pgfpathlineto{\pgfqpoint{5.465014in}{0.863393in}}%
\pgfpathlineto{\pgfqpoint{5.466082in}{0.811005in}}%
\pgfpathlineto{\pgfqpoint{5.467149in}{0.839694in}}%
\pgfpathlineto{\pgfqpoint{5.469277in}{0.818029in}}%
\pgfpathlineto{\pgfqpoint{5.470339in}{0.855123in}}%
\pgfpathlineto{\pgfqpoint{5.471399in}{0.817856in}}%
\pgfpathlineto{\pgfqpoint{5.472458in}{0.837191in}}%
\pgfpathlineto{\pgfqpoint{5.474571in}{0.831044in}}%
\pgfpathlineto{\pgfqpoint{5.475625in}{0.831715in}}%
\pgfpathlineto{\pgfqpoint{5.476678in}{0.820226in}}%
\pgfpathlineto{\pgfqpoint{5.477729in}{0.827335in}}%
\pgfpathlineto{\pgfqpoint{5.478778in}{0.824382in}}%
\pgfpathlineto{\pgfqpoint{5.480873in}{0.803750in}}%
\pgfpathlineto{\pgfqpoint{5.481918in}{0.802744in}}%
\pgfpathlineto{\pgfqpoint{5.482961in}{0.792280in}}%
\pgfpathlineto{\pgfqpoint{5.485043in}{0.826366in}}%
\pgfpathlineto{\pgfqpoint{5.486082in}{0.813213in}}%
\pgfpathlineto{\pgfqpoint{5.487120in}{0.835769in}}%
\pgfpathlineto{\pgfqpoint{5.488156in}{0.818114in}}%
\pgfpathlineto{\pgfqpoint{5.490223in}{0.843300in}}%
\pgfpathlineto{\pgfqpoint{5.491255in}{0.798463in}}%
\pgfpathlineto{\pgfqpoint{5.492285in}{0.830542in}}%
\pgfpathlineto{\pgfqpoint{5.493313in}{0.823966in}}%
\pgfpathlineto{\pgfqpoint{5.494340in}{0.807286in}}%
\pgfpathlineto{\pgfqpoint{5.495366in}{0.818398in}}%
\pgfpathlineto{\pgfqpoint{5.496390in}{0.785426in}}%
\pgfpathlineto{\pgfqpoint{5.497413in}{0.798728in}}%
\pgfpathlineto{\pgfqpoint{5.498434in}{0.787020in}}%
\pgfpathlineto{\pgfqpoint{5.500472in}{0.828589in}}%
\pgfpathlineto{\pgfqpoint{5.501489in}{0.827183in}}%
\pgfpathlineto{\pgfqpoint{5.502505in}{0.840690in}}%
\pgfpathlineto{\pgfqpoint{5.503519in}{0.824038in}}%
\pgfpathlineto{\pgfqpoint{5.504532in}{0.847797in}}%
\pgfpathlineto{\pgfqpoint{5.505543in}{0.825945in}}%
\pgfpathlineto{\pgfqpoint{5.506553in}{0.837917in}}%
\pgfpathlineto{\pgfqpoint{5.507561in}{0.835680in}}%
\pgfpathlineto{\pgfqpoint{5.508568in}{0.822857in}}%
\pgfpathlineto{\pgfqpoint{5.509574in}{0.846529in}}%
\pgfpathlineto{\pgfqpoint{5.510578in}{0.846431in}}%
\pgfpathlineto{\pgfqpoint{5.511581in}{0.833392in}}%
\pgfpathlineto{\pgfqpoint{5.512582in}{0.851954in}}%
\pgfpathlineto{\pgfqpoint{5.513582in}{0.813169in}}%
\pgfpathlineto{\pgfqpoint{5.514581in}{0.846132in}}%
\pgfpathlineto{\pgfqpoint{5.516574in}{0.819860in}}%
\pgfpathlineto{\pgfqpoint{5.517569in}{0.829967in}}%
\pgfpathlineto{\pgfqpoint{5.518562in}{0.816732in}}%
\pgfpathlineto{\pgfqpoint{5.519554in}{0.827849in}}%
\pgfpathlineto{\pgfqpoint{5.520544in}{0.824522in}}%
\pgfpathlineto{\pgfqpoint{5.521533in}{0.816951in}}%
\pgfpathlineto{\pgfqpoint{5.523507in}{0.817952in}}%
\pgfpathlineto{\pgfqpoint{5.524492in}{0.812207in}}%
\pgfpathlineto{\pgfqpoint{5.525476in}{0.818294in}}%
\pgfpathlineto{\pgfqpoint{5.526458in}{0.840364in}}%
\pgfpathlineto{\pgfqpoint{5.527439in}{0.840295in}}%
\pgfpathlineto{\pgfqpoint{5.528419in}{0.851838in}}%
\pgfpathlineto{\pgfqpoint{5.529397in}{0.848975in}}%
\pgfpathlineto{\pgfqpoint{5.530374in}{0.860615in}}%
\pgfpathlineto{\pgfqpoint{5.531350in}{0.860146in}}%
\pgfpathlineto{\pgfqpoint{5.534269in}{0.846028in}}%
\pgfpathlineto{\pgfqpoint{5.535240in}{0.846440in}}%
\pgfpathlineto{\pgfqpoint{5.536209in}{0.866970in}}%
\pgfpathlineto{\pgfqpoint{5.538143in}{0.840559in}}%
\pgfpathlineto{\pgfqpoint{5.539109in}{0.855201in}}%
\pgfpathlineto{\pgfqpoint{5.540073in}{0.837569in}}%
\pgfpathlineto{\pgfqpoint{5.541035in}{0.857090in}}%
\pgfpathlineto{\pgfqpoint{5.541997in}{0.808679in}}%
\pgfpathlineto{\pgfqpoint{5.542957in}{0.828382in}}%
\pgfpathlineto{\pgfqpoint{5.543916in}{0.820647in}}%
\pgfpathlineto{\pgfqpoint{5.544873in}{0.800966in}}%
\pgfpathlineto{\pgfqpoint{5.545830in}{0.839767in}}%
\pgfpathlineto{\pgfqpoint{5.546785in}{0.789048in}}%
\pgfpathlineto{\pgfqpoint{5.547738in}{0.840138in}}%
\pgfpathlineto{\pgfqpoint{5.549642in}{0.816479in}}%
\pgfpathlineto{\pgfqpoint{5.550592in}{0.850992in}}%
\pgfpathlineto{\pgfqpoint{5.551541in}{0.811645in}}%
\pgfpathlineto{\pgfqpoint{5.553435in}{0.842254in}}%
\pgfpathlineto{\pgfqpoint{5.554380in}{0.816215in}}%
\pgfpathlineto{\pgfqpoint{5.555324in}{0.843125in}}%
\pgfpathlineto{\pgfqpoint{5.557207in}{0.822669in}}%
\pgfpathlineto{\pgfqpoint{5.558148in}{0.837837in}}%
\pgfpathlineto{\pgfqpoint{5.559086in}{0.836502in}}%
\pgfpathlineto{\pgfqpoint{5.560024in}{0.818008in}}%
\pgfpathlineto{\pgfqpoint{5.560961in}{0.846925in}}%
\pgfpathlineto{\pgfqpoint{5.561896in}{0.820213in}}%
\pgfpathlineto{\pgfqpoint{5.562830in}{0.841950in}}%
\pgfpathlineto{\pgfqpoint{5.563763in}{0.839737in}}%
\pgfpathlineto{\pgfqpoint{5.564694in}{0.856946in}}%
\pgfpathlineto{\pgfqpoint{5.565625in}{0.836134in}}%
\pgfpathlineto{\pgfqpoint{5.566554in}{0.871671in}}%
\pgfpathlineto{\pgfqpoint{5.567482in}{0.833904in}}%
\pgfpathlineto{\pgfqpoint{5.568409in}{0.862192in}}%
\pgfpathlineto{\pgfqpoint{5.569334in}{0.835552in}}%
\pgfpathlineto{\pgfqpoint{5.570259in}{0.841274in}}%
\pgfpathlineto{\pgfqpoint{5.571182in}{0.806133in}}%
\pgfpathlineto{\pgfqpoint{5.572104in}{0.841514in}}%
\pgfpathlineto{\pgfqpoint{5.573025in}{0.787461in}}%
\pgfpathlineto{\pgfqpoint{5.573945in}{0.828406in}}%
\pgfpathlineto{\pgfqpoint{5.574863in}{0.821897in}}%
\pgfpathlineto{\pgfqpoint{5.575781in}{0.816158in}}%
\pgfpathlineto{\pgfqpoint{5.576697in}{0.829094in}}%
\pgfpathlineto{\pgfqpoint{5.577612in}{0.813224in}}%
\pgfpathlineto{\pgfqpoint{5.578526in}{0.845252in}}%
\pgfpathlineto{\pgfqpoint{5.579439in}{0.799989in}}%
\pgfpathlineto{\pgfqpoint{5.580350in}{0.843256in}}%
\pgfpathlineto{\pgfqpoint{5.581261in}{0.804491in}}%
\pgfpathlineto{\pgfqpoint{5.582170in}{0.805736in}}%
\pgfpathlineto{\pgfqpoint{5.583078in}{0.814846in}}%
\pgfpathlineto{\pgfqpoint{5.583985in}{0.797730in}}%
\pgfpathlineto{\pgfqpoint{5.584891in}{0.801987in}}%
\pgfpathlineto{\pgfqpoint{5.587602in}{0.843773in}}%
\pgfpathlineto{\pgfqpoint{5.588503in}{0.826922in}}%
\pgfpathlineto{\pgfqpoint{5.589404in}{0.844850in}}%
\pgfpathlineto{\pgfqpoint{5.591201in}{0.826064in}}%
\pgfpathlineto{\pgfqpoint{5.592098in}{0.845479in}}%
\pgfpathlineto{\pgfqpoint{5.592993in}{0.798195in}}%
\pgfpathlineto{\pgfqpoint{5.593888in}{0.847138in}}%
\pgfpathlineto{\pgfqpoint{5.594782in}{0.837535in}}%
\pgfpathlineto{\pgfqpoint{5.595674in}{0.836334in}}%
\pgfpathlineto{\pgfqpoint{5.596565in}{0.843531in}}%
\pgfpathlineto{\pgfqpoint{5.597456in}{0.811235in}}%
\pgfpathlineto{\pgfqpoint{5.598345in}{0.852532in}}%
\pgfpathlineto{\pgfqpoint{5.600120in}{0.810953in}}%
\pgfpathlineto{\pgfqpoint{5.601006in}{0.821946in}}%
\pgfpathlineto{\pgfqpoint{5.601890in}{0.793007in}}%
\pgfpathlineto{\pgfqpoint{5.602774in}{0.829836in}}%
\pgfpathlineto{\pgfqpoint{5.604538in}{0.813161in}}%
\pgfpathlineto{\pgfqpoint{5.606298in}{0.827416in}}%
\pgfpathlineto{\pgfqpoint{5.607176in}{0.844549in}}%
\pgfpathlineto{\pgfqpoint{5.608054in}{0.837886in}}%
\pgfpathlineto{\pgfqpoint{5.608930in}{0.846903in}}%
\pgfpathlineto{\pgfqpoint{5.609805in}{0.828169in}}%
\pgfpathlineto{\pgfqpoint{5.610679in}{0.877628in}}%
\pgfpathlineto{\pgfqpoint{5.611552in}{0.821396in}}%
\pgfpathlineto{\pgfqpoint{5.612424in}{0.845976in}}%
\pgfpathlineto{\pgfqpoint{5.613295in}{0.835948in}}%
\pgfpathlineto{\pgfqpoint{5.614165in}{0.801267in}}%
\pgfpathlineto{\pgfqpoint{5.615033in}{0.843686in}}%
\pgfpathlineto{\pgfqpoint{5.615901in}{0.780394in}}%
\pgfpathlineto{\pgfqpoint{5.616768in}{0.825809in}}%
\pgfpathlineto{\pgfqpoint{5.617633in}{0.741441in}}%
\pgfpathlineto{\pgfqpoint{5.618498in}{0.825801in}}%
\pgfpathlineto{\pgfqpoint{5.620224in}{0.766531in}}%
\pgfpathlineto{\pgfqpoint{5.621086in}{0.805438in}}%
\pgfpathlineto{\pgfqpoint{5.621946in}{0.770988in}}%
\pgfpathlineto{\pgfqpoint{5.623664in}{0.828722in}}%
\pgfpathlineto{\pgfqpoint{5.624522in}{0.795871in}}%
\pgfpathlineto{\pgfqpoint{5.626234in}{0.835565in}}%
\pgfpathlineto{\pgfqpoint{5.627088in}{0.812877in}}%
\pgfpathlineto{\pgfqpoint{5.627941in}{0.835875in}}%
\pgfpathlineto{\pgfqpoint{5.628794in}{0.807925in}}%
\pgfpathlineto{\pgfqpoint{5.630495in}{0.851858in}}%
\pgfpathlineto{\pgfqpoint{5.631345in}{0.812016in}}%
\pgfpathlineto{\pgfqpoint{5.632193in}{0.838756in}}%
\pgfpathlineto{\pgfqpoint{5.633887in}{0.823161in}}%
\pgfpathlineto{\pgfqpoint{5.634732in}{0.837021in}}%
\pgfpathlineto{\pgfqpoint{5.635577in}{0.825868in}}%
\pgfpathlineto{\pgfqpoint{5.636420in}{0.830559in}}%
\pgfpathlineto{\pgfqpoint{5.637262in}{0.853565in}}%
\pgfpathlineto{\pgfqpoint{5.638104in}{0.843086in}}%
\pgfpathlineto{\pgfqpoint{5.638944in}{0.882888in}}%
\pgfpathlineto{\pgfqpoint{5.639784in}{0.840630in}}%
\pgfpathlineto{\pgfqpoint{5.640622in}{0.885007in}}%
\pgfpathlineto{\pgfqpoint{5.641460in}{0.843198in}}%
\pgfpathlineto{\pgfqpoint{5.642296in}{0.876569in}}%
\pgfpathlineto{\pgfqpoint{5.643132in}{0.842009in}}%
\pgfpathlineto{\pgfqpoint{5.643967in}{0.868481in}}%
\pgfpathlineto{\pgfqpoint{5.644800in}{0.865699in}}%
\pgfpathlineto{\pgfqpoint{5.645633in}{0.881652in}}%
\pgfpathlineto{\pgfqpoint{5.646465in}{0.867409in}}%
\pgfpathlineto{\pgfqpoint{5.648125in}{0.904307in}}%
\pgfpathlineto{\pgfqpoint{5.648954in}{0.868849in}}%
\pgfpathlineto{\pgfqpoint{5.649782in}{0.895634in}}%
\pgfpathlineto{\pgfqpoint{5.650609in}{0.847501in}}%
\pgfpathlineto{\pgfqpoint{5.651435in}{0.861825in}}%
\pgfpathlineto{\pgfqpoint{5.652260in}{0.867554in}}%
\pgfpathlineto{\pgfqpoint{5.653084in}{0.794708in}}%
\pgfpathlineto{\pgfqpoint{5.653908in}{0.861819in}}%
\pgfpathlineto{\pgfqpoint{5.654730in}{0.811817in}}%
\pgfpathlineto{\pgfqpoint{5.656372in}{0.878028in}}%
\pgfpathlineto{\pgfqpoint{5.657191in}{0.816775in}}%
\pgfpathlineto{\pgfqpoint{5.658010in}{0.855913in}}%
\pgfpathlineto{\pgfqpoint{5.658827in}{0.848633in}}%
\pgfpathlineto{\pgfqpoint{5.659644in}{0.844347in}}%
\pgfpathlineto{\pgfqpoint{5.660460in}{0.852820in}}%
\pgfpathlineto{\pgfqpoint{5.661274in}{0.840026in}}%
\pgfpathlineto{\pgfqpoint{5.662088in}{0.874408in}}%
\pgfpathlineto{\pgfqpoint{5.663713in}{0.846683in}}%
\pgfpathlineto{\pgfqpoint{5.664525in}{0.847086in}}%
\pgfpathlineto{\pgfqpoint{5.665335in}{0.798385in}}%
\pgfpathlineto{\pgfqpoint{5.666144in}{0.840449in}}%
\pgfpathlineto{\pgfqpoint{5.666953in}{0.795288in}}%
\pgfpathlineto{\pgfqpoint{5.667760in}{0.838844in}}%
\pgfpathlineto{\pgfqpoint{5.668567in}{0.802778in}}%
\pgfpathlineto{\pgfqpoint{5.669373in}{0.840915in}}%
\pgfpathlineto{\pgfqpoint{5.670177in}{0.837012in}}%
\pgfpathlineto{\pgfqpoint{5.670981in}{0.845823in}}%
\pgfpathlineto{\pgfqpoint{5.672587in}{0.808197in}}%
\pgfpathlineto{\pgfqpoint{5.673388in}{0.865057in}}%
\pgfpathlineto{\pgfqpoint{5.674188in}{0.798102in}}%
\pgfpathlineto{\pgfqpoint{5.674988in}{0.840293in}}%
\pgfpathlineto{\pgfqpoint{5.675786in}{0.797434in}}%
\pgfpathlineto{\pgfqpoint{5.676584in}{0.851353in}}%
\pgfpathlineto{\pgfqpoint{5.677381in}{0.821214in}}%
\pgfpathlineto{\pgfqpoint{5.678177in}{0.872512in}}%
\pgfpathlineto{\pgfqpoint{5.678972in}{0.786111in}}%
\pgfpathlineto{\pgfqpoint{5.679766in}{0.910160in}}%
\pgfpathlineto{\pgfqpoint{5.680559in}{0.861482in}}%
\pgfpathlineto{\pgfqpoint{5.681352in}{0.827614in}}%
\pgfpathlineto{\pgfqpoint{5.682143in}{0.872426in}}%
\pgfpathlineto{\pgfqpoint{5.682934in}{0.811279in}}%
\pgfpathlineto{\pgfqpoint{5.683724in}{0.821693in}}%
\pgfpathlineto{\pgfqpoint{5.684513in}{0.828914in}}%
\pgfpathlineto{\pgfqpoint{5.685301in}{0.799521in}}%
\pgfpathlineto{\pgfqpoint{5.686875in}{0.854083in}}%
\pgfpathlineto{\pgfqpoint{5.687660in}{0.793263in}}%
\pgfpathlineto{\pgfqpoint{5.689229in}{0.836590in}}%
\pgfpathlineto{\pgfqpoint{5.690012in}{0.783904in}}%
\pgfpathlineto{\pgfqpoint{5.690794in}{0.841849in}}%
\pgfpathlineto{\pgfqpoint{5.691575in}{0.829400in}}%
\pgfpathlineto{\pgfqpoint{5.692356in}{0.784751in}}%
\pgfpathlineto{\pgfqpoint{5.693914in}{0.834596in}}%
\pgfpathlineto{\pgfqpoint{5.694692in}{0.807151in}}%
\pgfpathlineto{\pgfqpoint{5.695469in}{0.837551in}}%
\pgfpathlineto{\pgfqpoint{5.696245in}{0.833444in}}%
\pgfusepath{stroke}%
\end{pgfscope}%
\begin{pgfscope}%
\pgfpathrectangle{\pgfqpoint{0.517836in}{0.420092in}}{\pgfqpoint{5.425000in}{0.770000in}}%
\pgfusepath{clip}%
\pgfsetrectcap%
\pgfsetroundjoin%
\pgfsetlinewidth{0.501875pt}%
\definecolor{currentstroke}{rgb}{0.839216,0.152941,0.156863}%
\pgfsetstrokecolor{currentstroke}%
\pgfsetdash{}{0pt}%
\pgfpathmoveto{\pgfqpoint{1.232459in}{1.200092in}}%
\pgfpathlineto{\pgfqpoint{1.264557in}{1.051914in}}%
\pgfpathlineto{\pgfqpoint{1.557113in}{0.699928in}}%
\pgfpathlineto{\pgfqpoint{1.764686in}{0.586895in}}%
\pgfpathlineto{\pgfqpoint{1.925691in}{0.537343in}}%
\pgfpathlineto{\pgfqpoint{2.057243in}{0.515141in}}%
\pgfpathlineto{\pgfqpoint{2.168468in}{0.506352in}}%
\pgfpathlineto{\pgfqpoint{2.264815in}{0.500266in}}%
\pgfpathlineto{\pgfqpoint{2.349800in}{0.497223in}}%
\pgfpathlineto{\pgfqpoint{2.425821in}{0.495892in}}%
\pgfpathlineto{\pgfqpoint{2.494590in}{0.496257in}}%
\pgfpathlineto{\pgfqpoint{2.615126in}{0.501467in}}%
\pgfpathlineto{\pgfqpoint{2.668597in}{0.507774in}}%
\pgfpathlineto{\pgfqpoint{2.718378in}{0.519784in}}%
\pgfpathlineto{\pgfqpoint{2.764944in}{0.538880in}}%
\pgfpathlineto{\pgfqpoint{2.808687in}{0.581710in}}%
\pgfpathlineto{\pgfqpoint{2.849929in}{0.681909in}}%
\pgfpathlineto{\pgfqpoint{2.888940in}{1.064213in}}%
\pgfpathlineto{\pgfqpoint{2.894024in}{1.200092in}}%
\pgfpathmoveto{\pgfqpoint{2.956856in}{1.200092in}}%
\pgfpathlineto{\pgfqpoint{2.961154in}{1.081443in}}%
\pgfpathlineto{\pgfqpoint{2.994720in}{0.784069in}}%
\pgfpathlineto{\pgfqpoint{3.026793in}{0.674330in}}%
\pgfpathlineto{\pgfqpoint{3.057501in}{0.622472in}}%
\pgfpathlineto{\pgfqpoint{3.086956in}{0.605554in}}%
\pgfpathlineto{\pgfqpoint{3.115255in}{0.617544in}}%
\pgfpathlineto{\pgfqpoint{3.142486in}{0.645384in}}%
\pgfpathlineto{\pgfqpoint{3.168726in}{0.714111in}}%
\pgfpathlineto{\pgfqpoint{3.194046in}{0.901884in}}%
\pgfpathlineto{\pgfqpoint{3.212648in}{1.200092in}}%
\pgfpathmoveto{\pgfqpoint{3.236868in}{1.200092in}}%
\pgfpathlineto{\pgfqpoint{3.242166in}{1.172991in}}%
\pgfpathlineto{\pgfqpoint{3.265074in}{0.899292in}}%
\pgfpathlineto{\pgfqpoint{3.287276in}{0.750882in}}%
\pgfpathlineto{\pgfqpoint{3.308816in}{0.690154in}}%
\pgfpathlineto{\pgfqpoint{3.329732in}{0.675698in}}%
\pgfpathlineto{\pgfqpoint{3.350058in}{0.696236in}}%
\pgfpathlineto{\pgfqpoint{3.369827in}{0.740259in}}%
\pgfpathlineto{\pgfqpoint{3.426079in}{1.069928in}}%
\pgfpathlineto{\pgfqpoint{3.443896in}{1.154286in}}%
\pgfpathlineto{\pgfqpoint{3.461283in}{1.052526in}}%
\pgfpathlineto{\pgfqpoint{3.478261in}{0.922624in}}%
\pgfpathlineto{\pgfqpoint{3.494849in}{0.822966in}}%
\pgfpathlineto{\pgfqpoint{3.511064in}{0.789527in}}%
\pgfpathlineto{\pgfqpoint{3.526922in}{0.784428in}}%
\pgfpathlineto{\pgfqpoint{3.542440in}{0.817874in}}%
\pgfpathlineto{\pgfqpoint{3.557631in}{0.854937in}}%
\pgfpathlineto{\pgfqpoint{3.572508in}{0.907889in}}%
\pgfpathlineto{\pgfqpoint{3.587085in}{0.991816in}}%
\pgfpathlineto{\pgfqpoint{3.601373in}{1.007237in}}%
\pgfpathlineto{\pgfqpoint{3.615384in}{1.002496in}}%
\pgfpathlineto{\pgfqpoint{3.642615in}{0.846189in}}%
\pgfpathlineto{\pgfqpoint{3.655855in}{0.805508in}}%
\pgfpathlineto{\pgfqpoint{3.668855in}{0.781366in}}%
\pgfpathlineto{\pgfqpoint{3.681626in}{0.772636in}}%
\pgfpathlineto{\pgfqpoint{3.694175in}{0.802491in}}%
\pgfpathlineto{\pgfqpoint{3.706509in}{0.848359in}}%
\pgfpathlineto{\pgfqpoint{3.718636in}{0.856740in}}%
\pgfpathlineto{\pgfqpoint{3.730563in}{0.890877in}}%
\pgfpathlineto{\pgfqpoint{3.742295in}{0.952726in}}%
\pgfpathlineto{\pgfqpoint{3.765203in}{0.873265in}}%
\pgfpathlineto{\pgfqpoint{3.776390in}{0.881923in}}%
\pgfpathlineto{\pgfqpoint{3.787406in}{0.894890in}}%
\pgfpathlineto{\pgfqpoint{3.798256in}{0.882473in}}%
\pgfpathlineto{\pgfqpoint{3.808946in}{0.856874in}}%
\pgfpathlineto{\pgfqpoint{3.819479in}{0.869281in}}%
\pgfpathlineto{\pgfqpoint{3.829861in}{0.889157in}}%
\pgfpathlineto{\pgfqpoint{3.840096in}{0.913543in}}%
\pgfpathlineto{\pgfqpoint{3.850187in}{0.897214in}}%
\pgfpathlineto{\pgfqpoint{3.860140in}{0.888240in}}%
\pgfpathlineto{\pgfqpoint{3.869957in}{0.908899in}}%
\pgfpathlineto{\pgfqpoint{3.879642in}{0.906206in}}%
\pgfpathlineto{\pgfqpoint{3.889199in}{0.912194in}}%
\pgfpathlineto{\pgfqpoint{3.898631in}{0.883876in}}%
\pgfpathlineto{\pgfqpoint{3.917133in}{0.873064in}}%
\pgfpathlineto{\pgfqpoint{3.926209in}{0.871310in}}%
\pgfpathlineto{\pgfqpoint{3.935172in}{0.845685in}}%
\pgfpathlineto{\pgfqpoint{3.944025in}{0.900701in}}%
\pgfpathlineto{\pgfqpoint{3.952771in}{0.908744in}}%
\pgfpathlineto{\pgfqpoint{3.961412in}{0.881496in}}%
\pgfpathlineto{\pgfqpoint{3.969951in}{0.870727in}}%
\pgfpathlineto{\pgfqpoint{3.986732in}{0.885615in}}%
\pgfpathlineto{\pgfqpoint{3.994978in}{0.880533in}}%
\pgfpathlineto{\pgfqpoint{4.003131in}{0.888087in}}%
\pgfpathlineto{\pgfqpoint{4.011193in}{0.897945in}}%
\pgfpathlineto{\pgfqpoint{4.019166in}{0.866144in}}%
\pgfpathlineto{\pgfqpoint{4.027052in}{0.889396in}}%
\pgfpathlineto{\pgfqpoint{4.034852in}{0.905288in}}%
\pgfpathlineto{\pgfqpoint{4.042569in}{0.887835in}}%
\pgfpathlineto{\pgfqpoint{4.050204in}{0.874669in}}%
\pgfpathlineto{\pgfqpoint{4.057760in}{0.883379in}}%
\pgfpathlineto{\pgfqpoint{4.065237in}{0.866729in}}%
\pgfpathlineto{\pgfqpoint{4.072637in}{0.879102in}}%
\pgfpathlineto{\pgfqpoint{4.079963in}{0.906983in}}%
\pgfpathlineto{\pgfqpoint{4.087214in}{0.898546in}}%
\pgfpathlineto{\pgfqpoint{4.094394in}{0.873119in}}%
\pgfpathlineto{\pgfqpoint{4.101503in}{0.872914in}}%
\pgfpathlineto{\pgfqpoint{4.108542in}{0.896806in}}%
\pgfpathlineto{\pgfqpoint{4.115513in}{0.900888in}}%
\pgfpathlineto{\pgfqpoint{4.122418in}{0.892484in}}%
\pgfpathlineto{\pgfqpoint{4.129257in}{0.880768in}}%
\pgfpathlineto{\pgfqpoint{4.136032in}{0.892585in}}%
\pgfpathlineto{\pgfqpoint{4.142744in}{0.878650in}}%
\pgfpathlineto{\pgfqpoint{4.149394in}{0.907040in}}%
\pgfpathlineto{\pgfqpoint{4.155984in}{0.915169in}}%
\pgfpathlineto{\pgfqpoint{4.162514in}{0.878210in}}%
\pgfpathlineto{\pgfqpoint{4.168985in}{0.867412in}}%
\pgfpathlineto{\pgfqpoint{4.175398in}{0.869447in}}%
\pgfpathlineto{\pgfqpoint{4.181756in}{0.936860in}}%
\pgfpathlineto{\pgfqpoint{4.188057in}{0.909046in}}%
\pgfpathlineto{\pgfqpoint{4.194304in}{0.847915in}}%
\pgfpathlineto{\pgfqpoint{4.200498in}{0.869319in}}%
\pgfpathlineto{\pgfqpoint{4.206639in}{0.907299in}}%
\pgfpathlineto{\pgfqpoint{4.212727in}{0.913902in}}%
\pgfpathlineto{\pgfqpoint{4.224753in}{0.874575in}}%
\pgfpathlineto{\pgfqpoint{4.230692in}{0.874297in}}%
\pgfpathlineto{\pgfqpoint{4.236582in}{0.898119in}}%
\pgfpathlineto{\pgfqpoint{4.242424in}{0.897981in}}%
\pgfpathlineto{\pgfqpoint{4.248220in}{0.887100in}}%
\pgfpathlineto{\pgfqpoint{4.253969in}{0.908160in}}%
\pgfpathlineto{\pgfqpoint{4.259673in}{0.860118in}}%
\pgfpathlineto{\pgfqpoint{4.270947in}{0.890174in}}%
\pgfpathlineto{\pgfqpoint{4.276519in}{0.917349in}}%
\pgfpathlineto{\pgfqpoint{4.282048in}{0.874215in}}%
\pgfpathlineto{\pgfqpoint{4.287535in}{0.864322in}}%
\pgfpathlineto{\pgfqpoint{4.292981in}{0.905405in}}%
\pgfpathlineto{\pgfqpoint{4.298385in}{0.904356in}}%
\pgfpathlineto{\pgfqpoint{4.303750in}{0.897652in}}%
\pgfpathlineto{\pgfqpoint{4.309075in}{0.880959in}}%
\pgfpathlineto{\pgfqpoint{4.314361in}{0.884937in}}%
\pgfpathlineto{\pgfqpoint{4.319608in}{0.891621in}}%
\pgfpathlineto{\pgfqpoint{4.324818in}{0.926947in}}%
\pgfpathlineto{\pgfqpoint{4.329990in}{0.905769in}}%
\pgfpathlineto{\pgfqpoint{4.335126in}{0.847899in}}%
\pgfpathlineto{\pgfqpoint{4.340225in}{0.883810in}}%
\pgfpathlineto{\pgfqpoint{4.345289in}{0.899133in}}%
\pgfpathlineto{\pgfqpoint{4.350317in}{0.903212in}}%
\pgfpathlineto{\pgfqpoint{4.355310in}{0.861593in}}%
\pgfpathlineto{\pgfqpoint{4.360269in}{0.888327in}}%
\pgfpathlineto{\pgfqpoint{4.365194in}{0.904088in}}%
\pgfpathlineto{\pgfqpoint{4.370086in}{0.875876in}}%
\pgfpathlineto{\pgfqpoint{4.374945in}{0.914484in}}%
\pgfpathlineto{\pgfqpoint{4.379771in}{0.912013in}}%
\pgfpathlineto{\pgfqpoint{4.384565in}{0.855896in}}%
\pgfpathlineto{\pgfqpoint{4.389328in}{0.906783in}}%
\pgfpathlineto{\pgfqpoint{4.394059in}{0.893785in}}%
\pgfpathlineto{\pgfqpoint{4.398760in}{0.892789in}}%
\pgfpathlineto{\pgfqpoint{4.403430in}{0.889648in}}%
\pgfpathlineto{\pgfqpoint{4.408070in}{0.892871in}}%
\pgfpathlineto{\pgfqpoint{4.412681in}{0.890522in}}%
\pgfpathlineto{\pgfqpoint{4.417262in}{0.920944in}}%
\pgfpathlineto{\pgfqpoint{4.421814in}{0.866372in}}%
\pgfpathlineto{\pgfqpoint{4.426338in}{0.849615in}}%
\pgfpathlineto{\pgfqpoint{4.430833in}{0.947846in}}%
\pgfpathlineto{\pgfqpoint{4.435301in}{0.888129in}}%
\pgfpathlineto{\pgfqpoint{4.439741in}{0.846525in}}%
\pgfpathlineto{\pgfqpoint{4.444154in}{0.876451in}}%
\pgfpathlineto{\pgfqpoint{4.448541in}{0.972674in}}%
\pgfpathlineto{\pgfqpoint{4.457234in}{0.821606in}}%
\pgfpathlineto{\pgfqpoint{4.461542in}{0.874935in}}%
\pgfpathlineto{\pgfqpoint{4.465824in}{0.992607in}}%
\pgfpathlineto{\pgfqpoint{4.470081in}{0.864414in}}%
\pgfpathlineto{\pgfqpoint{4.474312in}{0.829858in}}%
\pgfpathlineto{\pgfqpoint{4.478520in}{0.891889in}}%
\pgfpathlineto{\pgfqpoint{4.482702in}{0.980628in}}%
\pgfpathlineto{\pgfqpoint{4.486861in}{0.876778in}}%
\pgfpathlineto{\pgfqpoint{4.490996in}{0.829482in}}%
\pgfpathlineto{\pgfqpoint{4.495107in}{0.924686in}}%
\pgfpathlineto{\pgfqpoint{4.499195in}{0.960854in}}%
\pgfpathlineto{\pgfqpoint{4.503260in}{0.895543in}}%
\pgfpathlineto{\pgfqpoint{4.507303in}{0.850902in}}%
\pgfpathlineto{\pgfqpoint{4.515320in}{0.946729in}}%
\pgfpathlineto{\pgfqpoint{4.519295in}{0.868786in}}%
\pgfpathlineto{\pgfqpoint{4.523249in}{0.846692in}}%
\pgfpathlineto{\pgfqpoint{4.531092in}{0.917068in}}%
\pgfpathlineto{\pgfqpoint{4.534981in}{0.864071in}}%
\pgfpathlineto{\pgfqpoint{4.538850in}{0.902376in}}%
\pgfpathlineto{\pgfqpoint{4.542698in}{0.868610in}}%
\pgfpathlineto{\pgfqpoint{4.546526in}{0.911867in}}%
\pgfpathlineto{\pgfqpoint{4.550334in}{0.926313in}}%
\pgfpathlineto{\pgfqpoint{4.554121in}{0.888532in}}%
\pgfpathlineto{\pgfqpoint{4.557889in}{0.883737in}}%
\pgfpathlineto{\pgfqpoint{4.561637in}{0.886311in}}%
\pgfpathlineto{\pgfqpoint{4.565366in}{0.933397in}}%
\pgfpathlineto{\pgfqpoint{4.569076in}{0.858157in}}%
\pgfpathlineto{\pgfqpoint{4.572767in}{0.887168in}}%
\pgfpathlineto{\pgfqpoint{4.576438in}{0.899128in}}%
\pgfpathlineto{\pgfqpoint{4.580092in}{0.945718in}}%
\pgfpathlineto{\pgfqpoint{4.583727in}{0.857217in}}%
\pgfpathlineto{\pgfqpoint{4.587344in}{0.892194in}}%
\pgfpathlineto{\pgfqpoint{4.590942in}{1.023134in}}%
\pgfpathlineto{\pgfqpoint{4.594523in}{0.814965in}}%
\pgfpathlineto{\pgfqpoint{4.598086in}{0.815781in}}%
\pgfpathlineto{\pgfqpoint{4.601632in}{0.975830in}}%
\pgfpathlineto{\pgfqpoint{4.605160in}{0.925289in}}%
\pgfpathlineto{\pgfqpoint{4.608671in}{0.824036in}}%
\pgfpathlineto{\pgfqpoint{4.612165in}{0.849513in}}%
\pgfpathlineto{\pgfqpoint{4.615643in}{0.961301in}}%
\pgfpathlineto{\pgfqpoint{4.619103in}{0.880765in}}%
\pgfpathlineto{\pgfqpoint{4.622547in}{0.843056in}}%
\pgfpathlineto{\pgfqpoint{4.625975in}{0.882257in}}%
\pgfpathlineto{\pgfqpoint{4.629387in}{0.939126in}}%
\pgfpathlineto{\pgfqpoint{4.632782in}{0.867245in}}%
\pgfpathlineto{\pgfqpoint{4.639525in}{0.895730in}}%
\pgfpathlineto{\pgfqpoint{4.642874in}{0.957732in}}%
\pgfpathlineto{\pgfqpoint{4.646206in}{0.857832in}}%
\pgfpathlineto{\pgfqpoint{4.649524in}{0.835496in}}%
\pgfpathlineto{\pgfqpoint{4.652826in}{0.969536in}}%
\pgfpathlineto{\pgfqpoint{4.656113in}{0.841000in}}%
\pgfpathlineto{\pgfqpoint{4.659385in}{0.864151in}}%
\pgfpathlineto{\pgfqpoint{4.662643in}{0.895238in}}%
\pgfpathlineto{\pgfqpoint{4.665886in}{0.943166in}}%
\pgfpathlineto{\pgfqpoint{4.669114in}{0.820146in}}%
\pgfpathlineto{\pgfqpoint{4.672328in}{0.834874in}}%
\pgfpathlineto{\pgfqpoint{4.675528in}{0.950411in}}%
\pgfpathlineto{\pgfqpoint{4.681885in}{0.812092in}}%
\pgfpathlineto{\pgfqpoint{4.685043in}{0.913903in}}%
\pgfpathlineto{\pgfqpoint{4.688186in}{0.947862in}}%
\pgfpathlineto{\pgfqpoint{4.691317in}{0.838133in}}%
\pgfpathlineto{\pgfqpoint{4.694434in}{0.861116in}}%
\pgfpathlineto{\pgfqpoint{4.697537in}{0.953152in}}%
\pgfpathlineto{\pgfqpoint{4.700627in}{0.860427in}}%
\pgfpathlineto{\pgfqpoint{4.703704in}{0.839829in}}%
\pgfpathlineto{\pgfqpoint{4.706768in}{0.868859in}}%
\pgfpathlineto{\pgfqpoint{4.709819in}{0.949847in}}%
\pgfpathlineto{\pgfqpoint{4.712857in}{0.809999in}}%
\pgfpathlineto{\pgfqpoint{4.715882in}{0.859251in}}%
\pgfpathlineto{\pgfqpoint{4.718895in}{0.991793in}}%
\pgfpathlineto{\pgfqpoint{4.721895in}{0.841264in}}%
\pgfpathlineto{\pgfqpoint{4.724883in}{0.822764in}}%
\pgfpathlineto{\pgfqpoint{4.727858in}{0.972741in}}%
\pgfpathlineto{\pgfqpoint{4.730821in}{0.899468in}}%
\pgfpathlineto{\pgfqpoint{4.733772in}{0.859561in}}%
\pgfpathlineto{\pgfqpoint{4.736711in}{0.854760in}}%
\pgfpathlineto{\pgfqpoint{4.739638in}{0.961414in}}%
\pgfpathlineto{\pgfqpoint{4.742554in}{0.862708in}}%
\pgfpathlineto{\pgfqpoint{4.745457in}{0.850548in}}%
\pgfpathlineto{\pgfqpoint{4.748349in}{0.901548in}}%
\pgfpathlineto{\pgfqpoint{4.751230in}{0.903811in}}%
\pgfpathlineto{\pgfqpoint{4.754098in}{0.857728in}}%
\pgfpathlineto{\pgfqpoint{4.756956in}{0.876837in}}%
\pgfpathlineto{\pgfqpoint{4.759802in}{0.931814in}}%
\pgfpathlineto{\pgfqpoint{4.762637in}{0.867596in}}%
\pgfpathlineto{\pgfqpoint{4.765461in}{0.846532in}}%
\pgfpathlineto{\pgfqpoint{4.768274in}{0.898904in}}%
\pgfpathlineto{\pgfqpoint{4.771077in}{0.897689in}}%
\pgfpathlineto{\pgfqpoint{4.773868in}{0.848044in}}%
\pgfpathlineto{\pgfqpoint{4.776648in}{0.871395in}}%
\pgfpathlineto{\pgfqpoint{4.779418in}{0.918212in}}%
\pgfpathlineto{\pgfqpoint{4.782177in}{0.853403in}}%
\pgfpathlineto{\pgfqpoint{4.784926in}{0.838357in}}%
\pgfpathlineto{\pgfqpoint{4.787664in}{0.950686in}}%
\pgfpathlineto{\pgfqpoint{4.790392in}{0.827250in}}%
\pgfpathlineto{\pgfqpoint{4.795817in}{0.919815in}}%
\pgfpathlineto{\pgfqpoint{4.798515in}{0.900340in}}%
\pgfpathlineto{\pgfqpoint{4.801202in}{0.845238in}}%
\pgfpathlineto{\pgfqpoint{4.803879in}{0.879475in}}%
\pgfpathlineto{\pgfqpoint{4.806547in}{0.884012in}}%
\pgfpathlineto{\pgfqpoint{4.811852in}{0.880337in}}%
\pgfpathlineto{\pgfqpoint{4.814490in}{0.918357in}}%
\pgfpathlineto{\pgfqpoint{4.817119in}{0.867947in}}%
\pgfpathlineto{\pgfqpoint{4.819738in}{0.894687in}}%
\pgfpathlineto{\pgfqpoint{4.822347in}{0.859889in}}%
\pgfpathlineto{\pgfqpoint{4.824947in}{0.909009in}}%
\pgfpathlineto{\pgfqpoint{4.827538in}{0.866532in}}%
\pgfpathlineto{\pgfqpoint{4.830120in}{0.893446in}}%
\pgfpathlineto{\pgfqpoint{4.832692in}{0.906343in}}%
\pgfpathlineto{\pgfqpoint{4.835255in}{0.846200in}}%
\pgfpathlineto{\pgfqpoint{4.837809in}{0.888649in}}%
\pgfpathlineto{\pgfqpoint{4.840354in}{0.907176in}}%
\pgfpathlineto{\pgfqpoint{4.842891in}{0.845775in}}%
\pgfpathlineto{\pgfqpoint{4.845418in}{0.928393in}}%
\pgfpathlineto{\pgfqpoint{4.847936in}{0.906522in}}%
\pgfpathlineto{\pgfqpoint{4.850446in}{0.859065in}}%
\pgfpathlineto{\pgfqpoint{4.852947in}{0.911043in}}%
\pgfpathlineto{\pgfqpoint{4.855439in}{0.889943in}}%
\pgfpathlineto{\pgfqpoint{4.857923in}{0.893921in}}%
\pgfpathlineto{\pgfqpoint{4.860398in}{0.853155in}}%
\pgfpathlineto{\pgfqpoint{4.862865in}{0.938581in}}%
\pgfpathlineto{\pgfqpoint{4.865323in}{0.827298in}}%
\pgfpathlineto{\pgfqpoint{4.867773in}{0.866662in}}%
\pgfpathlineto{\pgfqpoint{4.870215in}{0.924889in}}%
\pgfpathlineto{\pgfqpoint{4.872649in}{0.822057in}}%
\pgfpathlineto{\pgfqpoint{4.875074in}{0.882985in}}%
\pgfpathlineto{\pgfqpoint{4.877491in}{0.976116in}}%
\pgfpathlineto{\pgfqpoint{4.879900in}{0.798513in}}%
\pgfpathlineto{\pgfqpoint{4.882301in}{0.837811in}}%
\pgfpathlineto{\pgfqpoint{4.884695in}{1.049475in}}%
\pgfpathlineto{\pgfqpoint{4.887080in}{0.759041in}}%
\pgfpathlineto{\pgfqpoint{4.889457in}{0.799512in}}%
\pgfpathlineto{\pgfqpoint{4.891827in}{1.018509in}}%
\pgfpathlineto{\pgfqpoint{4.894189in}{0.845354in}}%
\pgfpathlineto{\pgfqpoint{4.896543in}{0.803044in}}%
\pgfpathlineto{\pgfqpoint{4.901228in}{0.999896in}}%
\pgfpathlineto{\pgfqpoint{4.903559in}{0.810761in}}%
\pgfpathlineto{\pgfqpoint{4.905883in}{0.807113in}}%
\pgfpathlineto{\pgfqpoint{4.908199in}{0.952090in}}%
\pgfpathlineto{\pgfqpoint{4.910508in}{0.932130in}}%
\pgfpathlineto{\pgfqpoint{4.912810in}{0.793948in}}%
\pgfpathlineto{\pgfqpoint{4.915104in}{0.845067in}}%
\pgfpathlineto{\pgfqpoint{4.917391in}{0.974228in}}%
\pgfpathlineto{\pgfqpoint{4.919671in}{0.848674in}}%
\pgfpathlineto{\pgfqpoint{4.921943in}{0.847009in}}%
\pgfpathlineto{\pgfqpoint{4.924209in}{0.882498in}}%
\pgfpathlineto{\pgfqpoint{4.926467in}{0.885085in}}%
\pgfpathlineto{\pgfqpoint{4.928718in}{0.894429in}}%
\pgfpathlineto{\pgfqpoint{4.930963in}{0.849267in}}%
\pgfpathlineto{\pgfqpoint{4.933200in}{0.902628in}}%
\pgfpathlineto{\pgfqpoint{4.935430in}{0.918078in}}%
\pgfpathlineto{\pgfqpoint{4.937654in}{0.896393in}}%
\pgfpathlineto{\pgfqpoint{4.939871in}{0.891850in}}%
\pgfpathlineto{\pgfqpoint{4.942081in}{0.873963in}}%
\pgfpathlineto{\pgfqpoint{4.944284in}{0.830835in}}%
\pgfpathlineto{\pgfqpoint{4.946480in}{0.878158in}}%
\pgfpathlineto{\pgfqpoint{4.948670in}{0.876099in}}%
\pgfpathlineto{\pgfqpoint{4.950853in}{0.883262in}}%
\pgfpathlineto{\pgfqpoint{4.953030in}{0.910383in}}%
\pgfpathlineto{\pgfqpoint{4.955200in}{0.890814in}}%
\pgfpathlineto{\pgfqpoint{4.957363in}{0.857191in}}%
\pgfpathlineto{\pgfqpoint{4.959520in}{0.867398in}}%
\pgfpathlineto{\pgfqpoint{4.961671in}{0.867244in}}%
\pgfpathlineto{\pgfqpoint{4.963815in}{0.872755in}}%
\pgfpathlineto{\pgfqpoint{4.965953in}{0.890327in}}%
\pgfpathlineto{\pgfqpoint{4.968085in}{0.873900in}}%
\pgfpathlineto{\pgfqpoint{4.970210in}{0.947917in}}%
\pgfpathlineto{\pgfqpoint{4.972329in}{0.848055in}}%
\pgfpathlineto{\pgfqpoint{4.974442in}{0.803492in}}%
\pgfpathlineto{\pgfqpoint{4.976548in}{0.935368in}}%
\pgfpathlineto{\pgfqpoint{4.980743in}{0.806693in}}%
\pgfpathlineto{\pgfqpoint{4.984914in}{0.969206in}}%
\pgfpathlineto{\pgfqpoint{4.986990in}{0.782996in}}%
\pgfpathlineto{\pgfqpoint{4.991125in}{0.937606in}}%
\pgfpathlineto{\pgfqpoint{4.993184in}{0.839413in}}%
\pgfpathlineto{\pgfqpoint{4.995237in}{0.837621in}}%
\pgfpathlineto{\pgfqpoint{4.997284in}{0.949794in}}%
\pgfpathlineto{\pgfqpoint{4.999325in}{0.852487in}}%
\pgfpathlineto{\pgfqpoint{5.001360in}{0.814555in}}%
\pgfpathlineto{\pgfqpoint{5.003390in}{0.877544in}}%
\pgfpathlineto{\pgfqpoint{5.005414in}{0.893772in}}%
\pgfpathlineto{\pgfqpoint{5.007432in}{0.822396in}}%
\pgfpathlineto{\pgfqpoint{5.009445in}{0.878154in}}%
\pgfpathlineto{\pgfqpoint{5.011452in}{0.974306in}}%
\pgfpathlineto{\pgfqpoint{5.013453in}{0.822239in}}%
\pgfpathlineto{\pgfqpoint{5.015449in}{0.828388in}}%
\pgfpathlineto{\pgfqpoint{5.017439in}{0.953207in}}%
\pgfpathlineto{\pgfqpoint{5.021404in}{0.787511in}}%
\pgfpathlineto{\pgfqpoint{5.023378in}{0.900831in}}%
\pgfpathlineto{\pgfqpoint{5.025347in}{0.901974in}}%
\pgfpathlineto{\pgfqpoint{5.027310in}{0.824546in}}%
\pgfpathlineto{\pgfqpoint{5.029268in}{0.882059in}}%
\pgfpathlineto{\pgfqpoint{5.031221in}{0.895831in}}%
\pgfpathlineto{\pgfqpoint{5.035111in}{0.834346in}}%
\pgfpathlineto{\pgfqpoint{5.037048in}{0.901803in}}%
\pgfpathlineto{\pgfqpoint{5.040906in}{0.825597in}}%
\pgfpathlineto{\pgfqpoint{5.042828in}{0.957291in}}%
\pgfpathlineto{\pgfqpoint{5.044744in}{0.857624in}}%
\pgfpathlineto{\pgfqpoint{5.046655in}{0.834635in}}%
\pgfpathlineto{\pgfqpoint{5.048562in}{0.912674in}}%
\pgfpathlineto{\pgfqpoint{5.050463in}{0.870932in}}%
\pgfpathlineto{\pgfqpoint{5.052359in}{0.858241in}}%
\pgfpathlineto{\pgfqpoint{5.054251in}{0.892649in}}%
\pgfpathlineto{\pgfqpoint{5.056137in}{0.832985in}}%
\pgfpathlineto{\pgfqpoint{5.058018in}{0.909949in}}%
\pgfpathlineto{\pgfqpoint{5.061767in}{0.841493in}}%
\pgfpathlineto{\pgfqpoint{5.063633in}{0.894183in}}%
\pgfpathlineto{\pgfqpoint{5.065495in}{0.886283in}}%
\pgfpathlineto{\pgfqpoint{5.067353in}{0.823326in}}%
\pgfpathlineto{\pgfqpoint{5.069205in}{0.873093in}}%
\pgfpathlineto{\pgfqpoint{5.071053in}{0.886090in}}%
\pgfpathlineto{\pgfqpoint{5.072896in}{0.861442in}}%
\pgfpathlineto{\pgfqpoint{5.074734in}{0.863160in}}%
\pgfpathlineto{\pgfqpoint{5.076568in}{0.862305in}}%
\pgfpathlineto{\pgfqpoint{5.078397in}{0.844262in}}%
\pgfpathlineto{\pgfqpoint{5.082041in}{0.881691in}}%
\pgfpathlineto{\pgfqpoint{5.083856in}{0.821503in}}%
\pgfpathlineto{\pgfqpoint{5.085667in}{0.912147in}}%
\pgfpathlineto{\pgfqpoint{5.087473in}{0.838725in}}%
\pgfpathlineto{\pgfqpoint{5.089274in}{0.858340in}}%
\pgfpathlineto{\pgfqpoint{5.091071in}{0.840589in}}%
\pgfpathlineto{\pgfqpoint{5.092864in}{0.886780in}}%
\pgfpathlineto{\pgfqpoint{5.096436in}{0.840634in}}%
\pgfpathlineto{\pgfqpoint{5.098215in}{0.938991in}}%
\pgfpathlineto{\pgfqpoint{5.099990in}{0.811420in}}%
\pgfpathlineto{\pgfqpoint{5.101761in}{0.839426in}}%
\pgfpathlineto{\pgfqpoint{5.103527in}{0.928174in}}%
\pgfpathlineto{\pgfqpoint{5.105289in}{0.796748in}}%
\pgfpathlineto{\pgfqpoint{5.108800in}{0.937478in}}%
\pgfpathlineto{\pgfqpoint{5.110550in}{0.805284in}}%
\pgfpathlineto{\pgfqpoint{5.112295in}{0.835312in}}%
\pgfpathlineto{\pgfqpoint{5.114035in}{0.984033in}}%
\pgfpathlineto{\pgfqpoint{5.115772in}{0.780431in}}%
\pgfpathlineto{\pgfqpoint{5.117504in}{0.817462in}}%
\pgfpathlineto{\pgfqpoint{5.119232in}{0.980971in}}%
\pgfpathlineto{\pgfqpoint{5.120957in}{0.774073in}}%
\pgfpathlineto{\pgfqpoint{5.122677in}{0.798638in}}%
\pgfpathlineto{\pgfqpoint{5.124392in}{0.920304in}}%
\pgfpathlineto{\pgfqpoint{5.126104in}{0.937648in}}%
\pgfpathlineto{\pgfqpoint{5.127812in}{0.785234in}}%
\pgfpathlineto{\pgfqpoint{5.129516in}{0.829702in}}%
\pgfpathlineto{\pgfqpoint{5.131215in}{0.989150in}}%
\pgfpathlineto{\pgfqpoint{5.132911in}{0.830971in}}%
\pgfpathlineto{\pgfqpoint{5.134603in}{0.825043in}}%
\pgfpathlineto{\pgfqpoint{5.137975in}{0.874993in}}%
\pgfpathlineto{\pgfqpoint{5.139655in}{0.874704in}}%
\pgfpathlineto{\pgfqpoint{5.141331in}{0.814875in}}%
\pgfpathlineto{\pgfqpoint{5.143003in}{0.884497in}}%
\pgfpathlineto{\pgfqpoint{5.144671in}{0.892967in}}%
\pgfpathlineto{\pgfqpoint{5.147996in}{0.822706in}}%
\pgfpathlineto{\pgfqpoint{5.151306in}{0.908962in}}%
\pgfpathlineto{\pgfqpoint{5.152955in}{0.826577in}}%
\pgfpathlineto{\pgfqpoint{5.154601in}{0.836188in}}%
\pgfpathlineto{\pgfqpoint{5.156242in}{0.911370in}}%
\pgfpathlineto{\pgfqpoint{5.159515in}{0.802388in}}%
\pgfpathlineto{\pgfqpoint{5.161145in}{0.894377in}}%
\pgfpathlineto{\pgfqpoint{5.162772in}{0.891440in}}%
\pgfpathlineto{\pgfqpoint{5.164395in}{0.841408in}}%
\pgfpathlineto{\pgfqpoint{5.166015in}{0.875045in}}%
\pgfpathlineto{\pgfqpoint{5.167631in}{0.875155in}}%
\pgfpathlineto{\pgfqpoint{5.169243in}{0.852224in}}%
\pgfpathlineto{\pgfqpoint{5.170852in}{0.916597in}}%
\pgfpathlineto{\pgfqpoint{5.172457in}{0.883647in}}%
\pgfpathlineto{\pgfqpoint{5.174059in}{0.800465in}}%
\pgfpathlineto{\pgfqpoint{5.175657in}{0.915275in}}%
\pgfpathlineto{\pgfqpoint{5.178843in}{0.818823in}}%
\pgfpathlineto{\pgfqpoint{5.180430in}{0.854937in}}%
\pgfpathlineto{\pgfqpoint{5.182014in}{0.932954in}}%
\pgfpathlineto{\pgfqpoint{5.183595in}{0.808093in}}%
\pgfpathlineto{\pgfqpoint{5.185172in}{0.835254in}}%
\pgfpathlineto{\pgfqpoint{5.186746in}{0.920348in}}%
\pgfpathlineto{\pgfqpoint{5.188316in}{0.828778in}}%
\pgfpathlineto{\pgfqpoint{5.189883in}{0.839478in}}%
\pgfpathlineto{\pgfqpoint{5.191446in}{0.889276in}}%
\pgfpathlineto{\pgfqpoint{5.193006in}{0.901796in}}%
\pgfpathlineto{\pgfqpoint{5.194563in}{0.822110in}}%
\pgfpathlineto{\pgfqpoint{5.196116in}{0.831294in}}%
\pgfpathlineto{\pgfqpoint{5.197666in}{0.912743in}}%
\pgfpathlineto{\pgfqpoint{5.200756in}{0.838779in}}%
\pgfpathlineto{\pgfqpoint{5.202296in}{0.889045in}}%
\pgfpathlineto{\pgfqpoint{5.203833in}{0.898122in}}%
\pgfpathlineto{\pgfqpoint{5.206897in}{0.806290in}}%
\pgfpathlineto{\pgfqpoint{5.209948in}{0.918845in}}%
\pgfpathlineto{\pgfqpoint{5.211469in}{0.815955in}}%
\pgfpathlineto{\pgfqpoint{5.214500in}{0.920058in}}%
\pgfpathlineto{\pgfqpoint{5.216011in}{0.889591in}}%
\pgfpathlineto{\pgfqpoint{5.217519in}{0.759862in}}%
\pgfpathlineto{\pgfqpoint{5.219024in}{0.907151in}}%
\pgfpathlineto{\pgfqpoint{5.220526in}{0.921160in}}%
\pgfpathlineto{\pgfqpoint{5.222024in}{0.816120in}}%
\pgfpathlineto{\pgfqpoint{5.223520in}{0.810991in}}%
\pgfpathlineto{\pgfqpoint{5.225012in}{0.949939in}}%
\pgfpathlineto{\pgfqpoint{5.226501in}{0.814844in}}%
\pgfpathlineto{\pgfqpoint{5.227987in}{0.795387in}}%
\pgfpathlineto{\pgfqpoint{5.229470in}{0.897362in}}%
\pgfpathlineto{\pgfqpoint{5.230950in}{0.916068in}}%
\pgfpathlineto{\pgfqpoint{5.232427in}{0.773216in}}%
\pgfpathlineto{\pgfqpoint{5.235373in}{0.947985in}}%
\pgfpathlineto{\pgfqpoint{5.236841in}{0.827552in}}%
\pgfpathlineto{\pgfqpoint{5.238306in}{0.809664in}}%
\pgfpathlineto{\pgfqpoint{5.239768in}{0.925871in}}%
\pgfpathlineto{\pgfqpoint{5.241227in}{0.838171in}}%
\pgfpathlineto{\pgfqpoint{5.242683in}{0.874113in}}%
\pgfpathlineto{\pgfqpoint{5.244136in}{0.839701in}}%
\pgfpathlineto{\pgfqpoint{5.245587in}{0.865016in}}%
\pgfpathlineto{\pgfqpoint{5.247034in}{0.868183in}}%
\pgfpathlineto{\pgfqpoint{5.248478in}{0.843143in}}%
\pgfpathlineto{\pgfqpoint{5.249920in}{0.868220in}}%
\pgfpathlineto{\pgfqpoint{5.251359in}{0.842463in}}%
\pgfpathlineto{\pgfqpoint{5.252795in}{0.871222in}}%
\pgfpathlineto{\pgfqpoint{5.254228in}{0.852582in}}%
\pgfpathlineto{\pgfqpoint{5.255658in}{0.861905in}}%
\pgfpathlineto{\pgfqpoint{5.257085in}{0.891625in}}%
\pgfpathlineto{\pgfqpoint{5.259932in}{0.842978in}}%
\pgfpathlineto{\pgfqpoint{5.261351in}{0.906063in}}%
\pgfpathlineto{\pgfqpoint{5.262767in}{0.812468in}}%
\pgfpathlineto{\pgfqpoint{5.264180in}{0.816909in}}%
\pgfpathlineto{\pgfqpoint{5.265591in}{0.942223in}}%
\pgfpathlineto{\pgfqpoint{5.268404in}{0.781669in}}%
\pgfpathlineto{\pgfqpoint{5.271206in}{0.950220in}}%
\pgfpathlineto{\pgfqpoint{5.272603in}{0.813039in}}%
\pgfpathlineto{\pgfqpoint{5.273997in}{0.788638in}}%
\pgfpathlineto{\pgfqpoint{5.275389in}{0.983615in}}%
\pgfpathlineto{\pgfqpoint{5.278164in}{0.796730in}}%
\pgfpathlineto{\pgfqpoint{5.280928in}{0.900414in}}%
\pgfpathlineto{\pgfqpoint{5.283682in}{0.822836in}}%
\pgfpathlineto{\pgfqpoint{5.285055in}{0.845280in}}%
\pgfpathlineto{\pgfqpoint{5.286426in}{0.956482in}}%
\pgfpathlineto{\pgfqpoint{5.287794in}{0.825803in}}%
\pgfpathlineto{\pgfqpoint{5.289159in}{0.823911in}}%
\pgfpathlineto{\pgfqpoint{5.291882in}{0.923900in}}%
\pgfpathlineto{\pgfqpoint{5.293239in}{0.806887in}}%
\pgfpathlineto{\pgfqpoint{5.294594in}{0.849768in}}%
\pgfpathlineto{\pgfqpoint{5.295947in}{0.930256in}}%
\pgfpathlineto{\pgfqpoint{5.298644in}{0.808220in}}%
\pgfpathlineto{\pgfqpoint{5.301331in}{0.924589in}}%
\pgfpathlineto{\pgfqpoint{5.304008in}{0.810429in}}%
\pgfpathlineto{\pgfqpoint{5.305343in}{0.931891in}}%
\pgfpathlineto{\pgfqpoint{5.308006in}{0.830645in}}%
\pgfpathlineto{\pgfqpoint{5.309333in}{0.848155in}}%
\pgfpathlineto{\pgfqpoint{5.310659in}{0.899756in}}%
\pgfpathlineto{\pgfqpoint{5.313302in}{0.817814in}}%
\pgfpathlineto{\pgfqpoint{5.314619in}{0.905508in}}%
\pgfpathlineto{\pgfqpoint{5.317248in}{0.842028in}}%
\pgfpathlineto{\pgfqpoint{5.318559in}{0.910135in}}%
\pgfpathlineto{\pgfqpoint{5.319867in}{0.894671in}}%
\pgfpathlineto{\pgfqpoint{5.321173in}{0.840634in}}%
\pgfpathlineto{\pgfqpoint{5.322477in}{0.887175in}}%
\pgfpathlineto{\pgfqpoint{5.323778in}{0.860796in}}%
\pgfpathlineto{\pgfqpoint{5.325077in}{0.862259in}}%
\pgfpathlineto{\pgfqpoint{5.326373in}{0.891165in}}%
\pgfpathlineto{\pgfqpoint{5.328959in}{0.812527in}}%
\pgfpathlineto{\pgfqpoint{5.330249in}{0.891359in}}%
\pgfpathlineto{\pgfqpoint{5.331536in}{0.884878in}}%
\pgfpathlineto{\pgfqpoint{5.332821in}{0.839409in}}%
\pgfpathlineto{\pgfqpoint{5.334104in}{0.849246in}}%
\pgfpathlineto{\pgfqpoint{5.335384in}{0.918669in}}%
\pgfpathlineto{\pgfqpoint{5.337939in}{0.806223in}}%
\pgfpathlineto{\pgfqpoint{5.340484in}{0.888317in}}%
\pgfpathlineto{\pgfqpoint{5.341753in}{0.824127in}}%
\pgfpathlineto{\pgfqpoint{5.343020in}{0.870803in}}%
\pgfpathlineto{\pgfqpoint{5.344285in}{0.966541in}}%
\pgfpathlineto{\pgfqpoint{5.345547in}{0.827858in}}%
\pgfpathlineto{\pgfqpoint{5.346807in}{0.818324in}}%
\pgfpathlineto{\pgfqpoint{5.348065in}{0.960010in}}%
\pgfpathlineto{\pgfqpoint{5.349321in}{0.802514in}}%
\pgfpathlineto{\pgfqpoint{5.350575in}{0.790175in}}%
\pgfpathlineto{\pgfqpoint{5.351827in}{0.914478in}}%
\pgfpathlineto{\pgfqpoint{5.354323in}{0.808024in}}%
\pgfpathlineto{\pgfqpoint{5.355569in}{0.898274in}}%
\pgfpathlineto{\pgfqpoint{5.356811in}{0.902672in}}%
\pgfpathlineto{\pgfqpoint{5.359291in}{0.836531in}}%
\pgfpathlineto{\pgfqpoint{5.360528in}{0.908547in}}%
\pgfpathlineto{\pgfqpoint{5.361762in}{0.905034in}}%
\pgfpathlineto{\pgfqpoint{5.362994in}{0.816784in}}%
\pgfpathlineto{\pgfqpoint{5.365453in}{0.896881in}}%
\pgfpathlineto{\pgfqpoint{5.366679in}{0.820572in}}%
\pgfpathlineto{\pgfqpoint{5.367903in}{0.832191in}}%
\pgfpathlineto{\pgfqpoint{5.369125in}{0.908259in}}%
\pgfpathlineto{\pgfqpoint{5.370344in}{0.831434in}}%
\pgfpathlineto{\pgfqpoint{5.372778in}{0.881355in}}%
\pgfpathlineto{\pgfqpoint{5.373992in}{0.863779in}}%
\pgfpathlineto{\pgfqpoint{5.375203in}{0.808714in}}%
\pgfpathlineto{\pgfqpoint{5.377621in}{0.884938in}}%
\pgfpathlineto{\pgfqpoint{5.378826in}{0.836333in}}%
\pgfpathlineto{\pgfqpoint{5.380030in}{0.841990in}}%
\pgfpathlineto{\pgfqpoint{5.381231in}{0.910820in}}%
\pgfpathlineto{\pgfqpoint{5.382431in}{0.827882in}}%
\pgfpathlineto{\pgfqpoint{5.383628in}{0.889208in}}%
\pgfpathlineto{\pgfqpoint{5.384824in}{0.848678in}}%
\pgfpathlineto{\pgfqpoint{5.386018in}{0.895478in}}%
\pgfpathlineto{\pgfqpoint{5.388399in}{0.828572in}}%
\pgfpathlineto{\pgfqpoint{5.389587in}{0.873422in}}%
\pgfpathlineto{\pgfqpoint{5.390772in}{0.847842in}}%
\pgfpathlineto{\pgfqpoint{5.391956in}{0.850816in}}%
\pgfpathlineto{\pgfqpoint{5.393138in}{0.906546in}}%
\pgfpathlineto{\pgfqpoint{5.394318in}{0.834099in}}%
\pgfpathlineto{\pgfqpoint{5.395496in}{0.863791in}}%
\pgfpathlineto{\pgfqpoint{5.396672in}{0.963264in}}%
\pgfpathlineto{\pgfqpoint{5.397846in}{0.809280in}}%
\pgfpathlineto{\pgfqpoint{5.399018in}{0.818397in}}%
\pgfpathlineto{\pgfqpoint{5.400189in}{0.960545in}}%
\pgfpathlineto{\pgfqpoint{5.401357in}{0.790538in}}%
\pgfpathlineto{\pgfqpoint{5.402524in}{0.800239in}}%
\pgfpathlineto{\pgfqpoint{5.403689in}{0.957658in}}%
\pgfpathlineto{\pgfqpoint{5.406012in}{0.841764in}}%
\pgfpathlineto{\pgfqpoint{5.407171in}{0.858494in}}%
\pgfpathlineto{\pgfqpoint{5.408329in}{0.910680in}}%
\pgfpathlineto{\pgfqpoint{5.409484in}{0.821862in}}%
\pgfpathlineto{\pgfqpoint{5.412939in}{0.895255in}}%
\pgfpathlineto{\pgfqpoint{5.414087in}{0.822623in}}%
\pgfpathlineto{\pgfqpoint{5.415233in}{0.826190in}}%
\pgfpathlineto{\pgfqpoint{5.416378in}{0.927877in}}%
\pgfpathlineto{\pgfqpoint{5.417520in}{0.802679in}}%
\pgfpathlineto{\pgfqpoint{5.418661in}{0.840512in}}%
\pgfpathlineto{\pgfqpoint{5.419800in}{0.958837in}}%
\pgfpathlineto{\pgfqpoint{5.420937in}{0.835506in}}%
\pgfpathlineto{\pgfqpoint{5.422073in}{0.847083in}}%
\pgfpathlineto{\pgfqpoint{5.424338in}{0.904925in}}%
\pgfpathlineto{\pgfqpoint{5.425468in}{0.826855in}}%
\pgfpathlineto{\pgfqpoint{5.427723in}{0.884673in}}%
\pgfpathlineto{\pgfqpoint{5.429971in}{0.840961in}}%
\pgfpathlineto{\pgfqpoint{5.431092in}{0.834318in}}%
\pgfpathlineto{\pgfqpoint{5.433329in}{0.905599in}}%
\pgfpathlineto{\pgfqpoint{5.434445in}{0.823487in}}%
\pgfpathlineto{\pgfqpoint{5.435560in}{0.853543in}}%
\pgfpathlineto{\pgfqpoint{5.436672in}{0.916149in}}%
\pgfpathlineto{\pgfqpoint{5.438892in}{0.797280in}}%
\pgfpathlineto{\pgfqpoint{5.440000in}{0.963934in}}%
\pgfpathlineto{\pgfqpoint{5.442210in}{0.779032in}}%
\pgfpathlineto{\pgfqpoint{5.444413in}{0.974826in}}%
\pgfpathlineto{\pgfqpoint{5.446609in}{0.792029in}}%
\pgfpathlineto{\pgfqpoint{5.447705in}{0.932569in}}%
\pgfpathlineto{\pgfqpoint{5.450982in}{0.829854in}}%
\pgfpathlineto{\pgfqpoint{5.453159in}{0.924119in}}%
\pgfpathlineto{\pgfqpoint{5.454245in}{0.808794in}}%
\pgfpathlineto{\pgfqpoint{5.456412in}{0.888557in}}%
\pgfpathlineto{\pgfqpoint{5.457492in}{0.884657in}}%
\pgfpathlineto{\pgfqpoint{5.458572in}{0.819934in}}%
\pgfpathlineto{\pgfqpoint{5.459650in}{0.885909in}}%
\pgfpathlineto{\pgfqpoint{5.461800in}{0.842250in}}%
\pgfpathlineto{\pgfqpoint{5.462873in}{0.886878in}}%
\pgfpathlineto{\pgfqpoint{5.463944in}{0.847487in}}%
\pgfpathlineto{\pgfqpoint{5.465014in}{0.847586in}}%
\pgfpathlineto{\pgfqpoint{5.466082in}{0.886126in}}%
\pgfpathlineto{\pgfqpoint{5.468214in}{0.846773in}}%
\pgfpathlineto{\pgfqpoint{5.470339in}{0.883675in}}%
\pgfpathlineto{\pgfqpoint{5.471399in}{0.846924in}}%
\pgfpathlineto{\pgfqpoint{5.472458in}{0.866084in}}%
\pgfpathlineto{\pgfqpoint{5.473515in}{0.859933in}}%
\pgfpathlineto{\pgfqpoint{5.474571in}{0.811891in}}%
\pgfpathlineto{\pgfqpoint{5.475625in}{0.848239in}}%
\pgfpathlineto{\pgfqpoint{5.476678in}{0.923161in}}%
\pgfpathlineto{\pgfqpoint{5.478778in}{0.819333in}}%
\pgfpathlineto{\pgfqpoint{5.479826in}{1.015157in}}%
\pgfpathlineto{\pgfqpoint{5.481918in}{0.784022in}}%
\pgfpathlineto{\pgfqpoint{5.484003in}{0.921309in}}%
\pgfpathlineto{\pgfqpoint{5.485043in}{0.802219in}}%
\pgfpathlineto{\pgfqpoint{5.486082in}{0.817794in}}%
\pgfpathlineto{\pgfqpoint{5.487120in}{0.922043in}}%
\pgfpathlineto{\pgfqpoint{5.488156in}{0.912493in}}%
\pgfpathlineto{\pgfqpoint{5.489190in}{0.803945in}}%
\pgfpathlineto{\pgfqpoint{5.490223in}{0.820160in}}%
\pgfpathlineto{\pgfqpoint{5.491255in}{0.950230in}}%
\pgfpathlineto{\pgfqpoint{5.493313in}{0.805812in}}%
\pgfpathlineto{\pgfqpoint{5.495366in}{0.942341in}}%
\pgfpathlineto{\pgfqpoint{5.497413in}{0.776081in}}%
\pgfpathlineto{\pgfqpoint{5.499454in}{0.945295in}}%
\pgfpathlineto{\pgfqpoint{5.500472in}{0.825482in}}%
\pgfpathlineto{\pgfqpoint{5.501489in}{0.834792in}}%
\pgfpathlineto{\pgfqpoint{5.503519in}{0.937551in}}%
\pgfpathlineto{\pgfqpoint{5.505543in}{0.811773in}}%
\pgfpathlineto{\pgfqpoint{5.507561in}{0.872644in}}%
\pgfpathlineto{\pgfqpoint{5.508568in}{0.847623in}}%
\pgfpathlineto{\pgfqpoint{5.509574in}{0.857954in}}%
\pgfpathlineto{\pgfqpoint{5.510578in}{0.880664in}}%
\pgfpathlineto{\pgfqpoint{5.512582in}{0.854303in}}%
\pgfpathlineto{\pgfqpoint{5.513582in}{0.903006in}}%
\pgfpathlineto{\pgfqpoint{5.515578in}{0.824476in}}%
\pgfpathlineto{\pgfqpoint{5.516574in}{0.940530in}}%
\pgfpathlineto{\pgfqpoint{5.518562in}{0.803660in}}%
\pgfpathlineto{\pgfqpoint{5.520544in}{0.932852in}}%
\pgfpathlineto{\pgfqpoint{5.521533in}{0.803902in}}%
\pgfpathlineto{\pgfqpoint{5.522521in}{0.821393in}}%
\pgfpathlineto{\pgfqpoint{5.523507in}{0.942792in}}%
\pgfpathlineto{\pgfqpoint{5.525476in}{0.803748in}}%
\pgfpathlineto{\pgfqpoint{5.527439in}{0.897330in}}%
\pgfpathlineto{\pgfqpoint{5.529397in}{0.840069in}}%
\pgfpathlineto{\pgfqpoint{5.530374in}{0.911749in}}%
\pgfpathlineto{\pgfqpoint{5.533298in}{0.823457in}}%
\pgfpathlineto{\pgfqpoint{5.534269in}{0.907623in}}%
\pgfpathlineto{\pgfqpoint{5.535240in}{0.846087in}}%
\pgfpathlineto{\pgfqpoint{5.537177in}{0.901100in}}%
\pgfpathlineto{\pgfqpoint{5.537177in}{0.901100in}}%
\pgfusepath{stroke}%
\end{pgfscope}%
\begin{pgfscope}%
\pgfpathrectangle{\pgfqpoint{0.517836in}{0.420092in}}{\pgfqpoint{5.425000in}{0.770000in}}%
\pgfusepath{clip}%
\pgfsetrectcap%
\pgfsetroundjoin%
\pgfsetlinewidth{0.501875pt}%
\definecolor{currentstroke}{rgb}{0.580392,0.403922,0.741176}%
\pgfsetstrokecolor{currentstroke}%
\pgfsetdash{}{0pt}%
\pgfpathmoveto{\pgfqpoint{0.507836in}{0.853848in}}%
\pgfpathlineto{\pgfqpoint{0.764427in}{0.853901in}}%
\pgfpathlineto{\pgfqpoint{1.264557in}{0.500857in}}%
\pgfpathlineto{\pgfqpoint{1.557113in}{0.462267in}}%
\pgfpathlineto{\pgfqpoint{1.764686in}{0.513794in}}%
\pgfpathlineto{\pgfqpoint{1.925691in}{0.473010in}}%
\pgfpathlineto{\pgfqpoint{2.057243in}{0.456199in}}%
\pgfpathlineto{\pgfqpoint{2.168468in}{0.450260in}}%
\pgfpathlineto{\pgfqpoint{2.264815in}{0.453093in}}%
\pgfpathlineto{\pgfqpoint{2.349800in}{0.446940in}}%
\pgfpathlineto{\pgfqpoint{2.425821in}{0.437728in}}%
\pgfpathlineto{\pgfqpoint{2.494590in}{0.431749in}}%
\pgfpathlineto{\pgfqpoint{2.615126in}{0.434723in}}%
\pgfpathlineto{\pgfqpoint{2.668597in}{0.436718in}}%
\pgfpathlineto{\pgfqpoint{2.718378in}{0.436650in}}%
\pgfpathlineto{\pgfqpoint{2.764944in}{0.434957in}}%
\pgfpathlineto{\pgfqpoint{2.808687in}{0.431025in}}%
\pgfpathlineto{\pgfqpoint{2.849929in}{0.430954in}}%
\pgfpathlineto{\pgfqpoint{2.925950in}{0.428390in}}%
\pgfpathlineto{\pgfqpoint{2.994720in}{0.426852in}}%
\pgfpathlineto{\pgfqpoint{3.057501in}{0.427038in}}%
\pgfpathlineto{\pgfqpoint{3.086956in}{0.431216in}}%
\pgfpathlineto{\pgfqpoint{3.142486in}{0.433091in}}%
\pgfpathlineto{\pgfqpoint{3.168726in}{0.431138in}}%
\pgfpathlineto{\pgfqpoint{3.194046in}{0.431260in}}%
\pgfpathlineto{\pgfqpoint{3.242166in}{0.428975in}}%
\pgfpathlineto{\pgfqpoint{3.265074in}{0.425834in}}%
\pgfpathlineto{\pgfqpoint{3.369827in}{0.425421in}}%
\pgfpathlineto{\pgfqpoint{3.407812in}{0.424358in}}%
\pgfpathlineto{\pgfqpoint{3.601373in}{0.423353in}}%
\pgfpathlineto{\pgfqpoint{3.655855in}{0.423459in}}%
\pgfpathlineto{\pgfqpoint{3.694175in}{0.422647in}}%
\pgfpathlineto{\pgfqpoint{3.706509in}{0.424220in}}%
\pgfpathlineto{\pgfqpoint{3.718636in}{0.423311in}}%
\pgfpathlineto{\pgfqpoint{3.742295in}{0.422974in}}%
\pgfpathlineto{\pgfqpoint{3.840096in}{0.422676in}}%
\pgfpathlineto{\pgfqpoint{3.898631in}{0.423918in}}%
\pgfpathlineto{\pgfqpoint{3.907941in}{0.424955in}}%
\pgfpathlineto{\pgfqpoint{3.917133in}{0.423963in}}%
\pgfpathlineto{\pgfqpoint{3.935172in}{0.424859in}}%
\pgfpathlineto{\pgfqpoint{3.944025in}{0.424071in}}%
\pgfpathlineto{\pgfqpoint{3.969951in}{0.425680in}}%
\pgfpathlineto{\pgfqpoint{3.978390in}{0.424242in}}%
\pgfpathlineto{\pgfqpoint{3.994978in}{0.424675in}}%
\pgfpathlineto{\pgfqpoint{4.019166in}{0.423946in}}%
\pgfpathlineto{\pgfqpoint{4.027052in}{0.425029in}}%
\pgfpathlineto{\pgfqpoint{4.115513in}{0.421913in}}%
\pgfpathlineto{\pgfqpoint{4.129257in}{0.421973in}}%
\pgfpathlineto{\pgfqpoint{4.175398in}{0.422032in}}%
\pgfpathlineto{\pgfqpoint{4.212727in}{0.421909in}}%
\pgfpathlineto{\pgfqpoint{4.303750in}{0.421674in}}%
\pgfpathlineto{\pgfqpoint{4.345289in}{0.422178in}}%
\pgfpathlineto{\pgfqpoint{4.350317in}{0.422638in}}%
\pgfpathlineto{\pgfqpoint{4.360269in}{0.421808in}}%
\pgfpathlineto{\pgfqpoint{4.379771in}{0.421609in}}%
\pgfpathlineto{\pgfqpoint{4.384565in}{0.423363in}}%
\pgfpathlineto{\pgfqpoint{4.394059in}{0.422035in}}%
\pgfpathlineto{\pgfqpoint{4.430833in}{0.421682in}}%
\pgfpathlineto{\pgfqpoint{4.439741in}{0.421986in}}%
\pgfpathlineto{\pgfqpoint{4.457234in}{0.421828in}}%
\pgfpathlineto{\pgfqpoint{4.478520in}{0.422053in}}%
\pgfpathlineto{\pgfqpoint{4.503260in}{0.422752in}}%
\pgfpathlineto{\pgfqpoint{4.507303in}{0.422085in}}%
\pgfpathlineto{\pgfqpoint{4.515320in}{0.422223in}}%
\pgfpathlineto{\pgfqpoint{4.534981in}{0.421781in}}%
\pgfpathlineto{\pgfqpoint{4.538850in}{0.423326in}}%
\pgfpathlineto{\pgfqpoint{4.550334in}{0.422043in}}%
\pgfpathlineto{\pgfqpoint{4.572767in}{0.421835in}}%
\pgfpathlineto{\pgfqpoint{4.576438in}{0.424398in}}%
\pgfpathlineto{\pgfqpoint{4.605160in}{0.423226in}}%
\pgfpathlineto{\pgfqpoint{4.622547in}{0.422819in}}%
\pgfpathlineto{\pgfqpoint{4.656113in}{0.422145in}}%
\pgfpathlineto{\pgfqpoint{4.659385in}{0.429273in}}%
\pgfpathlineto{\pgfqpoint{4.662643in}{0.426611in}}%
\pgfpathlineto{\pgfqpoint{4.672328in}{0.423645in}}%
\pgfpathlineto{\pgfqpoint{4.694434in}{0.422426in}}%
\pgfpathlineto{\pgfqpoint{4.715882in}{0.422150in}}%
\pgfpathlineto{\pgfqpoint{4.724883in}{0.422182in}}%
\pgfpathlineto{\pgfqpoint{4.727858in}{0.425970in}}%
\pgfpathlineto{\pgfqpoint{4.736711in}{0.423338in}}%
\pgfpathlineto{\pgfqpoint{4.739638in}{0.423007in}}%
\pgfpathlineto{\pgfqpoint{4.742554in}{0.429577in}}%
\pgfpathlineto{\pgfqpoint{4.745457in}{0.424238in}}%
\pgfpathlineto{\pgfqpoint{4.751230in}{0.423028in}}%
\pgfpathlineto{\pgfqpoint{4.762637in}{0.422306in}}%
\pgfpathlineto{\pgfqpoint{4.765461in}{0.430419in}}%
\pgfpathlineto{\pgfqpoint{4.768274in}{0.425101in}}%
\pgfpathlineto{\pgfqpoint{4.771077in}{0.429138in}}%
\pgfpathlineto{\pgfqpoint{4.787664in}{0.423923in}}%
\pgfpathlineto{\pgfqpoint{4.845418in}{0.421791in}}%
\pgfpathlineto{\pgfqpoint{4.850446in}{0.421990in}}%
\pgfpathlineto{\pgfqpoint{4.879900in}{0.421796in}}%
\pgfpathlineto{\pgfqpoint{4.882301in}{0.427807in}}%
\pgfpathlineto{\pgfqpoint{4.884695in}{0.424194in}}%
\pgfpathlineto{\pgfqpoint{4.891827in}{0.422338in}}%
\pgfpathlineto{\pgfqpoint{4.921943in}{0.422494in}}%
\pgfpathlineto{\pgfqpoint{4.933200in}{0.422077in}}%
\pgfpathlineto{\pgfqpoint{4.942081in}{0.422458in}}%
\pgfpathlineto{\pgfqpoint{4.950853in}{0.422130in}}%
\pgfpathlineto{\pgfqpoint{4.953030in}{0.423893in}}%
\pgfpathlineto{\pgfqpoint{4.957363in}{0.423061in}}%
\pgfpathlineto{\pgfqpoint{4.965953in}{0.422312in}}%
\pgfpathlineto{\pgfqpoint{4.968085in}{0.422251in}}%
\pgfpathlineto{\pgfqpoint{4.970210in}{0.430567in}}%
\pgfpathlineto{\pgfqpoint{4.976548in}{0.428623in}}%
\pgfpathlineto{\pgfqpoint{4.984914in}{0.426204in}}%
\pgfpathlineto{\pgfqpoint{4.995237in}{0.424138in}}%
\pgfpathlineto{\pgfqpoint{5.001360in}{0.423263in}}%
\pgfpathlineto{\pgfqpoint{5.003390in}{0.423866in}}%
\pgfpathlineto{\pgfqpoint{5.009445in}{0.422727in}}%
\pgfpathlineto{\pgfqpoint{5.015449in}{0.422932in}}%
\pgfpathlineto{\pgfqpoint{5.021404in}{0.422649in}}%
\pgfpathlineto{\pgfqpoint{5.023378in}{0.441083in}}%
\pgfpathlineto{\pgfqpoint{5.027310in}{0.431941in}}%
\pgfpathlineto{\pgfqpoint{5.031221in}{0.427607in}}%
\pgfpathlineto{\pgfqpoint{5.035111in}{0.425581in}}%
\pgfpathlineto{\pgfqpoint{5.040906in}{0.423796in}}%
\pgfpathlineto{\pgfqpoint{5.044744in}{0.423255in}}%
\pgfpathlineto{\pgfqpoint{5.063633in}{0.422817in}}%
\pgfpathlineto{\pgfqpoint{5.067353in}{0.422488in}}%
\pgfpathlineto{\pgfqpoint{5.087473in}{0.422842in}}%
\pgfpathlineto{\pgfqpoint{5.089274in}{0.423249in}}%
\pgfpathlineto{\pgfqpoint{5.092864in}{0.422022in}}%
\pgfpathlineto{\pgfqpoint{5.119232in}{0.422578in}}%
\pgfpathlineto{\pgfqpoint{5.131215in}{0.421992in}}%
\pgfpathlineto{\pgfqpoint{5.151306in}{0.421778in}}%
\pgfpathlineto{\pgfqpoint{5.154601in}{0.423131in}}%
\pgfpathlineto{\pgfqpoint{5.159515in}{0.422654in}}%
\pgfpathlineto{\pgfqpoint{5.167631in}{0.421851in}}%
\pgfpathlineto{\pgfqpoint{5.169243in}{0.428583in}}%
\pgfpathlineto{\pgfqpoint{5.170852in}{0.424058in}}%
\pgfpathlineto{\pgfqpoint{5.175657in}{0.422245in}}%
\pgfpathlineto{\pgfqpoint{5.185172in}{0.421946in}}%
\pgfpathlineto{\pgfqpoint{5.189883in}{0.422259in}}%
\pgfpathlineto{\pgfqpoint{5.191446in}{0.518954in}}%
\pgfpathlineto{\pgfqpoint{5.196116in}{0.473583in}}%
\pgfpathlineto{\pgfqpoint{5.200756in}{0.446960in}}%
\pgfpathlineto{\pgfqpoint{5.202296in}{0.444900in}}%
\pgfpathlineto{\pgfqpoint{5.203833in}{0.439748in}}%
\pgfpathlineto{\pgfqpoint{5.205367in}{0.438805in}}%
\pgfpathlineto{\pgfqpoint{5.208424in}{0.434496in}}%
\pgfpathlineto{\pgfqpoint{5.217519in}{0.427176in}}%
\pgfpathlineto{\pgfqpoint{5.220526in}{0.426679in}}%
\pgfpathlineto{\pgfqpoint{5.226501in}{0.424804in}}%
\pgfpathlineto{\pgfqpoint{5.236841in}{0.423875in}}%
\pgfpathlineto{\pgfqpoint{5.238306in}{0.427744in}}%
\pgfpathlineto{\pgfqpoint{5.241227in}{0.424309in}}%
\pgfpathlineto{\pgfqpoint{5.245587in}{0.422951in}}%
\pgfpathlineto{\pgfqpoint{5.261351in}{0.422195in}}%
\pgfpathlineto{\pgfqpoint{5.265591in}{0.422243in}}%
\pgfpathlineto{\pgfqpoint{5.283682in}{0.422099in}}%
\pgfpathlineto{\pgfqpoint{5.289159in}{0.422450in}}%
\pgfpathlineto{\pgfqpoint{5.304008in}{0.422860in}}%
\pgfpathlineto{\pgfqpoint{5.305343in}{0.428898in}}%
\pgfpathlineto{\pgfqpoint{5.311981in}{0.424573in}}%
\pgfpathlineto{\pgfqpoint{5.326373in}{0.422242in}}%
\pgfpathlineto{\pgfqpoint{5.337939in}{0.421985in}}%
\pgfpathlineto{\pgfqpoint{5.339212in}{0.423817in}}%
\pgfpathlineto{\pgfqpoint{5.340484in}{0.423008in}}%
\pgfpathlineto{\pgfqpoint{5.341753in}{0.429337in}}%
\pgfpathlineto{\pgfqpoint{5.344285in}{0.423045in}}%
\pgfpathlineto{\pgfqpoint{5.345547in}{0.426080in}}%
\pgfpathlineto{\pgfqpoint{5.349321in}{0.424920in}}%
\pgfpathlineto{\pgfqpoint{5.350575in}{0.564747in}}%
\pgfpathlineto{\pgfqpoint{5.353076in}{0.458391in}}%
\pgfpathlineto{\pgfqpoint{5.355569in}{0.438238in}}%
\pgfpathlineto{\pgfqpoint{5.358052in}{0.431906in}}%
\pgfpathlineto{\pgfqpoint{5.360528in}{0.427498in}}%
\pgfpathlineto{\pgfqpoint{5.361762in}{0.427885in}}%
\pgfpathlineto{\pgfqpoint{5.365453in}{0.425431in}}%
\pgfpathlineto{\pgfqpoint{5.371562in}{0.423946in}}%
\pgfpathlineto{\pgfqpoint{5.412939in}{0.422024in}}%
\pgfpathlineto{\pgfqpoint{5.414087in}{0.422036in}}%
\pgfpathlineto{\pgfqpoint{5.414087in}{0.422036in}}%
\pgfusepath{stroke}%
\end{pgfscope}%
\begin{pgfscope}%
\pgfsetrectcap%
\pgfsetmiterjoin%
\pgfsetlinewidth{0.803000pt}%
\definecolor{currentstroke}{rgb}{0.000000,0.000000,0.000000}%
\pgfsetstrokecolor{currentstroke}%
\pgfsetdash{}{0pt}%
\pgfpathmoveto{\pgfqpoint{0.517836in}{0.420092in}}%
\pgfpathlineto{\pgfqpoint{0.517836in}{1.190092in}}%
\pgfusepath{stroke}%
\end{pgfscope}%
\begin{pgfscope}%
\pgfsetrectcap%
\pgfsetmiterjoin%
\pgfsetlinewidth{0.803000pt}%
\definecolor{currentstroke}{rgb}{0.000000,0.000000,0.000000}%
\pgfsetstrokecolor{currentstroke}%
\pgfsetdash{}{0pt}%
\pgfpathmoveto{\pgfqpoint{5.942836in}{0.420092in}}%
\pgfpathlineto{\pgfqpoint{5.942836in}{1.190092in}}%
\pgfusepath{stroke}%
\end{pgfscope}%
\begin{pgfscope}%
\pgfsetrectcap%
\pgfsetmiterjoin%
\pgfsetlinewidth{0.803000pt}%
\definecolor{currentstroke}{rgb}{0.000000,0.000000,0.000000}%
\pgfsetstrokecolor{currentstroke}%
\pgfsetdash{}{0pt}%
\pgfpathmoveto{\pgfqpoint{0.517836in}{0.420092in}}%
\pgfpathlineto{\pgfqpoint{5.942836in}{0.420092in}}%
\pgfusepath{stroke}%
\end{pgfscope}%
\begin{pgfscope}%
\pgfsetrectcap%
\pgfsetmiterjoin%
\pgfsetlinewidth{0.803000pt}%
\definecolor{currentstroke}{rgb}{0.000000,0.000000,0.000000}%
\pgfsetstrokecolor{currentstroke}%
\pgfsetdash{}{0pt}%
\pgfpathmoveto{\pgfqpoint{0.517836in}{1.190092in}}%
\pgfpathlineto{\pgfqpoint{5.942836in}{1.190092in}}%
\pgfusepath{stroke}%
\end{pgfscope}%
\begin{pgfscope}%
\pgfsetbuttcap%
\pgfsetmiterjoin%
\definecolor{currentfill}{rgb}{1.000000,1.000000,1.000000}%
\pgfsetfillcolor{currentfill}%
\pgfsetfillopacity{0.800000}%
\pgfsetlinewidth{1.003750pt}%
\definecolor{currentstroke}{rgb}{0.800000,0.800000,0.800000}%
\pgfsetstrokecolor{currentstroke}%
\pgfsetstrokeopacity{0.800000}%
\pgfsetdash{}{0pt}%
\pgfpathmoveto{\pgfqpoint{0.576170in}{0.511854in}}%
\pgfpathlineto{\pgfqpoint{1.405988in}{0.511854in}}%
\pgfpathquadraticcurveto{\pgfqpoint{1.422654in}{0.511854in}}{\pgfqpoint{1.422654in}{0.528521in}}%
\pgfpathlineto{\pgfqpoint{1.422654in}{1.131759in}}%
\pgfpathquadraticcurveto{\pgfqpoint{1.422654in}{1.148426in}}{\pgfqpoint{1.405988in}{1.148426in}}%
\pgfpathlineto{\pgfqpoint{0.576170in}{1.148426in}}%
\pgfpathquadraticcurveto{\pgfqpoint{0.559503in}{1.148426in}}{\pgfqpoint{0.559503in}{1.131759in}}%
\pgfpathlineto{\pgfqpoint{0.559503in}{0.528521in}}%
\pgfpathquadraticcurveto{\pgfqpoint{0.559503in}{0.511854in}}{\pgfqpoint{0.576170in}{0.511854in}}%
\pgfpathlineto{\pgfqpoint{0.576170in}{0.511854in}}%
\pgfpathclose%
\pgfusepath{stroke,fill}%
\end{pgfscope}%
\begin{pgfscope}%
\pgfsetrectcap%
\pgfsetroundjoin%
\pgfsetlinewidth{0.501875pt}%
\definecolor{currentstroke}{rgb}{0.121569,0.466667,0.705882}%
\pgfsetstrokecolor{currentstroke}%
\pgfsetdash{}{0pt}%
\pgfpathmoveto{\pgfqpoint{0.592836in}{1.080945in}}%
\pgfpathlineto{\pgfqpoint{0.676170in}{1.080945in}}%
\pgfpathlineto{\pgfqpoint{0.759503in}{1.080945in}}%
\pgfusepath{stroke}%
\end{pgfscope}%
\begin{pgfscope}%
\definecolor{textcolor}{rgb}{0.000000,0.000000,0.000000}%
\pgfsetstrokecolor{textcolor}%
\pgfsetfillcolor{textcolor}%
\pgftext[x=0.826170in,y=1.051778in,left,base]{\color{textcolor}{\rmfamily\fontsize{6.000000}{7.200000}\selectfont\catcode`\^=\active\def^{\ifmmode\sp\else\^{}\fi}\catcode`\%=\active\def%{\%}NRC+PT}}%
\end{pgfscope}%
\begin{pgfscope}%
\pgfsetrectcap%
\pgfsetroundjoin%
\pgfsetlinewidth{0.501875pt}%
\definecolor{currentstroke}{rgb}{1.000000,0.498039,0.054902}%
\pgfsetstrokecolor{currentstroke}%
\pgfsetdash{}{0pt}%
\pgfpathmoveto{\pgfqpoint{0.592836in}{0.958631in}}%
\pgfpathlineto{\pgfqpoint{0.676170in}{0.958631in}}%
\pgfpathlineto{\pgfqpoint{0.759503in}{0.958631in}}%
\pgfusepath{stroke}%
\end{pgfscope}%
\begin{pgfscope}%
\definecolor{textcolor}{rgb}{0.000000,0.000000,0.000000}%
\pgfsetstrokecolor{textcolor}%
\pgfsetfillcolor{textcolor}%
\pgftext[x=0.826170in,y=0.929464in,left,base]{\color{textcolor}{\rmfamily\fontsize{6.000000}{7.200000}\selectfont\catcode`\^=\active\def^{\ifmmode\sp\else\^{}\fi}\catcode`\%=\active\def%{\%}NRC+PT+SL}}%
\end{pgfscope}%
\begin{pgfscope}%
\pgfsetrectcap%
\pgfsetroundjoin%
\pgfsetlinewidth{0.501875pt}%
\definecolor{currentstroke}{rgb}{0.172549,0.627451,0.172549}%
\pgfsetstrokecolor{currentstroke}%
\pgfsetdash{}{0pt}%
\pgfpathmoveto{\pgfqpoint{0.592836in}{0.836317in}}%
\pgfpathlineto{\pgfqpoint{0.676170in}{0.836317in}}%
\pgfpathlineto{\pgfqpoint{0.759503in}{0.836317in}}%
\pgfusepath{stroke}%
\end{pgfscope}%
\begin{pgfscope}%
\definecolor{textcolor}{rgb}{0.000000,0.000000,0.000000}%
\pgfsetstrokecolor{textcolor}%
\pgfsetfillcolor{textcolor}%
\pgftext[x=0.826170in,y=0.807150in,left,base]{\color{textcolor}{\rmfamily\fontsize{6.000000}{7.200000}\selectfont\catcode`\^=\active\def^{\ifmmode\sp\else\^{}\fi}\catcode`\%=\active\def%{\%}NRC+BT}}%
\end{pgfscope}%
\begin{pgfscope}%
\pgfsetrectcap%
\pgfsetroundjoin%
\pgfsetlinewidth{0.501875pt}%
\definecolor{currentstroke}{rgb}{0.839216,0.152941,0.156863}%
\pgfsetstrokecolor{currentstroke}%
\pgfsetdash{}{0pt}%
\pgfpathmoveto{\pgfqpoint{0.592836in}{0.714002in}}%
\pgfpathlineto{\pgfqpoint{0.676170in}{0.714002in}}%
\pgfpathlineto{\pgfqpoint{0.759503in}{0.714002in}}%
\pgfusepath{stroke}%
\end{pgfscope}%
\begin{pgfscope}%
\definecolor{textcolor}{rgb}{0.000000,0.000000,0.000000}%
\pgfsetstrokecolor{textcolor}%
\pgfsetfillcolor{textcolor}%
\pgftext[x=0.826170in,y=0.684836in,left,base]{\color{textcolor}{\rmfamily\fontsize{6.000000}{7.200000}\selectfont\catcode`\^=\active\def^{\ifmmode\sp\else\^{}\fi}\catcode`\%=\active\def%{\%}NRC+LT}}%
\end{pgfscope}%
\begin{pgfscope}%
\pgfsetrectcap%
\pgfsetroundjoin%
\pgfsetlinewidth{0.501875pt}%
\definecolor{currentstroke}{rgb}{0.580392,0.403922,0.741176}%
\pgfsetstrokecolor{currentstroke}%
\pgfsetdash{}{0pt}%
\pgfpathmoveto{\pgfqpoint{0.592836in}{0.591688in}}%
\pgfpathlineto{\pgfqpoint{0.676170in}{0.591688in}}%
\pgfpathlineto{\pgfqpoint{0.759503in}{0.591688in}}%
\pgfusepath{stroke}%
\end{pgfscope}%
\begin{pgfscope}%
\definecolor{textcolor}{rgb}{0.000000,0.000000,0.000000}%
\pgfsetstrokecolor{textcolor}%
\pgfsetfillcolor{textcolor}%
\pgftext[x=0.826170in,y=0.562521in,left,base]{\color{textcolor}{\rmfamily\fontsize{6.000000}{7.200000}\selectfont\catcode`\^=\active\def^{\ifmmode\sp\else\^{}\fi}\catcode`\%=\active\def%{\%}NRC+SPPC}}%
\end{pgfscope}%
\end{pgfpicture}%
\makeatother%
\endgroup%

    \caption{Comparison of training convergence. Because the training is online, there is no ground truth validation step. The loss thus mainly indicates the variance of the training set. BT has the highest training error because it samples both the light and the camera, SPPC has the lowest training error because it keeps estimates in a fixed ring buffer for accumulation.}
    \label{fig:convergence}
\end{figure}

\section{Performance and Optimization}

\begin{figure}[htb!]
    \centering
    %% Creator: Matplotlib, PGF backend
%%
%% To include the figure in your LaTeX document, write
%%   \input{<filename>.pgf}
%%
%% Make sure the required packages are loaded in your preamble
%%   \usepackage{pgf}
%%
%% Also ensure that all the required font packages are loaded; for instance,
%% the lmodern package is sometimes necessary when using math font.
%%   \usepackage{lmodern}
%%
%% Figures using additional raster images can only be included by \input if
%% they are in the same directory as the main LaTeX file. For loading figures
%% from other directories you can use the `import` package
%%   \usepackage{import}
%%
%% and then include the figures with
%%   \import{<path to file>}{<filename>.pgf}
%%
%% Matplotlib used the following preamble
%%   \def\mathdefault#1{#1}
%%   \everymath=\expandafter{\the\everymath\displaystyle}
%%   \IfFileExists{scrextend.sty}{
%%     \usepackage[fontsize=10.000000pt]{scrextend}
%%   }{
%%     \renewcommand{\normalsize}{\fontsize{10.000000}{12.000000}\selectfont}
%%     \normalsize
%%   }
%%   
%%   \ifdefined\pdftexversion\else  % non-pdftex case.
%%     \usepackage{fontspec}
%%     \setmainfont{DejaVuSerif.ttf}[Path=\detokenize{/opt/homebrew/Cellar/python-matplotlib/3.10.5/libexec/lib/python3.13/site-packages/matplotlib/mpl-data/fonts/ttf/}]
%%     \setsansfont{DejaVuSans.ttf}[Path=\detokenize{/opt/homebrew/Cellar/python-matplotlib/3.10.5/libexec/lib/python3.13/site-packages/matplotlib/mpl-data/fonts/ttf/}]
%%     \setmonofont{DejaVuSansMono.ttf}[Path=\detokenize{/opt/homebrew/Cellar/python-matplotlib/3.10.5/libexec/lib/python3.13/site-packages/matplotlib/mpl-data/fonts/ttf/}]
%%   \fi
%%   \makeatletter\@ifpackageloaded{underscore}{}{\usepackage[strings]{underscore}}\makeatother
%%
\begingroup%
\makeatletter%
\begin{pgfpicture}%
\pgfpathrectangle{\pgfpointorigin}{\pgfqpoint{6.118122in}{3.116660in}}%
\pgfusepath{use as bounding box, clip}%
\begin{pgfscope}%
\pgfsetbuttcap%
\pgfsetmiterjoin%
\definecolor{currentfill}{rgb}{1.000000,1.000000,1.000000}%
\pgfsetfillcolor{currentfill}%
\pgfsetlinewidth{0.000000pt}%
\definecolor{currentstroke}{rgb}{1.000000,1.000000,1.000000}%
\pgfsetstrokecolor{currentstroke}%
\pgfsetdash{}{0pt}%
\pgfpathmoveto{\pgfqpoint{0.000000in}{0.000000in}}%
\pgfpathlineto{\pgfqpoint{6.118122in}{0.000000in}}%
\pgfpathlineto{\pgfqpoint{6.118122in}{3.116660in}}%
\pgfpathlineto{\pgfqpoint{0.000000in}{3.116660in}}%
\pgfpathlineto{\pgfqpoint{0.000000in}{0.000000in}}%
\pgfpathclose%
\pgfusepath{fill}%
\end{pgfscope}%
\begin{pgfscope}%
\pgfsetbuttcap%
\pgfsetmiterjoin%
\definecolor{currentfill}{rgb}{1.000000,1.000000,1.000000}%
\pgfsetfillcolor{currentfill}%
\pgfsetlinewidth{0.000000pt}%
\definecolor{currentstroke}{rgb}{0.000000,0.000000,0.000000}%
\pgfsetstrokecolor{currentstroke}%
\pgfsetstrokeopacity{0.000000}%
\pgfsetdash{}{0pt}%
\pgfpathmoveto{\pgfqpoint{2.060179in}{0.294761in}}%
\pgfpathlineto{\pgfqpoint{6.018122in}{0.294761in}}%
\pgfpathlineto{\pgfqpoint{6.018122in}{3.016660in}}%
\pgfpathlineto{\pgfqpoint{2.060179in}{3.016660in}}%
\pgfpathlineto{\pgfqpoint{2.060179in}{0.294761in}}%
\pgfpathclose%
\pgfusepath{fill}%
\end{pgfscope}%
\begin{pgfscope}%
\pgfpathrectangle{\pgfqpoint{2.060179in}{0.294761in}}{\pgfqpoint{3.957943in}{2.721899in}}%
\pgfusepath{clip}%
\pgfsetbuttcap%
\pgfsetmiterjoin%
\definecolor{currentfill}{rgb}{0.710588,0.710588,0.710588}%
\pgfsetfillcolor{currentfill}%
\pgfsetlinewidth{0.000000pt}%
\definecolor{currentstroke}{rgb}{0.000000,0.000000,0.000000}%
\pgfsetstrokecolor{currentstroke}%
\pgfsetstrokeopacity{0.000000}%
\pgfsetdash{}{0pt}%
\pgfpathmoveto{\pgfqpoint{2.240085in}{0.294761in}}%
\pgfpathlineto{\pgfqpoint{2.516864in}{0.294761in}}%
\pgfpathlineto{\pgfqpoint{2.516864in}{1.573829in}}%
\pgfpathlineto{\pgfqpoint{2.240085in}{1.573829in}}%
\pgfpathlineto{\pgfqpoint{2.240085in}{0.294761in}}%
\pgfpathclose%
\pgfusepath{fill}%
\end{pgfscope}%
\begin{pgfscope}%
\pgfpathrectangle{\pgfqpoint{2.060179in}{0.294761in}}{\pgfqpoint{3.957943in}{2.721899in}}%
\pgfusepath{clip}%
\pgfsetbuttcap%
\pgfsetmiterjoin%
\definecolor{currentfill}{rgb}{0.854902,0.439216,0.839216}%
\pgfsetfillcolor{currentfill}%
\pgfsetlinewidth{0.000000pt}%
\definecolor{currentstroke}{rgb}{0.000000,0.000000,0.000000}%
\pgfsetstrokecolor{currentstroke}%
\pgfsetstrokeopacity{0.000000}%
\pgfsetdash{}{0pt}%
\pgfpathmoveto{\pgfqpoint{2.240085in}{1.573829in}}%
\pgfpathlineto{\pgfqpoint{2.516864in}{1.573829in}}%
\pgfpathlineto{\pgfqpoint{2.516864in}{1.574949in}}%
\pgfpathlineto{\pgfqpoint{2.240085in}{1.574949in}}%
\pgfpathlineto{\pgfqpoint{2.240085in}{1.573829in}}%
\pgfpathclose%
\pgfusepath{fill}%
\end{pgfscope}%
\begin{pgfscope}%
\pgfpathrectangle{\pgfqpoint{2.060179in}{0.294761in}}{\pgfqpoint{3.957943in}{2.721899in}}%
\pgfusepath{clip}%
\pgfsetbuttcap%
\pgfsetmiterjoin%
\definecolor{currentfill}{rgb}{0.814118,0.883922,0.949804}%
\pgfsetfillcolor{currentfill}%
\pgfsetlinewidth{0.000000pt}%
\definecolor{currentstroke}{rgb}{0.000000,0.000000,0.000000}%
\pgfsetstrokecolor{currentstroke}%
\pgfsetstrokeopacity{0.000000}%
\pgfsetdash{}{0pt}%
\pgfpathmoveto{\pgfqpoint{2.793644in}{0.294761in}}%
\pgfpathlineto{\pgfqpoint{3.070423in}{0.294761in}}%
\pgfpathlineto{\pgfqpoint{3.070423in}{0.349064in}}%
\pgfpathlineto{\pgfqpoint{2.793644in}{0.349064in}}%
\pgfpathlineto{\pgfqpoint{2.793644in}{0.294761in}}%
\pgfpathclose%
\pgfusepath{fill}%
\end{pgfscope}%
\begin{pgfscope}%
\pgfpathrectangle{\pgfqpoint{2.060179in}{0.294761in}}{\pgfqpoint{3.957943in}{2.721899in}}%
\pgfusepath{clip}%
\pgfsetbuttcap%
\pgfsetmiterjoin%
\definecolor{currentfill}{rgb}{0.887059,0.887059,0.887059}%
\pgfsetfillcolor{currentfill}%
\pgfsetlinewidth{0.000000pt}%
\definecolor{currentstroke}{rgb}{0.000000,0.000000,0.000000}%
\pgfsetstrokecolor{currentstroke}%
\pgfsetstrokeopacity{0.000000}%
\pgfsetdash{}{0pt}%
\pgfpathmoveto{\pgfqpoint{2.793644in}{0.349064in}}%
\pgfpathlineto{\pgfqpoint{3.070423in}{0.349064in}}%
\pgfpathlineto{\pgfqpoint{3.070423in}{0.609113in}}%
\pgfpathlineto{\pgfqpoint{2.793644in}{0.609113in}}%
\pgfpathlineto{\pgfqpoint{2.793644in}{0.349064in}}%
\pgfpathclose%
\pgfusepath{fill}%
\end{pgfscope}%
\begin{pgfscope}%
\pgfpathrectangle{\pgfqpoint{2.060179in}{0.294761in}}{\pgfqpoint{3.957943in}{2.721899in}}%
\pgfusepath{clip}%
\pgfsetbuttcap%
\pgfsetmiterjoin%
\definecolor{currentfill}{rgb}{0.710588,0.710588,0.710588}%
\pgfsetfillcolor{currentfill}%
\pgfsetlinewidth{0.000000pt}%
\definecolor{currentstroke}{rgb}{0.000000,0.000000,0.000000}%
\pgfsetstrokecolor{currentstroke}%
\pgfsetstrokeopacity{0.000000}%
\pgfsetdash{}{0pt}%
\pgfpathmoveto{\pgfqpoint{2.793644in}{0.609113in}}%
\pgfpathlineto{\pgfqpoint{3.070423in}{0.609113in}}%
\pgfpathlineto{\pgfqpoint{3.070423in}{0.870621in}}%
\pgfpathlineto{\pgfqpoint{2.793644in}{0.870621in}}%
\pgfpathlineto{\pgfqpoint{2.793644in}{0.609113in}}%
\pgfpathclose%
\pgfusepath{fill}%
\end{pgfscope}%
\begin{pgfscope}%
\pgfpathrectangle{\pgfqpoint{2.060179in}{0.294761in}}{\pgfqpoint{3.957943in}{2.721899in}}%
\pgfusepath{clip}%
\pgfsetbuttcap%
\pgfsetmiterjoin%
\definecolor{currentfill}{rgb}{0.478431,0.478431,0.478431}%
\pgfsetfillcolor{currentfill}%
\pgfsetlinewidth{0.000000pt}%
\definecolor{currentstroke}{rgb}{0.000000,0.000000,0.000000}%
\pgfsetstrokecolor{currentstroke}%
\pgfsetstrokeopacity{0.000000}%
\pgfsetdash{}{0pt}%
\pgfpathmoveto{\pgfqpoint{2.793644in}{0.870621in}}%
\pgfpathlineto{\pgfqpoint{3.070423in}{0.870621in}}%
\pgfpathlineto{\pgfqpoint{3.070423in}{1.122947in}}%
\pgfpathlineto{\pgfqpoint{2.793644in}{1.122947in}}%
\pgfpathlineto{\pgfqpoint{2.793644in}{0.870621in}}%
\pgfpathclose%
\pgfusepath{fill}%
\end{pgfscope}%
\begin{pgfscope}%
\pgfpathrectangle{\pgfqpoint{2.060179in}{0.294761in}}{\pgfqpoint{3.957943in}{2.721899in}}%
\pgfusepath{clip}%
\pgfsetbuttcap%
\pgfsetmiterjoin%
\definecolor{currentfill}{rgb}{1.000000,0.752941,0.796078}%
\pgfsetfillcolor{currentfill}%
\pgfsetlinewidth{0.000000pt}%
\definecolor{currentstroke}{rgb}{0.000000,0.000000,0.000000}%
\pgfsetstrokecolor{currentstroke}%
\pgfsetstrokeopacity{0.000000}%
\pgfsetdash{}{0pt}%
\pgfpathmoveto{\pgfqpoint{2.793644in}{1.122947in}}%
\pgfpathlineto{\pgfqpoint{3.070423in}{1.122947in}}%
\pgfpathlineto{\pgfqpoint{3.070423in}{1.138107in}}%
\pgfpathlineto{\pgfqpoint{2.793644in}{1.138107in}}%
\pgfpathlineto{\pgfqpoint{2.793644in}{1.122947in}}%
\pgfpathclose%
\pgfusepath{fill}%
\end{pgfscope}%
\begin{pgfscope}%
\pgfpathrectangle{\pgfqpoint{2.060179in}{0.294761in}}{\pgfqpoint{3.957943in}{2.721899in}}%
\pgfusepath{clip}%
\pgfsetbuttcap%
\pgfsetmiterjoin%
\definecolor{currentfill}{rgb}{0.854902,0.439216,0.839216}%
\pgfsetfillcolor{currentfill}%
\pgfsetlinewidth{0.000000pt}%
\definecolor{currentstroke}{rgb}{0.000000,0.000000,0.000000}%
\pgfsetstrokecolor{currentstroke}%
\pgfsetstrokeopacity{0.000000}%
\pgfsetdash{}{0pt}%
\pgfpathmoveto{\pgfqpoint{2.793644in}{1.138107in}}%
\pgfpathlineto{\pgfqpoint{3.070423in}{1.138107in}}%
\pgfpathlineto{\pgfqpoint{3.070423in}{1.140912in}}%
\pgfpathlineto{\pgfqpoint{2.793644in}{1.140912in}}%
\pgfpathlineto{\pgfqpoint{2.793644in}{1.138107in}}%
\pgfpathclose%
\pgfusepath{fill}%
\end{pgfscope}%
\begin{pgfscope}%
\pgfpathrectangle{\pgfqpoint{2.060179in}{0.294761in}}{\pgfqpoint{3.957943in}{2.721899in}}%
\pgfusepath{clip}%
\pgfsetbuttcap%
\pgfsetmiterjoin%
\definecolor{currentfill}{rgb}{0.814118,0.883922,0.949804}%
\pgfsetfillcolor{currentfill}%
\pgfsetlinewidth{0.000000pt}%
\definecolor{currentstroke}{rgb}{0.000000,0.000000,0.000000}%
\pgfsetstrokecolor{currentstroke}%
\pgfsetstrokeopacity{0.000000}%
\pgfsetdash{}{0pt}%
\pgfpathmoveto{\pgfqpoint{3.347202in}{0.294761in}}%
\pgfpathlineto{\pgfqpoint{3.623981in}{0.294761in}}%
\pgfpathlineto{\pgfqpoint{3.623981in}{0.353790in}}%
\pgfpathlineto{\pgfqpoint{3.347202in}{0.353790in}}%
\pgfpathlineto{\pgfqpoint{3.347202in}{0.294761in}}%
\pgfpathclose%
\pgfusepath{fill}%
\end{pgfscope}%
\begin{pgfscope}%
\pgfpathrectangle{\pgfqpoint{2.060179in}{0.294761in}}{\pgfqpoint{3.957943in}{2.721899in}}%
\pgfusepath{clip}%
\pgfsetbuttcap%
\pgfsetmiterjoin%
\definecolor{currentfill}{rgb}{0.827451,0.932549,0.803137}%
\pgfsetfillcolor{currentfill}%
\pgfsetlinewidth{0.000000pt}%
\definecolor{currentstroke}{rgb}{0.000000,0.000000,0.000000}%
\pgfsetstrokecolor{currentstroke}%
\pgfsetstrokeopacity{0.000000}%
\pgfsetdash{}{0pt}%
\pgfpathmoveto{\pgfqpoint{3.347202in}{0.353790in}}%
\pgfpathlineto{\pgfqpoint{3.623981in}{0.353790in}}%
\pgfpathlineto{\pgfqpoint{3.623981in}{0.385450in}}%
\pgfpathlineto{\pgfqpoint{3.347202in}{0.385450in}}%
\pgfpathlineto{\pgfqpoint{3.347202in}{0.353790in}}%
\pgfpathclose%
\pgfusepath{fill}%
\end{pgfscope}%
\begin{pgfscope}%
\pgfpathrectangle{\pgfqpoint{2.060179in}{0.294761in}}{\pgfqpoint{3.957943in}{2.721899in}}%
\pgfusepath{clip}%
\pgfsetbuttcap%
\pgfsetmiterjoin%
\definecolor{currentfill}{rgb}{0.451765,0.767090,0.461207}%
\pgfsetfillcolor{currentfill}%
\pgfsetlinewidth{0.000000pt}%
\definecolor{currentstroke}{rgb}{0.000000,0.000000,0.000000}%
\pgfsetstrokecolor{currentstroke}%
\pgfsetstrokeopacity{0.000000}%
\pgfsetdash{}{0pt}%
\pgfpathmoveto{\pgfqpoint{3.347202in}{0.385450in}}%
\pgfpathlineto{\pgfqpoint{3.623981in}{0.385450in}}%
\pgfpathlineto{\pgfqpoint{3.623981in}{0.387717in}}%
\pgfpathlineto{\pgfqpoint{3.347202in}{0.387717in}}%
\pgfpathlineto{\pgfqpoint{3.347202in}{0.385450in}}%
\pgfpathclose%
\pgfusepath{fill}%
\end{pgfscope}%
\begin{pgfscope}%
\pgfpathrectangle{\pgfqpoint{2.060179in}{0.294761in}}{\pgfqpoint{3.957943in}{2.721899in}}%
\pgfusepath{clip}%
\pgfsetbuttcap%
\pgfsetmiterjoin%
\definecolor{currentfill}{rgb}{0.887059,0.887059,0.887059}%
\pgfsetfillcolor{currentfill}%
\pgfsetlinewidth{0.000000pt}%
\definecolor{currentstroke}{rgb}{0.000000,0.000000,0.000000}%
\pgfsetstrokecolor{currentstroke}%
\pgfsetstrokeopacity{0.000000}%
\pgfsetdash{}{0pt}%
\pgfpathmoveto{\pgfqpoint{3.347202in}{0.387717in}}%
\pgfpathlineto{\pgfqpoint{3.623981in}{0.387717in}}%
\pgfpathlineto{\pgfqpoint{3.623981in}{0.648979in}}%
\pgfpathlineto{\pgfqpoint{3.347202in}{0.648979in}}%
\pgfpathlineto{\pgfqpoint{3.347202in}{0.387717in}}%
\pgfpathclose%
\pgfusepath{fill}%
\end{pgfscope}%
\begin{pgfscope}%
\pgfpathrectangle{\pgfqpoint{2.060179in}{0.294761in}}{\pgfqpoint{3.957943in}{2.721899in}}%
\pgfusepath{clip}%
\pgfsetbuttcap%
\pgfsetmiterjoin%
\definecolor{currentfill}{rgb}{0.710588,0.710588,0.710588}%
\pgfsetfillcolor{currentfill}%
\pgfsetlinewidth{0.000000pt}%
\definecolor{currentstroke}{rgb}{0.000000,0.000000,0.000000}%
\pgfsetstrokecolor{currentstroke}%
\pgfsetstrokeopacity{0.000000}%
\pgfsetdash{}{0pt}%
\pgfpathmoveto{\pgfqpoint{3.347202in}{0.648979in}}%
\pgfpathlineto{\pgfqpoint{3.623981in}{0.648979in}}%
\pgfpathlineto{\pgfqpoint{3.623981in}{0.911745in}}%
\pgfpathlineto{\pgfqpoint{3.347202in}{0.911745in}}%
\pgfpathlineto{\pgfqpoint{3.347202in}{0.648979in}}%
\pgfpathclose%
\pgfusepath{fill}%
\end{pgfscope}%
\begin{pgfscope}%
\pgfpathrectangle{\pgfqpoint{2.060179in}{0.294761in}}{\pgfqpoint{3.957943in}{2.721899in}}%
\pgfusepath{clip}%
\pgfsetbuttcap%
\pgfsetmiterjoin%
\definecolor{currentfill}{rgb}{0.478431,0.478431,0.478431}%
\pgfsetfillcolor{currentfill}%
\pgfsetlinewidth{0.000000pt}%
\definecolor{currentstroke}{rgb}{0.000000,0.000000,0.000000}%
\pgfsetstrokecolor{currentstroke}%
\pgfsetstrokeopacity{0.000000}%
\pgfsetdash{}{0pt}%
\pgfpathmoveto{\pgfqpoint{3.347202in}{0.911745in}}%
\pgfpathlineto{\pgfqpoint{3.623981in}{0.911745in}}%
\pgfpathlineto{\pgfqpoint{3.623981in}{1.162842in}}%
\pgfpathlineto{\pgfqpoint{3.347202in}{1.162842in}}%
\pgfpathlineto{\pgfqpoint{3.347202in}{0.911745in}}%
\pgfpathclose%
\pgfusepath{fill}%
\end{pgfscope}%
\begin{pgfscope}%
\pgfpathrectangle{\pgfqpoint{2.060179in}{0.294761in}}{\pgfqpoint{3.957943in}{2.721899in}}%
\pgfusepath{clip}%
\pgfsetbuttcap%
\pgfsetmiterjoin%
\definecolor{currentfill}{rgb}{1.000000,0.752941,0.796078}%
\pgfsetfillcolor{currentfill}%
\pgfsetlinewidth{0.000000pt}%
\definecolor{currentstroke}{rgb}{0.000000,0.000000,0.000000}%
\pgfsetstrokecolor{currentstroke}%
\pgfsetstrokeopacity{0.000000}%
\pgfsetdash{}{0pt}%
\pgfpathmoveto{\pgfqpoint{3.347202in}{1.162842in}}%
\pgfpathlineto{\pgfqpoint{3.623981in}{1.162842in}}%
\pgfpathlineto{\pgfqpoint{3.623981in}{1.178131in}}%
\pgfpathlineto{\pgfqpoint{3.347202in}{1.178131in}}%
\pgfpathlineto{\pgfqpoint{3.347202in}{1.162842in}}%
\pgfpathclose%
\pgfusepath{fill}%
\end{pgfscope}%
\begin{pgfscope}%
\pgfpathrectangle{\pgfqpoint{2.060179in}{0.294761in}}{\pgfqpoint{3.957943in}{2.721899in}}%
\pgfusepath{clip}%
\pgfsetbuttcap%
\pgfsetmiterjoin%
\definecolor{currentfill}{rgb}{0.854902,0.439216,0.839216}%
\pgfsetfillcolor{currentfill}%
\pgfsetlinewidth{0.000000pt}%
\definecolor{currentstroke}{rgb}{0.000000,0.000000,0.000000}%
\pgfsetstrokecolor{currentstroke}%
\pgfsetstrokeopacity{0.000000}%
\pgfsetdash{}{0pt}%
\pgfpathmoveto{\pgfqpoint{3.347202in}{1.178131in}}%
\pgfpathlineto{\pgfqpoint{3.623981in}{1.178131in}}%
\pgfpathlineto{\pgfqpoint{3.623981in}{1.181388in}}%
\pgfpathlineto{\pgfqpoint{3.347202in}{1.181388in}}%
\pgfpathlineto{\pgfqpoint{3.347202in}{1.178131in}}%
\pgfpathclose%
\pgfusepath{fill}%
\end{pgfscope}%
\begin{pgfscope}%
\pgfpathrectangle{\pgfqpoint{2.060179in}{0.294761in}}{\pgfqpoint{3.957943in}{2.721899in}}%
\pgfusepath{clip}%
\pgfsetbuttcap%
\pgfsetmiterjoin%
\definecolor{currentfill}{rgb}{0.579608,0.770196,0.873725}%
\pgfsetfillcolor{currentfill}%
\pgfsetlinewidth{0.000000pt}%
\definecolor{currentstroke}{rgb}{0.000000,0.000000,0.000000}%
\pgfsetstrokecolor{currentstroke}%
\pgfsetstrokeopacity{0.000000}%
\pgfsetdash{}{0pt}%
\pgfpathmoveto{\pgfqpoint{3.900761in}{0.294761in}}%
\pgfpathlineto{\pgfqpoint{4.177540in}{0.294761in}}%
\pgfpathlineto{\pgfqpoint{4.177540in}{0.349030in}}%
\pgfpathlineto{\pgfqpoint{3.900761in}{0.349030in}}%
\pgfpathlineto{\pgfqpoint{3.900761in}{0.294761in}}%
\pgfpathclose%
\pgfusepath{fill}%
\end{pgfscope}%
\begin{pgfscope}%
\pgfpathrectangle{\pgfqpoint{2.060179in}{0.294761in}}{\pgfqpoint{3.957943in}{2.721899in}}%
\pgfusepath{clip}%
\pgfsetbuttcap%
\pgfsetmiterjoin%
\definecolor{currentfill}{rgb}{0.887059,0.887059,0.887059}%
\pgfsetfillcolor{currentfill}%
\pgfsetlinewidth{0.000000pt}%
\definecolor{currentstroke}{rgb}{0.000000,0.000000,0.000000}%
\pgfsetstrokecolor{currentstroke}%
\pgfsetstrokeopacity{0.000000}%
\pgfsetdash{}{0pt}%
\pgfpathmoveto{\pgfqpoint{3.900761in}{0.349030in}}%
\pgfpathlineto{\pgfqpoint{4.177540in}{0.349030in}}%
\pgfpathlineto{\pgfqpoint{4.177540in}{0.610218in}}%
\pgfpathlineto{\pgfqpoint{3.900761in}{0.610218in}}%
\pgfpathlineto{\pgfqpoint{3.900761in}{0.349030in}}%
\pgfpathclose%
\pgfusepath{fill}%
\end{pgfscope}%
\begin{pgfscope}%
\pgfpathrectangle{\pgfqpoint{2.060179in}{0.294761in}}{\pgfqpoint{3.957943in}{2.721899in}}%
\pgfusepath{clip}%
\pgfsetbuttcap%
\pgfsetmiterjoin%
\definecolor{currentfill}{rgb}{0.710588,0.710588,0.710588}%
\pgfsetfillcolor{currentfill}%
\pgfsetlinewidth{0.000000pt}%
\definecolor{currentstroke}{rgb}{0.000000,0.000000,0.000000}%
\pgfsetstrokecolor{currentstroke}%
\pgfsetstrokeopacity{0.000000}%
\pgfsetdash{}{0pt}%
\pgfpathmoveto{\pgfqpoint{3.900761in}{0.610218in}}%
\pgfpathlineto{\pgfqpoint{4.177540in}{0.610218in}}%
\pgfpathlineto{\pgfqpoint{4.177540in}{0.872367in}}%
\pgfpathlineto{\pgfqpoint{3.900761in}{0.872367in}}%
\pgfpathlineto{\pgfqpoint{3.900761in}{0.610218in}}%
\pgfpathclose%
\pgfusepath{fill}%
\end{pgfscope}%
\begin{pgfscope}%
\pgfpathrectangle{\pgfqpoint{2.060179in}{0.294761in}}{\pgfqpoint{3.957943in}{2.721899in}}%
\pgfusepath{clip}%
\pgfsetbuttcap%
\pgfsetmiterjoin%
\definecolor{currentfill}{rgb}{0.478431,0.478431,0.478431}%
\pgfsetfillcolor{currentfill}%
\pgfsetlinewidth{0.000000pt}%
\definecolor{currentstroke}{rgb}{0.000000,0.000000,0.000000}%
\pgfsetstrokecolor{currentstroke}%
\pgfsetstrokeopacity{0.000000}%
\pgfsetdash{}{0pt}%
\pgfpathmoveto{\pgfqpoint{3.900761in}{0.872367in}}%
\pgfpathlineto{\pgfqpoint{4.177540in}{0.872367in}}%
\pgfpathlineto{\pgfqpoint{4.177540in}{1.124070in}}%
\pgfpathlineto{\pgfqpoint{3.900761in}{1.124070in}}%
\pgfpathlineto{\pgfqpoint{3.900761in}{0.872367in}}%
\pgfpathclose%
\pgfusepath{fill}%
\end{pgfscope}%
\begin{pgfscope}%
\pgfpathrectangle{\pgfqpoint{2.060179in}{0.294761in}}{\pgfqpoint{3.957943in}{2.721899in}}%
\pgfusepath{clip}%
\pgfsetbuttcap%
\pgfsetmiterjoin%
\definecolor{currentfill}{rgb}{1.000000,0.752941,0.796078}%
\pgfsetfillcolor{currentfill}%
\pgfsetlinewidth{0.000000pt}%
\definecolor{currentstroke}{rgb}{0.000000,0.000000,0.000000}%
\pgfsetstrokecolor{currentstroke}%
\pgfsetstrokeopacity{0.000000}%
\pgfsetdash{}{0pt}%
\pgfpathmoveto{\pgfqpoint{3.900761in}{1.124070in}}%
\pgfpathlineto{\pgfqpoint{4.177540in}{1.124070in}}%
\pgfpathlineto{\pgfqpoint{4.177540in}{1.139155in}}%
\pgfpathlineto{\pgfqpoint{3.900761in}{1.139155in}}%
\pgfpathlineto{\pgfqpoint{3.900761in}{1.124070in}}%
\pgfpathclose%
\pgfusepath{fill}%
\end{pgfscope}%
\begin{pgfscope}%
\pgfpathrectangle{\pgfqpoint{2.060179in}{0.294761in}}{\pgfqpoint{3.957943in}{2.721899in}}%
\pgfusepath{clip}%
\pgfsetbuttcap%
\pgfsetmiterjoin%
\definecolor{currentfill}{rgb}{0.854902,0.439216,0.839216}%
\pgfsetfillcolor{currentfill}%
\pgfsetlinewidth{0.000000pt}%
\definecolor{currentstroke}{rgb}{0.000000,0.000000,0.000000}%
\pgfsetstrokecolor{currentstroke}%
\pgfsetstrokeopacity{0.000000}%
\pgfsetdash{}{0pt}%
\pgfpathmoveto{\pgfqpoint{3.900761in}{1.139155in}}%
\pgfpathlineto{\pgfqpoint{4.177540in}{1.139155in}}%
\pgfpathlineto{\pgfqpoint{4.177540in}{1.141937in}}%
\pgfpathlineto{\pgfqpoint{3.900761in}{1.141937in}}%
\pgfpathlineto{\pgfqpoint{3.900761in}{1.139155in}}%
\pgfpathclose%
\pgfusepath{fill}%
\end{pgfscope}%
\begin{pgfscope}%
\pgfpathrectangle{\pgfqpoint{2.060179in}{0.294761in}}{\pgfqpoint{3.957943in}{2.721899in}}%
\pgfusepath{clip}%
\pgfsetbuttcap%
\pgfsetmiterjoin%
\definecolor{currentfill}{rgb}{0.579608,0.770196,0.873725}%
\pgfsetfillcolor{currentfill}%
\pgfsetlinewidth{0.000000pt}%
\definecolor{currentstroke}{rgb}{0.000000,0.000000,0.000000}%
\pgfsetstrokecolor{currentstroke}%
\pgfsetstrokeopacity{0.000000}%
\pgfsetdash{}{0pt}%
\pgfpathmoveto{\pgfqpoint{4.454319in}{0.294761in}}%
\pgfpathlineto{\pgfqpoint{4.731098in}{0.294761in}}%
\pgfpathlineto{\pgfqpoint{4.731098in}{0.531043in}}%
\pgfpathlineto{\pgfqpoint{4.454319in}{0.531043in}}%
\pgfpathlineto{\pgfqpoint{4.454319in}{0.294761in}}%
\pgfpathclose%
\pgfusepath{fill}%
\end{pgfscope}%
\begin{pgfscope}%
\pgfpathrectangle{\pgfqpoint{2.060179in}{0.294761in}}{\pgfqpoint{3.957943in}{2.721899in}}%
\pgfusepath{clip}%
\pgfsetbuttcap%
\pgfsetmiterjoin%
\definecolor{currentfill}{rgb}{0.887059,0.887059,0.887059}%
\pgfsetfillcolor{currentfill}%
\pgfsetlinewidth{0.000000pt}%
\definecolor{currentstroke}{rgb}{0.000000,0.000000,0.000000}%
\pgfsetstrokecolor{currentstroke}%
\pgfsetstrokeopacity{0.000000}%
\pgfsetdash{}{0pt}%
\pgfpathmoveto{\pgfqpoint{4.454319in}{0.531043in}}%
\pgfpathlineto{\pgfqpoint{4.731098in}{0.531043in}}%
\pgfpathlineto{\pgfqpoint{4.731098in}{0.791954in}}%
\pgfpathlineto{\pgfqpoint{4.454319in}{0.791954in}}%
\pgfpathlineto{\pgfqpoint{4.454319in}{0.531043in}}%
\pgfpathclose%
\pgfusepath{fill}%
\end{pgfscope}%
\begin{pgfscope}%
\pgfpathrectangle{\pgfqpoint{2.060179in}{0.294761in}}{\pgfqpoint{3.957943in}{2.721899in}}%
\pgfusepath{clip}%
\pgfsetbuttcap%
\pgfsetmiterjoin%
\definecolor{currentfill}{rgb}{0.710588,0.710588,0.710588}%
\pgfsetfillcolor{currentfill}%
\pgfsetlinewidth{0.000000pt}%
\definecolor{currentstroke}{rgb}{0.000000,0.000000,0.000000}%
\pgfsetstrokecolor{currentstroke}%
\pgfsetstrokeopacity{0.000000}%
\pgfsetdash{}{0pt}%
\pgfpathmoveto{\pgfqpoint{4.454319in}{0.791954in}}%
\pgfpathlineto{\pgfqpoint{4.731098in}{0.791954in}}%
\pgfpathlineto{\pgfqpoint{4.731098in}{1.053039in}}%
\pgfpathlineto{\pgfqpoint{4.454319in}{1.053039in}}%
\pgfpathlineto{\pgfqpoint{4.454319in}{0.791954in}}%
\pgfpathclose%
\pgfusepath{fill}%
\end{pgfscope}%
\begin{pgfscope}%
\pgfpathrectangle{\pgfqpoint{2.060179in}{0.294761in}}{\pgfqpoint{3.957943in}{2.721899in}}%
\pgfusepath{clip}%
\pgfsetbuttcap%
\pgfsetmiterjoin%
\definecolor{currentfill}{rgb}{0.478431,0.478431,0.478431}%
\pgfsetfillcolor{currentfill}%
\pgfsetlinewidth{0.000000pt}%
\definecolor{currentstroke}{rgb}{0.000000,0.000000,0.000000}%
\pgfsetstrokecolor{currentstroke}%
\pgfsetstrokeopacity{0.000000}%
\pgfsetdash{}{0pt}%
\pgfpathmoveto{\pgfqpoint{4.454319in}{1.053039in}}%
\pgfpathlineto{\pgfqpoint{4.731098in}{1.053039in}}%
\pgfpathlineto{\pgfqpoint{4.731098in}{1.303719in}}%
\pgfpathlineto{\pgfqpoint{4.454319in}{1.303719in}}%
\pgfpathlineto{\pgfqpoint{4.454319in}{1.053039in}}%
\pgfpathclose%
\pgfusepath{fill}%
\end{pgfscope}%
\begin{pgfscope}%
\pgfpathrectangle{\pgfqpoint{2.060179in}{0.294761in}}{\pgfqpoint{3.957943in}{2.721899in}}%
\pgfusepath{clip}%
\pgfsetbuttcap%
\pgfsetmiterjoin%
\definecolor{currentfill}{rgb}{1.000000,0.752941,0.796078}%
\pgfsetfillcolor{currentfill}%
\pgfsetlinewidth{0.000000pt}%
\definecolor{currentstroke}{rgb}{0.000000,0.000000,0.000000}%
\pgfsetstrokecolor{currentstroke}%
\pgfsetstrokeopacity{0.000000}%
\pgfsetdash{}{0pt}%
\pgfpathmoveto{\pgfqpoint{4.454319in}{1.303719in}}%
\pgfpathlineto{\pgfqpoint{4.731098in}{1.303719in}}%
\pgfpathlineto{\pgfqpoint{4.731098in}{1.318856in}}%
\pgfpathlineto{\pgfqpoint{4.454319in}{1.318856in}}%
\pgfpathlineto{\pgfqpoint{4.454319in}{1.303719in}}%
\pgfpathclose%
\pgfusepath{fill}%
\end{pgfscope}%
\begin{pgfscope}%
\pgfpathrectangle{\pgfqpoint{2.060179in}{0.294761in}}{\pgfqpoint{3.957943in}{2.721899in}}%
\pgfusepath{clip}%
\pgfsetbuttcap%
\pgfsetmiterjoin%
\definecolor{currentfill}{rgb}{0.854902,0.439216,0.839216}%
\pgfsetfillcolor{currentfill}%
\pgfsetlinewidth{0.000000pt}%
\definecolor{currentstroke}{rgb}{0.000000,0.000000,0.000000}%
\pgfsetstrokecolor{currentstroke}%
\pgfsetstrokeopacity{0.000000}%
\pgfsetdash{}{0pt}%
\pgfpathmoveto{\pgfqpoint{4.454319in}{1.318856in}}%
\pgfpathlineto{\pgfqpoint{4.731098in}{1.318856in}}%
\pgfpathlineto{\pgfqpoint{4.731098in}{1.321715in}}%
\pgfpathlineto{\pgfqpoint{4.454319in}{1.321715in}}%
\pgfpathlineto{\pgfqpoint{4.454319in}{1.318856in}}%
\pgfpathclose%
\pgfusepath{fill}%
\end{pgfscope}%
\begin{pgfscope}%
\pgfpathrectangle{\pgfqpoint{2.060179in}{0.294761in}}{\pgfqpoint{3.957943in}{2.721899in}}%
\pgfusepath{clip}%
\pgfsetbuttcap%
\pgfsetmiterjoin%
\definecolor{currentfill}{rgb}{0.993725,0.850196,0.704314}%
\pgfsetfillcolor{currentfill}%
\pgfsetlinewidth{0.000000pt}%
\definecolor{currentstroke}{rgb}{0.000000,0.000000,0.000000}%
\pgfsetstrokecolor{currentstroke}%
\pgfsetstrokeopacity{0.000000}%
\pgfsetdash{}{0pt}%
\pgfpathmoveto{\pgfqpoint{5.007877in}{0.294761in}}%
\pgfpathlineto{\pgfqpoint{5.284657in}{0.294761in}}%
\pgfpathlineto{\pgfqpoint{5.284657in}{0.305464in}}%
\pgfpathlineto{\pgfqpoint{5.007877in}{0.305464in}}%
\pgfpathlineto{\pgfqpoint{5.007877in}{0.294761in}}%
\pgfpathclose%
\pgfusepath{fill}%
\end{pgfscope}%
\begin{pgfscope}%
\pgfpathrectangle{\pgfqpoint{2.060179in}{0.294761in}}{\pgfqpoint{3.957943in}{2.721899in}}%
\pgfusepath{clip}%
\pgfsetbuttcap%
\pgfsetmiterjoin%
\definecolor{currentfill}{rgb}{0.992157,0.710065,0.464437}%
\pgfsetfillcolor{currentfill}%
\pgfsetlinewidth{0.000000pt}%
\definecolor{currentstroke}{rgb}{0.000000,0.000000,0.000000}%
\pgfsetstrokecolor{currentstroke}%
\pgfsetstrokeopacity{0.000000}%
\pgfsetdash{}{0pt}%
\pgfpathmoveto{\pgfqpoint{5.007877in}{0.305464in}}%
\pgfpathlineto{\pgfqpoint{5.284657in}{0.305464in}}%
\pgfpathlineto{\pgfqpoint{5.284657in}{0.366332in}}%
\pgfpathlineto{\pgfqpoint{5.007877in}{0.366332in}}%
\pgfpathlineto{\pgfqpoint{5.007877in}{0.305464in}}%
\pgfpathclose%
\pgfusepath{fill}%
\end{pgfscope}%
\begin{pgfscope}%
\pgfpathrectangle{\pgfqpoint{2.060179in}{0.294761in}}{\pgfqpoint{3.957943in}{2.721899in}}%
\pgfusepath{clip}%
\pgfsetbuttcap%
\pgfsetmiterjoin%
\definecolor{currentfill}{rgb}{0.991419,0.550727,0.232772}%
\pgfsetfillcolor{currentfill}%
\pgfsetlinewidth{0.000000pt}%
\definecolor{currentstroke}{rgb}{0.000000,0.000000,0.000000}%
\pgfsetstrokecolor{currentstroke}%
\pgfsetstrokeopacity{0.000000}%
\pgfsetdash{}{0pt}%
\pgfpathmoveto{\pgfqpoint{5.007877in}{0.366332in}}%
\pgfpathlineto{\pgfqpoint{5.284657in}{0.366332in}}%
\pgfpathlineto{\pgfqpoint{5.284657in}{0.584410in}}%
\pgfpathlineto{\pgfqpoint{5.007877in}{0.584410in}}%
\pgfpathlineto{\pgfqpoint{5.007877in}{0.366332in}}%
\pgfpathclose%
\pgfusepath{fill}%
\end{pgfscope}%
\begin{pgfscope}%
\pgfpathrectangle{\pgfqpoint{2.060179in}{0.294761in}}{\pgfqpoint{3.957943in}{2.721899in}}%
\pgfusepath{clip}%
\pgfsetbuttcap%
\pgfsetmiterjoin%
\definecolor{currentfill}{rgb}{0.925536,0.384867,0.059839}%
\pgfsetfillcolor{currentfill}%
\pgfsetlinewidth{0.000000pt}%
\definecolor{currentstroke}{rgb}{0.000000,0.000000,0.000000}%
\pgfsetstrokecolor{currentstroke}%
\pgfsetstrokeopacity{0.000000}%
\pgfsetdash{}{0pt}%
\pgfpathmoveto{\pgfqpoint{5.007877in}{0.584410in}}%
\pgfpathlineto{\pgfqpoint{5.284657in}{0.584410in}}%
\pgfpathlineto{\pgfqpoint{5.284657in}{0.594303in}}%
\pgfpathlineto{\pgfqpoint{5.007877in}{0.594303in}}%
\pgfpathlineto{\pgfqpoint{5.007877in}{0.584410in}}%
\pgfpathclose%
\pgfusepath{fill}%
\end{pgfscope}%
\begin{pgfscope}%
\pgfpathrectangle{\pgfqpoint{2.060179in}{0.294761in}}{\pgfqpoint{3.957943in}{2.721899in}}%
\pgfusepath{clip}%
\pgfsetbuttcap%
\pgfsetmiterjoin%
\definecolor{currentfill}{rgb}{0.887059,0.887059,0.887059}%
\pgfsetfillcolor{currentfill}%
\pgfsetlinewidth{0.000000pt}%
\definecolor{currentstroke}{rgb}{0.000000,0.000000,0.000000}%
\pgfsetstrokecolor{currentstroke}%
\pgfsetstrokeopacity{0.000000}%
\pgfsetdash{}{0pt}%
\pgfpathmoveto{\pgfqpoint{5.007877in}{0.594303in}}%
\pgfpathlineto{\pgfqpoint{5.284657in}{0.594303in}}%
\pgfpathlineto{\pgfqpoint{5.284657in}{0.913296in}}%
\pgfpathlineto{\pgfqpoint{5.007877in}{0.913296in}}%
\pgfpathlineto{\pgfqpoint{5.007877in}{0.594303in}}%
\pgfpathclose%
\pgfusepath{fill}%
\end{pgfscope}%
\begin{pgfscope}%
\pgfpathrectangle{\pgfqpoint{2.060179in}{0.294761in}}{\pgfqpoint{3.957943in}{2.721899in}}%
\pgfusepath{clip}%
\pgfsetbuttcap%
\pgfsetmiterjoin%
\definecolor{currentfill}{rgb}{0.710588,0.710588,0.710588}%
\pgfsetfillcolor{currentfill}%
\pgfsetlinewidth{0.000000pt}%
\definecolor{currentstroke}{rgb}{0.000000,0.000000,0.000000}%
\pgfsetstrokecolor{currentstroke}%
\pgfsetstrokeopacity{0.000000}%
\pgfsetdash{}{0pt}%
\pgfpathmoveto{\pgfqpoint{5.007877in}{0.913296in}}%
\pgfpathlineto{\pgfqpoint{5.284657in}{0.913296in}}%
\pgfpathlineto{\pgfqpoint{5.284657in}{1.226070in}}%
\pgfpathlineto{\pgfqpoint{5.007877in}{1.226070in}}%
\pgfpathlineto{\pgfqpoint{5.007877in}{0.913296in}}%
\pgfpathclose%
\pgfusepath{fill}%
\end{pgfscope}%
\begin{pgfscope}%
\pgfpathrectangle{\pgfqpoint{2.060179in}{0.294761in}}{\pgfqpoint{3.957943in}{2.721899in}}%
\pgfusepath{clip}%
\pgfsetbuttcap%
\pgfsetmiterjoin%
\definecolor{currentfill}{rgb}{0.478431,0.478431,0.478431}%
\pgfsetfillcolor{currentfill}%
\pgfsetlinewidth{0.000000pt}%
\definecolor{currentstroke}{rgb}{0.000000,0.000000,0.000000}%
\pgfsetstrokecolor{currentstroke}%
\pgfsetstrokeopacity{0.000000}%
\pgfsetdash{}{0pt}%
\pgfpathmoveto{\pgfqpoint{5.007877in}{1.226070in}}%
\pgfpathlineto{\pgfqpoint{5.284657in}{1.226070in}}%
\pgfpathlineto{\pgfqpoint{5.284657in}{1.498117in}}%
\pgfpathlineto{\pgfqpoint{5.007877in}{1.498117in}}%
\pgfpathlineto{\pgfqpoint{5.007877in}{1.226070in}}%
\pgfpathclose%
\pgfusepath{fill}%
\end{pgfscope}%
\begin{pgfscope}%
\pgfpathrectangle{\pgfqpoint{2.060179in}{0.294761in}}{\pgfqpoint{3.957943in}{2.721899in}}%
\pgfusepath{clip}%
\pgfsetbuttcap%
\pgfsetmiterjoin%
\definecolor{currentfill}{rgb}{1.000000,0.752941,0.796078}%
\pgfsetfillcolor{currentfill}%
\pgfsetlinewidth{0.000000pt}%
\definecolor{currentstroke}{rgb}{0.000000,0.000000,0.000000}%
\pgfsetstrokecolor{currentstroke}%
\pgfsetstrokeopacity{0.000000}%
\pgfsetdash{}{0pt}%
\pgfpathmoveto{\pgfqpoint{5.007877in}{1.498117in}}%
\pgfpathlineto{\pgfqpoint{5.284657in}{1.498117in}}%
\pgfpathlineto{\pgfqpoint{5.284657in}{1.529590in}}%
\pgfpathlineto{\pgfqpoint{5.007877in}{1.529590in}}%
\pgfpathlineto{\pgfqpoint{5.007877in}{1.498117in}}%
\pgfpathclose%
\pgfusepath{fill}%
\end{pgfscope}%
\begin{pgfscope}%
\pgfpathrectangle{\pgfqpoint{2.060179in}{0.294761in}}{\pgfqpoint{3.957943in}{2.721899in}}%
\pgfusepath{clip}%
\pgfsetbuttcap%
\pgfsetmiterjoin%
\definecolor{currentfill}{rgb}{0.854902,0.439216,0.839216}%
\pgfsetfillcolor{currentfill}%
\pgfsetlinewidth{0.000000pt}%
\definecolor{currentstroke}{rgb}{0.000000,0.000000,0.000000}%
\pgfsetstrokecolor{currentstroke}%
\pgfsetstrokeopacity{0.000000}%
\pgfsetdash{}{0pt}%
\pgfpathmoveto{\pgfqpoint{5.007877in}{1.529590in}}%
\pgfpathlineto{\pgfqpoint{5.284657in}{1.529590in}}%
\pgfpathlineto{\pgfqpoint{5.284657in}{1.542749in}}%
\pgfpathlineto{\pgfqpoint{5.007877in}{1.542749in}}%
\pgfpathlineto{\pgfqpoint{5.007877in}{1.529590in}}%
\pgfpathclose%
\pgfusepath{fill}%
\end{pgfscope}%
\begin{pgfscope}%
\pgfpathrectangle{\pgfqpoint{2.060179in}{0.294761in}}{\pgfqpoint{3.957943in}{2.721899in}}%
\pgfusepath{clip}%
\pgfsetbuttcap%
\pgfsetmiterjoin%
\definecolor{currentfill}{rgb}{0.993725,0.850196,0.704314}%
\pgfsetfillcolor{currentfill}%
\pgfsetlinewidth{0.000000pt}%
\definecolor{currentstroke}{rgb}{0.000000,0.000000,0.000000}%
\pgfsetstrokecolor{currentstroke}%
\pgfsetstrokeopacity{0.000000}%
\pgfsetdash{}{0pt}%
\pgfpathmoveto{\pgfqpoint{5.561436in}{0.294761in}}%
\pgfpathlineto{\pgfqpoint{5.838215in}{0.294761in}}%
\pgfpathlineto{\pgfqpoint{5.838215in}{0.307965in}}%
\pgfpathlineto{\pgfqpoint{5.561436in}{0.307965in}}%
\pgfpathlineto{\pgfqpoint{5.561436in}{0.294761in}}%
\pgfpathclose%
\pgfusepath{fill}%
\end{pgfscope}%
\begin{pgfscope}%
\pgfpathrectangle{\pgfqpoint{2.060179in}{0.294761in}}{\pgfqpoint{3.957943in}{2.721899in}}%
\pgfusepath{clip}%
\pgfsetbuttcap%
\pgfsetmiterjoin%
\definecolor{currentfill}{rgb}{0.992157,0.710065,0.464437}%
\pgfsetfillcolor{currentfill}%
\pgfsetlinewidth{0.000000pt}%
\definecolor{currentstroke}{rgb}{0.000000,0.000000,0.000000}%
\pgfsetstrokecolor{currentstroke}%
\pgfsetstrokeopacity{0.000000}%
\pgfsetdash{}{0pt}%
\pgfpathmoveto{\pgfqpoint{5.561436in}{0.307965in}}%
\pgfpathlineto{\pgfqpoint{5.838215in}{0.307965in}}%
\pgfpathlineto{\pgfqpoint{5.838215in}{0.336346in}}%
\pgfpathlineto{\pgfqpoint{5.561436in}{0.336346in}}%
\pgfpathlineto{\pgfqpoint{5.561436in}{0.307965in}}%
\pgfpathclose%
\pgfusepath{fill}%
\end{pgfscope}%
\begin{pgfscope}%
\pgfpathrectangle{\pgfqpoint{2.060179in}{0.294761in}}{\pgfqpoint{3.957943in}{2.721899in}}%
\pgfusepath{clip}%
\pgfsetbuttcap%
\pgfsetmiterjoin%
\definecolor{currentfill}{rgb}{0.991419,0.550727,0.232772}%
\pgfsetfillcolor{currentfill}%
\pgfsetlinewidth{0.000000pt}%
\definecolor{currentstroke}{rgb}{0.000000,0.000000,0.000000}%
\pgfsetstrokecolor{currentstroke}%
\pgfsetstrokeopacity{0.000000}%
\pgfsetdash{}{0pt}%
\pgfpathmoveto{\pgfqpoint{5.561436in}{0.336346in}}%
\pgfpathlineto{\pgfqpoint{5.838215in}{0.336346in}}%
\pgfpathlineto{\pgfqpoint{5.838215in}{2.646643in}}%
\pgfpathlineto{\pgfqpoint{5.561436in}{2.646643in}}%
\pgfpathlineto{\pgfqpoint{5.561436in}{0.336346in}}%
\pgfpathclose%
\pgfusepath{fill}%
\end{pgfscope}%
\begin{pgfscope}%
\pgfpathrectangle{\pgfqpoint{2.060179in}{0.294761in}}{\pgfqpoint{3.957943in}{2.721899in}}%
\pgfusepath{clip}%
\pgfsetbuttcap%
\pgfsetmiterjoin%
\definecolor{currentfill}{rgb}{0.854902,0.439216,0.839216}%
\pgfsetfillcolor{currentfill}%
\pgfsetlinewidth{0.000000pt}%
\definecolor{currentstroke}{rgb}{0.000000,0.000000,0.000000}%
\pgfsetstrokecolor{currentstroke}%
\pgfsetstrokeopacity{0.000000}%
\pgfsetdash{}{0pt}%
\pgfpathmoveto{\pgfqpoint{5.561436in}{2.646643in}}%
\pgfpathlineto{\pgfqpoint{5.838215in}{2.646643in}}%
\pgfpathlineto{\pgfqpoint{5.838215in}{2.651383in}}%
\pgfpathlineto{\pgfqpoint{5.561436in}{2.651383in}}%
\pgfpathlineto{\pgfqpoint{5.561436in}{2.646643in}}%
\pgfpathclose%
\pgfusepath{fill}%
\end{pgfscope}%
\begin{pgfscope}%
\pgfsetbuttcap%
\pgfsetroundjoin%
\definecolor{currentfill}{rgb}{0.000000,0.000000,0.000000}%
\pgfsetfillcolor{currentfill}%
\pgfsetlinewidth{0.803000pt}%
\definecolor{currentstroke}{rgb}{0.000000,0.000000,0.000000}%
\pgfsetstrokecolor{currentstroke}%
\pgfsetdash{}{0pt}%
\pgfsys@defobject{currentmarker}{\pgfqpoint{0.000000in}{-0.048611in}}{\pgfqpoint{0.000000in}{0.000000in}}{%
\pgfpathmoveto{\pgfqpoint{0.000000in}{0.000000in}}%
\pgfpathlineto{\pgfqpoint{0.000000in}{-0.048611in}}%
\pgfusepath{stroke,fill}%
}%
\begin{pgfscope}%
\pgfsys@transformshift{2.378475in}{0.294761in}%
\pgfsys@useobject{currentmarker}{}%
\end{pgfscope}%
\end{pgfscope}%
\begin{pgfscope}%
\definecolor{textcolor}{rgb}{0.000000,0.000000,0.000000}%
\pgfsetstrokecolor{textcolor}%
\pgfsetfillcolor{textcolor}%
\pgftext[x=2.378475in,y=0.197539in,,top]{\color{textcolor}{\rmfamily\fontsize{6.940000}{8.328000}\selectfont\catcode`\^=\active\def^{\ifmmode\sp\else\^{}\fi}\catcode`\%=\active\def%{\%}PT}}%
\end{pgfscope}%
\begin{pgfscope}%
\pgfsetbuttcap%
\pgfsetroundjoin%
\definecolor{currentfill}{rgb}{0.000000,0.000000,0.000000}%
\pgfsetfillcolor{currentfill}%
\pgfsetlinewidth{0.803000pt}%
\definecolor{currentstroke}{rgb}{0.000000,0.000000,0.000000}%
\pgfsetstrokecolor{currentstroke}%
\pgfsetdash{}{0pt}%
\pgfsys@defobject{currentmarker}{\pgfqpoint{0.000000in}{-0.048611in}}{\pgfqpoint{0.000000in}{0.000000in}}{%
\pgfpathmoveto{\pgfqpoint{0.000000in}{0.000000in}}%
\pgfpathlineto{\pgfqpoint{0.000000in}{-0.048611in}}%
\pgfusepath{stroke,fill}%
}%
\begin{pgfscope}%
\pgfsys@transformshift{2.932033in}{0.294761in}%
\pgfsys@useobject{currentmarker}{}%
\end{pgfscope}%
\end{pgfscope}%
\begin{pgfscope}%
\definecolor{textcolor}{rgb}{0.000000,0.000000,0.000000}%
\pgfsetstrokecolor{textcolor}%
\pgfsetfillcolor{textcolor}%
\pgftext[x=2.932033in,y=0.197539in,,top]{\color{textcolor}{\rmfamily\fontsize{6.940000}{8.328000}\selectfont\catcode`\^=\active\def^{\ifmmode\sp\else\^{}\fi}\catcode`\%=\active\def%{\%}NRC+PT}}%
\end{pgfscope}%
\begin{pgfscope}%
\pgfsetbuttcap%
\pgfsetroundjoin%
\definecolor{currentfill}{rgb}{0.000000,0.000000,0.000000}%
\pgfsetfillcolor{currentfill}%
\pgfsetlinewidth{0.803000pt}%
\definecolor{currentstroke}{rgb}{0.000000,0.000000,0.000000}%
\pgfsetstrokecolor{currentstroke}%
\pgfsetdash{}{0pt}%
\pgfsys@defobject{currentmarker}{\pgfqpoint{0.000000in}{-0.048611in}}{\pgfqpoint{0.000000in}{0.000000in}}{%
\pgfpathmoveto{\pgfqpoint{0.000000in}{0.000000in}}%
\pgfpathlineto{\pgfqpoint{0.000000in}{-0.048611in}}%
\pgfusepath{stroke,fill}%
}%
\begin{pgfscope}%
\pgfsys@transformshift{3.485592in}{0.294761in}%
\pgfsys@useobject{currentmarker}{}%
\end{pgfscope}%
\end{pgfscope}%
\begin{pgfscope}%
\definecolor{textcolor}{rgb}{0.000000,0.000000,0.000000}%
\pgfsetstrokecolor{textcolor}%
\pgfsetfillcolor{textcolor}%
\pgftext[x=3.485592in,y=0.197539in,,top]{\color{textcolor}{\rmfamily\fontsize{6.940000}{8.328000}\selectfont\catcode`\^=\active\def^{\ifmmode\sp\else\^{}\fi}\catcode`\%=\active\def%{\%}NRC+PT+SL}}%
\end{pgfscope}%
\begin{pgfscope}%
\pgfsetbuttcap%
\pgfsetroundjoin%
\definecolor{currentfill}{rgb}{0.000000,0.000000,0.000000}%
\pgfsetfillcolor{currentfill}%
\pgfsetlinewidth{0.803000pt}%
\definecolor{currentstroke}{rgb}{0.000000,0.000000,0.000000}%
\pgfsetstrokecolor{currentstroke}%
\pgfsetdash{}{0pt}%
\pgfsys@defobject{currentmarker}{\pgfqpoint{0.000000in}{-0.048611in}}{\pgfqpoint{0.000000in}{0.000000in}}{%
\pgfpathmoveto{\pgfqpoint{0.000000in}{0.000000in}}%
\pgfpathlineto{\pgfqpoint{0.000000in}{-0.048611in}}%
\pgfusepath{stroke,fill}%
}%
\begin{pgfscope}%
\pgfsys@transformshift{4.039150in}{0.294761in}%
\pgfsys@useobject{currentmarker}{}%
\end{pgfscope}%
\end{pgfscope}%
\begin{pgfscope}%
\definecolor{textcolor}{rgb}{0.000000,0.000000,0.000000}%
\pgfsetstrokecolor{textcolor}%
\pgfsetfillcolor{textcolor}%
\pgftext[x=4.039150in,y=0.197539in,,top]{\color{textcolor}{\rmfamily\fontsize{6.940000}{8.328000}\selectfont\catcode`\^=\active\def^{\ifmmode\sp\else\^{}\fi}\catcode`\%=\active\def%{\%}NRC+BT}}%
\end{pgfscope}%
\begin{pgfscope}%
\pgfsetbuttcap%
\pgfsetroundjoin%
\definecolor{currentfill}{rgb}{0.000000,0.000000,0.000000}%
\pgfsetfillcolor{currentfill}%
\pgfsetlinewidth{0.803000pt}%
\definecolor{currentstroke}{rgb}{0.000000,0.000000,0.000000}%
\pgfsetstrokecolor{currentstroke}%
\pgfsetdash{}{0pt}%
\pgfsys@defobject{currentmarker}{\pgfqpoint{0.000000in}{-0.048611in}}{\pgfqpoint{0.000000in}{0.000000in}}{%
\pgfpathmoveto{\pgfqpoint{0.000000in}{0.000000in}}%
\pgfpathlineto{\pgfqpoint{0.000000in}{-0.048611in}}%
\pgfusepath{stroke,fill}%
}%
\begin{pgfscope}%
\pgfsys@transformshift{4.592709in}{0.294761in}%
\pgfsys@useobject{currentmarker}{}%
\end{pgfscope}%
\end{pgfscope}%
\begin{pgfscope}%
\definecolor{textcolor}{rgb}{0.000000,0.000000,0.000000}%
\pgfsetstrokecolor{textcolor}%
\pgfsetfillcolor{textcolor}%
\pgftext[x=4.592709in,y=0.197539in,,top]{\color{textcolor}{\rmfamily\fontsize{6.940000}{8.328000}\selectfont\catcode`\^=\active\def^{\ifmmode\sp\else\^{}\fi}\catcode`\%=\active\def%{\%}NRC+LT}}%
\end{pgfscope}%
\begin{pgfscope}%
\pgfsetbuttcap%
\pgfsetroundjoin%
\definecolor{currentfill}{rgb}{0.000000,0.000000,0.000000}%
\pgfsetfillcolor{currentfill}%
\pgfsetlinewidth{0.803000pt}%
\definecolor{currentstroke}{rgb}{0.000000,0.000000,0.000000}%
\pgfsetstrokecolor{currentstroke}%
\pgfsetdash{}{0pt}%
\pgfsys@defobject{currentmarker}{\pgfqpoint{0.000000in}{-0.048611in}}{\pgfqpoint{0.000000in}{0.000000in}}{%
\pgfpathmoveto{\pgfqpoint{0.000000in}{0.000000in}}%
\pgfpathlineto{\pgfqpoint{0.000000in}{-0.048611in}}%
\pgfusepath{stroke,fill}%
}%
\begin{pgfscope}%
\pgfsys@transformshift{5.146267in}{0.294761in}%
\pgfsys@useobject{currentmarker}{}%
\end{pgfscope}%
\end{pgfscope}%
\begin{pgfscope}%
\definecolor{textcolor}{rgb}{0.000000,0.000000,0.000000}%
\pgfsetstrokecolor{textcolor}%
\pgfsetfillcolor{textcolor}%
\pgftext[x=5.146267in,y=0.197539in,,top]{\color{textcolor}{\rmfamily\fontsize{6.940000}{8.328000}\selectfont\catcode`\^=\active\def^{\ifmmode\sp\else\^{}\fi}\catcode`\%=\active\def%{\%}NRC+SPPC}}%
\end{pgfscope}%
\begin{pgfscope}%
\pgfsetbuttcap%
\pgfsetroundjoin%
\definecolor{currentfill}{rgb}{0.000000,0.000000,0.000000}%
\pgfsetfillcolor{currentfill}%
\pgfsetlinewidth{0.803000pt}%
\definecolor{currentstroke}{rgb}{0.000000,0.000000,0.000000}%
\pgfsetstrokecolor{currentstroke}%
\pgfsetdash{}{0pt}%
\pgfsys@defobject{currentmarker}{\pgfqpoint{0.000000in}{-0.048611in}}{\pgfqpoint{0.000000in}{0.000000in}}{%
\pgfpathmoveto{\pgfqpoint{0.000000in}{0.000000in}}%
\pgfpathlineto{\pgfqpoint{0.000000in}{-0.048611in}}%
\pgfusepath{stroke,fill}%
}%
\begin{pgfscope}%
\pgfsys@transformshift{5.699826in}{0.294761in}%
\pgfsys@useobject{currentmarker}{}%
\end{pgfscope}%
\end{pgfscope}%
\begin{pgfscope}%
\definecolor{textcolor}{rgb}{0.000000,0.000000,0.000000}%
\pgfsetstrokecolor{textcolor}%
\pgfsetfillcolor{textcolor}%
\pgftext[x=5.699826in,y=0.197539in,,top]{\color{textcolor}{\rmfamily\fontsize{6.940000}{8.328000}\selectfont\catcode`\^=\active\def^{\ifmmode\sp\else\^{}\fi}\catcode`\%=\active\def%{\%}PM ($1/4$)}}%
\end{pgfscope}%
\begin{pgfscope}%
\pgfsetbuttcap%
\pgfsetroundjoin%
\definecolor{currentfill}{rgb}{0.000000,0.000000,0.000000}%
\pgfsetfillcolor{currentfill}%
\pgfsetlinewidth{0.803000pt}%
\definecolor{currentstroke}{rgb}{0.000000,0.000000,0.000000}%
\pgfsetstrokecolor{currentstroke}%
\pgfsetdash{}{0pt}%
\pgfsys@defobject{currentmarker}{\pgfqpoint{-0.048611in}{0.000000in}}{\pgfqpoint{-0.000000in}{0.000000in}}{%
\pgfpathmoveto{\pgfqpoint{-0.000000in}{0.000000in}}%
\pgfpathlineto{\pgfqpoint{-0.048611in}{0.000000in}}%
\pgfusepath{stroke,fill}%
}%
\begin{pgfscope}%
\pgfsys@transformshift{2.060179in}{0.294761in}%
\pgfsys@useobject{currentmarker}{}%
\end{pgfscope}%
\end{pgfscope}%
\begin{pgfscope}%
\definecolor{textcolor}{rgb}{0.000000,0.000000,0.000000}%
\pgfsetstrokecolor{textcolor}%
\pgfsetfillcolor{textcolor}%
\pgftext[x=1.893512in, y=0.241999in, left, base]{\color{textcolor}{\rmfamily\fontsize{10.000000}{12.000000}\selectfont\catcode`\^=\active\def^{\ifmmode\sp\else\^{}\fi}\catcode`\%=\active\def%{\%}$\mathdefault{0}$}}%
\end{pgfscope}%
\begin{pgfscope}%
\pgfsetbuttcap%
\pgfsetroundjoin%
\definecolor{currentfill}{rgb}{0.000000,0.000000,0.000000}%
\pgfsetfillcolor{currentfill}%
\pgfsetlinewidth{0.803000pt}%
\definecolor{currentstroke}{rgb}{0.000000,0.000000,0.000000}%
\pgfsetstrokecolor{currentstroke}%
\pgfsetdash{}{0pt}%
\pgfsys@defobject{currentmarker}{\pgfqpoint{-0.048611in}{0.000000in}}{\pgfqpoint{-0.000000in}{0.000000in}}{%
\pgfpathmoveto{\pgfqpoint{-0.000000in}{0.000000in}}%
\pgfpathlineto{\pgfqpoint{-0.048611in}{0.000000in}}%
\pgfusepath{stroke,fill}%
}%
\begin{pgfscope}%
\pgfsys@transformshift{2.060179in}{0.613637in}%
\pgfsys@useobject{currentmarker}{}%
\end{pgfscope}%
\end{pgfscope}%
\begin{pgfscope}%
\definecolor{textcolor}{rgb}{0.000000,0.000000,0.000000}%
\pgfsetstrokecolor{textcolor}%
\pgfsetfillcolor{textcolor}%
\pgftext[x=1.893512in, y=0.560876in, left, base]{\color{textcolor}{\rmfamily\fontsize{10.000000}{12.000000}\selectfont\catcode`\^=\active\def^{\ifmmode\sp\else\^{}\fi}\catcode`\%=\active\def%{\%}$\mathdefault{5}$}}%
\end{pgfscope}%
\begin{pgfscope}%
\pgfsetbuttcap%
\pgfsetroundjoin%
\definecolor{currentfill}{rgb}{0.000000,0.000000,0.000000}%
\pgfsetfillcolor{currentfill}%
\pgfsetlinewidth{0.803000pt}%
\definecolor{currentstroke}{rgb}{0.000000,0.000000,0.000000}%
\pgfsetstrokecolor{currentstroke}%
\pgfsetdash{}{0pt}%
\pgfsys@defobject{currentmarker}{\pgfqpoint{-0.048611in}{0.000000in}}{\pgfqpoint{-0.000000in}{0.000000in}}{%
\pgfpathmoveto{\pgfqpoint{-0.000000in}{0.000000in}}%
\pgfpathlineto{\pgfqpoint{-0.048611in}{0.000000in}}%
\pgfusepath{stroke,fill}%
}%
\begin{pgfscope}%
\pgfsys@transformshift{2.060179in}{0.932514in}%
\pgfsys@useobject{currentmarker}{}%
\end{pgfscope}%
\end{pgfscope}%
\begin{pgfscope}%
\definecolor{textcolor}{rgb}{0.000000,0.000000,0.000000}%
\pgfsetstrokecolor{textcolor}%
\pgfsetfillcolor{textcolor}%
\pgftext[x=1.824067in, y=0.879752in, left, base]{\color{textcolor}{\rmfamily\fontsize{10.000000}{12.000000}\selectfont\catcode`\^=\active\def^{\ifmmode\sp\else\^{}\fi}\catcode`\%=\active\def%{\%}$\mathdefault{10}$}}%
\end{pgfscope}%
\begin{pgfscope}%
\pgfsetbuttcap%
\pgfsetroundjoin%
\definecolor{currentfill}{rgb}{0.000000,0.000000,0.000000}%
\pgfsetfillcolor{currentfill}%
\pgfsetlinewidth{0.803000pt}%
\definecolor{currentstroke}{rgb}{0.000000,0.000000,0.000000}%
\pgfsetstrokecolor{currentstroke}%
\pgfsetdash{}{0pt}%
\pgfsys@defobject{currentmarker}{\pgfqpoint{-0.048611in}{0.000000in}}{\pgfqpoint{-0.000000in}{0.000000in}}{%
\pgfpathmoveto{\pgfqpoint{-0.000000in}{0.000000in}}%
\pgfpathlineto{\pgfqpoint{-0.048611in}{0.000000in}}%
\pgfusepath{stroke,fill}%
}%
\begin{pgfscope}%
\pgfsys@transformshift{2.060179in}{1.251391in}%
\pgfsys@useobject{currentmarker}{}%
\end{pgfscope}%
\end{pgfscope}%
\begin{pgfscope}%
\definecolor{textcolor}{rgb}{0.000000,0.000000,0.000000}%
\pgfsetstrokecolor{textcolor}%
\pgfsetfillcolor{textcolor}%
\pgftext[x=1.824067in, y=1.198629in, left, base]{\color{textcolor}{\rmfamily\fontsize{10.000000}{12.000000}\selectfont\catcode`\^=\active\def^{\ifmmode\sp\else\^{}\fi}\catcode`\%=\active\def%{\%}$\mathdefault{15}$}}%
\end{pgfscope}%
\begin{pgfscope}%
\pgfsetbuttcap%
\pgfsetroundjoin%
\definecolor{currentfill}{rgb}{0.000000,0.000000,0.000000}%
\pgfsetfillcolor{currentfill}%
\pgfsetlinewidth{0.803000pt}%
\definecolor{currentstroke}{rgb}{0.000000,0.000000,0.000000}%
\pgfsetstrokecolor{currentstroke}%
\pgfsetdash{}{0pt}%
\pgfsys@defobject{currentmarker}{\pgfqpoint{-0.048611in}{0.000000in}}{\pgfqpoint{-0.000000in}{0.000000in}}{%
\pgfpathmoveto{\pgfqpoint{-0.000000in}{0.000000in}}%
\pgfpathlineto{\pgfqpoint{-0.048611in}{0.000000in}}%
\pgfusepath{stroke,fill}%
}%
\begin{pgfscope}%
\pgfsys@transformshift{2.060179in}{1.570267in}%
\pgfsys@useobject{currentmarker}{}%
\end{pgfscope}%
\end{pgfscope}%
\begin{pgfscope}%
\definecolor{textcolor}{rgb}{0.000000,0.000000,0.000000}%
\pgfsetstrokecolor{textcolor}%
\pgfsetfillcolor{textcolor}%
\pgftext[x=1.824067in, y=1.517506in, left, base]{\color{textcolor}{\rmfamily\fontsize{10.000000}{12.000000}\selectfont\catcode`\^=\active\def^{\ifmmode\sp\else\^{}\fi}\catcode`\%=\active\def%{\%}$\mathdefault{20}$}}%
\end{pgfscope}%
\begin{pgfscope}%
\pgfsetbuttcap%
\pgfsetroundjoin%
\definecolor{currentfill}{rgb}{0.000000,0.000000,0.000000}%
\pgfsetfillcolor{currentfill}%
\pgfsetlinewidth{0.803000pt}%
\definecolor{currentstroke}{rgb}{0.000000,0.000000,0.000000}%
\pgfsetstrokecolor{currentstroke}%
\pgfsetdash{}{0pt}%
\pgfsys@defobject{currentmarker}{\pgfqpoint{-0.048611in}{0.000000in}}{\pgfqpoint{-0.000000in}{0.000000in}}{%
\pgfpathmoveto{\pgfqpoint{-0.000000in}{0.000000in}}%
\pgfpathlineto{\pgfqpoint{-0.048611in}{0.000000in}}%
\pgfusepath{stroke,fill}%
}%
\begin{pgfscope}%
\pgfsys@transformshift{2.060179in}{1.889144in}%
\pgfsys@useobject{currentmarker}{}%
\end{pgfscope}%
\end{pgfscope}%
\begin{pgfscope}%
\definecolor{textcolor}{rgb}{0.000000,0.000000,0.000000}%
\pgfsetstrokecolor{textcolor}%
\pgfsetfillcolor{textcolor}%
\pgftext[x=1.824067in, y=1.836382in, left, base]{\color{textcolor}{\rmfamily\fontsize{10.000000}{12.000000}\selectfont\catcode`\^=\active\def^{\ifmmode\sp\else\^{}\fi}\catcode`\%=\active\def%{\%}$\mathdefault{25}$}}%
\end{pgfscope}%
\begin{pgfscope}%
\pgfsetbuttcap%
\pgfsetroundjoin%
\definecolor{currentfill}{rgb}{0.000000,0.000000,0.000000}%
\pgfsetfillcolor{currentfill}%
\pgfsetlinewidth{0.803000pt}%
\definecolor{currentstroke}{rgb}{0.000000,0.000000,0.000000}%
\pgfsetstrokecolor{currentstroke}%
\pgfsetdash{}{0pt}%
\pgfsys@defobject{currentmarker}{\pgfqpoint{-0.048611in}{0.000000in}}{\pgfqpoint{-0.000000in}{0.000000in}}{%
\pgfpathmoveto{\pgfqpoint{-0.000000in}{0.000000in}}%
\pgfpathlineto{\pgfqpoint{-0.048611in}{0.000000in}}%
\pgfusepath{stroke,fill}%
}%
\begin{pgfscope}%
\pgfsys@transformshift{2.060179in}{2.208020in}%
\pgfsys@useobject{currentmarker}{}%
\end{pgfscope}%
\end{pgfscope}%
\begin{pgfscope}%
\definecolor{textcolor}{rgb}{0.000000,0.000000,0.000000}%
\pgfsetstrokecolor{textcolor}%
\pgfsetfillcolor{textcolor}%
\pgftext[x=1.824067in, y=2.155259in, left, base]{\color{textcolor}{\rmfamily\fontsize{10.000000}{12.000000}\selectfont\catcode`\^=\active\def^{\ifmmode\sp\else\^{}\fi}\catcode`\%=\active\def%{\%}$\mathdefault{30}$}}%
\end{pgfscope}%
\begin{pgfscope}%
\pgfsetbuttcap%
\pgfsetroundjoin%
\definecolor{currentfill}{rgb}{0.000000,0.000000,0.000000}%
\pgfsetfillcolor{currentfill}%
\pgfsetlinewidth{0.803000pt}%
\definecolor{currentstroke}{rgb}{0.000000,0.000000,0.000000}%
\pgfsetstrokecolor{currentstroke}%
\pgfsetdash{}{0pt}%
\pgfsys@defobject{currentmarker}{\pgfqpoint{-0.048611in}{0.000000in}}{\pgfqpoint{-0.000000in}{0.000000in}}{%
\pgfpathmoveto{\pgfqpoint{-0.000000in}{0.000000in}}%
\pgfpathlineto{\pgfqpoint{-0.048611in}{0.000000in}}%
\pgfusepath{stroke,fill}%
}%
\begin{pgfscope}%
\pgfsys@transformshift{2.060179in}{2.526897in}%
\pgfsys@useobject{currentmarker}{}%
\end{pgfscope}%
\end{pgfscope}%
\begin{pgfscope}%
\definecolor{textcolor}{rgb}{0.000000,0.000000,0.000000}%
\pgfsetstrokecolor{textcolor}%
\pgfsetfillcolor{textcolor}%
\pgftext[x=1.824067in, y=2.474135in, left, base]{\color{textcolor}{\rmfamily\fontsize{10.000000}{12.000000}\selectfont\catcode`\^=\active\def^{\ifmmode\sp\else\^{}\fi}\catcode`\%=\active\def%{\%}$\mathdefault{35}$}}%
\end{pgfscope}%
\begin{pgfscope}%
\pgfsetbuttcap%
\pgfsetroundjoin%
\definecolor{currentfill}{rgb}{0.000000,0.000000,0.000000}%
\pgfsetfillcolor{currentfill}%
\pgfsetlinewidth{0.803000pt}%
\definecolor{currentstroke}{rgb}{0.000000,0.000000,0.000000}%
\pgfsetstrokecolor{currentstroke}%
\pgfsetdash{}{0pt}%
\pgfsys@defobject{currentmarker}{\pgfqpoint{-0.048611in}{0.000000in}}{\pgfqpoint{-0.000000in}{0.000000in}}{%
\pgfpathmoveto{\pgfqpoint{-0.000000in}{0.000000in}}%
\pgfpathlineto{\pgfqpoint{-0.048611in}{0.000000in}}%
\pgfusepath{stroke,fill}%
}%
\begin{pgfscope}%
\pgfsys@transformshift{2.060179in}{2.845773in}%
\pgfsys@useobject{currentmarker}{}%
\end{pgfscope}%
\end{pgfscope}%
\begin{pgfscope}%
\definecolor{textcolor}{rgb}{0.000000,0.000000,0.000000}%
\pgfsetstrokecolor{textcolor}%
\pgfsetfillcolor{textcolor}%
\pgftext[x=1.824067in, y=2.793012in, left, base]{\color{textcolor}{\rmfamily\fontsize{10.000000}{12.000000}\selectfont\catcode`\^=\active\def^{\ifmmode\sp\else\^{}\fi}\catcode`\%=\active\def%{\%}$\mathdefault{40}$}}%
\end{pgfscope}%
\begin{pgfscope}%
\definecolor{textcolor}{rgb}{0.000000,0.000000,0.000000}%
\pgfsetstrokecolor{textcolor}%
\pgfsetfillcolor{textcolor}%
\pgftext[x=1.768512in,y=1.655710in,,bottom,rotate=90.000000]{\color{textcolor}{\rmfamily\fontsize{10.000000}{12.000000}\selectfont\catcode`\^=\active\def^{\ifmmode\sp\else\^{}\fi}\catcode`\%=\active\def%{\%}Time (ms)}}%
\end{pgfscope}%
\begin{pgfscope}%
\pgfsetrectcap%
\pgfsetmiterjoin%
\pgfsetlinewidth{0.803000pt}%
\definecolor{currentstroke}{rgb}{0.000000,0.000000,0.000000}%
\pgfsetstrokecolor{currentstroke}%
\pgfsetdash{}{0pt}%
\pgfpathmoveto{\pgfqpoint{2.060179in}{0.294761in}}%
\pgfpathlineto{\pgfqpoint{2.060179in}{3.016660in}}%
\pgfusepath{stroke}%
\end{pgfscope}%
\begin{pgfscope}%
\pgfsetrectcap%
\pgfsetmiterjoin%
\pgfsetlinewidth{0.803000pt}%
\definecolor{currentstroke}{rgb}{0.000000,0.000000,0.000000}%
\pgfsetstrokecolor{currentstroke}%
\pgfsetdash{}{0pt}%
\pgfpathmoveto{\pgfqpoint{6.018122in}{0.294761in}}%
\pgfpathlineto{\pgfqpoint{6.018122in}{3.016660in}}%
\pgfusepath{stroke}%
\end{pgfscope}%
\begin{pgfscope}%
\pgfsetrectcap%
\pgfsetmiterjoin%
\pgfsetlinewidth{0.803000pt}%
\definecolor{currentstroke}{rgb}{0.000000,0.000000,0.000000}%
\pgfsetstrokecolor{currentstroke}%
\pgfsetdash{}{0pt}%
\pgfpathmoveto{\pgfqpoint{2.060179in}{0.294761in}}%
\pgfpathlineto{\pgfqpoint{6.018122in}{0.294761in}}%
\pgfusepath{stroke}%
\end{pgfscope}%
\begin{pgfscope}%
\pgfsetrectcap%
\pgfsetmiterjoin%
\pgfsetlinewidth{0.803000pt}%
\definecolor{currentstroke}{rgb}{0.000000,0.000000,0.000000}%
\pgfsetstrokecolor{currentstroke}%
\pgfsetdash{}{0pt}%
\pgfpathmoveto{\pgfqpoint{2.060179in}{3.016660in}}%
\pgfpathlineto{\pgfqpoint{6.018122in}{3.016660in}}%
\pgfusepath{stroke}%
\end{pgfscope}%
\begin{pgfscope}%
\definecolor{textcolor}{rgb}{0.000000,0.000000,0.000000}%
\pgfsetstrokecolor{textcolor}%
\pgfsetfillcolor{textcolor}%
\pgftext[x=2.378475in,y=0.934295in,,]{\color{textcolor}{\rmfamily\fontsize{5.790000}{6.948000}\selectfont\catcode`\^=\active\def^{\ifmmode\sp\else\^{}\fi}\catcode`\%=\active\def%{\%}20.06 ms}}%
\end{pgfscope}%
\begin{pgfscope}%
\definecolor{textcolor}{rgb}{0.000000,0.000000,0.000000}%
\pgfsetstrokecolor{textcolor}%
\pgfsetfillcolor{textcolor}%
\pgftext[x=2.378475in,y=1.602167in,,]{\color{textcolor}{\rmfamily\fontsize{5.790000}{6.948000}\selectfont\catcode`\^=\active\def^{\ifmmode\sp\else\^{}\fi}\catcode`\%=\active\def%{\%}0.02 ms}}%
\end{pgfscope}%
\begin{pgfscope}%
\definecolor{textcolor}{rgb}{0.000000,0.000000,0.000000}%
\pgfsetstrokecolor{textcolor}%
\pgfsetfillcolor{textcolor}%
\pgftext[x=2.378475in,y=1.690644in,,bottom]{\color{textcolor}{\rmfamily\fontsize{8.330000}{9.996000}\bfseries\selectfont\catcode`\^=\active\def^{\ifmmode\sp\else\^{}\fi}\catcode`\%=\active\def%{\%}20.07 ms}}%
\end{pgfscope}%
\begin{pgfscope}%
\definecolor{textcolor}{rgb}{0.000000,0.000000,0.000000}%
\pgfsetstrokecolor{textcolor}%
\pgfsetfillcolor{textcolor}%
\pgftext[x=2.932033in,y=0.321912in,,]{\color{textcolor}{\rmfamily\fontsize{5.790000}{6.948000}\selectfont\catcode`\^=\active\def^{\ifmmode\sp\else\^{}\fi}\catcode`\%=\active\def%{\%}0.85 ms}}%
\end{pgfscope}%
\begin{pgfscope}%
\definecolor{textcolor}{rgb}{0.000000,0.000000,0.000000}%
\pgfsetstrokecolor{textcolor}%
\pgfsetfillcolor{textcolor}%
\pgftext[x=2.932033in,y=0.479089in,,]{\color{textcolor}{\rmfamily\fontsize{5.790000}{6.948000}\selectfont\catcode`\^=\active\def^{\ifmmode\sp\else\^{}\fi}\catcode`\%=\active\def%{\%}4.08 ms}}%
\end{pgfscope}%
\begin{pgfscope}%
\definecolor{textcolor}{rgb}{0.000000,0.000000,0.000000}%
\pgfsetstrokecolor{textcolor}%
\pgfsetfillcolor{textcolor}%
\pgftext[x=2.932033in,y=0.739867in,,]{\color{textcolor}{\rmfamily\fontsize{5.790000}{6.948000}\selectfont\catcode`\^=\active\def^{\ifmmode\sp\else\^{}\fi}\catcode`\%=\active\def%{\%}4.10 ms}}%
\end{pgfscope}%
\begin{pgfscope}%
\definecolor{textcolor}{rgb}{0.000000,0.000000,0.000000}%
\pgfsetstrokecolor{textcolor}%
\pgfsetfillcolor{textcolor}%
\pgftext[x=2.932033in,y=0.996784in,,]{\color{textcolor}{\rmfamily\fontsize{5.790000}{6.948000}\selectfont\catcode`\^=\active\def^{\ifmmode\sp\else\^{}\fi}\catcode`\%=\active\def%{\%}3.96 ms}}%
\end{pgfscope}%
\begin{pgfscope}%
\definecolor{textcolor}{rgb}{0.000000,0.000000,0.000000}%
\pgfsetstrokecolor{textcolor}%
\pgfsetfillcolor{textcolor}%
\pgftext[x=2.932033in,y=1.130527in,,]{\color{textcolor}{\rmfamily\fontsize{5.790000}{6.948000}\selectfont\catcode`\^=\active\def^{\ifmmode\sp\else\^{}\fi}\catcode`\%=\active\def%{\%}0.24 ms}}%
\end{pgfscope}%
\begin{pgfscope}%
\definecolor{textcolor}{rgb}{0.000000,0.000000,0.000000}%
\pgfsetstrokecolor{textcolor}%
\pgfsetfillcolor{textcolor}%
\pgftext[x=2.932033in,y=1.167287in,,]{\color{textcolor}{\rmfamily\fontsize{5.790000}{6.948000}\selectfont\catcode`\^=\active\def^{\ifmmode\sp\else\^{}\fi}\catcode`\%=\active\def%{\%}0.04 ms}}%
\end{pgfscope}%
\begin{pgfscope}%
\definecolor{textcolor}{rgb}{0.000000,0.000000,0.000000}%
\pgfsetstrokecolor{textcolor}%
\pgfsetfillcolor{textcolor}%
\pgftext[x=2.932033in,y=1.256607in,,bottom]{\color{textcolor}{\rmfamily\fontsize{8.330000}{9.996000}\bfseries\selectfont\catcode`\^=\active\def^{\ifmmode\sp\else\^{}\fi}\catcode`\%=\active\def%{\%}13.27 ms}}%
\end{pgfscope}%
\begin{pgfscope}%
\definecolor{textcolor}{rgb}{0.000000,0.000000,0.000000}%
\pgfsetstrokecolor{textcolor}%
\pgfsetfillcolor{textcolor}%
\pgftext[x=3.485592in,y=0.324275in,,]{\color{textcolor}{\rmfamily\fontsize{5.790000}{6.948000}\selectfont\catcode`\^=\active\def^{\ifmmode\sp\else\^{}\fi}\catcode`\%=\active\def%{\%}0.93 ms}}%
\end{pgfscope}%
\begin{pgfscope}%
\definecolor{textcolor}{rgb}{0.000000,0.000000,0.000000}%
\pgfsetstrokecolor{textcolor}%
\pgfsetfillcolor{textcolor}%
\pgftext[x=3.485592in,y=0.369620in,,]{\color{textcolor}{\rmfamily\fontsize{5.790000}{6.948000}\selectfont\catcode`\^=\active\def^{\ifmmode\sp\else\^{}\fi}\catcode`\%=\active\def%{\%}0.50 ms}}%
\end{pgfscope}%
\begin{pgfscope}%
\definecolor{textcolor}{rgb}{0.000000,0.000000,0.000000}%
\pgfsetstrokecolor{textcolor}%
\pgfsetfillcolor{textcolor}%
\pgftext[x=3.485592in,y=0.414361in,,]{\color{textcolor}{\rmfamily\fontsize{5.790000}{6.948000}\selectfont\catcode`\^=\active\def^{\ifmmode\sp\else\^{}\fi}\catcode`\%=\active\def%{\%}0.04 ms}}%
\end{pgfscope}%
\begin{pgfscope}%
\definecolor{textcolor}{rgb}{0.000000,0.000000,0.000000}%
\pgfsetstrokecolor{textcolor}%
\pgfsetfillcolor{textcolor}%
\pgftext[x=3.485592in,y=0.518348in,,]{\color{textcolor}{\rmfamily\fontsize{5.790000}{6.948000}\selectfont\catcode`\^=\active\def^{\ifmmode\sp\else\^{}\fi}\catcode`\%=\active\def%{\%}4.10 ms}}%
\end{pgfscope}%
\begin{pgfscope}%
\definecolor{textcolor}{rgb}{0.000000,0.000000,0.000000}%
\pgfsetstrokecolor{textcolor}%
\pgfsetfillcolor{textcolor}%
\pgftext[x=3.485592in,y=0.780362in,,]{\color{textcolor}{\rmfamily\fontsize{5.790000}{6.948000}\selectfont\catcode`\^=\active\def^{\ifmmode\sp\else\^{}\fi}\catcode`\%=\active\def%{\%}4.12 ms}}%
\end{pgfscope}%
\begin{pgfscope}%
\definecolor{textcolor}{rgb}{0.000000,0.000000,0.000000}%
\pgfsetstrokecolor{textcolor}%
\pgfsetfillcolor{textcolor}%
\pgftext[x=3.485592in,y=1.037294in,,]{\color{textcolor}{\rmfamily\fontsize{5.790000}{6.948000}\selectfont\catcode`\^=\active\def^{\ifmmode\sp\else\^{}\fi}\catcode`\%=\active\def%{\%}3.94 ms}}%
\end{pgfscope}%
\begin{pgfscope}%
\definecolor{textcolor}{rgb}{0.000000,0.000000,0.000000}%
\pgfsetstrokecolor{textcolor}%
\pgfsetfillcolor{textcolor}%
\pgftext[x=3.485592in,y=1.170486in,,]{\color{textcolor}{\rmfamily\fontsize{5.790000}{6.948000}\selectfont\catcode`\^=\active\def^{\ifmmode\sp\else\^{}\fi}\catcode`\%=\active\def%{\%}0.24 ms}}%
\end{pgfscope}%
\begin{pgfscope}%
\definecolor{textcolor}{rgb}{0.000000,0.000000,0.000000}%
\pgfsetstrokecolor{textcolor}%
\pgfsetfillcolor{textcolor}%
\pgftext[x=3.485592in,y=1.207537in,,]{\color{textcolor}{\rmfamily\fontsize{5.790000}{6.948000}\selectfont\catcode`\^=\active\def^{\ifmmode\sp\else\^{}\fi}\catcode`\%=\active\def%{\%}0.05 ms}}%
\end{pgfscope}%
\begin{pgfscope}%
\definecolor{textcolor}{rgb}{0.000000,0.000000,0.000000}%
\pgfsetstrokecolor{textcolor}%
\pgfsetfillcolor{textcolor}%
\pgftext[x=3.485592in,y=1.297083in,,bottom]{\color{textcolor}{\rmfamily\fontsize{8.330000}{9.996000}\bfseries\selectfont\catcode`\^=\active\def^{\ifmmode\sp\else\^{}\fi}\catcode`\%=\active\def%{\%}13.90 ms}}%
\end{pgfscope}%
\begin{pgfscope}%
\definecolor{textcolor}{rgb}{0.000000,0.000000,0.000000}%
\pgfsetstrokecolor{textcolor}%
\pgfsetfillcolor{textcolor}%
\pgftext[x=4.039150in,y=0.321896in,,]{\color{textcolor}{\rmfamily\fontsize{5.790000}{6.948000}\selectfont\catcode`\^=\active\def^{\ifmmode\sp\else\^{}\fi}\catcode`\%=\active\def%{\%}0.85 ms}}%
\end{pgfscope}%
\begin{pgfscope}%
\definecolor{textcolor}{rgb}{0.000000,0.000000,0.000000}%
\pgfsetstrokecolor{textcolor}%
\pgfsetfillcolor{textcolor}%
\pgftext[x=4.039150in,y=0.479624in,,]{\color{textcolor}{\rmfamily\fontsize{5.790000}{6.948000}\selectfont\catcode`\^=\active\def^{\ifmmode\sp\else\^{}\fi}\catcode`\%=\active\def%{\%}4.10 ms}}%
\end{pgfscope}%
\begin{pgfscope}%
\definecolor{textcolor}{rgb}{0.000000,0.000000,0.000000}%
\pgfsetstrokecolor{textcolor}%
\pgfsetfillcolor{textcolor}%
\pgftext[x=4.039150in,y=0.741292in,,]{\color{textcolor}{\rmfamily\fontsize{5.790000}{6.948000}\selectfont\catcode`\^=\active\def^{\ifmmode\sp\else\^{}\fi}\catcode`\%=\active\def%{\%}4.11 ms}}%
\end{pgfscope}%
\begin{pgfscope}%
\definecolor{textcolor}{rgb}{0.000000,0.000000,0.000000}%
\pgfsetstrokecolor{textcolor}%
\pgfsetfillcolor{textcolor}%
\pgftext[x=4.039150in,y=0.998218in,,]{\color{textcolor}{\rmfamily\fontsize{5.790000}{6.948000}\selectfont\catcode`\^=\active\def^{\ifmmode\sp\else\^{}\fi}\catcode`\%=\active\def%{\%}3.95 ms}}%
\end{pgfscope}%
\begin{pgfscope}%
\definecolor{textcolor}{rgb}{0.000000,0.000000,0.000000}%
\pgfsetstrokecolor{textcolor}%
\pgfsetfillcolor{textcolor}%
\pgftext[x=4.039150in,y=1.131613in,,]{\color{textcolor}{\rmfamily\fontsize{5.790000}{6.948000}\selectfont\catcode`\^=\active\def^{\ifmmode\sp\else\^{}\fi}\catcode`\%=\active\def%{\%}0.24 ms}}%
\end{pgfscope}%
\begin{pgfscope}%
\definecolor{textcolor}{rgb}{0.000000,0.000000,0.000000}%
\pgfsetstrokecolor{textcolor}%
\pgfsetfillcolor{textcolor}%
\pgftext[x=4.039150in,y=1.168324in,,]{\color{textcolor}{\rmfamily\fontsize{5.790000}{6.948000}\selectfont\catcode`\^=\active\def^{\ifmmode\sp\else\^{}\fi}\catcode`\%=\active\def%{\%}0.04 ms}}%
\end{pgfscope}%
\begin{pgfscope}%
\definecolor{textcolor}{rgb}{0.000000,0.000000,0.000000}%
\pgfsetstrokecolor{textcolor}%
\pgfsetfillcolor{textcolor}%
\pgftext[x=4.039150in,y=1.257631in,,bottom]{\color{textcolor}{\rmfamily\fontsize{8.330000}{9.996000}\bfseries\selectfont\catcode`\^=\active\def^{\ifmmode\sp\else\^{}\fi}\catcode`\%=\active\def%{\%}13.28 ms}}%
\end{pgfscope}%
\begin{pgfscope}%
\definecolor{textcolor}{rgb}{0.000000,0.000000,0.000000}%
\pgfsetstrokecolor{textcolor}%
\pgfsetfillcolor{textcolor}%
\pgftext[x=4.592709in,y=0.412902in,,]{\color{textcolor}{\rmfamily\fontsize{5.790000}{6.948000}\selectfont\catcode`\^=\active\def^{\ifmmode\sp\else\^{}\fi}\catcode`\%=\active\def%{\%}3.70 ms}}%
\end{pgfscope}%
\begin{pgfscope}%
\definecolor{textcolor}{rgb}{0.000000,0.000000,0.000000}%
\pgfsetstrokecolor{textcolor}%
\pgfsetfillcolor{textcolor}%
\pgftext[x=4.592709in,y=0.661499in,,]{\color{textcolor}{\rmfamily\fontsize{5.790000}{6.948000}\selectfont\catcode`\^=\active\def^{\ifmmode\sp\else\^{}\fi}\catcode`\%=\active\def%{\%}4.09 ms}}%
\end{pgfscope}%
\begin{pgfscope}%
\definecolor{textcolor}{rgb}{0.000000,0.000000,0.000000}%
\pgfsetstrokecolor{textcolor}%
\pgfsetfillcolor{textcolor}%
\pgftext[x=4.592709in,y=0.922496in,,]{\color{textcolor}{\rmfamily\fontsize{5.790000}{6.948000}\selectfont\catcode`\^=\active\def^{\ifmmode\sp\else\^{}\fi}\catcode`\%=\active\def%{\%}4.09 ms}}%
\end{pgfscope}%
\begin{pgfscope}%
\definecolor{textcolor}{rgb}{0.000000,0.000000,0.000000}%
\pgfsetstrokecolor{textcolor}%
\pgfsetfillcolor{textcolor}%
\pgftext[x=4.592709in,y=1.178379in,,]{\color{textcolor}{\rmfamily\fontsize{5.790000}{6.948000}\selectfont\catcode`\^=\active\def^{\ifmmode\sp\else\^{}\fi}\catcode`\%=\active\def%{\%}3.93 ms}}%
\end{pgfscope}%
\begin{pgfscope}%
\definecolor{textcolor}{rgb}{0.000000,0.000000,0.000000}%
\pgfsetstrokecolor{textcolor}%
\pgfsetfillcolor{textcolor}%
\pgftext[x=4.592709in,y=1.311288in,,]{\color{textcolor}{\rmfamily\fontsize{5.790000}{6.948000}\selectfont\catcode`\^=\active\def^{\ifmmode\sp\else\^{}\fi}\catcode`\%=\active\def%{\%}0.24 ms}}%
\end{pgfscope}%
\begin{pgfscope}%
\definecolor{textcolor}{rgb}{0.000000,0.000000,0.000000}%
\pgfsetstrokecolor{textcolor}%
\pgfsetfillcolor{textcolor}%
\pgftext[x=4.592709in,y=1.348063in,,]{\color{textcolor}{\rmfamily\fontsize{5.790000}{6.948000}\selectfont\catcode`\^=\active\def^{\ifmmode\sp\else\^{}\fi}\catcode`\%=\active\def%{\%}0.04 ms}}%
\end{pgfscope}%
\begin{pgfscope}%
\definecolor{textcolor}{rgb}{0.000000,0.000000,0.000000}%
\pgfsetstrokecolor{textcolor}%
\pgfsetfillcolor{textcolor}%
\pgftext[x=4.592709in,y=1.437409in,,bottom]{\color{textcolor}{\rmfamily\fontsize{8.330000}{9.996000}\bfseries\selectfont\catcode`\^=\active\def^{\ifmmode\sp\else\^{}\fi}\catcode`\%=\active\def%{\%}16.10 ms}}%
\end{pgfscope}%
\begin{pgfscope}%
\definecolor{textcolor}{rgb}{0.000000,0.000000,0.000000}%
\pgfsetstrokecolor{textcolor}%
\pgfsetfillcolor{textcolor}%
\pgftext[x=5.146267in,y=0.300112in,,]{\color{textcolor}{\rmfamily\fontsize{5.790000}{6.948000}\selectfont\catcode`\^=\active\def^{\ifmmode\sp\else\^{}\fi}\catcode`\%=\active\def%{\%}0.17 ms}}%
\end{pgfscope}%
\begin{pgfscope}%
\definecolor{textcolor}{rgb}{0.000000,0.000000,0.000000}%
\pgfsetstrokecolor{textcolor}%
\pgfsetfillcolor{textcolor}%
\pgftext[x=5.146267in,y=0.335898in,,]{\color{textcolor}{\rmfamily\fontsize{5.790000}{6.948000}\selectfont\catcode`\^=\active\def^{\ifmmode\sp\else\^{}\fi}\catcode`\%=\active\def%{\%}0.95 ms}}%
\end{pgfscope}%
\begin{pgfscope}%
\definecolor{textcolor}{rgb}{0.000000,0.000000,0.000000}%
\pgfsetstrokecolor{textcolor}%
\pgfsetfillcolor{textcolor}%
\pgftext[x=5.146267in,y=0.475371in,,]{\color{textcolor}{\rmfamily\fontsize{5.790000}{6.948000}\selectfont\catcode`\^=\active\def^{\ifmmode\sp\else\^{}\fi}\catcode`\%=\active\def%{\%}3.42 ms}}%
\end{pgfscope}%
\begin{pgfscope}%
\definecolor{textcolor}{rgb}{0.000000,0.000000,0.000000}%
\pgfsetstrokecolor{textcolor}%
\pgfsetfillcolor{textcolor}%
\pgftext[x=5.146267in,y=0.589357in,,]{\color{textcolor}{\rmfamily\fontsize{5.790000}{6.948000}\selectfont\catcode`\^=\active\def^{\ifmmode\sp\else\^{}\fi}\catcode`\%=\active\def%{\%}0.16 ms}}%
\end{pgfscope}%
\begin{pgfscope}%
\definecolor{textcolor}{rgb}{0.000000,0.000000,0.000000}%
\pgfsetstrokecolor{textcolor}%
\pgfsetfillcolor{textcolor}%
\pgftext[x=5.146267in,y=0.753800in,,]{\color{textcolor}{\rmfamily\fontsize{5.790000}{6.948000}\selectfont\catcode`\^=\active\def^{\ifmmode\sp\else\^{}\fi}\catcode`\%=\active\def%{\%}5.00 ms}}%
\end{pgfscope}%
\begin{pgfscope}%
\definecolor{textcolor}{rgb}{0.000000,0.000000,0.000000}%
\pgfsetstrokecolor{textcolor}%
\pgfsetfillcolor{textcolor}%
\pgftext[x=5.146267in,y=1.069683in,,]{\color{textcolor}{\rmfamily\fontsize{5.790000}{6.948000}\selectfont\catcode`\^=\active\def^{\ifmmode\sp\else\^{}\fi}\catcode`\%=\active\def%{\%}4.90 ms}}%
\end{pgfscope}%
\begin{pgfscope}%
\definecolor{textcolor}{rgb}{0.000000,0.000000,0.000000}%
\pgfsetstrokecolor{textcolor}%
\pgfsetfillcolor{textcolor}%
\pgftext[x=5.146267in,y=1.362093in,,]{\color{textcolor}{\rmfamily\fontsize{5.790000}{6.948000}\selectfont\catcode`\^=\active\def^{\ifmmode\sp\else\^{}\fi}\catcode`\%=\active\def%{\%}4.27 ms}}%
\end{pgfscope}%
\begin{pgfscope}%
\definecolor{textcolor}{rgb}{0.000000,0.000000,0.000000}%
\pgfsetstrokecolor{textcolor}%
\pgfsetfillcolor{textcolor}%
\pgftext[x=5.146267in,y=1.513853in,,]{\color{textcolor}{\rmfamily\fontsize{5.790000}{6.948000}\selectfont\catcode`\^=\active\def^{\ifmmode\sp\else\^{}\fi}\catcode`\%=\active\def%{\%}0.49 ms}}%
\end{pgfscope}%
\begin{pgfscope}%
\definecolor{textcolor}{rgb}{0.000000,0.000000,0.000000}%
\pgfsetstrokecolor{textcolor}%
\pgfsetfillcolor{textcolor}%
\pgftext[x=5.146267in,y=1.536170in,,]{\color{textcolor}{\rmfamily\fontsize{5.790000}{6.948000}\selectfont\catcode`\^=\active\def^{\ifmmode\sp\else\^{}\fi}\catcode`\%=\active\def%{\%}0.21 ms}}%
\end{pgfscope}%
\begin{pgfscope}%
\definecolor{textcolor}{rgb}{0.000000,0.000000,0.000000}%
\pgfsetstrokecolor{textcolor}%
\pgfsetfillcolor{textcolor}%
\pgftext[x=5.146267in,y=1.658443in,,bottom]{\color{textcolor}{\rmfamily\fontsize{8.330000}{9.996000}\bfseries\selectfont\catcode`\^=\active\def^{\ifmmode\sp\else\^{}\fi}\catcode`\%=\active\def%{\%}19.57 ms}}%
\end{pgfscope}%
\begin{pgfscope}%
\definecolor{textcolor}{rgb}{0.000000,0.000000,0.000000}%
\pgfsetstrokecolor{textcolor}%
\pgfsetfillcolor{textcolor}%
\pgftext[x=5.699826in,y=0.301363in,,]{\color{textcolor}{\rmfamily\fontsize{5.790000}{6.948000}\selectfont\catcode`\^=\active\def^{\ifmmode\sp\else\^{}\fi}\catcode`\%=\active\def%{\%}0.21 ms}}%
\end{pgfscope}%
\begin{pgfscope}%
\definecolor{textcolor}{rgb}{0.000000,0.000000,0.000000}%
\pgfsetstrokecolor{textcolor}%
\pgfsetfillcolor{textcolor}%
\pgftext[x=5.699826in,y=0.322155in,,]{\color{textcolor}{\rmfamily\fontsize{5.790000}{6.948000}\selectfont\catcode`\^=\active\def^{\ifmmode\sp\else\^{}\fi}\catcode`\%=\active\def%{\%}0.45 ms}}%
\end{pgfscope}%
\begin{pgfscope}%
\definecolor{textcolor}{rgb}{0.000000,0.000000,0.000000}%
\pgfsetstrokecolor{textcolor}%
\pgfsetfillcolor{textcolor}%
\pgftext[x=5.699826in,y=1.491494in,,]{\color{textcolor}{\rmfamily\fontsize{5.790000}{6.948000}\selectfont\catcode`\^=\active\def^{\ifmmode\sp\else\^{}\fi}\catcode`\%=\active\def%{\%}36.23 ms}}%
\end{pgfscope}%
\begin{pgfscope}%
\definecolor{textcolor}{rgb}{0.000000,0.000000,0.000000}%
\pgfsetstrokecolor{textcolor}%
\pgfsetfillcolor{textcolor}%
\pgftext[x=5.699826in,y=2.676791in,,]{\color{textcolor}{\rmfamily\fontsize{5.790000}{6.948000}\selectfont\catcode`\^=\active\def^{\ifmmode\sp\else\^{}\fi}\catcode`\%=\active\def%{\%}0.07 ms}}%
\end{pgfscope}%
\begin{pgfscope}%
\definecolor{textcolor}{rgb}{0.000000,0.000000,0.000000}%
\pgfsetstrokecolor{textcolor}%
\pgfsetfillcolor{textcolor}%
\pgftext[x=5.699826in,y=2.767078in,,bottom]{\color{textcolor}{\rmfamily\fontsize{8.330000}{9.996000}\bfseries\selectfont\catcode`\^=\active\def^{\ifmmode\sp\else\^{}\fi}\catcode`\%=\active\def%{\%}36.95 ms}}%
\end{pgfscope}%
\begin{pgfscope}%
\pgfsetbuttcap%
\pgfsetmiterjoin%
\definecolor{currentfill}{rgb}{1.000000,1.000000,1.000000}%
\pgfsetfillcolor{currentfill}%
\pgfsetfillopacity{0.800000}%
\pgfsetlinewidth{1.003750pt}%
\definecolor{currentstroke}{rgb}{0.800000,0.800000,0.800000}%
\pgfsetstrokecolor{currentstroke}%
\pgfsetstrokeopacity{0.800000}%
\pgfsetdash{}{0pt}%
\pgfpathmoveto{\pgfqpoint{0.056292in}{0.775068in}}%
\pgfpathlineto{\pgfqpoint{1.473422in}{0.775068in}}%
\pgfpathquadraticcurveto{\pgfqpoint{1.489505in}{0.775068in}}{\pgfqpoint{1.489505in}{0.791151in}}%
\pgfpathlineto{\pgfqpoint{1.489505in}{2.325509in}}%
\pgfpathquadraticcurveto{\pgfqpoint{1.489505in}{2.341593in}}{\pgfqpoint{1.473422in}{2.341593in}}%
\pgfpathlineto{\pgfqpoint{0.056292in}{2.341593in}}%
\pgfpathquadraticcurveto{\pgfqpoint{0.040208in}{2.341593in}}{\pgfqpoint{0.040208in}{2.325509in}}%
\pgfpathlineto{\pgfqpoint{0.040208in}{0.791151in}}%
\pgfpathquadraticcurveto{\pgfqpoint{0.040208in}{0.775068in}}{\pgfqpoint{0.056292in}{0.775068in}}%
\pgfpathlineto{\pgfqpoint{0.056292in}{0.775068in}}%
\pgfpathclose%
\pgfusepath{stroke,fill}%
\end{pgfscope}%
\begin{pgfscope}%
\pgfsetbuttcap%
\pgfsetmiterjoin%
\definecolor{currentfill}{rgb}{0.993725,0.850196,0.704314}%
\pgfsetfillcolor{currentfill}%
\pgfsetlinewidth{0.000000pt}%
\definecolor{currentstroke}{rgb}{0.000000,0.000000,0.000000}%
\pgfsetstrokecolor{currentstroke}%
\pgfsetstrokeopacity{0.000000}%
\pgfsetdash{}{0pt}%
\pgfpathmoveto{\pgfqpoint{0.072375in}{2.248328in}}%
\pgfpathlineto{\pgfqpoint{0.233208in}{2.248328in}}%
\pgfpathlineto{\pgfqpoint{0.233208in}{2.304620in}}%
\pgfpathlineto{\pgfqpoint{0.072375in}{2.304620in}}%
\pgfpathlineto{\pgfqpoint{0.072375in}{2.248328in}}%
\pgfpathclose%
\pgfusepath{fill}%
\end{pgfscope}%
\begin{pgfscope}%
\definecolor{textcolor}{rgb}{0.000000,0.000000,0.000000}%
\pgfsetstrokecolor{textcolor}%
\pgfsetfillcolor{textcolor}%
\pgftext[x=0.297542in,y=2.248328in,left,base]{\color{textcolor}{\rmfamily\fontsize{5.790000}{6.948000}\selectfont\catcode`\^=\active\def^{\ifmmode\sp\else\^{}\fi}\catcode`\%=\active\def%{\%}photonQueryGeneration}}%
\end{pgfscope}%
\begin{pgfscope}%
\pgfsetbuttcap%
\pgfsetmiterjoin%
\definecolor{currentfill}{rgb}{0.992157,0.710065,0.464437}%
\pgfsetfillcolor{currentfill}%
\pgfsetlinewidth{0.000000pt}%
\definecolor{currentstroke}{rgb}{0.000000,0.000000,0.000000}%
\pgfsetstrokecolor{currentstroke}%
\pgfsetstrokeopacity{0.000000}%
\pgfsetdash{}{0pt}%
\pgfpathmoveto{\pgfqpoint{0.072375in}{2.129157in}}%
\pgfpathlineto{\pgfqpoint{0.233208in}{2.129157in}}%
\pgfpathlineto{\pgfqpoint{0.233208in}{2.185448in}}%
\pgfpathlineto{\pgfqpoint{0.072375in}{2.185448in}}%
\pgfpathlineto{\pgfqpoint{0.072375in}{2.129157in}}%
\pgfpathclose%
\pgfusepath{fill}%
\end{pgfscope}%
\begin{pgfscope}%
\definecolor{textcolor}{rgb}{0.000000,0.000000,0.000000}%
\pgfsetstrokecolor{textcolor}%
\pgfsetfillcolor{textcolor}%
\pgftext[x=0.297542in,y=2.129157in,left,base]{\color{textcolor}{\rmfamily\fontsize{5.790000}{6.948000}\selectfont\catcode`\^=\active\def^{\ifmmode\sp\else\^{}\fi}\catcode`\%=\active\def%{\%}photonQueryMapBuildTime}}%
\end{pgfscope}%
\begin{pgfscope}%
\pgfsetbuttcap%
\pgfsetmiterjoin%
\definecolor{currentfill}{rgb}{0.991419,0.550727,0.232772}%
\pgfsetfillcolor{currentfill}%
\pgfsetlinewidth{0.000000pt}%
\definecolor{currentstroke}{rgb}{0.000000,0.000000,0.000000}%
\pgfsetstrokecolor{currentstroke}%
\pgfsetstrokeopacity{0.000000}%
\pgfsetdash{}{0pt}%
\pgfpathmoveto{\pgfqpoint{0.072375in}{2.009985in}}%
\pgfpathlineto{\pgfqpoint{0.233208in}{2.009985in}}%
\pgfpathlineto{\pgfqpoint{0.233208in}{2.066277in}}%
\pgfpathlineto{\pgfqpoint{0.072375in}{2.066277in}}%
\pgfpathlineto{\pgfqpoint{0.072375in}{2.009985in}}%
\pgfpathclose%
\pgfusepath{fill}%
\end{pgfscope}%
\begin{pgfscope}%
\definecolor{textcolor}{rgb}{0.000000,0.000000,0.000000}%
\pgfsetstrokecolor{textcolor}%
\pgfsetfillcolor{textcolor}%
\pgftext[x=0.297542in,y=2.009985in,left,base]{\color{textcolor}{\rmfamily\fontsize{5.790000}{6.948000}\selectfont\catcode`\^=\active\def^{\ifmmode\sp\else\^{}\fi}\catcode`\%=\active\def%{\%}photonGeneration}}%
\end{pgfscope}%
\begin{pgfscope}%
\pgfsetbuttcap%
\pgfsetmiterjoin%
\definecolor{currentfill}{rgb}{0.925536,0.384867,0.059839}%
\pgfsetfillcolor{currentfill}%
\pgfsetlinewidth{0.000000pt}%
\definecolor{currentstroke}{rgb}{0.000000,0.000000,0.000000}%
\pgfsetstrokecolor{currentstroke}%
\pgfsetstrokeopacity{0.000000}%
\pgfsetdash{}{0pt}%
\pgfpathmoveto{\pgfqpoint{0.072375in}{1.891952in}}%
\pgfpathlineto{\pgfqpoint{0.233208in}{1.891952in}}%
\pgfpathlineto{\pgfqpoint{0.233208in}{1.948243in}}%
\pgfpathlineto{\pgfqpoint{0.072375in}{1.948243in}}%
\pgfpathlineto{\pgfqpoint{0.072375in}{1.891952in}}%
\pgfpathclose%
\pgfusepath{fill}%
\end{pgfscope}%
\begin{pgfscope}%
\definecolor{textcolor}{rgb}{0.000000,0.000000,0.000000}%
\pgfsetstrokecolor{textcolor}%
\pgfsetfillcolor{textcolor}%
\pgftext[x=0.297542in,y=1.891952in,left,base]{\color{textcolor}{\rmfamily\fontsize{5.790000}{6.948000}\selectfont\catcode`\^=\active\def^{\ifmmode\sp\else\^{}\fi}\catcode`\%=\active\def%{\%}photonPostprocessing}}%
\end{pgfscope}%
\begin{pgfscope}%
\pgfsetbuttcap%
\pgfsetmiterjoin%
\definecolor{currentfill}{rgb}{0.814118,0.883922,0.949804}%
\pgfsetfillcolor{currentfill}%
\pgfsetlinewidth{0.000000pt}%
\definecolor{currentstroke}{rgb}{0.000000,0.000000,0.000000}%
\pgfsetstrokecolor{currentstroke}%
\pgfsetstrokeopacity{0.000000}%
\pgfsetdash{}{0pt}%
\pgfpathmoveto{\pgfqpoint{0.072375in}{1.772780in}}%
\pgfpathlineto{\pgfqpoint{0.233208in}{1.772780in}}%
\pgfpathlineto{\pgfqpoint{0.233208in}{1.829072in}}%
\pgfpathlineto{\pgfqpoint{0.072375in}{1.829072in}}%
\pgfpathlineto{\pgfqpoint{0.072375in}{1.772780in}}%
\pgfpathclose%
\pgfusepath{fill}%
\end{pgfscope}%
\begin{pgfscope}%
\definecolor{textcolor}{rgb}{0.000000,0.000000,0.000000}%
\pgfsetstrokecolor{textcolor}%
\pgfsetfillcolor{textcolor}%
\pgftext[x=0.297542in,y=1.772780in,left,base]{\color{textcolor}{\rmfamily\fontsize{5.790000}{6.948000}\selectfont\catcode`\^=\active\def^{\ifmmode\sp\else\^{}\fi}\catcode`\%=\active\def%{\%}forwardSampleGeneration}}%
\end{pgfscope}%
\begin{pgfscope}%
\pgfsetbuttcap%
\pgfsetmiterjoin%
\definecolor{currentfill}{rgb}{0.579608,0.770196,0.873725}%
\pgfsetfillcolor{currentfill}%
\pgfsetlinewidth{0.000000pt}%
\definecolor{currentstroke}{rgb}{0.000000,0.000000,0.000000}%
\pgfsetstrokecolor{currentstroke}%
\pgfsetstrokeopacity{0.000000}%
\pgfsetdash{}{0pt}%
\pgfpathmoveto{\pgfqpoint{0.072375in}{1.654747in}}%
\pgfpathlineto{\pgfqpoint{0.233208in}{1.654747in}}%
\pgfpathlineto{\pgfqpoint{0.233208in}{1.711039in}}%
\pgfpathlineto{\pgfqpoint{0.072375in}{1.711039in}}%
\pgfpathlineto{\pgfqpoint{0.072375in}{1.654747in}}%
\pgfpathclose%
\pgfusepath{fill}%
\end{pgfscope}%
\begin{pgfscope}%
\definecolor{textcolor}{rgb}{0.000000,0.000000,0.000000}%
\pgfsetstrokecolor{textcolor}%
\pgfsetfillcolor{textcolor}%
\pgftext[x=0.297542in,y=1.654747in,left,base]{\color{textcolor}{\rmfamily\fontsize{5.790000}{6.948000}\selectfont\catcode`\^=\active\def^{\ifmmode\sp\else\^{}\fi}\catcode`\%=\active\def%{\%}backwardSampleGeneration}}%
\end{pgfscope}%
\begin{pgfscope}%
\pgfsetbuttcap%
\pgfsetmiterjoin%
\definecolor{currentfill}{rgb}{0.827451,0.932549,0.803137}%
\pgfsetfillcolor{currentfill}%
\pgfsetlinewidth{0.000000pt}%
\definecolor{currentstroke}{rgb}{0.000000,0.000000,0.000000}%
\pgfsetstrokecolor{currentstroke}%
\pgfsetstrokeopacity{0.000000}%
\pgfsetdash{}{0pt}%
\pgfpathmoveto{\pgfqpoint{0.072375in}{1.536714in}}%
\pgfpathlineto{\pgfqpoint{0.233208in}{1.536714in}}%
\pgfpathlineto{\pgfqpoint{0.233208in}{1.593006in}}%
\pgfpathlineto{\pgfqpoint{0.072375in}{1.593006in}}%
\pgfpathlineto{\pgfqpoint{0.072375in}{1.536714in}}%
\pgfpathclose%
\pgfusepath{fill}%
\end{pgfscope}%
\begin{pgfscope}%
\definecolor{textcolor}{rgb}{0.000000,0.000000,0.000000}%
\pgfsetstrokecolor{textcolor}%
\pgfsetfillcolor{textcolor}%
\pgftext[x=0.297542in,y=1.536714in,left,base]{\color{textcolor}{\rmfamily\fontsize{5.790000}{6.948000}\selectfont\catcode`\^=\active\def^{\ifmmode\sp\else\^{}\fi}\catcode`\%=\active\def%{\%}selfLearningInference}}%
\end{pgfscope}%
\begin{pgfscope}%
\pgfsetbuttcap%
\pgfsetmiterjoin%
\definecolor{currentfill}{rgb}{0.451765,0.767090,0.461207}%
\pgfsetfillcolor{currentfill}%
\pgfsetlinewidth{0.000000pt}%
\definecolor{currentstroke}{rgb}{0.000000,0.000000,0.000000}%
\pgfsetstrokecolor{currentstroke}%
\pgfsetstrokeopacity{0.000000}%
\pgfsetdash{}{0pt}%
\pgfpathmoveto{\pgfqpoint{0.072375in}{1.417542in}}%
\pgfpathlineto{\pgfqpoint{0.233208in}{1.417542in}}%
\pgfpathlineto{\pgfqpoint{0.233208in}{1.473834in}}%
\pgfpathlineto{\pgfqpoint{0.072375in}{1.473834in}}%
\pgfpathlineto{\pgfqpoint{0.072375in}{1.417542in}}%
\pgfpathclose%
\pgfusepath{fill}%
\end{pgfscope}%
\begin{pgfscope}%
\definecolor{textcolor}{rgb}{0.000000,0.000000,0.000000}%
\pgfsetstrokecolor{textcolor}%
\pgfsetfillcolor{textcolor}%
\pgftext[x=0.297542in,y=1.417542in,left,base]{\color{textcolor}{\rmfamily\fontsize{5.790000}{6.948000}\selectfont\catcode`\^=\active\def^{\ifmmode\sp\else\^{}\fi}\catcode`\%=\active\def%{\%}selfLearningPostprocessing}}%
\end{pgfscope}%
\begin{pgfscope}%
\pgfsetbuttcap%
\pgfsetmiterjoin%
\definecolor{currentfill}{rgb}{0.887059,0.887059,0.887059}%
\pgfsetfillcolor{currentfill}%
\pgfsetlinewidth{0.000000pt}%
\definecolor{currentstroke}{rgb}{0.000000,0.000000,0.000000}%
\pgfsetstrokecolor{currentstroke}%
\pgfsetstrokeopacity{0.000000}%
\pgfsetdash{}{0pt}%
\pgfpathmoveto{\pgfqpoint{0.072375in}{1.298371in}}%
\pgfpathlineto{\pgfqpoint{0.233208in}{1.298371in}}%
\pgfpathlineto{\pgfqpoint{0.233208in}{1.354662in}}%
\pgfpathlineto{\pgfqpoint{0.072375in}{1.354662in}}%
\pgfpathlineto{\pgfqpoint{0.072375in}{1.298371in}}%
\pgfpathclose%
\pgfusepath{fill}%
\end{pgfscope}%
\begin{pgfscope}%
\definecolor{textcolor}{rgb}{0.000000,0.000000,0.000000}%
\pgfsetstrokecolor{textcolor}%
\pgfsetfillcolor{textcolor}%
\pgftext[x=0.297542in,y=1.298371in,left,base]{\color{textcolor}{\rmfamily\fontsize{5.790000}{6.948000}\selectfont\catcode`\^=\active\def^{\ifmmode\sp\else\^{}\fi}\catcode`\%=\active\def%{\%}training}}%
\end{pgfscope}%
\begin{pgfscope}%
\pgfsetbuttcap%
\pgfsetmiterjoin%
\definecolor{currentfill}{rgb}{0.710588,0.710588,0.710588}%
\pgfsetfillcolor{currentfill}%
\pgfsetlinewidth{0.000000pt}%
\definecolor{currentstroke}{rgb}{0.000000,0.000000,0.000000}%
\pgfsetstrokecolor{currentstroke}%
\pgfsetstrokeopacity{0.000000}%
\pgfsetdash{}{0pt}%
\pgfpathmoveto{\pgfqpoint{0.072375in}{1.179199in}}%
\pgfpathlineto{\pgfqpoint{0.233208in}{1.179199in}}%
\pgfpathlineto{\pgfqpoint{0.233208in}{1.235491in}}%
\pgfpathlineto{\pgfqpoint{0.072375in}{1.235491in}}%
\pgfpathlineto{\pgfqpoint{0.072375in}{1.179199in}}%
\pgfpathclose%
\pgfusepath{fill}%
\end{pgfscope}%
\begin{pgfscope}%
\definecolor{textcolor}{rgb}{0.000000,0.000000,0.000000}%
\pgfsetstrokecolor{textcolor}%
\pgfsetfillcolor{textcolor}%
\pgftext[x=0.297542in,y=1.179199in,left,base]{\color{textcolor}{\rmfamily\fontsize{5.790000}{6.948000}\selectfont\catcode`\^=\active\def^{\ifmmode\sp\else\^{}\fi}\catcode`\%=\active\def%{\%}pathtracing}}%
\end{pgfscope}%
\begin{pgfscope}%
\pgfsetbuttcap%
\pgfsetmiterjoin%
\definecolor{currentfill}{rgb}{0.478431,0.478431,0.478431}%
\pgfsetfillcolor{currentfill}%
\pgfsetlinewidth{0.000000pt}%
\definecolor{currentstroke}{rgb}{0.000000,0.000000,0.000000}%
\pgfsetstrokecolor{currentstroke}%
\pgfsetstrokeopacity{0.000000}%
\pgfsetdash{}{0pt}%
\pgfpathmoveto{\pgfqpoint{0.072375in}{1.060027in}}%
\pgfpathlineto{\pgfqpoint{0.233208in}{1.060027in}}%
\pgfpathlineto{\pgfqpoint{0.233208in}{1.116319in}}%
\pgfpathlineto{\pgfqpoint{0.072375in}{1.116319in}}%
\pgfpathlineto{\pgfqpoint{0.072375in}{1.060027in}}%
\pgfpathclose%
\pgfusepath{fill}%
\end{pgfscope}%
\begin{pgfscope}%
\definecolor{textcolor}{rgb}{0.000000,0.000000,0.000000}%
\pgfsetstrokecolor{textcolor}%
\pgfsetfillcolor{textcolor}%
\pgftext[x=0.297542in,y=1.060027in,left,base]{\color{textcolor}{\rmfamily\fontsize{5.790000}{6.948000}\selectfont\catcode`\^=\active\def^{\ifmmode\sp\else\^{}\fi}\catcode`\%=\active\def%{\%}inference}}%
\end{pgfscope}%
\begin{pgfscope}%
\pgfsetbuttcap%
\pgfsetmiterjoin%
\definecolor{currentfill}{rgb}{1.000000,0.752941,0.796078}%
\pgfsetfillcolor{currentfill}%
\pgfsetlinewidth{0.000000pt}%
\definecolor{currentstroke}{rgb}{0.000000,0.000000,0.000000}%
\pgfsetstrokecolor{currentstroke}%
\pgfsetstrokeopacity{0.000000}%
\pgfsetdash{}{0pt}%
\pgfpathmoveto{\pgfqpoint{0.072375in}{0.941994in}}%
\pgfpathlineto{\pgfqpoint{0.233208in}{0.941994in}}%
\pgfpathlineto{\pgfqpoint{0.233208in}{0.998286in}}%
\pgfpathlineto{\pgfqpoint{0.072375in}{0.998286in}}%
\pgfpathlineto{\pgfqpoint{0.072375in}{0.941994in}}%
\pgfpathclose%
\pgfusepath{fill}%
\end{pgfscope}%
\begin{pgfscope}%
\definecolor{textcolor}{rgb}{0.000000,0.000000,0.000000}%
\pgfsetstrokecolor{textcolor}%
\pgfsetfillcolor{textcolor}%
\pgftext[x=0.297542in,y=0.941994in,left,base]{\color{textcolor}{\rmfamily\fontsize{5.790000}{6.948000}\selectfont\catcode`\^=\active\def^{\ifmmode\sp\else\^{}\fi}\catcode`\%=\active\def%{\%}visualization}}%
\end{pgfscope}%
\begin{pgfscope}%
\pgfsetbuttcap%
\pgfsetmiterjoin%
\definecolor{currentfill}{rgb}{0.854902,0.439216,0.839216}%
\pgfsetfillcolor{currentfill}%
\pgfsetlinewidth{0.000000pt}%
\definecolor{currentstroke}{rgb}{0.000000,0.000000,0.000000}%
\pgfsetstrokecolor{currentstroke}%
\pgfsetstrokeopacity{0.000000}%
\pgfsetdash{}{0pt}%
\pgfpathmoveto{\pgfqpoint{0.072375in}{0.823961in}}%
\pgfpathlineto{\pgfqpoint{0.233208in}{0.823961in}}%
\pgfpathlineto{\pgfqpoint{0.233208in}{0.880253in}}%
\pgfpathlineto{\pgfqpoint{0.072375in}{0.880253in}}%
\pgfpathlineto{\pgfqpoint{0.072375in}{0.823961in}}%
\pgfpathclose%
\pgfusepath{fill}%
\end{pgfscope}%
\begin{pgfscope}%
\definecolor{textcolor}{rgb}{0.000000,0.000000,0.000000}%
\pgfsetstrokecolor{textcolor}%
\pgfsetfillcolor{textcolor}%
\pgftext[x=0.297542in,y=0.823961in,left,base]{\color{textcolor}{\rmfamily\fontsize{5.790000}{6.948000}\selectfont\catcode`\^=\active\def^{\ifmmode\sp\else\^{}\fi}\catcode`\%=\active\def%{\%}other}}%
\end{pgfscope}%
\end{pgfpicture}%
\makeatother%
\endgroup%

    \caption{Performance breakdown of the different rendering techniques. Measured in 720p-equivalent $960^2$px resolution on an NVIDIA RTX 3060 Ti and averaged over 10s.}
    \label{fig:breakdown}
\end{figure}

\begin{figure}[htb!]
    \centering
    %% Creator: Matplotlib, PGF backend
%%
%% To include the figure in your LaTeX document, write
%%   \input{<filename>.pgf}
%%
%% Make sure the required packages are loaded in your preamble
%%   \usepackage{pgf}
%%
%% Also ensure that all the required font packages are loaded; for instance,
%% the lmodern package is sometimes necessary when using math font.
%%   \usepackage{lmodern}
%%
%% Figures using additional raster images can only be included by \input if
%% they are in the same directory as the main LaTeX file. For loading figures
%% from other directories you can use the `import` package
%%   \usepackage{import}
%%
%% and then include the figures with
%%   \import{<path to file>}{<filename>.pgf}
%%
%% Matplotlib used the following preamble
%%   \def\mathdefault#1{#1}
%%   \everymath=\expandafter{\the\everymath\displaystyle}
%%   \IfFileExists{scrextend.sty}{
%%     \usepackage[fontsize=10.000000pt]{scrextend}
%%   }{
%%     \renewcommand{\normalsize}{\fontsize{10.000000}{12.000000}\selectfont}
%%     \normalsize
%%   }
%%   
%%   \ifdefined\pdftexversion\else  % non-pdftex case.
%%     \usepackage{fontspec}
%%     \setmainfont{DejaVuSerif.ttf}[Path=\detokenize{/opt/homebrew/Cellar/python-matplotlib/3.10.5/libexec/lib/python3.13/site-packages/matplotlib/mpl-data/fonts/ttf/}]
%%     \setsansfont{DejaVuSans.ttf}[Path=\detokenize{/opt/homebrew/Cellar/python-matplotlib/3.10.5/libexec/lib/python3.13/site-packages/matplotlib/mpl-data/fonts/ttf/}]
%%     \setmonofont{DejaVuSansMono.ttf}[Path=\detokenize{/opt/homebrew/Cellar/python-matplotlib/3.10.5/libexec/lib/python3.13/site-packages/matplotlib/mpl-data/fonts/ttf/}]
%%   \fi
%%   \makeatletter\@ifpackageloaded{underscore}{}{\usepackage[strings]{underscore}}\makeatother
%%
\begingroup%
\makeatletter%
\begin{pgfpicture}%
\pgfpathrectangle{\pgfpointorigin}{\pgfqpoint{5.276080in}{3.490883in}}%
\pgfusepath{use as bounding box, clip}%
\begin{pgfscope}%
\pgfsetbuttcap%
\pgfsetmiterjoin%
\definecolor{currentfill}{rgb}{1.000000,1.000000,1.000000}%
\pgfsetfillcolor{currentfill}%
\pgfsetlinewidth{0.000000pt}%
\definecolor{currentstroke}{rgb}{1.000000,1.000000,1.000000}%
\pgfsetstrokecolor{currentstroke}%
\pgfsetdash{}{0pt}%
\pgfpathmoveto{\pgfqpoint{0.000000in}{0.000000in}}%
\pgfpathlineto{\pgfqpoint{5.276080in}{0.000000in}}%
\pgfpathlineto{\pgfqpoint{5.276080in}{3.490883in}}%
\pgfpathlineto{\pgfqpoint{0.000000in}{3.490883in}}%
\pgfpathlineto{\pgfqpoint{0.000000in}{0.000000in}}%
\pgfpathclose%
\pgfusepath{fill}%
\end{pgfscope}%
\begin{pgfscope}%
\pgfsetbuttcap%
\pgfsetmiterjoin%
\definecolor{currentfill}{rgb}{1.000000,1.000000,1.000000}%
\pgfsetfillcolor{currentfill}%
\pgfsetlinewidth{0.000000pt}%
\definecolor{currentstroke}{rgb}{0.000000,0.000000,0.000000}%
\pgfsetstrokecolor{currentstroke}%
\pgfsetstrokeopacity{0.000000}%
\pgfsetdash{}{0pt}%
\pgfpathmoveto{\pgfqpoint{0.526080in}{0.310883in}}%
\pgfpathlineto{\pgfqpoint{5.176080in}{0.310883in}}%
\pgfpathlineto{\pgfqpoint{5.176080in}{3.390883in}}%
\pgfpathlineto{\pgfqpoint{0.526080in}{3.390883in}}%
\pgfpathlineto{\pgfqpoint{0.526080in}{0.310883in}}%
\pgfpathclose%
\pgfusepath{fill}%
\end{pgfscope}%
\begin{pgfscope}%
\pgfpathrectangle{\pgfqpoint{0.526080in}{0.310883in}}{\pgfqpoint{4.650000in}{3.080000in}}%
\pgfusepath{clip}%
\pgfsetbuttcap%
\pgfsetmiterjoin%
\definecolor{currentfill}{rgb}{0.993725,0.850196,0.704314}%
\pgfsetfillcolor{currentfill}%
\pgfsetlinewidth{0.000000pt}%
\definecolor{currentstroke}{rgb}{0.000000,0.000000,0.000000}%
\pgfsetstrokecolor{currentstroke}%
\pgfsetstrokeopacity{0.000000}%
\pgfsetdash{}{0pt}%
\pgfpathmoveto{\pgfqpoint{0.737443in}{0.310883in}}%
\pgfpathlineto{\pgfqpoint{0.986107in}{0.310883in}}%
\pgfpathlineto{\pgfqpoint{0.986107in}{0.325638in}}%
\pgfpathlineto{\pgfqpoint{0.737443in}{0.325638in}}%
\pgfpathlineto{\pgfqpoint{0.737443in}{0.310883in}}%
\pgfpathclose%
\pgfusepath{fill}%
\end{pgfscope}%
\begin{pgfscope}%
\pgfpathrectangle{\pgfqpoint{0.526080in}{0.310883in}}{\pgfqpoint{4.650000in}{3.080000in}}%
\pgfusepath{clip}%
\pgfsetbuttcap%
\pgfsetmiterjoin%
\definecolor{currentfill}{rgb}{0.992157,0.710065,0.464437}%
\pgfsetfillcolor{currentfill}%
\pgfsetlinewidth{0.000000pt}%
\definecolor{currentstroke}{rgb}{0.000000,0.000000,0.000000}%
\pgfsetstrokecolor{currentstroke}%
\pgfsetstrokeopacity{0.000000}%
\pgfsetdash{}{0pt}%
\pgfpathmoveto{\pgfqpoint{0.737443in}{0.325638in}}%
\pgfpathlineto{\pgfqpoint{0.986107in}{0.325638in}}%
\pgfpathlineto{\pgfqpoint{0.986107in}{0.411774in}}%
\pgfpathlineto{\pgfqpoint{0.737443in}{0.411774in}}%
\pgfpathlineto{\pgfqpoint{0.737443in}{0.325638in}}%
\pgfpathclose%
\pgfusepath{fill}%
\end{pgfscope}%
\begin{pgfscope}%
\pgfpathrectangle{\pgfqpoint{0.526080in}{0.310883in}}{\pgfqpoint{4.650000in}{3.080000in}}%
\pgfusepath{clip}%
\pgfsetbuttcap%
\pgfsetmiterjoin%
\definecolor{currentfill}{rgb}{0.991419,0.550727,0.232772}%
\pgfsetfillcolor{currentfill}%
\pgfsetlinewidth{0.000000pt}%
\definecolor{currentstroke}{rgb}{0.000000,0.000000,0.000000}%
\pgfsetstrokecolor{currentstroke}%
\pgfsetstrokeopacity{0.000000}%
\pgfsetdash{}{0pt}%
\pgfpathmoveto{\pgfqpoint{0.737443in}{0.411774in}}%
\pgfpathlineto{\pgfqpoint{0.986107in}{0.411774in}}%
\pgfpathlineto{\pgfqpoint{0.986107in}{0.822985in}}%
\pgfpathlineto{\pgfqpoint{0.737443in}{0.822985in}}%
\pgfpathlineto{\pgfqpoint{0.737443in}{0.411774in}}%
\pgfpathclose%
\pgfusepath{fill}%
\end{pgfscope}%
\begin{pgfscope}%
\pgfpathrectangle{\pgfqpoint{0.526080in}{0.310883in}}{\pgfqpoint{4.650000in}{3.080000in}}%
\pgfusepath{clip}%
\pgfsetbuttcap%
\pgfsetmiterjoin%
\definecolor{currentfill}{rgb}{0.925536,0.384867,0.059839}%
\pgfsetfillcolor{currentfill}%
\pgfsetlinewidth{0.000000pt}%
\definecolor{currentstroke}{rgb}{0.000000,0.000000,0.000000}%
\pgfsetstrokecolor{currentstroke}%
\pgfsetstrokeopacity{0.000000}%
\pgfsetdash{}{0pt}%
\pgfpathmoveto{\pgfqpoint{0.737443in}{0.822985in}}%
\pgfpathlineto{\pgfqpoint{0.986107in}{0.822985in}}%
\pgfpathlineto{\pgfqpoint{0.986107in}{0.837052in}}%
\pgfpathlineto{\pgfqpoint{0.737443in}{0.837052in}}%
\pgfpathlineto{\pgfqpoint{0.737443in}{0.822985in}}%
\pgfpathclose%
\pgfusepath{fill}%
\end{pgfscope}%
\begin{pgfscope}%
\pgfpathrectangle{\pgfqpoint{0.526080in}{0.310883in}}{\pgfqpoint{4.650000in}{3.080000in}}%
\pgfusepath{clip}%
\pgfsetbuttcap%
\pgfsetmiterjoin%
\definecolor{currentfill}{rgb}{0.887059,0.887059,0.887059}%
\pgfsetfillcolor{currentfill}%
\pgfsetlinewidth{0.000000pt}%
\definecolor{currentstroke}{rgb}{0.000000,0.000000,0.000000}%
\pgfsetstrokecolor{currentstroke}%
\pgfsetstrokeopacity{0.000000}%
\pgfsetdash{}{0pt}%
\pgfpathmoveto{\pgfqpoint{0.737443in}{0.837052in}}%
\pgfpathlineto{\pgfqpoint{0.986107in}{0.837052in}}%
\pgfpathlineto{\pgfqpoint{0.986107in}{1.388583in}}%
\pgfpathlineto{\pgfqpoint{0.737443in}{1.388583in}}%
\pgfpathlineto{\pgfqpoint{0.737443in}{0.837052in}}%
\pgfpathclose%
\pgfusepath{fill}%
\end{pgfscope}%
\begin{pgfscope}%
\pgfpathrectangle{\pgfqpoint{0.526080in}{0.310883in}}{\pgfqpoint{4.650000in}{3.080000in}}%
\pgfusepath{clip}%
\pgfsetbuttcap%
\pgfsetmiterjoin%
\definecolor{currentfill}{rgb}{0.710588,0.710588,0.710588}%
\pgfsetfillcolor{currentfill}%
\pgfsetlinewidth{0.000000pt}%
\definecolor{currentstroke}{rgb}{0.000000,0.000000,0.000000}%
\pgfsetstrokecolor{currentstroke}%
\pgfsetstrokeopacity{0.000000}%
\pgfsetdash{}{0pt}%
\pgfpathmoveto{\pgfqpoint{0.737443in}{1.388583in}}%
\pgfpathlineto{\pgfqpoint{0.986107in}{1.388583in}}%
\pgfpathlineto{\pgfqpoint{0.986107in}{1.464690in}}%
\pgfpathlineto{\pgfqpoint{0.737443in}{1.464690in}}%
\pgfpathlineto{\pgfqpoint{0.737443in}{1.388583in}}%
\pgfpathclose%
\pgfusepath{fill}%
\end{pgfscope}%
\begin{pgfscope}%
\pgfpathrectangle{\pgfqpoint{0.526080in}{0.310883in}}{\pgfqpoint{4.650000in}{3.080000in}}%
\pgfusepath{clip}%
\pgfsetbuttcap%
\pgfsetmiterjoin%
\definecolor{currentfill}{rgb}{0.478431,0.478431,0.478431}%
\pgfsetfillcolor{currentfill}%
\pgfsetlinewidth{0.000000pt}%
\definecolor{currentstroke}{rgb}{0.000000,0.000000,0.000000}%
\pgfsetstrokecolor{currentstroke}%
\pgfsetstrokeopacity{0.000000}%
\pgfsetdash{}{0pt}%
\pgfpathmoveto{\pgfqpoint{0.737443in}{1.464690in}}%
\pgfpathlineto{\pgfqpoint{0.986107in}{1.464690in}}%
\pgfpathlineto{\pgfqpoint{0.986107in}{1.586181in}}%
\pgfpathlineto{\pgfqpoint{0.737443in}{1.586181in}}%
\pgfpathlineto{\pgfqpoint{0.737443in}{1.464690in}}%
\pgfpathclose%
\pgfusepath{fill}%
\end{pgfscope}%
\begin{pgfscope}%
\pgfpathrectangle{\pgfqpoint{0.526080in}{0.310883in}}{\pgfqpoint{4.650000in}{3.080000in}}%
\pgfusepath{clip}%
\pgfsetbuttcap%
\pgfsetmiterjoin%
\definecolor{currentfill}{rgb}{1.000000,0.752941,0.796078}%
\pgfsetfillcolor{currentfill}%
\pgfsetlinewidth{0.000000pt}%
\definecolor{currentstroke}{rgb}{0.000000,0.000000,0.000000}%
\pgfsetstrokecolor{currentstroke}%
\pgfsetstrokeopacity{0.000000}%
\pgfsetdash{}{0pt}%
\pgfpathmoveto{\pgfqpoint{0.737443in}{1.586181in}}%
\pgfpathlineto{\pgfqpoint{0.986107in}{1.586181in}}%
\pgfpathlineto{\pgfqpoint{0.986107in}{1.593505in}}%
\pgfpathlineto{\pgfqpoint{0.737443in}{1.593505in}}%
\pgfpathlineto{\pgfqpoint{0.737443in}{1.586181in}}%
\pgfpathclose%
\pgfusepath{fill}%
\end{pgfscope}%
\begin{pgfscope}%
\pgfpathrectangle{\pgfqpoint{0.526080in}{0.310883in}}{\pgfqpoint{4.650000in}{3.080000in}}%
\pgfusepath{clip}%
\pgfsetbuttcap%
\pgfsetmiterjoin%
\definecolor{currentfill}{rgb}{0.854902,0.439216,0.839216}%
\pgfsetfillcolor{currentfill}%
\pgfsetlinewidth{0.000000pt}%
\definecolor{currentstroke}{rgb}{0.000000,0.000000,0.000000}%
\pgfsetstrokecolor{currentstroke}%
\pgfsetstrokeopacity{0.000000}%
\pgfsetdash{}{0pt}%
\pgfpathmoveto{\pgfqpoint{0.737443in}{1.593505in}}%
\pgfpathlineto{\pgfqpoint{0.986107in}{1.593505in}}%
\pgfpathlineto{\pgfqpoint{0.986107in}{1.602281in}}%
\pgfpathlineto{\pgfqpoint{0.737443in}{1.602281in}}%
\pgfpathlineto{\pgfqpoint{0.737443in}{1.593505in}}%
\pgfpathclose%
\pgfusepath{fill}%
\end{pgfscope}%
\begin{pgfscope}%
\pgfpathrectangle{\pgfqpoint{0.526080in}{0.310883in}}{\pgfqpoint{4.650000in}{3.080000in}}%
\pgfusepath{clip}%
\pgfsetbuttcap%
\pgfsetmiterjoin%
\definecolor{currentfill}{rgb}{0.993725,0.850196,0.704314}%
\pgfsetfillcolor{currentfill}%
\pgfsetlinewidth{0.000000pt}%
\definecolor{currentstroke}{rgb}{0.000000,0.000000,0.000000}%
\pgfsetstrokecolor{currentstroke}%
\pgfsetstrokeopacity{0.000000}%
\pgfsetdash{}{0pt}%
\pgfpathmoveto{\pgfqpoint{1.234770in}{0.310883in}}%
\pgfpathlineto{\pgfqpoint{1.483433in}{0.310883in}}%
\pgfpathlineto{\pgfqpoint{1.483433in}{0.325604in}}%
\pgfpathlineto{\pgfqpoint{1.234770in}{0.325604in}}%
\pgfpathlineto{\pgfqpoint{1.234770in}{0.310883in}}%
\pgfpathclose%
\pgfusepath{fill}%
\end{pgfscope}%
\begin{pgfscope}%
\pgfpathrectangle{\pgfqpoint{0.526080in}{0.310883in}}{\pgfqpoint{4.650000in}{3.080000in}}%
\pgfusepath{clip}%
\pgfsetbuttcap%
\pgfsetmiterjoin%
\definecolor{currentfill}{rgb}{0.992157,0.710065,0.464437}%
\pgfsetfillcolor{currentfill}%
\pgfsetlinewidth{0.000000pt}%
\definecolor{currentstroke}{rgb}{0.000000,0.000000,0.000000}%
\pgfsetstrokecolor{currentstroke}%
\pgfsetstrokeopacity{0.000000}%
\pgfsetdash{}{0pt}%
\pgfpathmoveto{\pgfqpoint{1.234770in}{0.325604in}}%
\pgfpathlineto{\pgfqpoint{1.483433in}{0.325604in}}%
\pgfpathlineto{\pgfqpoint{1.483433in}{0.415579in}}%
\pgfpathlineto{\pgfqpoint{1.234770in}{0.415579in}}%
\pgfpathlineto{\pgfqpoint{1.234770in}{0.325604in}}%
\pgfpathclose%
\pgfusepath{fill}%
\end{pgfscope}%
\begin{pgfscope}%
\pgfpathrectangle{\pgfqpoint{0.526080in}{0.310883in}}{\pgfqpoint{4.650000in}{3.080000in}}%
\pgfusepath{clip}%
\pgfsetbuttcap%
\pgfsetmiterjoin%
\definecolor{currentfill}{rgb}{0.991419,0.550727,0.232772}%
\pgfsetfillcolor{currentfill}%
\pgfsetlinewidth{0.000000pt}%
\definecolor{currentstroke}{rgb}{0.000000,0.000000,0.000000}%
\pgfsetstrokecolor{currentstroke}%
\pgfsetstrokeopacity{0.000000}%
\pgfsetdash{}{0pt}%
\pgfpathmoveto{\pgfqpoint{1.234770in}{0.415579in}}%
\pgfpathlineto{\pgfqpoint{1.483433in}{0.415579in}}%
\pgfpathlineto{\pgfqpoint{1.483433in}{0.844116in}}%
\pgfpathlineto{\pgfqpoint{1.234770in}{0.844116in}}%
\pgfpathlineto{\pgfqpoint{1.234770in}{0.415579in}}%
\pgfpathclose%
\pgfusepath{fill}%
\end{pgfscope}%
\begin{pgfscope}%
\pgfpathrectangle{\pgfqpoint{0.526080in}{0.310883in}}{\pgfqpoint{4.650000in}{3.080000in}}%
\pgfusepath{clip}%
\pgfsetbuttcap%
\pgfsetmiterjoin%
\definecolor{currentfill}{rgb}{0.925536,0.384867,0.059839}%
\pgfsetfillcolor{currentfill}%
\pgfsetlinewidth{0.000000pt}%
\definecolor{currentstroke}{rgb}{0.000000,0.000000,0.000000}%
\pgfsetstrokecolor{currentstroke}%
\pgfsetstrokeopacity{0.000000}%
\pgfsetdash{}{0pt}%
\pgfpathmoveto{\pgfqpoint{1.234770in}{0.844116in}}%
\pgfpathlineto{\pgfqpoint{1.483433in}{0.844116in}}%
\pgfpathlineto{\pgfqpoint{1.483433in}{0.858875in}}%
\pgfpathlineto{\pgfqpoint{1.234770in}{0.858875in}}%
\pgfpathlineto{\pgfqpoint{1.234770in}{0.844116in}}%
\pgfpathclose%
\pgfusepath{fill}%
\end{pgfscope}%
\begin{pgfscope}%
\pgfpathrectangle{\pgfqpoint{0.526080in}{0.310883in}}{\pgfqpoint{4.650000in}{3.080000in}}%
\pgfusepath{clip}%
\pgfsetbuttcap%
\pgfsetmiterjoin%
\definecolor{currentfill}{rgb}{0.887059,0.887059,0.887059}%
\pgfsetfillcolor{currentfill}%
\pgfsetlinewidth{0.000000pt}%
\definecolor{currentstroke}{rgb}{0.000000,0.000000,0.000000}%
\pgfsetstrokecolor{currentstroke}%
\pgfsetstrokeopacity{0.000000}%
\pgfsetdash{}{0pt}%
\pgfpathmoveto{\pgfqpoint{1.234770in}{0.858875in}}%
\pgfpathlineto{\pgfqpoint{1.483433in}{0.858875in}}%
\pgfpathlineto{\pgfqpoint{1.483433in}{1.333995in}}%
\pgfpathlineto{\pgfqpoint{1.234770in}{1.333995in}}%
\pgfpathlineto{\pgfqpoint{1.234770in}{0.858875in}}%
\pgfpathclose%
\pgfusepath{fill}%
\end{pgfscope}%
\begin{pgfscope}%
\pgfpathrectangle{\pgfqpoint{0.526080in}{0.310883in}}{\pgfqpoint{4.650000in}{3.080000in}}%
\pgfusepath{clip}%
\pgfsetbuttcap%
\pgfsetmiterjoin%
\definecolor{currentfill}{rgb}{0.710588,0.710588,0.710588}%
\pgfsetfillcolor{currentfill}%
\pgfsetlinewidth{0.000000pt}%
\definecolor{currentstroke}{rgb}{0.000000,0.000000,0.000000}%
\pgfsetstrokecolor{currentstroke}%
\pgfsetstrokeopacity{0.000000}%
\pgfsetdash{}{0pt}%
\pgfpathmoveto{\pgfqpoint{1.234770in}{1.333995in}}%
\pgfpathlineto{\pgfqpoint{1.483433in}{1.333995in}}%
\pgfpathlineto{\pgfqpoint{1.483433in}{1.471680in}}%
\pgfpathlineto{\pgfqpoint{1.234770in}{1.471680in}}%
\pgfpathlineto{\pgfqpoint{1.234770in}{1.333995in}}%
\pgfpathclose%
\pgfusepath{fill}%
\end{pgfscope}%
\begin{pgfscope}%
\pgfpathrectangle{\pgfqpoint{0.526080in}{0.310883in}}{\pgfqpoint{4.650000in}{3.080000in}}%
\pgfusepath{clip}%
\pgfsetbuttcap%
\pgfsetmiterjoin%
\definecolor{currentfill}{rgb}{0.478431,0.478431,0.478431}%
\pgfsetfillcolor{currentfill}%
\pgfsetlinewidth{0.000000pt}%
\definecolor{currentstroke}{rgb}{0.000000,0.000000,0.000000}%
\pgfsetstrokecolor{currentstroke}%
\pgfsetstrokeopacity{0.000000}%
\pgfsetdash{}{0pt}%
\pgfpathmoveto{\pgfqpoint{1.234770in}{1.471680in}}%
\pgfpathlineto{\pgfqpoint{1.483433in}{1.471680in}}%
\pgfpathlineto{\pgfqpoint{1.483433in}{1.704144in}}%
\pgfpathlineto{\pgfqpoint{1.234770in}{1.704144in}}%
\pgfpathlineto{\pgfqpoint{1.234770in}{1.471680in}}%
\pgfpathclose%
\pgfusepath{fill}%
\end{pgfscope}%
\begin{pgfscope}%
\pgfpathrectangle{\pgfqpoint{0.526080in}{0.310883in}}{\pgfqpoint{4.650000in}{3.080000in}}%
\pgfusepath{clip}%
\pgfsetbuttcap%
\pgfsetmiterjoin%
\definecolor{currentfill}{rgb}{1.000000,0.752941,0.796078}%
\pgfsetfillcolor{currentfill}%
\pgfsetlinewidth{0.000000pt}%
\definecolor{currentstroke}{rgb}{0.000000,0.000000,0.000000}%
\pgfsetstrokecolor{currentstroke}%
\pgfsetstrokeopacity{0.000000}%
\pgfsetdash{}{0pt}%
\pgfpathmoveto{\pgfqpoint{1.234770in}{1.704144in}}%
\pgfpathlineto{\pgfqpoint{1.483433in}{1.704144in}}%
\pgfpathlineto{\pgfqpoint{1.483433in}{1.718366in}}%
\pgfpathlineto{\pgfqpoint{1.234770in}{1.718366in}}%
\pgfpathlineto{\pgfqpoint{1.234770in}{1.704144in}}%
\pgfpathclose%
\pgfusepath{fill}%
\end{pgfscope}%
\begin{pgfscope}%
\pgfpathrectangle{\pgfqpoint{0.526080in}{0.310883in}}{\pgfqpoint{4.650000in}{3.080000in}}%
\pgfusepath{clip}%
\pgfsetbuttcap%
\pgfsetmiterjoin%
\definecolor{currentfill}{rgb}{0.854902,0.439216,0.839216}%
\pgfsetfillcolor{currentfill}%
\pgfsetlinewidth{0.000000pt}%
\definecolor{currentstroke}{rgb}{0.000000,0.000000,0.000000}%
\pgfsetstrokecolor{currentstroke}%
\pgfsetstrokeopacity{0.000000}%
\pgfsetdash{}{0pt}%
\pgfpathmoveto{\pgfqpoint{1.234770in}{1.718366in}}%
\pgfpathlineto{\pgfqpoint{1.483433in}{1.718366in}}%
\pgfpathlineto{\pgfqpoint{1.483433in}{1.726633in}}%
\pgfpathlineto{\pgfqpoint{1.234770in}{1.726633in}}%
\pgfpathlineto{\pgfqpoint{1.234770in}{1.718366in}}%
\pgfpathclose%
\pgfusepath{fill}%
\end{pgfscope}%
\begin{pgfscope}%
\pgfpathrectangle{\pgfqpoint{0.526080in}{0.310883in}}{\pgfqpoint{4.650000in}{3.080000in}}%
\pgfusepath{clip}%
\pgfsetbuttcap%
\pgfsetmiterjoin%
\definecolor{currentfill}{rgb}{0.993725,0.850196,0.704314}%
\pgfsetfillcolor{currentfill}%
\pgfsetlinewidth{0.000000pt}%
\definecolor{currentstroke}{rgb}{0.000000,0.000000,0.000000}%
\pgfsetstrokecolor{currentstroke}%
\pgfsetstrokeopacity{0.000000}%
\pgfsetdash{}{0pt}%
\pgfpathmoveto{\pgfqpoint{1.732096in}{0.310883in}}%
\pgfpathlineto{\pgfqpoint{1.980759in}{0.310883in}}%
\pgfpathlineto{\pgfqpoint{1.980759in}{0.325586in}}%
\pgfpathlineto{\pgfqpoint{1.732096in}{0.325586in}}%
\pgfpathlineto{\pgfqpoint{1.732096in}{0.310883in}}%
\pgfpathclose%
\pgfusepath{fill}%
\end{pgfscope}%
\begin{pgfscope}%
\pgfpathrectangle{\pgfqpoint{0.526080in}{0.310883in}}{\pgfqpoint{4.650000in}{3.080000in}}%
\pgfusepath{clip}%
\pgfsetbuttcap%
\pgfsetmiterjoin%
\definecolor{currentfill}{rgb}{0.992157,0.710065,0.464437}%
\pgfsetfillcolor{currentfill}%
\pgfsetlinewidth{0.000000pt}%
\definecolor{currentstroke}{rgb}{0.000000,0.000000,0.000000}%
\pgfsetstrokecolor{currentstroke}%
\pgfsetstrokeopacity{0.000000}%
\pgfsetdash{}{0pt}%
\pgfpathmoveto{\pgfqpoint{1.732096in}{0.325586in}}%
\pgfpathlineto{\pgfqpoint{1.980759in}{0.325586in}}%
\pgfpathlineto{\pgfqpoint{1.980759in}{0.415685in}}%
\pgfpathlineto{\pgfqpoint{1.732096in}{0.415685in}}%
\pgfpathlineto{\pgfqpoint{1.732096in}{0.325586in}}%
\pgfpathclose%
\pgfusepath{fill}%
\end{pgfscope}%
\begin{pgfscope}%
\pgfpathrectangle{\pgfqpoint{0.526080in}{0.310883in}}{\pgfqpoint{4.650000in}{3.080000in}}%
\pgfusepath{clip}%
\pgfsetbuttcap%
\pgfsetmiterjoin%
\definecolor{currentfill}{rgb}{0.991419,0.550727,0.232772}%
\pgfsetfillcolor{currentfill}%
\pgfsetlinewidth{0.000000pt}%
\definecolor{currentstroke}{rgb}{0.000000,0.000000,0.000000}%
\pgfsetstrokecolor{currentstroke}%
\pgfsetstrokeopacity{0.000000}%
\pgfsetdash{}{0pt}%
\pgfpathmoveto{\pgfqpoint{1.732096in}{0.415685in}}%
\pgfpathlineto{\pgfqpoint{1.980759in}{0.415685in}}%
\pgfpathlineto{\pgfqpoint{1.980759in}{0.844968in}}%
\pgfpathlineto{\pgfqpoint{1.732096in}{0.844968in}}%
\pgfpathlineto{\pgfqpoint{1.732096in}{0.415685in}}%
\pgfpathclose%
\pgfusepath{fill}%
\end{pgfscope}%
\begin{pgfscope}%
\pgfpathrectangle{\pgfqpoint{0.526080in}{0.310883in}}{\pgfqpoint{4.650000in}{3.080000in}}%
\pgfusepath{clip}%
\pgfsetbuttcap%
\pgfsetmiterjoin%
\definecolor{currentfill}{rgb}{0.925536,0.384867,0.059839}%
\pgfsetfillcolor{currentfill}%
\pgfsetlinewidth{0.000000pt}%
\definecolor{currentstroke}{rgb}{0.000000,0.000000,0.000000}%
\pgfsetstrokecolor{currentstroke}%
\pgfsetstrokeopacity{0.000000}%
\pgfsetdash{}{0pt}%
\pgfpathmoveto{\pgfqpoint{1.732096in}{0.844968in}}%
\pgfpathlineto{\pgfqpoint{1.980759in}{0.844968in}}%
\pgfpathlineto{\pgfqpoint{1.980759in}{0.859743in}}%
\pgfpathlineto{\pgfqpoint{1.732096in}{0.859743in}}%
\pgfpathlineto{\pgfqpoint{1.732096in}{0.844968in}}%
\pgfpathclose%
\pgfusepath{fill}%
\end{pgfscope}%
\begin{pgfscope}%
\pgfpathrectangle{\pgfqpoint{0.526080in}{0.310883in}}{\pgfqpoint{4.650000in}{3.080000in}}%
\pgfusepath{clip}%
\pgfsetbuttcap%
\pgfsetmiterjoin%
\definecolor{currentfill}{rgb}{0.887059,0.887059,0.887059}%
\pgfsetfillcolor{currentfill}%
\pgfsetlinewidth{0.000000pt}%
\definecolor{currentstroke}{rgb}{0.000000,0.000000,0.000000}%
\pgfsetstrokecolor{currentstroke}%
\pgfsetstrokeopacity{0.000000}%
\pgfsetdash{}{0pt}%
\pgfpathmoveto{\pgfqpoint{1.732096in}{0.859743in}}%
\pgfpathlineto{\pgfqpoint{1.980759in}{0.859743in}}%
\pgfpathlineto{\pgfqpoint{1.980759in}{1.334642in}}%
\pgfpathlineto{\pgfqpoint{1.732096in}{1.334642in}}%
\pgfpathlineto{\pgfqpoint{1.732096in}{0.859743in}}%
\pgfpathclose%
\pgfusepath{fill}%
\end{pgfscope}%
\begin{pgfscope}%
\pgfpathrectangle{\pgfqpoint{0.526080in}{0.310883in}}{\pgfqpoint{4.650000in}{3.080000in}}%
\pgfusepath{clip}%
\pgfsetbuttcap%
\pgfsetmiterjoin%
\definecolor{currentfill}{rgb}{0.710588,0.710588,0.710588}%
\pgfsetfillcolor{currentfill}%
\pgfsetlinewidth{0.000000pt}%
\definecolor{currentstroke}{rgb}{0.000000,0.000000,0.000000}%
\pgfsetstrokecolor{currentstroke}%
\pgfsetstrokeopacity{0.000000}%
\pgfsetdash{}{0pt}%
\pgfpathmoveto{\pgfqpoint{1.732096in}{1.334642in}}%
\pgfpathlineto{\pgfqpoint{1.980759in}{1.334642in}}%
\pgfpathlineto{\pgfqpoint{1.980759in}{1.532120in}}%
\pgfpathlineto{\pgfqpoint{1.732096in}{1.532120in}}%
\pgfpathlineto{\pgfqpoint{1.732096in}{1.334642in}}%
\pgfpathclose%
\pgfusepath{fill}%
\end{pgfscope}%
\begin{pgfscope}%
\pgfpathrectangle{\pgfqpoint{0.526080in}{0.310883in}}{\pgfqpoint{4.650000in}{3.080000in}}%
\pgfusepath{clip}%
\pgfsetbuttcap%
\pgfsetmiterjoin%
\definecolor{currentfill}{rgb}{0.478431,0.478431,0.478431}%
\pgfsetfillcolor{currentfill}%
\pgfsetlinewidth{0.000000pt}%
\definecolor{currentstroke}{rgb}{0.000000,0.000000,0.000000}%
\pgfsetstrokecolor{currentstroke}%
\pgfsetstrokeopacity{0.000000}%
\pgfsetdash{}{0pt}%
\pgfpathmoveto{\pgfqpoint{1.732096in}{1.532120in}}%
\pgfpathlineto{\pgfqpoint{1.980759in}{1.532120in}}%
\pgfpathlineto{\pgfqpoint{1.980759in}{1.875337in}}%
\pgfpathlineto{\pgfqpoint{1.732096in}{1.875337in}}%
\pgfpathlineto{\pgfqpoint{1.732096in}{1.532120in}}%
\pgfpathclose%
\pgfusepath{fill}%
\end{pgfscope}%
\begin{pgfscope}%
\pgfpathrectangle{\pgfqpoint{0.526080in}{0.310883in}}{\pgfqpoint{4.650000in}{3.080000in}}%
\pgfusepath{clip}%
\pgfsetbuttcap%
\pgfsetmiterjoin%
\definecolor{currentfill}{rgb}{1.000000,0.752941,0.796078}%
\pgfsetfillcolor{currentfill}%
\pgfsetlinewidth{0.000000pt}%
\definecolor{currentstroke}{rgb}{0.000000,0.000000,0.000000}%
\pgfsetstrokecolor{currentstroke}%
\pgfsetstrokeopacity{0.000000}%
\pgfsetdash{}{0pt}%
\pgfpathmoveto{\pgfqpoint{1.732096in}{1.875337in}}%
\pgfpathlineto{\pgfqpoint{1.980759in}{1.875337in}}%
\pgfpathlineto{\pgfqpoint{1.980759in}{1.896339in}}%
\pgfpathlineto{\pgfqpoint{1.732096in}{1.896339in}}%
\pgfpathlineto{\pgfqpoint{1.732096in}{1.875337in}}%
\pgfpathclose%
\pgfusepath{fill}%
\end{pgfscope}%
\begin{pgfscope}%
\pgfpathrectangle{\pgfqpoint{0.526080in}{0.310883in}}{\pgfqpoint{4.650000in}{3.080000in}}%
\pgfusepath{clip}%
\pgfsetbuttcap%
\pgfsetmiterjoin%
\definecolor{currentfill}{rgb}{0.854902,0.439216,0.839216}%
\pgfsetfillcolor{currentfill}%
\pgfsetlinewidth{0.000000pt}%
\definecolor{currentstroke}{rgb}{0.000000,0.000000,0.000000}%
\pgfsetstrokecolor{currentstroke}%
\pgfsetstrokeopacity{0.000000}%
\pgfsetdash{}{0pt}%
\pgfpathmoveto{\pgfqpoint{1.732096in}{1.896339in}}%
\pgfpathlineto{\pgfqpoint{1.980759in}{1.896339in}}%
\pgfpathlineto{\pgfqpoint{1.980759in}{1.904570in}}%
\pgfpathlineto{\pgfqpoint{1.732096in}{1.904570in}}%
\pgfpathlineto{\pgfqpoint{1.732096in}{1.896339in}}%
\pgfpathclose%
\pgfusepath{fill}%
\end{pgfscope}%
\begin{pgfscope}%
\pgfpathrectangle{\pgfqpoint{0.526080in}{0.310883in}}{\pgfqpoint{4.650000in}{3.080000in}}%
\pgfusepath{clip}%
\pgfsetbuttcap%
\pgfsetmiterjoin%
\definecolor{currentfill}{rgb}{0.993725,0.850196,0.704314}%
\pgfsetfillcolor{currentfill}%
\pgfsetlinewidth{0.000000pt}%
\definecolor{currentstroke}{rgb}{0.000000,0.000000,0.000000}%
\pgfsetstrokecolor{currentstroke}%
\pgfsetstrokeopacity{0.000000}%
\pgfsetdash{}{0pt}%
\pgfpathmoveto{\pgfqpoint{2.229422in}{0.310883in}}%
\pgfpathlineto{\pgfqpoint{2.478085in}{0.310883in}}%
\pgfpathlineto{\pgfqpoint{2.478085in}{0.325725in}}%
\pgfpathlineto{\pgfqpoint{2.229422in}{0.325725in}}%
\pgfpathlineto{\pgfqpoint{2.229422in}{0.310883in}}%
\pgfpathclose%
\pgfusepath{fill}%
\end{pgfscope}%
\begin{pgfscope}%
\pgfpathrectangle{\pgfqpoint{0.526080in}{0.310883in}}{\pgfqpoint{4.650000in}{3.080000in}}%
\pgfusepath{clip}%
\pgfsetbuttcap%
\pgfsetmiterjoin%
\definecolor{currentfill}{rgb}{0.992157,0.710065,0.464437}%
\pgfsetfillcolor{currentfill}%
\pgfsetlinewidth{0.000000pt}%
\definecolor{currentstroke}{rgb}{0.000000,0.000000,0.000000}%
\pgfsetstrokecolor{currentstroke}%
\pgfsetstrokeopacity{0.000000}%
\pgfsetdash{}{0pt}%
\pgfpathmoveto{\pgfqpoint{2.229422in}{0.325725in}}%
\pgfpathlineto{\pgfqpoint{2.478085in}{0.325725in}}%
\pgfpathlineto{\pgfqpoint{2.478085in}{0.416152in}}%
\pgfpathlineto{\pgfqpoint{2.229422in}{0.416152in}}%
\pgfpathlineto{\pgfqpoint{2.229422in}{0.325725in}}%
\pgfpathclose%
\pgfusepath{fill}%
\end{pgfscope}%
\begin{pgfscope}%
\pgfpathrectangle{\pgfqpoint{0.526080in}{0.310883in}}{\pgfqpoint{4.650000in}{3.080000in}}%
\pgfusepath{clip}%
\pgfsetbuttcap%
\pgfsetmiterjoin%
\definecolor{currentfill}{rgb}{0.991419,0.550727,0.232772}%
\pgfsetfillcolor{currentfill}%
\pgfsetlinewidth{0.000000pt}%
\definecolor{currentstroke}{rgb}{0.000000,0.000000,0.000000}%
\pgfsetstrokecolor{currentstroke}%
\pgfsetstrokeopacity{0.000000}%
\pgfsetdash{}{0pt}%
\pgfpathmoveto{\pgfqpoint{2.229422in}{0.416152in}}%
\pgfpathlineto{\pgfqpoint{2.478085in}{0.416152in}}%
\pgfpathlineto{\pgfqpoint{2.478085in}{0.846868in}}%
\pgfpathlineto{\pgfqpoint{2.229422in}{0.846868in}}%
\pgfpathlineto{\pgfqpoint{2.229422in}{0.416152in}}%
\pgfpathclose%
\pgfusepath{fill}%
\end{pgfscope}%
\begin{pgfscope}%
\pgfpathrectangle{\pgfqpoint{0.526080in}{0.310883in}}{\pgfqpoint{4.650000in}{3.080000in}}%
\pgfusepath{clip}%
\pgfsetbuttcap%
\pgfsetmiterjoin%
\definecolor{currentfill}{rgb}{0.925536,0.384867,0.059839}%
\pgfsetfillcolor{currentfill}%
\pgfsetlinewidth{0.000000pt}%
\definecolor{currentstroke}{rgb}{0.000000,0.000000,0.000000}%
\pgfsetstrokecolor{currentstroke}%
\pgfsetstrokeopacity{0.000000}%
\pgfsetdash{}{0pt}%
\pgfpathmoveto{\pgfqpoint{2.229422in}{0.846868in}}%
\pgfpathlineto{\pgfqpoint{2.478085in}{0.846868in}}%
\pgfpathlineto{\pgfqpoint{2.478085in}{0.861679in}}%
\pgfpathlineto{\pgfqpoint{2.229422in}{0.861679in}}%
\pgfpathlineto{\pgfqpoint{2.229422in}{0.846868in}}%
\pgfpathclose%
\pgfusepath{fill}%
\end{pgfscope}%
\begin{pgfscope}%
\pgfpathrectangle{\pgfqpoint{0.526080in}{0.310883in}}{\pgfqpoint{4.650000in}{3.080000in}}%
\pgfusepath{clip}%
\pgfsetbuttcap%
\pgfsetmiterjoin%
\definecolor{currentfill}{rgb}{0.887059,0.887059,0.887059}%
\pgfsetfillcolor{currentfill}%
\pgfsetlinewidth{0.000000pt}%
\definecolor{currentstroke}{rgb}{0.000000,0.000000,0.000000}%
\pgfsetstrokecolor{currentstroke}%
\pgfsetstrokeopacity{0.000000}%
\pgfsetdash{}{0pt}%
\pgfpathmoveto{\pgfqpoint{2.229422in}{0.861679in}}%
\pgfpathlineto{\pgfqpoint{2.478085in}{0.861679in}}%
\pgfpathlineto{\pgfqpoint{2.478085in}{1.336336in}}%
\pgfpathlineto{\pgfqpoint{2.229422in}{1.336336in}}%
\pgfpathlineto{\pgfqpoint{2.229422in}{0.861679in}}%
\pgfpathclose%
\pgfusepath{fill}%
\end{pgfscope}%
\begin{pgfscope}%
\pgfpathrectangle{\pgfqpoint{0.526080in}{0.310883in}}{\pgfqpoint{4.650000in}{3.080000in}}%
\pgfusepath{clip}%
\pgfsetbuttcap%
\pgfsetmiterjoin%
\definecolor{currentfill}{rgb}{0.710588,0.710588,0.710588}%
\pgfsetfillcolor{currentfill}%
\pgfsetlinewidth{0.000000pt}%
\definecolor{currentstroke}{rgb}{0.000000,0.000000,0.000000}%
\pgfsetstrokecolor{currentstroke}%
\pgfsetstrokeopacity{0.000000}%
\pgfsetdash{}{0pt}%
\pgfpathmoveto{\pgfqpoint{2.229422in}{1.336336in}}%
\pgfpathlineto{\pgfqpoint{2.478085in}{1.336336in}}%
\pgfpathlineto{\pgfqpoint{2.478085in}{1.590775in}}%
\pgfpathlineto{\pgfqpoint{2.229422in}{1.590775in}}%
\pgfpathlineto{\pgfqpoint{2.229422in}{1.336336in}}%
\pgfpathclose%
\pgfusepath{fill}%
\end{pgfscope}%
\begin{pgfscope}%
\pgfpathrectangle{\pgfqpoint{0.526080in}{0.310883in}}{\pgfqpoint{4.650000in}{3.080000in}}%
\pgfusepath{clip}%
\pgfsetbuttcap%
\pgfsetmiterjoin%
\definecolor{currentfill}{rgb}{0.478431,0.478431,0.478431}%
\pgfsetfillcolor{currentfill}%
\pgfsetlinewidth{0.000000pt}%
\definecolor{currentstroke}{rgb}{0.000000,0.000000,0.000000}%
\pgfsetstrokecolor{currentstroke}%
\pgfsetstrokeopacity{0.000000}%
\pgfsetdash{}{0pt}%
\pgfpathmoveto{\pgfqpoint{2.229422in}{1.590775in}}%
\pgfpathlineto{\pgfqpoint{2.478085in}{1.590775in}}%
\pgfpathlineto{\pgfqpoint{2.478085in}{2.042585in}}%
\pgfpathlineto{\pgfqpoint{2.229422in}{2.042585in}}%
\pgfpathlineto{\pgfqpoint{2.229422in}{1.590775in}}%
\pgfpathclose%
\pgfusepath{fill}%
\end{pgfscope}%
\begin{pgfscope}%
\pgfpathrectangle{\pgfqpoint{0.526080in}{0.310883in}}{\pgfqpoint{4.650000in}{3.080000in}}%
\pgfusepath{clip}%
\pgfsetbuttcap%
\pgfsetmiterjoin%
\definecolor{currentfill}{rgb}{1.000000,0.752941,0.796078}%
\pgfsetfillcolor{currentfill}%
\pgfsetlinewidth{0.000000pt}%
\definecolor{currentstroke}{rgb}{0.000000,0.000000,0.000000}%
\pgfsetstrokecolor{currentstroke}%
\pgfsetstrokeopacity{0.000000}%
\pgfsetdash{}{0pt}%
\pgfpathmoveto{\pgfqpoint{2.229422in}{2.042585in}}%
\pgfpathlineto{\pgfqpoint{2.478085in}{2.042585in}}%
\pgfpathlineto{\pgfqpoint{2.478085in}{2.070145in}}%
\pgfpathlineto{\pgfqpoint{2.229422in}{2.070145in}}%
\pgfpathlineto{\pgfqpoint{2.229422in}{2.042585in}}%
\pgfpathclose%
\pgfusepath{fill}%
\end{pgfscope}%
\begin{pgfscope}%
\pgfpathrectangle{\pgfqpoint{0.526080in}{0.310883in}}{\pgfqpoint{4.650000in}{3.080000in}}%
\pgfusepath{clip}%
\pgfsetbuttcap%
\pgfsetmiterjoin%
\definecolor{currentfill}{rgb}{0.854902,0.439216,0.839216}%
\pgfsetfillcolor{currentfill}%
\pgfsetlinewidth{0.000000pt}%
\definecolor{currentstroke}{rgb}{0.000000,0.000000,0.000000}%
\pgfsetstrokecolor{currentstroke}%
\pgfsetstrokeopacity{0.000000}%
\pgfsetdash{}{0pt}%
\pgfpathmoveto{\pgfqpoint{2.229422in}{2.070145in}}%
\pgfpathlineto{\pgfqpoint{2.478085in}{2.070145in}}%
\pgfpathlineto{\pgfqpoint{2.478085in}{2.078396in}}%
\pgfpathlineto{\pgfqpoint{2.229422in}{2.078396in}}%
\pgfpathlineto{\pgfqpoint{2.229422in}{2.070145in}}%
\pgfpathclose%
\pgfusepath{fill}%
\end{pgfscope}%
\begin{pgfscope}%
\pgfpathrectangle{\pgfqpoint{0.526080in}{0.310883in}}{\pgfqpoint{4.650000in}{3.080000in}}%
\pgfusepath{clip}%
\pgfsetbuttcap%
\pgfsetmiterjoin%
\definecolor{currentfill}{rgb}{0.993725,0.850196,0.704314}%
\pgfsetfillcolor{currentfill}%
\pgfsetlinewidth{0.000000pt}%
\definecolor{currentstroke}{rgb}{0.000000,0.000000,0.000000}%
\pgfsetstrokecolor{currentstroke}%
\pgfsetstrokeopacity{0.000000}%
\pgfsetdash{}{0pt}%
\pgfpathmoveto{\pgfqpoint{2.726748in}{0.310883in}}%
\pgfpathlineto{\pgfqpoint{2.975411in}{0.310883in}}%
\pgfpathlineto{\pgfqpoint{2.975411in}{0.325773in}}%
\pgfpathlineto{\pgfqpoint{2.726748in}{0.325773in}}%
\pgfpathlineto{\pgfqpoint{2.726748in}{0.310883in}}%
\pgfpathclose%
\pgfusepath{fill}%
\end{pgfscope}%
\begin{pgfscope}%
\pgfpathrectangle{\pgfqpoint{0.526080in}{0.310883in}}{\pgfqpoint{4.650000in}{3.080000in}}%
\pgfusepath{clip}%
\pgfsetbuttcap%
\pgfsetmiterjoin%
\definecolor{currentfill}{rgb}{0.992157,0.710065,0.464437}%
\pgfsetfillcolor{currentfill}%
\pgfsetlinewidth{0.000000pt}%
\definecolor{currentstroke}{rgb}{0.000000,0.000000,0.000000}%
\pgfsetstrokecolor{currentstroke}%
\pgfsetstrokeopacity{0.000000}%
\pgfsetdash{}{0pt}%
\pgfpathmoveto{\pgfqpoint{2.726748in}{0.325773in}}%
\pgfpathlineto{\pgfqpoint{2.975411in}{0.325773in}}%
\pgfpathlineto{\pgfqpoint{2.975411in}{0.416865in}}%
\pgfpathlineto{\pgfqpoint{2.726748in}{0.416865in}}%
\pgfpathlineto{\pgfqpoint{2.726748in}{0.325773in}}%
\pgfpathclose%
\pgfusepath{fill}%
\end{pgfscope}%
\begin{pgfscope}%
\pgfpathrectangle{\pgfqpoint{0.526080in}{0.310883in}}{\pgfqpoint{4.650000in}{3.080000in}}%
\pgfusepath{clip}%
\pgfsetbuttcap%
\pgfsetmiterjoin%
\definecolor{currentfill}{rgb}{0.991419,0.550727,0.232772}%
\pgfsetfillcolor{currentfill}%
\pgfsetlinewidth{0.000000pt}%
\definecolor{currentstroke}{rgb}{0.000000,0.000000,0.000000}%
\pgfsetstrokecolor{currentstroke}%
\pgfsetstrokeopacity{0.000000}%
\pgfsetdash{}{0pt}%
\pgfpathmoveto{\pgfqpoint{2.726748in}{0.416865in}}%
\pgfpathlineto{\pgfqpoint{2.975411in}{0.416865in}}%
\pgfpathlineto{\pgfqpoint{2.975411in}{0.849902in}}%
\pgfpathlineto{\pgfqpoint{2.726748in}{0.849902in}}%
\pgfpathlineto{\pgfqpoint{2.726748in}{0.416865in}}%
\pgfpathclose%
\pgfusepath{fill}%
\end{pgfscope}%
\begin{pgfscope}%
\pgfpathrectangle{\pgfqpoint{0.526080in}{0.310883in}}{\pgfqpoint{4.650000in}{3.080000in}}%
\pgfusepath{clip}%
\pgfsetbuttcap%
\pgfsetmiterjoin%
\definecolor{currentfill}{rgb}{0.925536,0.384867,0.059839}%
\pgfsetfillcolor{currentfill}%
\pgfsetlinewidth{0.000000pt}%
\definecolor{currentstroke}{rgb}{0.000000,0.000000,0.000000}%
\pgfsetstrokecolor{currentstroke}%
\pgfsetstrokeopacity{0.000000}%
\pgfsetdash{}{0pt}%
\pgfpathmoveto{\pgfqpoint{2.726748in}{0.849902in}}%
\pgfpathlineto{\pgfqpoint{2.975411in}{0.849902in}}%
\pgfpathlineto{\pgfqpoint{2.975411in}{0.864775in}}%
\pgfpathlineto{\pgfqpoint{2.726748in}{0.864775in}}%
\pgfpathlineto{\pgfqpoint{2.726748in}{0.849902in}}%
\pgfpathclose%
\pgfusepath{fill}%
\end{pgfscope}%
\begin{pgfscope}%
\pgfpathrectangle{\pgfqpoint{0.526080in}{0.310883in}}{\pgfqpoint{4.650000in}{3.080000in}}%
\pgfusepath{clip}%
\pgfsetbuttcap%
\pgfsetmiterjoin%
\definecolor{currentfill}{rgb}{0.887059,0.887059,0.887059}%
\pgfsetfillcolor{currentfill}%
\pgfsetlinewidth{0.000000pt}%
\definecolor{currentstroke}{rgb}{0.000000,0.000000,0.000000}%
\pgfsetstrokecolor{currentstroke}%
\pgfsetstrokeopacity{0.000000}%
\pgfsetdash{}{0pt}%
\pgfpathmoveto{\pgfqpoint{2.726748in}{0.864775in}}%
\pgfpathlineto{\pgfqpoint{2.975411in}{0.864775in}}%
\pgfpathlineto{\pgfqpoint{2.975411in}{1.339668in}}%
\pgfpathlineto{\pgfqpoint{2.726748in}{1.339668in}}%
\pgfpathlineto{\pgfqpoint{2.726748in}{0.864775in}}%
\pgfpathclose%
\pgfusepath{fill}%
\end{pgfscope}%
\begin{pgfscope}%
\pgfpathrectangle{\pgfqpoint{0.526080in}{0.310883in}}{\pgfqpoint{4.650000in}{3.080000in}}%
\pgfusepath{clip}%
\pgfsetbuttcap%
\pgfsetmiterjoin%
\definecolor{currentfill}{rgb}{0.710588,0.710588,0.710588}%
\pgfsetfillcolor{currentfill}%
\pgfsetlinewidth{0.000000pt}%
\definecolor{currentstroke}{rgb}{0.000000,0.000000,0.000000}%
\pgfsetstrokecolor{currentstroke}%
\pgfsetstrokeopacity{0.000000}%
\pgfsetdash{}{0pt}%
\pgfpathmoveto{\pgfqpoint{2.726748in}{1.339668in}}%
\pgfpathlineto{\pgfqpoint{2.975411in}{1.339668in}}%
\pgfpathlineto{\pgfqpoint{2.975411in}{1.658747in}}%
\pgfpathlineto{\pgfqpoint{2.726748in}{1.658747in}}%
\pgfpathlineto{\pgfqpoint{2.726748in}{1.339668in}}%
\pgfpathclose%
\pgfusepath{fill}%
\end{pgfscope}%
\begin{pgfscope}%
\pgfpathrectangle{\pgfqpoint{0.526080in}{0.310883in}}{\pgfqpoint{4.650000in}{3.080000in}}%
\pgfusepath{clip}%
\pgfsetbuttcap%
\pgfsetmiterjoin%
\definecolor{currentfill}{rgb}{0.478431,0.478431,0.478431}%
\pgfsetfillcolor{currentfill}%
\pgfsetlinewidth{0.000000pt}%
\definecolor{currentstroke}{rgb}{0.000000,0.000000,0.000000}%
\pgfsetstrokecolor{currentstroke}%
\pgfsetstrokeopacity{0.000000}%
\pgfsetdash{}{0pt}%
\pgfpathmoveto{\pgfqpoint{2.726748in}{1.658747in}}%
\pgfpathlineto{\pgfqpoint{2.975411in}{1.658747in}}%
\pgfpathlineto{\pgfqpoint{2.975411in}{2.221304in}}%
\pgfpathlineto{\pgfqpoint{2.726748in}{2.221304in}}%
\pgfpathlineto{\pgfqpoint{2.726748in}{1.658747in}}%
\pgfpathclose%
\pgfusepath{fill}%
\end{pgfscope}%
\begin{pgfscope}%
\pgfpathrectangle{\pgfqpoint{0.526080in}{0.310883in}}{\pgfqpoint{4.650000in}{3.080000in}}%
\pgfusepath{clip}%
\pgfsetbuttcap%
\pgfsetmiterjoin%
\definecolor{currentfill}{rgb}{1.000000,0.752941,0.796078}%
\pgfsetfillcolor{currentfill}%
\pgfsetlinewidth{0.000000pt}%
\definecolor{currentstroke}{rgb}{0.000000,0.000000,0.000000}%
\pgfsetstrokecolor{currentstroke}%
\pgfsetstrokeopacity{0.000000}%
\pgfsetdash{}{0pt}%
\pgfpathmoveto{\pgfqpoint{2.726748in}{2.221304in}}%
\pgfpathlineto{\pgfqpoint{2.975411in}{2.221304in}}%
\pgfpathlineto{\pgfqpoint{2.975411in}{2.255917in}}%
\pgfpathlineto{\pgfqpoint{2.726748in}{2.255917in}}%
\pgfpathlineto{\pgfqpoint{2.726748in}{2.221304in}}%
\pgfpathclose%
\pgfusepath{fill}%
\end{pgfscope}%
\begin{pgfscope}%
\pgfpathrectangle{\pgfqpoint{0.526080in}{0.310883in}}{\pgfqpoint{4.650000in}{3.080000in}}%
\pgfusepath{clip}%
\pgfsetbuttcap%
\pgfsetmiterjoin%
\definecolor{currentfill}{rgb}{0.854902,0.439216,0.839216}%
\pgfsetfillcolor{currentfill}%
\pgfsetlinewidth{0.000000pt}%
\definecolor{currentstroke}{rgb}{0.000000,0.000000,0.000000}%
\pgfsetstrokecolor{currentstroke}%
\pgfsetstrokeopacity{0.000000}%
\pgfsetdash{}{0pt}%
\pgfpathmoveto{\pgfqpoint{2.726748in}{2.255917in}}%
\pgfpathlineto{\pgfqpoint{2.975411in}{2.255917in}}%
\pgfpathlineto{\pgfqpoint{2.975411in}{2.264210in}}%
\pgfpathlineto{\pgfqpoint{2.726748in}{2.264210in}}%
\pgfpathlineto{\pgfqpoint{2.726748in}{2.255917in}}%
\pgfpathclose%
\pgfusepath{fill}%
\end{pgfscope}%
\begin{pgfscope}%
\pgfpathrectangle{\pgfqpoint{0.526080in}{0.310883in}}{\pgfqpoint{4.650000in}{3.080000in}}%
\pgfusepath{clip}%
\pgfsetbuttcap%
\pgfsetmiterjoin%
\definecolor{currentfill}{rgb}{0.993725,0.850196,0.704314}%
\pgfsetfillcolor{currentfill}%
\pgfsetlinewidth{0.000000pt}%
\definecolor{currentstroke}{rgb}{0.000000,0.000000,0.000000}%
\pgfsetstrokecolor{currentstroke}%
\pgfsetstrokeopacity{0.000000}%
\pgfsetdash{}{0pt}%
\pgfpathmoveto{\pgfqpoint{3.224075in}{0.310883in}}%
\pgfpathlineto{\pgfqpoint{3.472738in}{0.310883in}}%
\pgfpathlineto{\pgfqpoint{3.472738in}{0.325849in}}%
\pgfpathlineto{\pgfqpoint{3.224075in}{0.325849in}}%
\pgfpathlineto{\pgfqpoint{3.224075in}{0.310883in}}%
\pgfpathclose%
\pgfusepath{fill}%
\end{pgfscope}%
\begin{pgfscope}%
\pgfpathrectangle{\pgfqpoint{0.526080in}{0.310883in}}{\pgfqpoint{4.650000in}{3.080000in}}%
\pgfusepath{clip}%
\pgfsetbuttcap%
\pgfsetmiterjoin%
\definecolor{currentfill}{rgb}{0.992157,0.710065,0.464437}%
\pgfsetfillcolor{currentfill}%
\pgfsetlinewidth{0.000000pt}%
\definecolor{currentstroke}{rgb}{0.000000,0.000000,0.000000}%
\pgfsetstrokecolor{currentstroke}%
\pgfsetstrokeopacity{0.000000}%
\pgfsetdash{}{0pt}%
\pgfpathmoveto{\pgfqpoint{3.224075in}{0.325849in}}%
\pgfpathlineto{\pgfqpoint{3.472738in}{0.325849in}}%
\pgfpathlineto{\pgfqpoint{3.472738in}{0.417312in}}%
\pgfpathlineto{\pgfqpoint{3.224075in}{0.417312in}}%
\pgfpathlineto{\pgfqpoint{3.224075in}{0.325849in}}%
\pgfpathclose%
\pgfusepath{fill}%
\end{pgfscope}%
\begin{pgfscope}%
\pgfpathrectangle{\pgfqpoint{0.526080in}{0.310883in}}{\pgfqpoint{4.650000in}{3.080000in}}%
\pgfusepath{clip}%
\pgfsetbuttcap%
\pgfsetmiterjoin%
\definecolor{currentfill}{rgb}{0.991419,0.550727,0.232772}%
\pgfsetfillcolor{currentfill}%
\pgfsetlinewidth{0.000000pt}%
\definecolor{currentstroke}{rgb}{0.000000,0.000000,0.000000}%
\pgfsetstrokecolor{currentstroke}%
\pgfsetstrokeopacity{0.000000}%
\pgfsetdash{}{0pt}%
\pgfpathmoveto{\pgfqpoint{3.224075in}{0.417312in}}%
\pgfpathlineto{\pgfqpoint{3.472738in}{0.417312in}}%
\pgfpathlineto{\pgfqpoint{3.472738in}{0.852388in}}%
\pgfpathlineto{\pgfqpoint{3.224075in}{0.852388in}}%
\pgfpathlineto{\pgfqpoint{3.224075in}{0.417312in}}%
\pgfpathclose%
\pgfusepath{fill}%
\end{pgfscope}%
\begin{pgfscope}%
\pgfpathrectangle{\pgfqpoint{0.526080in}{0.310883in}}{\pgfqpoint{4.650000in}{3.080000in}}%
\pgfusepath{clip}%
\pgfsetbuttcap%
\pgfsetmiterjoin%
\definecolor{currentfill}{rgb}{0.925536,0.384867,0.059839}%
\pgfsetfillcolor{currentfill}%
\pgfsetlinewidth{0.000000pt}%
\definecolor{currentstroke}{rgb}{0.000000,0.000000,0.000000}%
\pgfsetstrokecolor{currentstroke}%
\pgfsetstrokeopacity{0.000000}%
\pgfsetdash{}{0pt}%
\pgfpathmoveto{\pgfqpoint{3.224075in}{0.852388in}}%
\pgfpathlineto{\pgfqpoint{3.472738in}{0.852388in}}%
\pgfpathlineto{\pgfqpoint{3.472738in}{0.867340in}}%
\pgfpathlineto{\pgfqpoint{3.224075in}{0.867340in}}%
\pgfpathlineto{\pgfqpoint{3.224075in}{0.852388in}}%
\pgfpathclose%
\pgfusepath{fill}%
\end{pgfscope}%
\begin{pgfscope}%
\pgfpathrectangle{\pgfqpoint{0.526080in}{0.310883in}}{\pgfqpoint{4.650000in}{3.080000in}}%
\pgfusepath{clip}%
\pgfsetbuttcap%
\pgfsetmiterjoin%
\definecolor{currentfill}{rgb}{0.887059,0.887059,0.887059}%
\pgfsetfillcolor{currentfill}%
\pgfsetlinewidth{0.000000pt}%
\definecolor{currentstroke}{rgb}{0.000000,0.000000,0.000000}%
\pgfsetstrokecolor{currentstroke}%
\pgfsetstrokeopacity{0.000000}%
\pgfsetdash{}{0pt}%
\pgfpathmoveto{\pgfqpoint{3.224075in}{0.867340in}}%
\pgfpathlineto{\pgfqpoint{3.472738in}{0.867340in}}%
\pgfpathlineto{\pgfqpoint{3.472738in}{1.342642in}}%
\pgfpathlineto{\pgfqpoint{3.224075in}{1.342642in}}%
\pgfpathlineto{\pgfqpoint{3.224075in}{0.867340in}}%
\pgfpathclose%
\pgfusepath{fill}%
\end{pgfscope}%
\begin{pgfscope}%
\pgfpathrectangle{\pgfqpoint{0.526080in}{0.310883in}}{\pgfqpoint{4.650000in}{3.080000in}}%
\pgfusepath{clip}%
\pgfsetbuttcap%
\pgfsetmiterjoin%
\definecolor{currentfill}{rgb}{0.710588,0.710588,0.710588}%
\pgfsetfillcolor{currentfill}%
\pgfsetlinewidth{0.000000pt}%
\definecolor{currentstroke}{rgb}{0.000000,0.000000,0.000000}%
\pgfsetstrokecolor{currentstroke}%
\pgfsetstrokeopacity{0.000000}%
\pgfsetdash{}{0pt}%
\pgfpathmoveto{\pgfqpoint{3.224075in}{1.342642in}}%
\pgfpathlineto{\pgfqpoint{3.472738in}{1.342642in}}%
\pgfpathlineto{\pgfqpoint{3.472738in}{1.716048in}}%
\pgfpathlineto{\pgfqpoint{3.224075in}{1.716048in}}%
\pgfpathlineto{\pgfqpoint{3.224075in}{1.342642in}}%
\pgfpathclose%
\pgfusepath{fill}%
\end{pgfscope}%
\begin{pgfscope}%
\pgfpathrectangle{\pgfqpoint{0.526080in}{0.310883in}}{\pgfqpoint{4.650000in}{3.080000in}}%
\pgfusepath{clip}%
\pgfsetbuttcap%
\pgfsetmiterjoin%
\definecolor{currentfill}{rgb}{0.478431,0.478431,0.478431}%
\pgfsetfillcolor{currentfill}%
\pgfsetlinewidth{0.000000pt}%
\definecolor{currentstroke}{rgb}{0.000000,0.000000,0.000000}%
\pgfsetstrokecolor{currentstroke}%
\pgfsetstrokeopacity{0.000000}%
\pgfsetdash{}{0pt}%
\pgfpathmoveto{\pgfqpoint{3.224075in}{1.716048in}}%
\pgfpathlineto{\pgfqpoint{3.472738in}{1.716048in}}%
\pgfpathlineto{\pgfqpoint{3.472738in}{2.390379in}}%
\pgfpathlineto{\pgfqpoint{3.224075in}{2.390379in}}%
\pgfpathlineto{\pgfqpoint{3.224075in}{1.716048in}}%
\pgfpathclose%
\pgfusepath{fill}%
\end{pgfscope}%
\begin{pgfscope}%
\pgfpathrectangle{\pgfqpoint{0.526080in}{0.310883in}}{\pgfqpoint{4.650000in}{3.080000in}}%
\pgfusepath{clip}%
\pgfsetbuttcap%
\pgfsetmiterjoin%
\definecolor{currentfill}{rgb}{1.000000,0.752941,0.796078}%
\pgfsetfillcolor{currentfill}%
\pgfsetlinewidth{0.000000pt}%
\definecolor{currentstroke}{rgb}{0.000000,0.000000,0.000000}%
\pgfsetstrokecolor{currentstroke}%
\pgfsetstrokeopacity{0.000000}%
\pgfsetdash{}{0pt}%
\pgfpathmoveto{\pgfqpoint{3.224075in}{2.390379in}}%
\pgfpathlineto{\pgfqpoint{3.472738in}{2.390379in}}%
\pgfpathlineto{\pgfqpoint{3.472738in}{2.431488in}}%
\pgfpathlineto{\pgfqpoint{3.224075in}{2.431488in}}%
\pgfpathlineto{\pgfqpoint{3.224075in}{2.390379in}}%
\pgfpathclose%
\pgfusepath{fill}%
\end{pgfscope}%
\begin{pgfscope}%
\pgfpathrectangle{\pgfqpoint{0.526080in}{0.310883in}}{\pgfqpoint{4.650000in}{3.080000in}}%
\pgfusepath{clip}%
\pgfsetbuttcap%
\pgfsetmiterjoin%
\definecolor{currentfill}{rgb}{0.854902,0.439216,0.839216}%
\pgfsetfillcolor{currentfill}%
\pgfsetlinewidth{0.000000pt}%
\definecolor{currentstroke}{rgb}{0.000000,0.000000,0.000000}%
\pgfsetstrokecolor{currentstroke}%
\pgfsetstrokeopacity{0.000000}%
\pgfsetdash{}{0pt}%
\pgfpathmoveto{\pgfqpoint{3.224075in}{2.431488in}}%
\pgfpathlineto{\pgfqpoint{3.472738in}{2.431488in}}%
\pgfpathlineto{\pgfqpoint{3.472738in}{2.439800in}}%
\pgfpathlineto{\pgfqpoint{3.224075in}{2.439800in}}%
\pgfpathlineto{\pgfqpoint{3.224075in}{2.431488in}}%
\pgfpathclose%
\pgfusepath{fill}%
\end{pgfscope}%
\begin{pgfscope}%
\pgfpathrectangle{\pgfqpoint{0.526080in}{0.310883in}}{\pgfqpoint{4.650000in}{3.080000in}}%
\pgfusepath{clip}%
\pgfsetbuttcap%
\pgfsetmiterjoin%
\definecolor{currentfill}{rgb}{0.993725,0.850196,0.704314}%
\pgfsetfillcolor{currentfill}%
\pgfsetlinewidth{0.000000pt}%
\definecolor{currentstroke}{rgb}{0.000000,0.000000,0.000000}%
\pgfsetstrokecolor{currentstroke}%
\pgfsetstrokeopacity{0.000000}%
\pgfsetdash{}{0pt}%
\pgfpathmoveto{\pgfqpoint{3.721401in}{0.310883in}}%
\pgfpathlineto{\pgfqpoint{3.970064in}{0.310883in}}%
\pgfpathlineto{\pgfqpoint{3.970064in}{0.325930in}}%
\pgfpathlineto{\pgfqpoint{3.721401in}{0.325930in}}%
\pgfpathlineto{\pgfqpoint{3.721401in}{0.310883in}}%
\pgfpathclose%
\pgfusepath{fill}%
\end{pgfscope}%
\begin{pgfscope}%
\pgfpathrectangle{\pgfqpoint{0.526080in}{0.310883in}}{\pgfqpoint{4.650000in}{3.080000in}}%
\pgfusepath{clip}%
\pgfsetbuttcap%
\pgfsetmiterjoin%
\definecolor{currentfill}{rgb}{0.992157,0.710065,0.464437}%
\pgfsetfillcolor{currentfill}%
\pgfsetlinewidth{0.000000pt}%
\definecolor{currentstroke}{rgb}{0.000000,0.000000,0.000000}%
\pgfsetstrokecolor{currentstroke}%
\pgfsetstrokeopacity{0.000000}%
\pgfsetdash{}{0pt}%
\pgfpathmoveto{\pgfqpoint{3.721401in}{0.325930in}}%
\pgfpathlineto{\pgfqpoint{3.970064in}{0.325930in}}%
\pgfpathlineto{\pgfqpoint{3.970064in}{0.417919in}}%
\pgfpathlineto{\pgfqpoint{3.721401in}{0.417919in}}%
\pgfpathlineto{\pgfqpoint{3.721401in}{0.325930in}}%
\pgfpathclose%
\pgfusepath{fill}%
\end{pgfscope}%
\begin{pgfscope}%
\pgfpathrectangle{\pgfqpoint{0.526080in}{0.310883in}}{\pgfqpoint{4.650000in}{3.080000in}}%
\pgfusepath{clip}%
\pgfsetbuttcap%
\pgfsetmiterjoin%
\definecolor{currentfill}{rgb}{0.991419,0.550727,0.232772}%
\pgfsetfillcolor{currentfill}%
\pgfsetlinewidth{0.000000pt}%
\definecolor{currentstroke}{rgb}{0.000000,0.000000,0.000000}%
\pgfsetstrokecolor{currentstroke}%
\pgfsetstrokeopacity{0.000000}%
\pgfsetdash{}{0pt}%
\pgfpathmoveto{\pgfqpoint{3.721401in}{0.417919in}}%
\pgfpathlineto{\pgfqpoint{3.970064in}{0.417919in}}%
\pgfpathlineto{\pgfqpoint{3.970064in}{0.855779in}}%
\pgfpathlineto{\pgfqpoint{3.721401in}{0.855779in}}%
\pgfpathlineto{\pgfqpoint{3.721401in}{0.417919in}}%
\pgfpathclose%
\pgfusepath{fill}%
\end{pgfscope}%
\begin{pgfscope}%
\pgfpathrectangle{\pgfqpoint{0.526080in}{0.310883in}}{\pgfqpoint{4.650000in}{3.080000in}}%
\pgfusepath{clip}%
\pgfsetbuttcap%
\pgfsetmiterjoin%
\definecolor{currentfill}{rgb}{0.925536,0.384867,0.059839}%
\pgfsetfillcolor{currentfill}%
\pgfsetlinewidth{0.000000pt}%
\definecolor{currentstroke}{rgb}{0.000000,0.000000,0.000000}%
\pgfsetstrokecolor{currentstroke}%
\pgfsetstrokeopacity{0.000000}%
\pgfsetdash{}{0pt}%
\pgfpathmoveto{\pgfqpoint{3.721401in}{0.855779in}}%
\pgfpathlineto{\pgfqpoint{3.970064in}{0.855779in}}%
\pgfpathlineto{\pgfqpoint{3.970064in}{0.870787in}}%
\pgfpathlineto{\pgfqpoint{3.721401in}{0.870787in}}%
\pgfpathlineto{\pgfqpoint{3.721401in}{0.855779in}}%
\pgfpathclose%
\pgfusepath{fill}%
\end{pgfscope}%
\begin{pgfscope}%
\pgfpathrectangle{\pgfqpoint{0.526080in}{0.310883in}}{\pgfqpoint{4.650000in}{3.080000in}}%
\pgfusepath{clip}%
\pgfsetbuttcap%
\pgfsetmiterjoin%
\definecolor{currentfill}{rgb}{0.887059,0.887059,0.887059}%
\pgfsetfillcolor{currentfill}%
\pgfsetlinewidth{0.000000pt}%
\definecolor{currentstroke}{rgb}{0.000000,0.000000,0.000000}%
\pgfsetstrokecolor{currentstroke}%
\pgfsetstrokeopacity{0.000000}%
\pgfsetdash{}{0pt}%
\pgfpathmoveto{\pgfqpoint{3.721401in}{0.870787in}}%
\pgfpathlineto{\pgfqpoint{3.970064in}{0.870787in}}%
\pgfpathlineto{\pgfqpoint{3.970064in}{1.346313in}}%
\pgfpathlineto{\pgfqpoint{3.721401in}{1.346313in}}%
\pgfpathlineto{\pgfqpoint{3.721401in}{0.870787in}}%
\pgfpathclose%
\pgfusepath{fill}%
\end{pgfscope}%
\begin{pgfscope}%
\pgfpathrectangle{\pgfqpoint{0.526080in}{0.310883in}}{\pgfqpoint{4.650000in}{3.080000in}}%
\pgfusepath{clip}%
\pgfsetbuttcap%
\pgfsetmiterjoin%
\definecolor{currentfill}{rgb}{0.710588,0.710588,0.710588}%
\pgfsetfillcolor{currentfill}%
\pgfsetlinewidth{0.000000pt}%
\definecolor{currentstroke}{rgb}{0.000000,0.000000,0.000000}%
\pgfsetstrokecolor{currentstroke}%
\pgfsetstrokeopacity{0.000000}%
\pgfsetdash{}{0pt}%
\pgfpathmoveto{\pgfqpoint{3.721401in}{1.346313in}}%
\pgfpathlineto{\pgfqpoint{3.970064in}{1.346313in}}%
\pgfpathlineto{\pgfqpoint{3.970064in}{1.785253in}}%
\pgfpathlineto{\pgfqpoint{3.721401in}{1.785253in}}%
\pgfpathlineto{\pgfqpoint{3.721401in}{1.346313in}}%
\pgfpathclose%
\pgfusepath{fill}%
\end{pgfscope}%
\begin{pgfscope}%
\pgfpathrectangle{\pgfqpoint{0.526080in}{0.310883in}}{\pgfqpoint{4.650000in}{3.080000in}}%
\pgfusepath{clip}%
\pgfsetbuttcap%
\pgfsetmiterjoin%
\definecolor{currentfill}{rgb}{0.478431,0.478431,0.478431}%
\pgfsetfillcolor{currentfill}%
\pgfsetlinewidth{0.000000pt}%
\definecolor{currentstroke}{rgb}{0.000000,0.000000,0.000000}%
\pgfsetstrokecolor{currentstroke}%
\pgfsetstrokeopacity{0.000000}%
\pgfsetdash{}{0pt}%
\pgfpathmoveto{\pgfqpoint{3.721401in}{1.785253in}}%
\pgfpathlineto{\pgfqpoint{3.970064in}{1.785253in}}%
\pgfpathlineto{\pgfqpoint{3.970064in}{2.573033in}}%
\pgfpathlineto{\pgfqpoint{3.721401in}{2.573033in}}%
\pgfpathlineto{\pgfqpoint{3.721401in}{1.785253in}}%
\pgfpathclose%
\pgfusepath{fill}%
\end{pgfscope}%
\begin{pgfscope}%
\pgfpathrectangle{\pgfqpoint{0.526080in}{0.310883in}}{\pgfqpoint{4.650000in}{3.080000in}}%
\pgfusepath{clip}%
\pgfsetbuttcap%
\pgfsetmiterjoin%
\definecolor{currentfill}{rgb}{1.000000,0.752941,0.796078}%
\pgfsetfillcolor{currentfill}%
\pgfsetlinewidth{0.000000pt}%
\definecolor{currentstroke}{rgb}{0.000000,0.000000,0.000000}%
\pgfsetstrokecolor{currentstroke}%
\pgfsetstrokeopacity{0.000000}%
\pgfsetdash{}{0pt}%
\pgfpathmoveto{\pgfqpoint{3.721401in}{2.573033in}}%
\pgfpathlineto{\pgfqpoint{3.970064in}{2.573033in}}%
\pgfpathlineto{\pgfqpoint{3.970064in}{2.621145in}}%
\pgfpathlineto{\pgfqpoint{3.721401in}{2.621145in}}%
\pgfpathlineto{\pgfqpoint{3.721401in}{2.573033in}}%
\pgfpathclose%
\pgfusepath{fill}%
\end{pgfscope}%
\begin{pgfscope}%
\pgfpathrectangle{\pgfqpoint{0.526080in}{0.310883in}}{\pgfqpoint{4.650000in}{3.080000in}}%
\pgfusepath{clip}%
\pgfsetbuttcap%
\pgfsetmiterjoin%
\definecolor{currentfill}{rgb}{0.854902,0.439216,0.839216}%
\pgfsetfillcolor{currentfill}%
\pgfsetlinewidth{0.000000pt}%
\definecolor{currentstroke}{rgb}{0.000000,0.000000,0.000000}%
\pgfsetstrokecolor{currentstroke}%
\pgfsetstrokeopacity{0.000000}%
\pgfsetdash{}{0pt}%
\pgfpathmoveto{\pgfqpoint{3.721401in}{2.621145in}}%
\pgfpathlineto{\pgfqpoint{3.970064in}{2.621145in}}%
\pgfpathlineto{\pgfqpoint{3.970064in}{2.629514in}}%
\pgfpathlineto{\pgfqpoint{3.721401in}{2.629514in}}%
\pgfpathlineto{\pgfqpoint{3.721401in}{2.621145in}}%
\pgfpathclose%
\pgfusepath{fill}%
\end{pgfscope}%
\begin{pgfscope}%
\pgfpathrectangle{\pgfqpoint{0.526080in}{0.310883in}}{\pgfqpoint{4.650000in}{3.080000in}}%
\pgfusepath{clip}%
\pgfsetbuttcap%
\pgfsetmiterjoin%
\definecolor{currentfill}{rgb}{0.993725,0.850196,0.704314}%
\pgfsetfillcolor{currentfill}%
\pgfsetlinewidth{0.000000pt}%
\definecolor{currentstroke}{rgb}{0.000000,0.000000,0.000000}%
\pgfsetstrokecolor{currentstroke}%
\pgfsetstrokeopacity{0.000000}%
\pgfsetdash{}{0pt}%
\pgfpathmoveto{\pgfqpoint{4.218727in}{0.310883in}}%
\pgfpathlineto{\pgfqpoint{4.467390in}{0.310883in}}%
\pgfpathlineto{\pgfqpoint{4.467390in}{0.326051in}}%
\pgfpathlineto{\pgfqpoint{4.218727in}{0.326051in}}%
\pgfpathlineto{\pgfqpoint{4.218727in}{0.310883in}}%
\pgfpathclose%
\pgfusepath{fill}%
\end{pgfscope}%
\begin{pgfscope}%
\pgfpathrectangle{\pgfqpoint{0.526080in}{0.310883in}}{\pgfqpoint{4.650000in}{3.080000in}}%
\pgfusepath{clip}%
\pgfsetbuttcap%
\pgfsetmiterjoin%
\definecolor{currentfill}{rgb}{0.992157,0.710065,0.464437}%
\pgfsetfillcolor{currentfill}%
\pgfsetlinewidth{0.000000pt}%
\definecolor{currentstroke}{rgb}{0.000000,0.000000,0.000000}%
\pgfsetstrokecolor{currentstroke}%
\pgfsetstrokeopacity{0.000000}%
\pgfsetdash{}{0pt}%
\pgfpathmoveto{\pgfqpoint{4.218727in}{0.326051in}}%
\pgfpathlineto{\pgfqpoint{4.467390in}{0.326051in}}%
\pgfpathlineto{\pgfqpoint{4.467390in}{0.418428in}}%
\pgfpathlineto{\pgfqpoint{4.218727in}{0.418428in}}%
\pgfpathlineto{\pgfqpoint{4.218727in}{0.326051in}}%
\pgfpathclose%
\pgfusepath{fill}%
\end{pgfscope}%
\begin{pgfscope}%
\pgfpathrectangle{\pgfqpoint{0.526080in}{0.310883in}}{\pgfqpoint{4.650000in}{3.080000in}}%
\pgfusepath{clip}%
\pgfsetbuttcap%
\pgfsetmiterjoin%
\definecolor{currentfill}{rgb}{0.991419,0.550727,0.232772}%
\pgfsetfillcolor{currentfill}%
\pgfsetlinewidth{0.000000pt}%
\definecolor{currentstroke}{rgb}{0.000000,0.000000,0.000000}%
\pgfsetstrokecolor{currentstroke}%
\pgfsetstrokeopacity{0.000000}%
\pgfsetdash{}{0pt}%
\pgfpathmoveto{\pgfqpoint{4.218727in}{0.418428in}}%
\pgfpathlineto{\pgfqpoint{4.467390in}{0.418428in}}%
\pgfpathlineto{\pgfqpoint{4.467390in}{0.857350in}}%
\pgfpathlineto{\pgfqpoint{4.218727in}{0.857350in}}%
\pgfpathlineto{\pgfqpoint{4.218727in}{0.418428in}}%
\pgfpathclose%
\pgfusepath{fill}%
\end{pgfscope}%
\begin{pgfscope}%
\pgfpathrectangle{\pgfqpoint{0.526080in}{0.310883in}}{\pgfqpoint{4.650000in}{3.080000in}}%
\pgfusepath{clip}%
\pgfsetbuttcap%
\pgfsetmiterjoin%
\definecolor{currentfill}{rgb}{0.925536,0.384867,0.059839}%
\pgfsetfillcolor{currentfill}%
\pgfsetlinewidth{0.000000pt}%
\definecolor{currentstroke}{rgb}{0.000000,0.000000,0.000000}%
\pgfsetstrokecolor{currentstroke}%
\pgfsetstrokeopacity{0.000000}%
\pgfsetdash{}{0pt}%
\pgfpathmoveto{\pgfqpoint{4.218727in}{0.857350in}}%
\pgfpathlineto{\pgfqpoint{4.467390in}{0.857350in}}%
\pgfpathlineto{\pgfqpoint{4.467390in}{0.872405in}}%
\pgfpathlineto{\pgfqpoint{4.218727in}{0.872405in}}%
\pgfpathlineto{\pgfqpoint{4.218727in}{0.857350in}}%
\pgfpathclose%
\pgfusepath{fill}%
\end{pgfscope}%
\begin{pgfscope}%
\pgfpathrectangle{\pgfqpoint{0.526080in}{0.310883in}}{\pgfqpoint{4.650000in}{3.080000in}}%
\pgfusepath{clip}%
\pgfsetbuttcap%
\pgfsetmiterjoin%
\definecolor{currentfill}{rgb}{0.887059,0.887059,0.887059}%
\pgfsetfillcolor{currentfill}%
\pgfsetlinewidth{0.000000pt}%
\definecolor{currentstroke}{rgb}{0.000000,0.000000,0.000000}%
\pgfsetstrokecolor{currentstroke}%
\pgfsetstrokeopacity{0.000000}%
\pgfsetdash{}{0pt}%
\pgfpathmoveto{\pgfqpoint{4.218727in}{0.872405in}}%
\pgfpathlineto{\pgfqpoint{4.467390in}{0.872405in}}%
\pgfpathlineto{\pgfqpoint{4.467390in}{1.347834in}}%
\pgfpathlineto{\pgfqpoint{4.218727in}{1.347834in}}%
\pgfpathlineto{\pgfqpoint{4.218727in}{0.872405in}}%
\pgfpathclose%
\pgfusepath{fill}%
\end{pgfscope}%
\begin{pgfscope}%
\pgfpathrectangle{\pgfqpoint{0.526080in}{0.310883in}}{\pgfqpoint{4.650000in}{3.080000in}}%
\pgfusepath{clip}%
\pgfsetbuttcap%
\pgfsetmiterjoin%
\definecolor{currentfill}{rgb}{0.710588,0.710588,0.710588}%
\pgfsetfillcolor{currentfill}%
\pgfsetlinewidth{0.000000pt}%
\definecolor{currentstroke}{rgb}{0.000000,0.000000,0.000000}%
\pgfsetstrokecolor{currentstroke}%
\pgfsetstrokeopacity{0.000000}%
\pgfsetdash{}{0pt}%
\pgfpathmoveto{\pgfqpoint{4.218727in}{1.347834in}}%
\pgfpathlineto{\pgfqpoint{4.467390in}{1.347834in}}%
\pgfpathlineto{\pgfqpoint{4.467390in}{1.846829in}}%
\pgfpathlineto{\pgfqpoint{4.218727in}{1.846829in}}%
\pgfpathlineto{\pgfqpoint{4.218727in}{1.347834in}}%
\pgfpathclose%
\pgfusepath{fill}%
\end{pgfscope}%
\begin{pgfscope}%
\pgfpathrectangle{\pgfqpoint{0.526080in}{0.310883in}}{\pgfqpoint{4.650000in}{3.080000in}}%
\pgfusepath{clip}%
\pgfsetbuttcap%
\pgfsetmiterjoin%
\definecolor{currentfill}{rgb}{0.478431,0.478431,0.478431}%
\pgfsetfillcolor{currentfill}%
\pgfsetlinewidth{0.000000pt}%
\definecolor{currentstroke}{rgb}{0.000000,0.000000,0.000000}%
\pgfsetstrokecolor{currentstroke}%
\pgfsetstrokeopacity{0.000000}%
\pgfsetdash{}{0pt}%
\pgfpathmoveto{\pgfqpoint{4.218727in}{1.846829in}}%
\pgfpathlineto{\pgfqpoint{4.467390in}{1.846829in}}%
\pgfpathlineto{\pgfqpoint{4.467390in}{2.736570in}}%
\pgfpathlineto{\pgfqpoint{4.218727in}{2.736570in}}%
\pgfpathlineto{\pgfqpoint{4.218727in}{1.846829in}}%
\pgfpathclose%
\pgfusepath{fill}%
\end{pgfscope}%
\begin{pgfscope}%
\pgfpathrectangle{\pgfqpoint{0.526080in}{0.310883in}}{\pgfqpoint{4.650000in}{3.080000in}}%
\pgfusepath{clip}%
\pgfsetbuttcap%
\pgfsetmiterjoin%
\definecolor{currentfill}{rgb}{1.000000,0.752941,0.796078}%
\pgfsetfillcolor{currentfill}%
\pgfsetlinewidth{0.000000pt}%
\definecolor{currentstroke}{rgb}{0.000000,0.000000,0.000000}%
\pgfsetstrokecolor{currentstroke}%
\pgfsetstrokeopacity{0.000000}%
\pgfsetdash{}{0pt}%
\pgfpathmoveto{\pgfqpoint{4.218727in}{2.736570in}}%
\pgfpathlineto{\pgfqpoint{4.467390in}{2.736570in}}%
\pgfpathlineto{\pgfqpoint{4.467390in}{2.791507in}}%
\pgfpathlineto{\pgfqpoint{4.218727in}{2.791507in}}%
\pgfpathlineto{\pgfqpoint{4.218727in}{2.736570in}}%
\pgfpathclose%
\pgfusepath{fill}%
\end{pgfscope}%
\begin{pgfscope}%
\pgfpathrectangle{\pgfqpoint{0.526080in}{0.310883in}}{\pgfqpoint{4.650000in}{3.080000in}}%
\pgfusepath{clip}%
\pgfsetbuttcap%
\pgfsetmiterjoin%
\definecolor{currentfill}{rgb}{0.854902,0.439216,0.839216}%
\pgfsetfillcolor{currentfill}%
\pgfsetlinewidth{0.000000pt}%
\definecolor{currentstroke}{rgb}{0.000000,0.000000,0.000000}%
\pgfsetstrokecolor{currentstroke}%
\pgfsetstrokeopacity{0.000000}%
\pgfsetdash{}{0pt}%
\pgfpathmoveto{\pgfqpoint{4.218727in}{2.791507in}}%
\pgfpathlineto{\pgfqpoint{4.467390in}{2.791507in}}%
\pgfpathlineto{\pgfqpoint{4.467390in}{2.799898in}}%
\pgfpathlineto{\pgfqpoint{4.218727in}{2.799898in}}%
\pgfpathlineto{\pgfqpoint{4.218727in}{2.791507in}}%
\pgfpathclose%
\pgfusepath{fill}%
\end{pgfscope}%
\begin{pgfscope}%
\pgfpathrectangle{\pgfqpoint{0.526080in}{0.310883in}}{\pgfqpoint{4.650000in}{3.080000in}}%
\pgfusepath{clip}%
\pgfsetbuttcap%
\pgfsetmiterjoin%
\definecolor{currentfill}{rgb}{0.993725,0.850196,0.704314}%
\pgfsetfillcolor{currentfill}%
\pgfsetlinewidth{0.000000pt}%
\definecolor{currentstroke}{rgb}{0.000000,0.000000,0.000000}%
\pgfsetstrokecolor{currentstroke}%
\pgfsetstrokeopacity{0.000000}%
\pgfsetdash{}{0pt}%
\pgfpathmoveto{\pgfqpoint{4.716053in}{0.310883in}}%
\pgfpathlineto{\pgfqpoint{4.964716in}{0.310883in}}%
\pgfpathlineto{\pgfqpoint{4.964716in}{0.326178in}}%
\pgfpathlineto{\pgfqpoint{4.716053in}{0.326178in}}%
\pgfpathlineto{\pgfqpoint{4.716053in}{0.310883in}}%
\pgfpathclose%
\pgfusepath{fill}%
\end{pgfscope}%
\begin{pgfscope}%
\pgfpathrectangle{\pgfqpoint{0.526080in}{0.310883in}}{\pgfqpoint{4.650000in}{3.080000in}}%
\pgfusepath{clip}%
\pgfsetbuttcap%
\pgfsetmiterjoin%
\definecolor{currentfill}{rgb}{0.992157,0.710065,0.464437}%
\pgfsetfillcolor{currentfill}%
\pgfsetlinewidth{0.000000pt}%
\definecolor{currentstroke}{rgb}{0.000000,0.000000,0.000000}%
\pgfsetstrokecolor{currentstroke}%
\pgfsetstrokeopacity{0.000000}%
\pgfsetdash{}{0pt}%
\pgfpathmoveto{\pgfqpoint{4.716053in}{0.326178in}}%
\pgfpathlineto{\pgfqpoint{4.964716in}{0.326178in}}%
\pgfpathlineto{\pgfqpoint{4.964716in}{0.419276in}}%
\pgfpathlineto{\pgfqpoint{4.716053in}{0.419276in}}%
\pgfpathlineto{\pgfqpoint{4.716053in}{0.326178in}}%
\pgfpathclose%
\pgfusepath{fill}%
\end{pgfscope}%
\begin{pgfscope}%
\pgfpathrectangle{\pgfqpoint{0.526080in}{0.310883in}}{\pgfqpoint{4.650000in}{3.080000in}}%
\pgfusepath{clip}%
\pgfsetbuttcap%
\pgfsetmiterjoin%
\definecolor{currentfill}{rgb}{0.991419,0.550727,0.232772}%
\pgfsetfillcolor{currentfill}%
\pgfsetlinewidth{0.000000pt}%
\definecolor{currentstroke}{rgb}{0.000000,0.000000,0.000000}%
\pgfsetstrokecolor{currentstroke}%
\pgfsetstrokeopacity{0.000000}%
\pgfsetdash{}{0pt}%
\pgfpathmoveto{\pgfqpoint{4.716053in}{0.419276in}}%
\pgfpathlineto{\pgfqpoint{4.964716in}{0.419276in}}%
\pgfpathlineto{\pgfqpoint{4.964716in}{0.859383in}}%
\pgfpathlineto{\pgfqpoint{4.716053in}{0.859383in}}%
\pgfpathlineto{\pgfqpoint{4.716053in}{0.419276in}}%
\pgfpathclose%
\pgfusepath{fill}%
\end{pgfscope}%
\begin{pgfscope}%
\pgfpathrectangle{\pgfqpoint{0.526080in}{0.310883in}}{\pgfqpoint{4.650000in}{3.080000in}}%
\pgfusepath{clip}%
\pgfsetbuttcap%
\pgfsetmiterjoin%
\definecolor{currentfill}{rgb}{0.925536,0.384867,0.059839}%
\pgfsetfillcolor{currentfill}%
\pgfsetlinewidth{0.000000pt}%
\definecolor{currentstroke}{rgb}{0.000000,0.000000,0.000000}%
\pgfsetstrokecolor{currentstroke}%
\pgfsetstrokeopacity{0.000000}%
\pgfsetdash{}{0pt}%
\pgfpathmoveto{\pgfqpoint{4.716053in}{0.859383in}}%
\pgfpathlineto{\pgfqpoint{4.964716in}{0.859383in}}%
\pgfpathlineto{\pgfqpoint{4.964716in}{0.874495in}}%
\pgfpathlineto{\pgfqpoint{4.716053in}{0.874495in}}%
\pgfpathlineto{\pgfqpoint{4.716053in}{0.859383in}}%
\pgfpathclose%
\pgfusepath{fill}%
\end{pgfscope}%
\begin{pgfscope}%
\pgfpathrectangle{\pgfqpoint{0.526080in}{0.310883in}}{\pgfqpoint{4.650000in}{3.080000in}}%
\pgfusepath{clip}%
\pgfsetbuttcap%
\pgfsetmiterjoin%
\definecolor{currentfill}{rgb}{0.887059,0.887059,0.887059}%
\pgfsetfillcolor{currentfill}%
\pgfsetlinewidth{0.000000pt}%
\definecolor{currentstroke}{rgb}{0.000000,0.000000,0.000000}%
\pgfsetstrokecolor{currentstroke}%
\pgfsetstrokeopacity{0.000000}%
\pgfsetdash{}{0pt}%
\pgfpathmoveto{\pgfqpoint{4.716053in}{0.874495in}}%
\pgfpathlineto{\pgfqpoint{4.964716in}{0.874495in}}%
\pgfpathlineto{\pgfqpoint{4.964716in}{1.349898in}}%
\pgfpathlineto{\pgfqpoint{4.716053in}{1.349898in}}%
\pgfpathlineto{\pgfqpoint{4.716053in}{0.874495in}}%
\pgfpathclose%
\pgfusepath{fill}%
\end{pgfscope}%
\begin{pgfscope}%
\pgfpathrectangle{\pgfqpoint{0.526080in}{0.310883in}}{\pgfqpoint{4.650000in}{3.080000in}}%
\pgfusepath{clip}%
\pgfsetbuttcap%
\pgfsetmiterjoin%
\definecolor{currentfill}{rgb}{0.710588,0.710588,0.710588}%
\pgfsetfillcolor{currentfill}%
\pgfsetlinewidth{0.000000pt}%
\definecolor{currentstroke}{rgb}{0.000000,0.000000,0.000000}%
\pgfsetstrokecolor{currentstroke}%
\pgfsetstrokeopacity{0.000000}%
\pgfsetdash{}{0pt}%
\pgfpathmoveto{\pgfqpoint{4.716053in}{1.349898in}}%
\pgfpathlineto{\pgfqpoint{4.964716in}{1.349898in}}%
\pgfpathlineto{\pgfqpoint{4.964716in}{1.903390in}}%
\pgfpathlineto{\pgfqpoint{4.716053in}{1.903390in}}%
\pgfpathlineto{\pgfqpoint{4.716053in}{1.349898in}}%
\pgfpathclose%
\pgfusepath{fill}%
\end{pgfscope}%
\begin{pgfscope}%
\pgfpathrectangle{\pgfqpoint{0.526080in}{0.310883in}}{\pgfqpoint{4.650000in}{3.080000in}}%
\pgfusepath{clip}%
\pgfsetbuttcap%
\pgfsetmiterjoin%
\definecolor{currentfill}{rgb}{0.478431,0.478431,0.478431}%
\pgfsetfillcolor{currentfill}%
\pgfsetlinewidth{0.000000pt}%
\definecolor{currentstroke}{rgb}{0.000000,0.000000,0.000000}%
\pgfsetstrokecolor{currentstroke}%
\pgfsetstrokeopacity{0.000000}%
\pgfsetdash{}{0pt}%
\pgfpathmoveto{\pgfqpoint{4.716053in}{1.903390in}}%
\pgfpathlineto{\pgfqpoint{4.964716in}{1.903390in}}%
\pgfpathlineto{\pgfqpoint{4.964716in}{2.907719in}}%
\pgfpathlineto{\pgfqpoint{4.716053in}{2.907719in}}%
\pgfpathlineto{\pgfqpoint{4.716053in}{1.903390in}}%
\pgfpathclose%
\pgfusepath{fill}%
\end{pgfscope}%
\begin{pgfscope}%
\pgfpathrectangle{\pgfqpoint{0.526080in}{0.310883in}}{\pgfqpoint{4.650000in}{3.080000in}}%
\pgfusepath{clip}%
\pgfsetbuttcap%
\pgfsetmiterjoin%
\definecolor{currentfill}{rgb}{1.000000,0.752941,0.796078}%
\pgfsetfillcolor{currentfill}%
\pgfsetlinewidth{0.000000pt}%
\definecolor{currentstroke}{rgb}{0.000000,0.000000,0.000000}%
\pgfsetstrokecolor{currentstroke}%
\pgfsetstrokeopacity{0.000000}%
\pgfsetdash{}{0pt}%
\pgfpathmoveto{\pgfqpoint{4.716053in}{2.907719in}}%
\pgfpathlineto{\pgfqpoint{4.964716in}{2.907719in}}%
\pgfpathlineto{\pgfqpoint{4.964716in}{2.969094in}}%
\pgfpathlineto{\pgfqpoint{4.716053in}{2.969094in}}%
\pgfpathlineto{\pgfqpoint{4.716053in}{2.907719in}}%
\pgfpathclose%
\pgfusepath{fill}%
\end{pgfscope}%
\begin{pgfscope}%
\pgfpathrectangle{\pgfqpoint{0.526080in}{0.310883in}}{\pgfqpoint{4.650000in}{3.080000in}}%
\pgfusepath{clip}%
\pgfsetbuttcap%
\pgfsetmiterjoin%
\definecolor{currentfill}{rgb}{0.854902,0.439216,0.839216}%
\pgfsetfillcolor{currentfill}%
\pgfsetlinewidth{0.000000pt}%
\definecolor{currentstroke}{rgb}{0.000000,0.000000,0.000000}%
\pgfsetstrokecolor{currentstroke}%
\pgfsetstrokeopacity{0.000000}%
\pgfsetdash{}{0pt}%
\pgfpathmoveto{\pgfqpoint{4.716053in}{2.969094in}}%
\pgfpathlineto{\pgfqpoint{4.964716in}{2.969094in}}%
\pgfpathlineto{\pgfqpoint{4.964716in}{2.977549in}}%
\pgfpathlineto{\pgfqpoint{4.716053in}{2.977549in}}%
\pgfpathlineto{\pgfqpoint{4.716053in}{2.969094in}}%
\pgfpathclose%
\pgfusepath{fill}%
\end{pgfscope}%
\begin{pgfscope}%
\pgfsetbuttcap%
\pgfsetroundjoin%
\definecolor{currentfill}{rgb}{0.000000,0.000000,0.000000}%
\pgfsetfillcolor{currentfill}%
\pgfsetlinewidth{0.803000pt}%
\definecolor{currentstroke}{rgb}{0.000000,0.000000,0.000000}%
\pgfsetstrokecolor{currentstroke}%
\pgfsetdash{}{0pt}%
\pgfsys@defobject{currentmarker}{\pgfqpoint{0.000000in}{-0.048611in}}{\pgfqpoint{0.000000in}{0.000000in}}{%
\pgfpathmoveto{\pgfqpoint{0.000000in}{0.000000in}}%
\pgfpathlineto{\pgfqpoint{0.000000in}{-0.048611in}}%
\pgfusepath{stroke,fill}%
}%
\begin{pgfscope}%
\pgfsys@transformshift{0.861775in}{0.310883in}%
\pgfsys@useobject{currentmarker}{}%
\end{pgfscope}%
\end{pgfscope}%
\begin{pgfscope}%
\definecolor{textcolor}{rgb}{0.000000,0.000000,0.000000}%
\pgfsetstrokecolor{textcolor}%
\pgfsetfillcolor{textcolor}%
\pgftext[x=0.861775in,y=0.213661in,,top]{\color{textcolor}{\rmfamily\fontsize{6.940000}{8.328000}\selectfont\catcode`\^=\active\def^{\ifmmode\sp\else\^{}\fi}\catcode`\%=\active\def%{\%}$480^2$px}}%
\end{pgfscope}%
\begin{pgfscope}%
\pgfsetbuttcap%
\pgfsetroundjoin%
\definecolor{currentfill}{rgb}{0.000000,0.000000,0.000000}%
\pgfsetfillcolor{currentfill}%
\pgfsetlinewidth{0.803000pt}%
\definecolor{currentstroke}{rgb}{0.000000,0.000000,0.000000}%
\pgfsetstrokecolor{currentstroke}%
\pgfsetdash{}{0pt}%
\pgfsys@defobject{currentmarker}{\pgfqpoint{0.000000in}{-0.048611in}}{\pgfqpoint{0.000000in}{0.000000in}}{%
\pgfpathmoveto{\pgfqpoint{0.000000in}{0.000000in}}%
\pgfpathlineto{\pgfqpoint{0.000000in}{-0.048611in}}%
\pgfusepath{stroke,fill}%
}%
\begin{pgfscope}%
\pgfsys@transformshift{1.359101in}{0.310883in}%
\pgfsys@useobject{currentmarker}{}%
\end{pgfscope}%
\end{pgfscope}%
\begin{pgfscope}%
\definecolor{textcolor}{rgb}{0.000000,0.000000,0.000000}%
\pgfsetstrokecolor{textcolor}%
\pgfsetfillcolor{textcolor}%
\pgftext[x=1.359101in,y=0.213661in,,top]{\color{textcolor}{\rmfamily\fontsize{6.940000}{8.328000}\selectfont\catcode`\^=\active\def^{\ifmmode\sp\else\^{}\fi}\catcode`\%=\active\def%{\%}$679^2$px}}%
\end{pgfscope}%
\begin{pgfscope}%
\pgfsetbuttcap%
\pgfsetroundjoin%
\definecolor{currentfill}{rgb}{0.000000,0.000000,0.000000}%
\pgfsetfillcolor{currentfill}%
\pgfsetlinewidth{0.803000pt}%
\definecolor{currentstroke}{rgb}{0.000000,0.000000,0.000000}%
\pgfsetstrokecolor{currentstroke}%
\pgfsetdash{}{0pt}%
\pgfsys@defobject{currentmarker}{\pgfqpoint{0.000000in}{-0.048611in}}{\pgfqpoint{0.000000in}{0.000000in}}{%
\pgfpathmoveto{\pgfqpoint{0.000000in}{0.000000in}}%
\pgfpathlineto{\pgfqpoint{0.000000in}{-0.048611in}}%
\pgfusepath{stroke,fill}%
}%
\begin{pgfscope}%
\pgfsys@transformshift{1.856427in}{0.310883in}%
\pgfsys@useobject{currentmarker}{}%
\end{pgfscope}%
\end{pgfscope}%
\begin{pgfscope}%
\definecolor{textcolor}{rgb}{0.000000,0.000000,0.000000}%
\pgfsetstrokecolor{textcolor}%
\pgfsetfillcolor{textcolor}%
\pgftext[x=1.856427in,y=0.213661in,,top]{\color{textcolor}{\rmfamily\fontsize{6.940000}{8.328000}\selectfont\catcode`\^=\active\def^{\ifmmode\sp\else\^{}\fi}\catcode`\%=\active\def%{\%}$831^2$px}}%
\end{pgfscope}%
\begin{pgfscope}%
\pgfsetbuttcap%
\pgfsetroundjoin%
\definecolor{currentfill}{rgb}{0.000000,0.000000,0.000000}%
\pgfsetfillcolor{currentfill}%
\pgfsetlinewidth{0.803000pt}%
\definecolor{currentstroke}{rgb}{0.000000,0.000000,0.000000}%
\pgfsetstrokecolor{currentstroke}%
\pgfsetdash{}{0pt}%
\pgfsys@defobject{currentmarker}{\pgfqpoint{0.000000in}{-0.048611in}}{\pgfqpoint{0.000000in}{0.000000in}}{%
\pgfpathmoveto{\pgfqpoint{0.000000in}{0.000000in}}%
\pgfpathlineto{\pgfqpoint{0.000000in}{-0.048611in}}%
\pgfusepath{stroke,fill}%
}%
\begin{pgfscope}%
\pgfsys@transformshift{2.353754in}{0.310883in}%
\pgfsys@useobject{currentmarker}{}%
\end{pgfscope}%
\end{pgfscope}%
\begin{pgfscope}%
\definecolor{textcolor}{rgb}{0.000000,0.000000,0.000000}%
\pgfsetstrokecolor{textcolor}%
\pgfsetfillcolor{textcolor}%
\pgftext[x=2.353754in,y=0.213661in,,top]{\color{textcolor}{\rmfamily\fontsize{6.940000}{8.328000}\selectfont\catcode`\^=\active\def^{\ifmmode\sp\else\^{}\fi}\catcode`\%=\active\def%{\%}$960^2$px}}%
\end{pgfscope}%
\begin{pgfscope}%
\pgfsetbuttcap%
\pgfsetroundjoin%
\definecolor{currentfill}{rgb}{0.000000,0.000000,0.000000}%
\pgfsetfillcolor{currentfill}%
\pgfsetlinewidth{0.803000pt}%
\definecolor{currentstroke}{rgb}{0.000000,0.000000,0.000000}%
\pgfsetstrokecolor{currentstroke}%
\pgfsetdash{}{0pt}%
\pgfsys@defobject{currentmarker}{\pgfqpoint{0.000000in}{-0.048611in}}{\pgfqpoint{0.000000in}{0.000000in}}{%
\pgfpathmoveto{\pgfqpoint{0.000000in}{0.000000in}}%
\pgfpathlineto{\pgfqpoint{0.000000in}{-0.048611in}}%
\pgfusepath{stroke,fill}%
}%
\begin{pgfscope}%
\pgfsys@transformshift{2.851080in}{0.310883in}%
\pgfsys@useobject{currentmarker}{}%
\end{pgfscope}%
\end{pgfscope}%
\begin{pgfscope}%
\definecolor{textcolor}{rgb}{0.000000,0.000000,0.000000}%
\pgfsetstrokecolor{textcolor}%
\pgfsetfillcolor{textcolor}%
\pgftext[x=2.851080in,y=0.213661in,,top]{\color{textcolor}{\rmfamily\fontsize{6.940000}{8.328000}\selectfont\catcode`\^=\active\def^{\ifmmode\sp\else\^{}\fi}\catcode`\%=\active\def%{\%}$1073^2$px}}%
\end{pgfscope}%
\begin{pgfscope}%
\pgfsetbuttcap%
\pgfsetroundjoin%
\definecolor{currentfill}{rgb}{0.000000,0.000000,0.000000}%
\pgfsetfillcolor{currentfill}%
\pgfsetlinewidth{0.803000pt}%
\definecolor{currentstroke}{rgb}{0.000000,0.000000,0.000000}%
\pgfsetstrokecolor{currentstroke}%
\pgfsetdash{}{0pt}%
\pgfsys@defobject{currentmarker}{\pgfqpoint{0.000000in}{-0.048611in}}{\pgfqpoint{0.000000in}{0.000000in}}{%
\pgfpathmoveto{\pgfqpoint{0.000000in}{0.000000in}}%
\pgfpathlineto{\pgfqpoint{0.000000in}{-0.048611in}}%
\pgfusepath{stroke,fill}%
}%
\begin{pgfscope}%
\pgfsys@transformshift{3.348406in}{0.310883in}%
\pgfsys@useobject{currentmarker}{}%
\end{pgfscope}%
\end{pgfscope}%
\begin{pgfscope}%
\definecolor{textcolor}{rgb}{0.000000,0.000000,0.000000}%
\pgfsetstrokecolor{textcolor}%
\pgfsetfillcolor{textcolor}%
\pgftext[x=3.348406in,y=0.213661in,,top]{\color{textcolor}{\rmfamily\fontsize{6.940000}{8.328000}\selectfont\catcode`\^=\active\def^{\ifmmode\sp\else\^{}\fi}\catcode`\%=\active\def%{\%}$1176^2$px}}%
\end{pgfscope}%
\begin{pgfscope}%
\pgfsetbuttcap%
\pgfsetroundjoin%
\definecolor{currentfill}{rgb}{0.000000,0.000000,0.000000}%
\pgfsetfillcolor{currentfill}%
\pgfsetlinewidth{0.803000pt}%
\definecolor{currentstroke}{rgb}{0.000000,0.000000,0.000000}%
\pgfsetstrokecolor{currentstroke}%
\pgfsetdash{}{0pt}%
\pgfsys@defobject{currentmarker}{\pgfqpoint{0.000000in}{-0.048611in}}{\pgfqpoint{0.000000in}{0.000000in}}{%
\pgfpathmoveto{\pgfqpoint{0.000000in}{0.000000in}}%
\pgfpathlineto{\pgfqpoint{0.000000in}{-0.048611in}}%
\pgfusepath{stroke,fill}%
}%
\begin{pgfscope}%
\pgfsys@transformshift{3.845732in}{0.310883in}%
\pgfsys@useobject{currentmarker}{}%
\end{pgfscope}%
\end{pgfscope}%
\begin{pgfscope}%
\definecolor{textcolor}{rgb}{0.000000,0.000000,0.000000}%
\pgfsetstrokecolor{textcolor}%
\pgfsetfillcolor{textcolor}%
\pgftext[x=3.845732in,y=0.213661in,,top]{\color{textcolor}{\rmfamily\fontsize{6.940000}{8.328000}\selectfont\catcode`\^=\active\def^{\ifmmode\sp\else\^{}\fi}\catcode`\%=\active\def%{\%}$1270^2$px}}%
\end{pgfscope}%
\begin{pgfscope}%
\pgfsetbuttcap%
\pgfsetroundjoin%
\definecolor{currentfill}{rgb}{0.000000,0.000000,0.000000}%
\pgfsetfillcolor{currentfill}%
\pgfsetlinewidth{0.803000pt}%
\definecolor{currentstroke}{rgb}{0.000000,0.000000,0.000000}%
\pgfsetstrokecolor{currentstroke}%
\pgfsetdash{}{0pt}%
\pgfsys@defobject{currentmarker}{\pgfqpoint{0.000000in}{-0.048611in}}{\pgfqpoint{0.000000in}{0.000000in}}{%
\pgfpathmoveto{\pgfqpoint{0.000000in}{0.000000in}}%
\pgfpathlineto{\pgfqpoint{0.000000in}{-0.048611in}}%
\pgfusepath{stroke,fill}%
}%
\begin{pgfscope}%
\pgfsys@transformshift{4.343058in}{0.310883in}%
\pgfsys@useobject{currentmarker}{}%
\end{pgfscope}%
\end{pgfscope}%
\begin{pgfscope}%
\definecolor{textcolor}{rgb}{0.000000,0.000000,0.000000}%
\pgfsetstrokecolor{textcolor}%
\pgfsetfillcolor{textcolor}%
\pgftext[x=4.343058in,y=0.213661in,,top]{\color{textcolor}{\rmfamily\fontsize{6.940000}{8.328000}\selectfont\catcode`\^=\active\def^{\ifmmode\sp\else\^{}\fi}\catcode`\%=\active\def%{\%}$1358^2$px}}%
\end{pgfscope}%
\begin{pgfscope}%
\pgfsetbuttcap%
\pgfsetroundjoin%
\definecolor{currentfill}{rgb}{0.000000,0.000000,0.000000}%
\pgfsetfillcolor{currentfill}%
\pgfsetlinewidth{0.803000pt}%
\definecolor{currentstroke}{rgb}{0.000000,0.000000,0.000000}%
\pgfsetstrokecolor{currentstroke}%
\pgfsetdash{}{0pt}%
\pgfsys@defobject{currentmarker}{\pgfqpoint{0.000000in}{-0.048611in}}{\pgfqpoint{0.000000in}{0.000000in}}{%
\pgfpathmoveto{\pgfqpoint{0.000000in}{0.000000in}}%
\pgfpathlineto{\pgfqpoint{0.000000in}{-0.048611in}}%
\pgfusepath{stroke,fill}%
}%
\begin{pgfscope}%
\pgfsys@transformshift{4.840385in}{0.310883in}%
\pgfsys@useobject{currentmarker}{}%
\end{pgfscope}%
\end{pgfscope}%
\begin{pgfscope}%
\definecolor{textcolor}{rgb}{0.000000,0.000000,0.000000}%
\pgfsetstrokecolor{textcolor}%
\pgfsetfillcolor{textcolor}%
\pgftext[x=4.840385in,y=0.213661in,,top]{\color{textcolor}{\rmfamily\fontsize{6.940000}{8.328000}\selectfont\catcode`\^=\active\def^{\ifmmode\sp\else\^{}\fi}\catcode`\%=\active\def%{\%}$1440^2$px}}%
\end{pgfscope}%
\begin{pgfscope}%
\pgfsetbuttcap%
\pgfsetroundjoin%
\definecolor{currentfill}{rgb}{0.000000,0.000000,0.000000}%
\pgfsetfillcolor{currentfill}%
\pgfsetlinewidth{0.803000pt}%
\definecolor{currentstroke}{rgb}{0.000000,0.000000,0.000000}%
\pgfsetstrokecolor{currentstroke}%
\pgfsetdash{}{0pt}%
\pgfsys@defobject{currentmarker}{\pgfqpoint{-0.048611in}{0.000000in}}{\pgfqpoint{-0.000000in}{0.000000in}}{%
\pgfpathmoveto{\pgfqpoint{-0.000000in}{0.000000in}}%
\pgfpathlineto{\pgfqpoint{-0.048611in}{0.000000in}}%
\pgfusepath{stroke,fill}%
}%
\begin{pgfscope}%
\pgfsys@transformshift{0.526080in}{0.310883in}%
\pgfsys@useobject{currentmarker}{}%
\end{pgfscope}%
\end{pgfscope}%
\begin{pgfscope}%
\definecolor{textcolor}{rgb}{0.000000,0.000000,0.000000}%
\pgfsetstrokecolor{textcolor}%
\pgfsetfillcolor{textcolor}%
\pgftext[x=0.359413in, y=0.258121in, left, base]{\color{textcolor}{\rmfamily\fontsize{10.000000}{12.000000}\selectfont\catcode`\^=\active\def^{\ifmmode\sp\else\^{}\fi}\catcode`\%=\active\def%{\%}$\mathdefault{0}$}}%
\end{pgfscope}%
\begin{pgfscope}%
\pgfsetbuttcap%
\pgfsetroundjoin%
\definecolor{currentfill}{rgb}{0.000000,0.000000,0.000000}%
\pgfsetfillcolor{currentfill}%
\pgfsetlinewidth{0.803000pt}%
\definecolor{currentstroke}{rgb}{0.000000,0.000000,0.000000}%
\pgfsetstrokecolor{currentstroke}%
\pgfsetdash{}{0pt}%
\pgfsys@defobject{currentmarker}{\pgfqpoint{-0.048611in}{0.000000in}}{\pgfqpoint{-0.000000in}{0.000000in}}{%
\pgfpathmoveto{\pgfqpoint{-0.000000in}{0.000000in}}%
\pgfpathlineto{\pgfqpoint{-0.048611in}{0.000000in}}%
\pgfusepath{stroke,fill}%
}%
\begin{pgfscope}%
\pgfsys@transformshift{0.526080in}{0.894021in}%
\pgfsys@useobject{currentmarker}{}%
\end{pgfscope}%
\end{pgfscope}%
\begin{pgfscope}%
\definecolor{textcolor}{rgb}{0.000000,0.000000,0.000000}%
\pgfsetstrokecolor{textcolor}%
\pgfsetfillcolor{textcolor}%
\pgftext[x=0.359413in, y=0.841259in, left, base]{\color{textcolor}{\rmfamily\fontsize{10.000000}{12.000000}\selectfont\catcode`\^=\active\def^{\ifmmode\sp\else\^{}\fi}\catcode`\%=\active\def%{\%}$\mathdefault{5}$}}%
\end{pgfscope}%
\begin{pgfscope}%
\pgfsetbuttcap%
\pgfsetroundjoin%
\definecolor{currentfill}{rgb}{0.000000,0.000000,0.000000}%
\pgfsetfillcolor{currentfill}%
\pgfsetlinewidth{0.803000pt}%
\definecolor{currentstroke}{rgb}{0.000000,0.000000,0.000000}%
\pgfsetstrokecolor{currentstroke}%
\pgfsetdash{}{0pt}%
\pgfsys@defobject{currentmarker}{\pgfqpoint{-0.048611in}{0.000000in}}{\pgfqpoint{-0.000000in}{0.000000in}}{%
\pgfpathmoveto{\pgfqpoint{-0.000000in}{0.000000in}}%
\pgfpathlineto{\pgfqpoint{-0.048611in}{0.000000in}}%
\pgfusepath{stroke,fill}%
}%
\begin{pgfscope}%
\pgfsys@transformshift{0.526080in}{1.477158in}%
\pgfsys@useobject{currentmarker}{}%
\end{pgfscope}%
\end{pgfscope}%
\begin{pgfscope}%
\definecolor{textcolor}{rgb}{0.000000,0.000000,0.000000}%
\pgfsetstrokecolor{textcolor}%
\pgfsetfillcolor{textcolor}%
\pgftext[x=0.289968in, y=1.424397in, left, base]{\color{textcolor}{\rmfamily\fontsize{10.000000}{12.000000}\selectfont\catcode`\^=\active\def^{\ifmmode\sp\else\^{}\fi}\catcode`\%=\active\def%{\%}$\mathdefault{10}$}}%
\end{pgfscope}%
\begin{pgfscope}%
\pgfsetbuttcap%
\pgfsetroundjoin%
\definecolor{currentfill}{rgb}{0.000000,0.000000,0.000000}%
\pgfsetfillcolor{currentfill}%
\pgfsetlinewidth{0.803000pt}%
\definecolor{currentstroke}{rgb}{0.000000,0.000000,0.000000}%
\pgfsetstrokecolor{currentstroke}%
\pgfsetdash{}{0pt}%
\pgfsys@defobject{currentmarker}{\pgfqpoint{-0.048611in}{0.000000in}}{\pgfqpoint{-0.000000in}{0.000000in}}{%
\pgfpathmoveto{\pgfqpoint{-0.000000in}{0.000000in}}%
\pgfpathlineto{\pgfqpoint{-0.048611in}{0.000000in}}%
\pgfusepath{stroke,fill}%
}%
\begin{pgfscope}%
\pgfsys@transformshift{0.526080in}{2.060296in}%
\pgfsys@useobject{currentmarker}{}%
\end{pgfscope}%
\end{pgfscope}%
\begin{pgfscope}%
\definecolor{textcolor}{rgb}{0.000000,0.000000,0.000000}%
\pgfsetstrokecolor{textcolor}%
\pgfsetfillcolor{textcolor}%
\pgftext[x=0.289968in, y=2.007535in, left, base]{\color{textcolor}{\rmfamily\fontsize{10.000000}{12.000000}\selectfont\catcode`\^=\active\def^{\ifmmode\sp\else\^{}\fi}\catcode`\%=\active\def%{\%}$\mathdefault{15}$}}%
\end{pgfscope}%
\begin{pgfscope}%
\pgfsetbuttcap%
\pgfsetroundjoin%
\definecolor{currentfill}{rgb}{0.000000,0.000000,0.000000}%
\pgfsetfillcolor{currentfill}%
\pgfsetlinewidth{0.803000pt}%
\definecolor{currentstroke}{rgb}{0.000000,0.000000,0.000000}%
\pgfsetstrokecolor{currentstroke}%
\pgfsetdash{}{0pt}%
\pgfsys@defobject{currentmarker}{\pgfqpoint{-0.048611in}{0.000000in}}{\pgfqpoint{-0.000000in}{0.000000in}}{%
\pgfpathmoveto{\pgfqpoint{-0.000000in}{0.000000in}}%
\pgfpathlineto{\pgfqpoint{-0.048611in}{0.000000in}}%
\pgfusepath{stroke,fill}%
}%
\begin{pgfscope}%
\pgfsys@transformshift{0.526080in}{2.643434in}%
\pgfsys@useobject{currentmarker}{}%
\end{pgfscope}%
\end{pgfscope}%
\begin{pgfscope}%
\definecolor{textcolor}{rgb}{0.000000,0.000000,0.000000}%
\pgfsetstrokecolor{textcolor}%
\pgfsetfillcolor{textcolor}%
\pgftext[x=0.289968in, y=2.590672in, left, base]{\color{textcolor}{\rmfamily\fontsize{10.000000}{12.000000}\selectfont\catcode`\^=\active\def^{\ifmmode\sp\else\^{}\fi}\catcode`\%=\active\def%{\%}$\mathdefault{20}$}}%
\end{pgfscope}%
\begin{pgfscope}%
\pgfsetbuttcap%
\pgfsetroundjoin%
\definecolor{currentfill}{rgb}{0.000000,0.000000,0.000000}%
\pgfsetfillcolor{currentfill}%
\pgfsetlinewidth{0.803000pt}%
\definecolor{currentstroke}{rgb}{0.000000,0.000000,0.000000}%
\pgfsetstrokecolor{currentstroke}%
\pgfsetdash{}{0pt}%
\pgfsys@defobject{currentmarker}{\pgfqpoint{-0.048611in}{0.000000in}}{\pgfqpoint{-0.000000in}{0.000000in}}{%
\pgfpathmoveto{\pgfqpoint{-0.000000in}{0.000000in}}%
\pgfpathlineto{\pgfqpoint{-0.048611in}{0.000000in}}%
\pgfusepath{stroke,fill}%
}%
\begin{pgfscope}%
\pgfsys@transformshift{0.526080in}{3.226572in}%
\pgfsys@useobject{currentmarker}{}%
\end{pgfscope}%
\end{pgfscope}%
\begin{pgfscope}%
\definecolor{textcolor}{rgb}{0.000000,0.000000,0.000000}%
\pgfsetstrokecolor{textcolor}%
\pgfsetfillcolor{textcolor}%
\pgftext[x=0.289968in, y=3.173810in, left, base]{\color{textcolor}{\rmfamily\fontsize{10.000000}{12.000000}\selectfont\catcode`\^=\active\def^{\ifmmode\sp\else\^{}\fi}\catcode`\%=\active\def%{\%}$\mathdefault{25}$}}%
\end{pgfscope}%
\begin{pgfscope}%
\definecolor{textcolor}{rgb}{0.000000,0.000000,0.000000}%
\pgfsetstrokecolor{textcolor}%
\pgfsetfillcolor{textcolor}%
\pgftext[x=0.234413in,y=1.850883in,,bottom,rotate=90.000000]{\color{textcolor}{\rmfamily\fontsize{10.000000}{12.000000}\selectfont\catcode`\^=\active\def^{\ifmmode\sp\else\^{}\fi}\catcode`\%=\active\def%{\%}Time (ms)}}%
\end{pgfscope}%
\begin{pgfscope}%
\pgfpathrectangle{\pgfqpoint{0.526080in}{0.310883in}}{\pgfqpoint{4.650000in}{3.080000in}}%
\pgfusepath{clip}%
\pgfsetbuttcap%
\pgfsetroundjoin%
\pgfsetlinewidth{1.505625pt}%
\definecolor{currentstroke}{rgb}{0.000000,0.000000,0.000000}%
\pgfsetstrokecolor{currentstroke}%
\pgfsetdash{{5.550000pt}{2.400000pt}}{0.000000pt}%
\pgfpathmoveto{\pgfqpoint{0.861775in}{0.840915in}}%
\pgfpathlineto{\pgfqpoint{1.359101in}{1.351794in}}%
\pgfpathlineto{\pgfqpoint{1.856427in}{1.851036in}}%
\pgfpathlineto{\pgfqpoint{2.353754in}{2.355584in}}%
\pgfpathlineto{\pgfqpoint{2.851080in}{2.873858in}}%
\pgfpathlineto{\pgfqpoint{3.348406in}{3.382787in}}%
\pgfpathlineto{\pgfqpoint{3.365752in}{3.400883in}}%
\pgfusepath{stroke}%
\end{pgfscope}%
\begin{pgfscope}%
\pgfpathrectangle{\pgfqpoint{0.526080in}{0.310883in}}{\pgfqpoint{4.650000in}{3.080000in}}%
\pgfusepath{clip}%
\pgfsetbuttcap%
\pgfsetroundjoin%
\definecolor{currentfill}{rgb}{0.000000,0.000000,0.000000}%
\pgfsetfillcolor{currentfill}%
\pgfsetlinewidth{1.003750pt}%
\definecolor{currentstroke}{rgb}{0.000000,0.000000,0.000000}%
\pgfsetstrokecolor{currentstroke}%
\pgfsetdash{}{0pt}%
\pgfsys@defobject{currentmarker}{\pgfqpoint{-0.020833in}{-0.020833in}}{\pgfqpoint{0.020833in}{0.020833in}}{%
\pgfpathmoveto{\pgfqpoint{0.000000in}{-0.020833in}}%
\pgfpathcurveto{\pgfqpoint{0.005525in}{-0.020833in}}{\pgfqpoint{0.010825in}{-0.018638in}}{\pgfqpoint{0.014731in}{-0.014731in}}%
\pgfpathcurveto{\pgfqpoint{0.018638in}{-0.010825in}}{\pgfqpoint{0.020833in}{-0.005525in}}{\pgfqpoint{0.020833in}{0.000000in}}%
\pgfpathcurveto{\pgfqpoint{0.020833in}{0.005525in}}{\pgfqpoint{0.018638in}{0.010825in}}{\pgfqpoint{0.014731in}{0.014731in}}%
\pgfpathcurveto{\pgfqpoint{0.010825in}{0.018638in}}{\pgfqpoint{0.005525in}{0.020833in}}{\pgfqpoint{0.000000in}{0.020833in}}%
\pgfpathcurveto{\pgfqpoint{-0.005525in}{0.020833in}}{\pgfqpoint{-0.010825in}{0.018638in}}{\pgfqpoint{-0.014731in}{0.014731in}}%
\pgfpathcurveto{\pgfqpoint{-0.018638in}{0.010825in}}{\pgfqpoint{-0.020833in}{0.005525in}}{\pgfqpoint{-0.020833in}{0.000000in}}%
\pgfpathcurveto{\pgfqpoint{-0.020833in}{-0.005525in}}{\pgfqpoint{-0.018638in}{-0.010825in}}{\pgfqpoint{-0.014731in}{-0.014731in}}%
\pgfpathcurveto{\pgfqpoint{-0.010825in}{-0.018638in}}{\pgfqpoint{-0.005525in}{-0.020833in}}{\pgfqpoint{0.000000in}{-0.020833in}}%
\pgfpathlineto{\pgfqpoint{0.000000in}{-0.020833in}}%
\pgfpathclose%
\pgfusepath{stroke,fill}%
}%
\begin{pgfscope}%
\pgfsys@transformshift{0.861775in}{0.840915in}%
\pgfsys@useobject{currentmarker}{}%
\end{pgfscope}%
\begin{pgfscope}%
\pgfsys@transformshift{1.359101in}{1.351794in}%
\pgfsys@useobject{currentmarker}{}%
\end{pgfscope}%
\begin{pgfscope}%
\pgfsys@transformshift{1.856427in}{1.851036in}%
\pgfsys@useobject{currentmarker}{}%
\end{pgfscope}%
\begin{pgfscope}%
\pgfsys@transformshift{2.353754in}{2.355584in}%
\pgfsys@useobject{currentmarker}{}%
\end{pgfscope}%
\begin{pgfscope}%
\pgfsys@transformshift{2.851080in}{2.873858in}%
\pgfsys@useobject{currentmarker}{}%
\end{pgfscope}%
\begin{pgfscope}%
\pgfsys@transformshift{3.348406in}{3.382787in}%
\pgfsys@useobject{currentmarker}{}%
\end{pgfscope}%
\begin{pgfscope}%
\pgfsys@transformshift{3.845732in}{3.901615in}%
\pgfsys@useobject{currentmarker}{}%
\end{pgfscope}%
\begin{pgfscope}%
\pgfsys@transformshift{4.343058in}{4.393195in}%
\pgfsys@useobject{currentmarker}{}%
\end{pgfscope}%
\begin{pgfscope}%
\pgfsys@transformshift{4.840385in}{4.892099in}%
\pgfsys@useobject{currentmarker}{}%
\end{pgfscope}%
\end{pgfscope}%
\begin{pgfscope}%
\pgfpathrectangle{\pgfqpoint{0.526080in}{0.310883in}}{\pgfqpoint{4.650000in}{3.080000in}}%
\pgfusepath{clip}%
\pgfsetbuttcap%
\pgfsetroundjoin%
\pgfsetlinewidth{1.505625pt}%
\definecolor{currentstroke}{rgb}{0.501961,0.000000,0.501961}%
\pgfsetstrokecolor{currentstroke}%
\pgfsetdash{{5.550000pt}{2.400000pt}}{0.000000pt}%
\pgfpathmoveto{\pgfqpoint{0.861775in}{1.068913in}}%
\pgfpathlineto{\pgfqpoint{1.359101in}{1.249065in}}%
\pgfpathlineto{\pgfqpoint{1.856427in}{1.391582in}}%
\pgfpathlineto{\pgfqpoint{2.353754in}{1.579337in}}%
\pgfpathlineto{\pgfqpoint{2.851080in}{1.720762in}}%
\pgfpathlineto{\pgfqpoint{3.348406in}{1.920763in}}%
\pgfpathlineto{\pgfqpoint{3.845732in}{2.083341in}}%
\pgfpathlineto{\pgfqpoint{4.343058in}{2.244634in}}%
\pgfpathlineto{\pgfqpoint{4.840385in}{2.444573in}}%
\pgfusepath{stroke}%
\end{pgfscope}%
\begin{pgfscope}%
\pgfpathrectangle{\pgfqpoint{0.526080in}{0.310883in}}{\pgfqpoint{4.650000in}{3.080000in}}%
\pgfusepath{clip}%
\pgfsetbuttcap%
\pgfsetroundjoin%
\definecolor{currentfill}{rgb}{0.501961,0.000000,0.501961}%
\pgfsetfillcolor{currentfill}%
\pgfsetlinewidth{1.003750pt}%
\definecolor{currentstroke}{rgb}{0.501961,0.000000,0.501961}%
\pgfsetstrokecolor{currentstroke}%
\pgfsetdash{}{0pt}%
\pgfsys@defobject{currentmarker}{\pgfqpoint{-0.020833in}{-0.020833in}}{\pgfqpoint{0.020833in}{0.020833in}}{%
\pgfpathmoveto{\pgfqpoint{0.000000in}{-0.020833in}}%
\pgfpathcurveto{\pgfqpoint{0.005525in}{-0.020833in}}{\pgfqpoint{0.010825in}{-0.018638in}}{\pgfqpoint{0.014731in}{-0.014731in}}%
\pgfpathcurveto{\pgfqpoint{0.018638in}{-0.010825in}}{\pgfqpoint{0.020833in}{-0.005525in}}{\pgfqpoint{0.020833in}{0.000000in}}%
\pgfpathcurveto{\pgfqpoint{0.020833in}{0.005525in}}{\pgfqpoint{0.018638in}{0.010825in}}{\pgfqpoint{0.014731in}{0.014731in}}%
\pgfpathcurveto{\pgfqpoint{0.010825in}{0.018638in}}{\pgfqpoint{0.005525in}{0.020833in}}{\pgfqpoint{0.000000in}{0.020833in}}%
\pgfpathcurveto{\pgfqpoint{-0.005525in}{0.020833in}}{\pgfqpoint{-0.010825in}{0.018638in}}{\pgfqpoint{-0.014731in}{0.014731in}}%
\pgfpathcurveto{\pgfqpoint{-0.018638in}{0.010825in}}{\pgfqpoint{-0.020833in}{0.005525in}}{\pgfqpoint{-0.020833in}{0.000000in}}%
\pgfpathcurveto{\pgfqpoint{-0.020833in}{-0.005525in}}{\pgfqpoint{-0.018638in}{-0.010825in}}{\pgfqpoint{-0.014731in}{-0.014731in}}%
\pgfpathcurveto{\pgfqpoint{-0.010825in}{-0.018638in}}{\pgfqpoint{-0.005525in}{-0.020833in}}{\pgfqpoint{0.000000in}{-0.020833in}}%
\pgfpathlineto{\pgfqpoint{0.000000in}{-0.020833in}}%
\pgfpathclose%
\pgfusepath{stroke,fill}%
}%
\begin{pgfscope}%
\pgfsys@transformshift{0.861775in}{1.068913in}%
\pgfsys@useobject{currentmarker}{}%
\end{pgfscope}%
\begin{pgfscope}%
\pgfsys@transformshift{1.359101in}{1.249065in}%
\pgfsys@useobject{currentmarker}{}%
\end{pgfscope}%
\begin{pgfscope}%
\pgfsys@transformshift{1.856427in}{1.391582in}%
\pgfsys@useobject{currentmarker}{}%
\end{pgfscope}%
\begin{pgfscope}%
\pgfsys@transformshift{2.353754in}{1.579337in}%
\pgfsys@useobject{currentmarker}{}%
\end{pgfscope}%
\begin{pgfscope}%
\pgfsys@transformshift{2.851080in}{1.720762in}%
\pgfsys@useobject{currentmarker}{}%
\end{pgfscope}%
\begin{pgfscope}%
\pgfsys@transformshift{3.348406in}{1.920763in}%
\pgfsys@useobject{currentmarker}{}%
\end{pgfscope}%
\begin{pgfscope}%
\pgfsys@transformshift{3.845732in}{2.083341in}%
\pgfsys@useobject{currentmarker}{}%
\end{pgfscope}%
\begin{pgfscope}%
\pgfsys@transformshift{4.343058in}{2.244634in}%
\pgfsys@useobject{currentmarker}{}%
\end{pgfscope}%
\begin{pgfscope}%
\pgfsys@transformshift{4.840385in}{2.444573in}%
\pgfsys@useobject{currentmarker}{}%
\end{pgfscope}%
\end{pgfscope}%
\begin{pgfscope}%
\pgfsetrectcap%
\pgfsetmiterjoin%
\pgfsetlinewidth{0.803000pt}%
\definecolor{currentstroke}{rgb}{0.000000,0.000000,0.000000}%
\pgfsetstrokecolor{currentstroke}%
\pgfsetdash{}{0pt}%
\pgfpathmoveto{\pgfqpoint{0.526080in}{0.310883in}}%
\pgfpathlineto{\pgfqpoint{0.526080in}{3.390883in}}%
\pgfusepath{stroke}%
\end{pgfscope}%
\begin{pgfscope}%
\pgfsetrectcap%
\pgfsetmiterjoin%
\pgfsetlinewidth{0.803000pt}%
\definecolor{currentstroke}{rgb}{0.000000,0.000000,0.000000}%
\pgfsetstrokecolor{currentstroke}%
\pgfsetdash{}{0pt}%
\pgfpathmoveto{\pgfqpoint{5.176080in}{0.310883in}}%
\pgfpathlineto{\pgfqpoint{5.176080in}{3.390883in}}%
\pgfusepath{stroke}%
\end{pgfscope}%
\begin{pgfscope}%
\pgfsetrectcap%
\pgfsetmiterjoin%
\pgfsetlinewidth{0.803000pt}%
\definecolor{currentstroke}{rgb}{0.000000,0.000000,0.000000}%
\pgfsetstrokecolor{currentstroke}%
\pgfsetdash{}{0pt}%
\pgfpathmoveto{\pgfqpoint{0.526080in}{0.310883in}}%
\pgfpathlineto{\pgfqpoint{5.176080in}{0.310883in}}%
\pgfusepath{stroke}%
\end{pgfscope}%
\begin{pgfscope}%
\pgfsetrectcap%
\pgfsetmiterjoin%
\pgfsetlinewidth{0.803000pt}%
\definecolor{currentstroke}{rgb}{0.000000,0.000000,0.000000}%
\pgfsetstrokecolor{currentstroke}%
\pgfsetdash{}{0pt}%
\pgfpathmoveto{\pgfqpoint{0.526080in}{3.390883in}}%
\pgfpathlineto{\pgfqpoint{5.176080in}{3.390883in}}%
\pgfusepath{stroke}%
\end{pgfscope}%
\begin{pgfscope}%
\definecolor{textcolor}{rgb}{0.000000,0.000000,0.000000}%
\pgfsetstrokecolor{textcolor}%
\pgfsetfillcolor{textcolor}%
\pgftext[x=0.861775in,y=0.346038in,,]{\color{textcolor}{\rmfamily\fontsize{5.790000}{6.948000}\selectfont\catcode`\^=\active\def^{\ifmmode\sp\else\^{}\fi}\catcode`\%=\active\def%{\%}0.13 ms}}%
\end{pgfscope}%
\begin{pgfscope}%
\definecolor{textcolor}{rgb}{0.000000,0.000000,0.000000}%
\pgfsetstrokecolor{textcolor}%
\pgfsetfillcolor{textcolor}%
\pgftext[x=0.861775in,y=0.368706in,,]{\color{textcolor}{\rmfamily\fontsize{5.790000}{6.948000}\selectfont\catcode`\^=\active\def^{\ifmmode\sp\else\^{}\fi}\catcode`\%=\active\def%{\%}0.74 ms}}%
\end{pgfscope}%
\begin{pgfscope}%
\definecolor{textcolor}{rgb}{0.000000,0.000000,0.000000}%
\pgfsetstrokecolor{textcolor}%
\pgfsetfillcolor{textcolor}%
\pgftext[x=0.861775in,y=0.617380in,,]{\color{textcolor}{\rmfamily\fontsize{5.790000}{6.948000}\selectfont\catcode`\^=\active\def^{\ifmmode\sp\else\^{}\fi}\catcode`\%=\active\def%{\%}3.53 ms}}%
\end{pgfscope}%
\begin{pgfscope}%
\definecolor{textcolor}{rgb}{0.000000,0.000000,0.000000}%
\pgfsetstrokecolor{textcolor}%
\pgfsetfillcolor{textcolor}%
\pgftext[x=0.861775in,y=0.857796in,,]{\color{textcolor}{\rmfamily\fontsize{5.790000}{6.948000}\selectfont\catcode`\^=\active\def^{\ifmmode\sp\else\^{}\fi}\catcode`\%=\active\def%{\%}0.12 ms}}%
\end{pgfscope}%
\begin{pgfscope}%
\definecolor{textcolor}{rgb}{0.000000,0.000000,0.000000}%
\pgfsetstrokecolor{textcolor}%
\pgfsetfillcolor{textcolor}%
\pgftext[x=0.861775in,y=1.112817in,,]{\color{textcolor}{\rmfamily\fontsize{5.790000}{6.948000}\selectfont\catcode`\^=\active\def^{\ifmmode\sp\else\^{}\fi}\catcode`\%=\active\def%{\%}4.73 ms}}%
\end{pgfscope}%
\begin{pgfscope}%
\definecolor{textcolor}{rgb}{0.000000,0.000000,0.000000}%
\pgfsetstrokecolor{textcolor}%
\pgfsetfillcolor{textcolor}%
\pgftext[x=0.861775in,y=1.426637in,,]{\color{textcolor}{\rmfamily\fontsize{5.790000}{6.948000}\selectfont\catcode`\^=\active\def^{\ifmmode\sp\else\^{}\fi}\catcode`\%=\active\def%{\%}0.65 ms}}%
\end{pgfscope}%
\begin{pgfscope}%
\definecolor{textcolor}{rgb}{0.000000,0.000000,0.000000}%
\pgfsetstrokecolor{textcolor}%
\pgfsetfillcolor{textcolor}%
\pgftext[x=0.861775in,y=1.525436in,,]{\color{textcolor}{\rmfamily\fontsize{5.790000}{6.948000}\selectfont\catcode`\^=\active\def^{\ifmmode\sp\else\^{}\fi}\catcode`\%=\active\def%{\%}1.04 ms}}%
\end{pgfscope}%
\begin{pgfscope}%
\definecolor{textcolor}{rgb}{0.000000,0.000000,0.000000}%
\pgfsetstrokecolor{textcolor}%
\pgfsetfillcolor{textcolor}%
\pgftext[x=0.861775in,y=1.617621in,,]{\color{textcolor}{\rmfamily\fontsize{5.790000}{6.948000}\selectfont\catcode`\^=\active\def^{\ifmmode\sp\else\^{}\fi}\catcode`\%=\active\def%{\%}0.06 ms}}%
\end{pgfscope}%
\begin{pgfscope}%
\definecolor{textcolor}{rgb}{0.000000,0.000000,0.000000}%
\pgfsetstrokecolor{textcolor}%
\pgfsetfillcolor{textcolor}%
\pgftext[x=0.861775in,y=1.625671in,,]{\color{textcolor}{\rmfamily\fontsize{5.790000}{6.948000}\selectfont\catcode`\^=\active\def^{\ifmmode\sp\else\^{}\fi}\catcode`\%=\active\def%{\%}0.08 ms}}%
\end{pgfscope}%
\begin{pgfscope}%
\definecolor{textcolor}{rgb}{0.000000,0.000000,0.000000}%
\pgfsetstrokecolor{textcolor}%
\pgfsetfillcolor{textcolor}%
\pgftext[x=0.861775in,y=1.717976in,,bottom]{\color{textcolor}{\rmfamily\fontsize{8.330000}{9.996000}\bfseries\selectfont\catcode`\^=\active\def^{\ifmmode\sp\else\^{}\fi}\catcode`\%=\active\def%{\%}11.07 ms}}%
\end{pgfscope}%
\begin{pgfscope}%
\definecolor{textcolor}{rgb}{0.000000,0.000000,0.000000}%
\pgfsetstrokecolor{textcolor}%
\pgfsetfillcolor{textcolor}%
\pgftext[x=1.359101in,y=0.346021in,,]{\color{textcolor}{\rmfamily\fontsize{5.790000}{6.948000}\selectfont\catcode`\^=\active\def^{\ifmmode\sp\else\^{}\fi}\catcode`\%=\active\def%{\%}0.13 ms}}%
\end{pgfscope}%
\begin{pgfscope}%
\definecolor{textcolor}{rgb}{0.000000,0.000000,0.000000}%
\pgfsetstrokecolor{textcolor}%
\pgfsetfillcolor{textcolor}%
\pgftext[x=1.359101in,y=0.370591in,,]{\color{textcolor}{\rmfamily\fontsize{5.790000}{6.948000}\selectfont\catcode`\^=\active\def^{\ifmmode\sp\else\^{}\fi}\catcode`\%=\active\def%{\%}0.77 ms}}%
\end{pgfscope}%
\begin{pgfscope}%
\definecolor{textcolor}{rgb}{0.000000,0.000000,0.000000}%
\pgfsetstrokecolor{textcolor}%
\pgfsetfillcolor{textcolor}%
\pgftext[x=1.359101in,y=0.629847in,,]{\color{textcolor}{\rmfamily\fontsize{5.790000}{6.948000}\selectfont\catcode`\^=\active\def^{\ifmmode\sp\else\^{}\fi}\catcode`\%=\active\def%{\%}3.67 ms}}%
\end{pgfscope}%
\begin{pgfscope}%
\definecolor{textcolor}{rgb}{0.000000,0.000000,0.000000}%
\pgfsetstrokecolor{textcolor}%
\pgfsetfillcolor{textcolor}%
\pgftext[x=1.359101in,y=0.879273in,,]{\color{textcolor}{\rmfamily\fontsize{5.790000}{6.948000}\selectfont\catcode`\^=\active\def^{\ifmmode\sp\else\^{}\fi}\catcode`\%=\active\def%{\%}0.13 ms}}%
\end{pgfscope}%
\begin{pgfscope}%
\definecolor{textcolor}{rgb}{0.000000,0.000000,0.000000}%
\pgfsetstrokecolor{textcolor}%
\pgfsetfillcolor{textcolor}%
\pgftext[x=1.359101in,y=1.096435in,,]{\color{textcolor}{\rmfamily\fontsize{5.790000}{6.948000}\selectfont\catcode`\^=\active\def^{\ifmmode\sp\else\^{}\fi}\catcode`\%=\active\def%{\%}4.07 ms}}%
\end{pgfscope}%
\begin{pgfscope}%
\definecolor{textcolor}{rgb}{0.000000,0.000000,0.000000}%
\pgfsetstrokecolor{textcolor}%
\pgfsetfillcolor{textcolor}%
\pgftext[x=1.359101in,y=1.402838in,,]{\color{textcolor}{\rmfamily\fontsize{5.790000}{6.948000}\selectfont\catcode`\^=\active\def^{\ifmmode\sp\else\^{}\fi}\catcode`\%=\active\def%{\%}1.18 ms}}%
\end{pgfscope}%
\begin{pgfscope}%
\definecolor{textcolor}{rgb}{0.000000,0.000000,0.000000}%
\pgfsetstrokecolor{textcolor}%
\pgfsetfillcolor{textcolor}%
\pgftext[x=1.359101in,y=1.587912in,,]{\color{textcolor}{\rmfamily\fontsize{5.790000}{6.948000}\selectfont\catcode`\^=\active\def^{\ifmmode\sp\else\^{}\fi}\catcode`\%=\active\def%{\%}1.99 ms}}%
\end{pgfscope}%
\begin{pgfscope}%
\definecolor{textcolor}{rgb}{0.000000,0.000000,0.000000}%
\pgfsetstrokecolor{textcolor}%
\pgfsetfillcolor{textcolor}%
\pgftext[x=1.359101in,y=1.739033in,,]{\color{textcolor}{\rmfamily\fontsize{5.790000}{6.948000}\selectfont\catcode`\^=\active\def^{\ifmmode\sp\else\^{}\fi}\catcode`\%=\active\def%{\%}0.12 ms}}%
\end{pgfscope}%
\begin{pgfscope}%
\definecolor{textcolor}{rgb}{0.000000,0.000000,0.000000}%
\pgfsetstrokecolor{textcolor}%
\pgfsetfillcolor{textcolor}%
\pgftext[x=1.359101in,y=1.750277in,,]{\color{textcolor}{\rmfamily\fontsize{5.790000}{6.948000}\selectfont\catcode`\^=\active\def^{\ifmmode\sp\else\^{}\fi}\catcode`\%=\active\def%{\%}0.07 ms}}%
\end{pgfscope}%
\begin{pgfscope}%
\definecolor{textcolor}{rgb}{0.000000,0.000000,0.000000}%
\pgfsetstrokecolor{textcolor}%
\pgfsetfillcolor{textcolor}%
\pgftext[x=1.359101in,y=1.842327in,,bottom]{\color{textcolor}{\rmfamily\fontsize{8.330000}{9.996000}\bfseries\selectfont\catcode`\^=\active\def^{\ifmmode\sp\else\^{}\fi}\catcode`\%=\active\def%{\%}12.14 ms}}%
\end{pgfscope}%
\begin{pgfscope}%
\definecolor{textcolor}{rgb}{0.000000,0.000000,0.000000}%
\pgfsetstrokecolor{textcolor}%
\pgfsetfillcolor{textcolor}%
\pgftext[x=1.856427in,y=0.346012in,,]{\color{textcolor}{\rmfamily\fontsize{5.790000}{6.948000}\selectfont\catcode`\^=\active\def^{\ifmmode\sp\else\^{}\fi}\catcode`\%=\active\def%{\%}0.13 ms}}%
\end{pgfscope}%
\begin{pgfscope}%
\definecolor{textcolor}{rgb}{0.000000,0.000000,0.000000}%
\pgfsetstrokecolor{textcolor}%
\pgfsetfillcolor{textcolor}%
\pgftext[x=1.856427in,y=0.370635in,,]{\color{textcolor}{\rmfamily\fontsize{5.790000}{6.948000}\selectfont\catcode`\^=\active\def^{\ifmmode\sp\else\^{}\fi}\catcode`\%=\active\def%{\%}0.77 ms}}%
\end{pgfscope}%
\begin{pgfscope}%
\definecolor{textcolor}{rgb}{0.000000,0.000000,0.000000}%
\pgfsetstrokecolor{textcolor}%
\pgfsetfillcolor{textcolor}%
\pgftext[x=1.856427in,y=0.630327in,,]{\color{textcolor}{\rmfamily\fontsize{5.790000}{6.948000}\selectfont\catcode`\^=\active\def^{\ifmmode\sp\else\^{}\fi}\catcode`\%=\active\def%{\%}3.68 ms}}%
\end{pgfscope}%
\begin{pgfscope}%
\definecolor{textcolor}{rgb}{0.000000,0.000000,0.000000}%
\pgfsetstrokecolor{textcolor}%
\pgfsetfillcolor{textcolor}%
\pgftext[x=1.856427in,y=0.880134in,,]{\color{textcolor}{\rmfamily\fontsize{5.790000}{6.948000}\selectfont\catcode`\^=\active\def^{\ifmmode\sp\else\^{}\fi}\catcode`\%=\active\def%{\%}0.13 ms}}%
\end{pgfscope}%
\begin{pgfscope}%
\definecolor{textcolor}{rgb}{0.000000,0.000000,0.000000}%
\pgfsetstrokecolor{textcolor}%
\pgfsetfillcolor{textcolor}%
\pgftext[x=1.856427in,y=1.097193in,,]{\color{textcolor}{\rmfamily\fontsize{5.790000}{6.948000}\selectfont\catcode`\^=\active\def^{\ifmmode\sp\else\^{}\fi}\catcode`\%=\active\def%{\%}4.07 ms}}%
\end{pgfscope}%
\begin{pgfscope}%
\definecolor{textcolor}{rgb}{0.000000,0.000000,0.000000}%
\pgfsetstrokecolor{textcolor}%
\pgfsetfillcolor{textcolor}%
\pgftext[x=1.856427in,y=1.433381in,,]{\color{textcolor}{\rmfamily\fontsize{5.790000}{6.948000}\selectfont\catcode`\^=\active\def^{\ifmmode\sp\else\^{}\fi}\catcode`\%=\active\def%{\%}1.69 ms}}%
\end{pgfscope}%
\begin{pgfscope}%
\definecolor{textcolor}{rgb}{0.000000,0.000000,0.000000}%
\pgfsetstrokecolor{textcolor}%
\pgfsetfillcolor{textcolor}%
\pgftext[x=1.856427in,y=1.703729in,,]{\color{textcolor}{\rmfamily\fontsize{5.790000}{6.948000}\selectfont\catcode`\^=\active\def^{\ifmmode\sp\else\^{}\fi}\catcode`\%=\active\def%{\%}2.94 ms}}%
\end{pgfscope}%
\begin{pgfscope}%
\definecolor{textcolor}{rgb}{0.000000,0.000000,0.000000}%
\pgfsetstrokecolor{textcolor}%
\pgfsetfillcolor{textcolor}%
\pgftext[x=1.856427in,y=1.885838in,,]{\color{textcolor}{\rmfamily\fontsize{5.790000}{6.948000}\selectfont\catcode`\^=\active\def^{\ifmmode\sp\else\^{}\fi}\catcode`\%=\active\def%{\%}0.18 ms}}%
\end{pgfscope}%
\begin{pgfscope}%
\definecolor{textcolor}{rgb}{0.000000,0.000000,0.000000}%
\pgfsetstrokecolor{textcolor}%
\pgfsetfillcolor{textcolor}%
\pgftext[x=1.856427in,y=1.928232in,,]{\color{textcolor}{\rmfamily\fontsize{5.790000}{6.948000}\selectfont\catcode`\^=\active\def^{\ifmmode\sp\else\^{}\fi}\catcode`\%=\active\def%{\%}0.07 ms}}%
\end{pgfscope}%
\begin{pgfscope}%
\definecolor{textcolor}{rgb}{0.000000,0.000000,0.000000}%
\pgfsetstrokecolor{textcolor}%
\pgfsetfillcolor{textcolor}%
\pgftext[x=1.856427in,y=2.020264in,,bottom]{\color{textcolor}{\rmfamily\fontsize{8.330000}{9.996000}\bfseries\selectfont\catcode`\^=\active\def^{\ifmmode\sp\else\^{}\fi}\catcode`\%=\active\def%{\%}13.66 ms}}%
\end{pgfscope}%
\begin{pgfscope}%
\definecolor{textcolor}{rgb}{0.000000,0.000000,0.000000}%
\pgfsetstrokecolor{textcolor}%
\pgfsetfillcolor{textcolor}%
\pgftext[x=2.353754in,y=0.346082in,,]{\color{textcolor}{\rmfamily\fontsize{5.790000}{6.948000}\selectfont\catcode`\^=\active\def^{\ifmmode\sp\else\^{}\fi}\catcode`\%=\active\def%{\%}0.13 ms}}%
\end{pgfscope}%
\begin{pgfscope}%
\definecolor{textcolor}{rgb}{0.000000,0.000000,0.000000}%
\pgfsetstrokecolor{textcolor}%
\pgfsetfillcolor{textcolor}%
\pgftext[x=2.353754in,y=0.370939in,,]{\color{textcolor}{\rmfamily\fontsize{5.790000}{6.948000}\selectfont\catcode`\^=\active\def^{\ifmmode\sp\else\^{}\fi}\catcode`\%=\active\def%{\%}0.78 ms}}%
\end{pgfscope}%
\begin{pgfscope}%
\definecolor{textcolor}{rgb}{0.000000,0.000000,0.000000}%
\pgfsetstrokecolor{textcolor}%
\pgfsetfillcolor{textcolor}%
\pgftext[x=2.353754in,y=0.631510in,,]{\color{textcolor}{\rmfamily\fontsize{5.790000}{6.948000}\selectfont\catcode`\^=\active\def^{\ifmmode\sp\else\^{}\fi}\catcode`\%=\active\def%{\%}3.69 ms}}%
\end{pgfscope}%
\begin{pgfscope}%
\definecolor{textcolor}{rgb}{0.000000,0.000000,0.000000}%
\pgfsetstrokecolor{textcolor}%
\pgfsetfillcolor{textcolor}%
\pgftext[x=2.353754in,y=0.882051in,,]{\color{textcolor}{\rmfamily\fontsize{5.790000}{6.948000}\selectfont\catcode`\^=\active\def^{\ifmmode\sp\else\^{}\fi}\catcode`\%=\active\def%{\%}0.13 ms}}%
\end{pgfscope}%
\begin{pgfscope}%
\definecolor{textcolor}{rgb}{0.000000,0.000000,0.000000}%
\pgfsetstrokecolor{textcolor}%
\pgfsetfillcolor{textcolor}%
\pgftext[x=2.353754in,y=1.099008in,,]{\color{textcolor}{\rmfamily\fontsize{5.790000}{6.948000}\selectfont\catcode`\^=\active\def^{\ifmmode\sp\else\^{}\fi}\catcode`\%=\active\def%{\%}4.07 ms}}%
\end{pgfscope}%
\begin{pgfscope}%
\definecolor{textcolor}{rgb}{0.000000,0.000000,0.000000}%
\pgfsetstrokecolor{textcolor}%
\pgfsetfillcolor{textcolor}%
\pgftext[x=2.353754in,y=1.463555in,,]{\color{textcolor}{\rmfamily\fontsize{5.790000}{6.948000}\selectfont\catcode`\^=\active\def^{\ifmmode\sp\else\^{}\fi}\catcode`\%=\active\def%{\%}2.18 ms}}%
\end{pgfscope}%
\begin{pgfscope}%
\definecolor{textcolor}{rgb}{0.000000,0.000000,0.000000}%
\pgfsetstrokecolor{textcolor}%
\pgfsetfillcolor{textcolor}%
\pgftext[x=2.353754in,y=1.816680in,,]{\color{textcolor}{\rmfamily\fontsize{5.790000}{6.948000}\selectfont\catcode`\^=\active\def^{\ifmmode\sp\else\^{}\fi}\catcode`\%=\active\def%{\%}3.87 ms}}%
\end{pgfscope}%
\begin{pgfscope}%
\definecolor{textcolor}{rgb}{0.000000,0.000000,0.000000}%
\pgfsetstrokecolor{textcolor}%
\pgfsetfillcolor{textcolor}%
\pgftext[x=2.353754in,y=2.056365in,,]{\color{textcolor}{\rmfamily\fontsize{5.790000}{6.948000}\selectfont\catcode`\^=\active\def^{\ifmmode\sp\else\^{}\fi}\catcode`\%=\active\def%{\%}0.24 ms}}%
\end{pgfscope}%
\begin{pgfscope}%
\definecolor{textcolor}{rgb}{0.000000,0.000000,0.000000}%
\pgfsetstrokecolor{textcolor}%
\pgfsetfillcolor{textcolor}%
\pgftext[x=2.353754in,y=2.102048in,,]{\color{textcolor}{\rmfamily\fontsize{5.790000}{6.948000}\selectfont\catcode`\^=\active\def^{\ifmmode\sp\else\^{}\fi}\catcode`\%=\active\def%{\%}0.07 ms}}%
\end{pgfscope}%
\begin{pgfscope}%
\definecolor{textcolor}{rgb}{0.000000,0.000000,0.000000}%
\pgfsetstrokecolor{textcolor}%
\pgfsetfillcolor{textcolor}%
\pgftext[x=2.353754in,y=2.194090in,,bottom]{\color{textcolor}{\rmfamily\fontsize{8.330000}{9.996000}\bfseries\selectfont\catcode`\^=\active\def^{\ifmmode\sp\else\^{}\fi}\catcode`\%=\active\def%{\%}15.16 ms}}%
\end{pgfscope}%
\begin{pgfscope}%
\definecolor{textcolor}{rgb}{0.000000,0.000000,0.000000}%
\pgfsetstrokecolor{textcolor}%
\pgfsetfillcolor{textcolor}%
\pgftext[x=2.851080in,y=0.346106in,,]{\color{textcolor}{\rmfamily\fontsize{5.790000}{6.948000}\selectfont\catcode`\^=\active\def^{\ifmmode\sp\else\^{}\fi}\catcode`\%=\active\def%{\%}0.13 ms}}%
\end{pgfscope}%
\begin{pgfscope}%
\definecolor{textcolor}{rgb}{0.000000,0.000000,0.000000}%
\pgfsetstrokecolor{textcolor}%
\pgfsetfillcolor{textcolor}%
\pgftext[x=2.851080in,y=0.371319in,,]{\color{textcolor}{\rmfamily\fontsize{5.790000}{6.948000}\selectfont\catcode`\^=\active\def^{\ifmmode\sp\else\^{}\fi}\catcode`\%=\active\def%{\%}0.78 ms}}%
\end{pgfscope}%
\begin{pgfscope}%
\definecolor{textcolor}{rgb}{0.000000,0.000000,0.000000}%
\pgfsetstrokecolor{textcolor}%
\pgfsetfillcolor{textcolor}%
\pgftext[x=2.851080in,y=0.633383in,,]{\color{textcolor}{\rmfamily\fontsize{5.790000}{6.948000}\selectfont\catcode`\^=\active\def^{\ifmmode\sp\else\^{}\fi}\catcode`\%=\active\def%{\%}3.71 ms}}%
\end{pgfscope}%
\begin{pgfscope}%
\definecolor{textcolor}{rgb}{0.000000,0.000000,0.000000}%
\pgfsetstrokecolor{textcolor}%
\pgfsetfillcolor{textcolor}%
\pgftext[x=2.851080in,y=0.885116in,,]{\color{textcolor}{\rmfamily\fontsize{5.790000}{6.948000}\selectfont\catcode`\^=\active\def^{\ifmmode\sp\else\^{}\fi}\catcode`\%=\active\def%{\%}0.13 ms}}%
\end{pgfscope}%
\begin{pgfscope}%
\definecolor{textcolor}{rgb}{0.000000,0.000000,0.000000}%
\pgfsetstrokecolor{textcolor}%
\pgfsetfillcolor{textcolor}%
\pgftext[x=2.851080in,y=1.102221in,,]{\color{textcolor}{\rmfamily\fontsize{5.790000}{6.948000}\selectfont\catcode`\^=\active\def^{\ifmmode\sp\else\^{}\fi}\catcode`\%=\active\def%{\%}4.07 ms}}%
\end{pgfscope}%
\begin{pgfscope}%
\definecolor{textcolor}{rgb}{0.000000,0.000000,0.000000}%
\pgfsetstrokecolor{textcolor}%
\pgfsetfillcolor{textcolor}%
\pgftext[x=2.851080in,y=1.499207in,,]{\color{textcolor}{\rmfamily\fontsize{5.790000}{6.948000}\selectfont\catcode`\^=\active\def^{\ifmmode\sp\else\^{}\fi}\catcode`\%=\active\def%{\%}2.74 ms}}%
\end{pgfscope}%
\begin{pgfscope}%
\definecolor{textcolor}{rgb}{0.000000,0.000000,0.000000}%
\pgfsetstrokecolor{textcolor}%
\pgfsetfillcolor{textcolor}%
\pgftext[x=2.851080in,y=1.940025in,,]{\color{textcolor}{\rmfamily\fontsize{5.790000}{6.948000}\selectfont\catcode`\^=\active\def^{\ifmmode\sp\else\^{}\fi}\catcode`\%=\active\def%{\%}4.82 ms}}%
\end{pgfscope}%
\begin{pgfscope}%
\definecolor{textcolor}{rgb}{0.000000,0.000000,0.000000}%
\pgfsetstrokecolor{textcolor}%
\pgfsetfillcolor{textcolor}%
\pgftext[x=2.851080in,y=2.238610in,,]{\color{textcolor}{\rmfamily\fontsize{5.790000}{6.948000}\selectfont\catcode`\^=\active\def^{\ifmmode\sp\else\^{}\fi}\catcode`\%=\active\def%{\%}0.30 ms}}%
\end{pgfscope}%
\begin{pgfscope}%
\definecolor{textcolor}{rgb}{0.000000,0.000000,0.000000}%
\pgfsetstrokecolor{textcolor}%
\pgfsetfillcolor{textcolor}%
\pgftext[x=2.851080in,y=2.287841in,,]{\color{textcolor}{\rmfamily\fontsize{5.790000}{6.948000}\selectfont\catcode`\^=\active\def^{\ifmmode\sp\else\^{}\fi}\catcode`\%=\active\def%{\%}0.07 ms}}%
\end{pgfscope}%
\begin{pgfscope}%
\definecolor{textcolor}{rgb}{0.000000,0.000000,0.000000}%
\pgfsetstrokecolor{textcolor}%
\pgfsetfillcolor{textcolor}%
\pgftext[x=2.851080in,y=2.379905in,,bottom]{\color{textcolor}{\rmfamily\fontsize{8.330000}{9.996000}\bfseries\selectfont\catcode`\^=\active\def^{\ifmmode\sp\else\^{}\fi}\catcode`\%=\active\def%{\%}16.75 ms}}%
\end{pgfscope}%
\begin{pgfscope}%
\definecolor{textcolor}{rgb}{0.000000,0.000000,0.000000}%
\pgfsetstrokecolor{textcolor}%
\pgfsetfillcolor{textcolor}%
\pgftext[x=3.348406in,y=0.346144in,,]{\color{textcolor}{\rmfamily\fontsize{5.790000}{6.948000}\selectfont\catcode`\^=\active\def^{\ifmmode\sp\else\^{}\fi}\catcode`\%=\active\def%{\%}0.13 ms}}%
\end{pgfscope}%
\begin{pgfscope}%
\definecolor{textcolor}{rgb}{0.000000,0.000000,0.000000}%
\pgfsetstrokecolor{textcolor}%
\pgfsetfillcolor{textcolor}%
\pgftext[x=3.348406in,y=0.371581in,,]{\color{textcolor}{\rmfamily\fontsize{5.790000}{6.948000}\selectfont\catcode`\^=\active\def^{\ifmmode\sp\else\^{}\fi}\catcode`\%=\active\def%{\%}0.78 ms}}%
\end{pgfscope}%
\begin{pgfscope}%
\definecolor{textcolor}{rgb}{0.000000,0.000000,0.000000}%
\pgfsetstrokecolor{textcolor}%
\pgfsetfillcolor{textcolor}%
\pgftext[x=3.348406in,y=0.634850in,,]{\color{textcolor}{\rmfamily\fontsize{5.790000}{6.948000}\selectfont\catcode`\^=\active\def^{\ifmmode\sp\else\^{}\fi}\catcode`\%=\active\def%{\%}3.73 ms}}%
\end{pgfscope}%
\begin{pgfscope}%
\definecolor{textcolor}{rgb}{0.000000,0.000000,0.000000}%
\pgfsetstrokecolor{textcolor}%
\pgfsetfillcolor{textcolor}%
\pgftext[x=3.348406in,y=0.887642in,,]{\color{textcolor}{\rmfamily\fontsize{5.790000}{6.948000}\selectfont\catcode`\^=\active\def^{\ifmmode\sp\else\^{}\fi}\catcode`\%=\active\def%{\%}0.13 ms}}%
\end{pgfscope}%
\begin{pgfscope}%
\definecolor{textcolor}{rgb}{0.000000,0.000000,0.000000}%
\pgfsetstrokecolor{textcolor}%
\pgfsetfillcolor{textcolor}%
\pgftext[x=3.348406in,y=1.104991in,,]{\color{textcolor}{\rmfamily\fontsize{5.790000}{6.948000}\selectfont\catcode`\^=\active\def^{\ifmmode\sp\else\^{}\fi}\catcode`\%=\active\def%{\%}4.08 ms}}%
\end{pgfscope}%
\begin{pgfscope}%
\definecolor{textcolor}{rgb}{0.000000,0.000000,0.000000}%
\pgfsetstrokecolor{textcolor}%
\pgfsetfillcolor{textcolor}%
\pgftext[x=3.348406in,y=1.529345in,,]{\color{textcolor}{\rmfamily\fontsize{5.790000}{6.948000}\selectfont\catcode`\^=\active\def^{\ifmmode\sp\else\^{}\fi}\catcode`\%=\active\def%{\%}3.20 ms}}%
\end{pgfscope}%
\begin{pgfscope}%
\definecolor{textcolor}{rgb}{0.000000,0.000000,0.000000}%
\pgfsetstrokecolor{textcolor}%
\pgfsetfillcolor{textcolor}%
\pgftext[x=3.348406in,y=2.053214in,,]{\color{textcolor}{\rmfamily\fontsize{5.790000}{6.948000}\selectfont\catcode`\^=\active\def^{\ifmmode\sp\else\^{}\fi}\catcode`\%=\active\def%{\%}5.78 ms}}%
\end{pgfscope}%
\begin{pgfscope}%
\definecolor{textcolor}{rgb}{0.000000,0.000000,0.000000}%
\pgfsetstrokecolor{textcolor}%
\pgfsetfillcolor{textcolor}%
\pgftext[x=3.348406in,y=2.410933in,,]{\color{textcolor}{\rmfamily\fontsize{5.790000}{6.948000}\selectfont\catcode`\^=\active\def^{\ifmmode\sp\else\^{}\fi}\catcode`\%=\active\def%{\%}0.35 ms}}%
\end{pgfscope}%
\begin{pgfscope}%
\definecolor{textcolor}{rgb}{0.000000,0.000000,0.000000}%
\pgfsetstrokecolor{textcolor}%
\pgfsetfillcolor{textcolor}%
\pgftext[x=3.348406in,y=2.463422in,,]{\color{textcolor}{\rmfamily\fontsize{5.790000}{6.948000}\selectfont\catcode`\^=\active\def^{\ifmmode\sp\else\^{}\fi}\catcode`\%=\active\def%{\%}0.07 ms}}%
\end{pgfscope}%
\begin{pgfscope}%
\definecolor{textcolor}{rgb}{0.000000,0.000000,0.000000}%
\pgfsetstrokecolor{textcolor}%
\pgfsetfillcolor{textcolor}%
\pgftext[x=3.348406in,y=2.555495in,,bottom]{\color{textcolor}{\rmfamily\fontsize{8.330000}{9.996000}\bfseries\selectfont\catcode`\^=\active\def^{\ifmmode\sp\else\^{}\fi}\catcode`\%=\active\def%{\%}18.25 ms}}%
\end{pgfscope}%
\begin{pgfscope}%
\definecolor{textcolor}{rgb}{0.000000,0.000000,0.000000}%
\pgfsetstrokecolor{textcolor}%
\pgfsetfillcolor{textcolor}%
\pgftext[x=3.845732in,y=0.346184in,,]{\color{textcolor}{\rmfamily\fontsize{5.790000}{6.948000}\selectfont\catcode`\^=\active\def^{\ifmmode\sp\else\^{}\fi}\catcode`\%=\active\def%{\%}0.13 ms}}%
\end{pgfscope}%
\begin{pgfscope}%
\definecolor{textcolor}{rgb}{0.000000,0.000000,0.000000}%
\pgfsetstrokecolor{textcolor}%
\pgfsetfillcolor{textcolor}%
\pgftext[x=3.845732in,y=0.371925in,,]{\color{textcolor}{\rmfamily\fontsize{5.790000}{6.948000}\selectfont\catcode`\^=\active\def^{\ifmmode\sp\else\^{}\fi}\catcode`\%=\active\def%{\%}0.79 ms}}%
\end{pgfscope}%
\begin{pgfscope}%
\definecolor{textcolor}{rgb}{0.000000,0.000000,0.000000}%
\pgfsetstrokecolor{textcolor}%
\pgfsetfillcolor{textcolor}%
\pgftext[x=3.845732in,y=0.636849in,,]{\color{textcolor}{\rmfamily\fontsize{5.790000}{6.948000}\selectfont\catcode`\^=\active\def^{\ifmmode\sp\else\^{}\fi}\catcode`\%=\active\def%{\%}3.75 ms}}%
\end{pgfscope}%
\begin{pgfscope}%
\definecolor{textcolor}{rgb}{0.000000,0.000000,0.000000}%
\pgfsetstrokecolor{textcolor}%
\pgfsetfillcolor{textcolor}%
\pgftext[x=3.845732in,y=0.891061in,,]{\color{textcolor}{\rmfamily\fontsize{5.790000}{6.948000}\selectfont\catcode`\^=\active\def^{\ifmmode\sp\else\^{}\fi}\catcode`\%=\active\def%{\%}0.13 ms}}%
\end{pgfscope}%
\begin{pgfscope}%
\definecolor{textcolor}{rgb}{0.000000,0.000000,0.000000}%
\pgfsetstrokecolor{textcolor}%
\pgfsetfillcolor{textcolor}%
\pgftext[x=3.845732in,y=1.108550in,,]{\color{textcolor}{\rmfamily\fontsize{5.790000}{6.948000}\selectfont\catcode`\^=\active\def^{\ifmmode\sp\else\^{}\fi}\catcode`\%=\active\def%{\%}4.08 ms}}%
\end{pgfscope}%
\begin{pgfscope}%
\definecolor{textcolor}{rgb}{0.000000,0.000000,0.000000}%
\pgfsetstrokecolor{textcolor}%
\pgfsetfillcolor{textcolor}%
\pgftext[x=3.845732in,y=1.565783in,,]{\color{textcolor}{\rmfamily\fontsize{5.790000}{6.948000}\selectfont\catcode`\^=\active\def^{\ifmmode\sp\else\^{}\fi}\catcode`\%=\active\def%{\%}3.76 ms}}%
\end{pgfscope}%
\begin{pgfscope}%
\definecolor{textcolor}{rgb}{0.000000,0.000000,0.000000}%
\pgfsetstrokecolor{textcolor}%
\pgfsetfillcolor{textcolor}%
\pgftext[x=3.845732in,y=2.179143in,,]{\color{textcolor}{\rmfamily\fontsize{5.790000}{6.948000}\selectfont\catcode`\^=\active\def^{\ifmmode\sp\else\^{}\fi}\catcode`\%=\active\def%{\%}6.75 ms}}%
\end{pgfscope}%
\begin{pgfscope}%
\definecolor{textcolor}{rgb}{0.000000,0.000000,0.000000}%
\pgfsetstrokecolor{textcolor}%
\pgfsetfillcolor{textcolor}%
\pgftext[x=3.845732in,y=2.597089in,,]{\color{textcolor}{\rmfamily\fontsize{5.790000}{6.948000}\selectfont\catcode`\^=\active\def^{\ifmmode\sp\else\^{}\fi}\catcode`\%=\active\def%{\%}0.41 ms}}%
\end{pgfscope}%
\begin{pgfscope}%
\definecolor{textcolor}{rgb}{0.000000,0.000000,0.000000}%
\pgfsetstrokecolor{textcolor}%
\pgfsetfillcolor{textcolor}%
\pgftext[x=3.845732in,y=2.653107in,,]{\color{textcolor}{\rmfamily\fontsize{5.790000}{6.948000}\selectfont\catcode`\^=\active\def^{\ifmmode\sp\else\^{}\fi}\catcode`\%=\active\def%{\%}0.07 ms}}%
\end{pgfscope}%
\begin{pgfscope}%
\definecolor{textcolor}{rgb}{0.000000,0.000000,0.000000}%
\pgfsetstrokecolor{textcolor}%
\pgfsetfillcolor{textcolor}%
\pgftext[x=3.845732in,y=2.745208in,,bottom]{\color{textcolor}{\rmfamily\fontsize{8.330000}{9.996000}\bfseries\selectfont\catcode`\^=\active\def^{\ifmmode\sp\else\^{}\fi}\catcode`\%=\active\def%{\%}19.88 ms}}%
\end{pgfscope}%
\begin{pgfscope}%
\definecolor{textcolor}{rgb}{0.000000,0.000000,0.000000}%
\pgfsetstrokecolor{textcolor}%
\pgfsetfillcolor{textcolor}%
\pgftext[x=4.343058in,y=0.346245in,,]{\color{textcolor}{\rmfamily\fontsize{5.790000}{6.948000}\selectfont\catcode`\^=\active\def^{\ifmmode\sp\else\^{}\fi}\catcode`\%=\active\def%{\%}0.13 ms}}%
\end{pgfscope}%
\begin{pgfscope}%
\definecolor{textcolor}{rgb}{0.000000,0.000000,0.000000}%
\pgfsetstrokecolor{textcolor}%
\pgfsetfillcolor{textcolor}%
\pgftext[x=4.343058in,y=0.372240in,,]{\color{textcolor}{\rmfamily\fontsize{5.790000}{6.948000}\selectfont\catcode`\^=\active\def^{\ifmmode\sp\else\^{}\fi}\catcode`\%=\active\def%{\%}0.79 ms}}%
\end{pgfscope}%
\begin{pgfscope}%
\definecolor{textcolor}{rgb}{0.000000,0.000000,0.000000}%
\pgfsetstrokecolor{textcolor}%
\pgfsetfillcolor{textcolor}%
\pgftext[x=4.343058in,y=0.637889in,,]{\color{textcolor}{\rmfamily\fontsize{5.790000}{6.948000}\selectfont\catcode`\^=\active\def^{\ifmmode\sp\else\^{}\fi}\catcode`\%=\active\def%{\%}3.76 ms}}%
\end{pgfscope}%
\begin{pgfscope}%
\definecolor{textcolor}{rgb}{0.000000,0.000000,0.000000}%
\pgfsetstrokecolor{textcolor}%
\pgfsetfillcolor{textcolor}%
\pgftext[x=4.343058in,y=0.892656in,,]{\color{textcolor}{\rmfamily\fontsize{5.790000}{6.948000}\selectfont\catcode`\^=\active\def^{\ifmmode\sp\else\^{}\fi}\catcode`\%=\active\def%{\%}0.13 ms}}%
\end{pgfscope}%
\begin{pgfscope}%
\definecolor{textcolor}{rgb}{0.000000,0.000000,0.000000}%
\pgfsetstrokecolor{textcolor}%
\pgfsetfillcolor{textcolor}%
\pgftext[x=4.343058in,y=1.110120in,,]{\color{textcolor}{\rmfamily\fontsize{5.790000}{6.948000}\selectfont\catcode`\^=\active\def^{\ifmmode\sp\else\^{}\fi}\catcode`\%=\active\def%{\%}4.08 ms}}%
\end{pgfscope}%
\begin{pgfscope}%
\definecolor{textcolor}{rgb}{0.000000,0.000000,0.000000}%
\pgfsetstrokecolor{textcolor}%
\pgfsetfillcolor{textcolor}%
\pgftext[x=4.343058in,y=1.597331in,,]{\color{textcolor}{\rmfamily\fontsize{5.790000}{6.948000}\selectfont\catcode`\^=\active\def^{\ifmmode\sp\else\^{}\fi}\catcode`\%=\active\def%{\%}4.28 ms}}%
\end{pgfscope}%
\begin{pgfscope}%
\definecolor{textcolor}{rgb}{0.000000,0.000000,0.000000}%
\pgfsetstrokecolor{textcolor}%
\pgfsetfillcolor{textcolor}%
\pgftext[x=4.343058in,y=2.291699in,,]{\color{textcolor}{\rmfamily\fontsize{5.790000}{6.948000}\selectfont\catcode`\^=\active\def^{\ifmmode\sp\else\^{}\fi}\catcode`\%=\active\def%{\%}7.63 ms}}%
\end{pgfscope}%
\begin{pgfscope}%
\definecolor{textcolor}{rgb}{0.000000,0.000000,0.000000}%
\pgfsetstrokecolor{textcolor}%
\pgfsetfillcolor{textcolor}%
\pgftext[x=4.343058in,y=2.764038in,,]{\color{textcolor}{\rmfamily\fontsize{5.790000}{6.948000}\selectfont\catcode`\^=\active\def^{\ifmmode\sp\else\^{}\fi}\catcode`\%=\active\def%{\%}0.47 ms}}%
\end{pgfscope}%
\begin{pgfscope}%
\definecolor{textcolor}{rgb}{0.000000,0.000000,0.000000}%
\pgfsetstrokecolor{textcolor}%
\pgfsetfillcolor{textcolor}%
\pgftext[x=4.343058in,y=2.823480in,,]{\color{textcolor}{\rmfamily\fontsize{5.790000}{6.948000}\selectfont\catcode`\^=\active\def^{\ifmmode\sp\else\^{}\fi}\catcode`\%=\active\def%{\%}0.07 ms}}%
\end{pgfscope}%
\begin{pgfscope}%
\definecolor{textcolor}{rgb}{0.000000,0.000000,0.000000}%
\pgfsetstrokecolor{textcolor}%
\pgfsetfillcolor{textcolor}%
\pgftext[x=4.343058in,y=2.915592in,,bottom]{\color{textcolor}{\rmfamily\fontsize{8.330000}{9.996000}\bfseries\selectfont\catcode`\^=\active\def^{\ifmmode\sp\else\^{}\fi}\catcode`\%=\active\def%{\%}21.34 ms}}%
\end{pgfscope}%
\begin{pgfscope}%
\definecolor{textcolor}{rgb}{0.000000,0.000000,0.000000}%
\pgfsetstrokecolor{textcolor}%
\pgfsetfillcolor{textcolor}%
\pgftext[x=4.840385in,y=0.346308in,,]{\color{textcolor}{\rmfamily\fontsize{5.790000}{6.948000}\selectfont\catcode`\^=\active\def^{\ifmmode\sp\else\^{}\fi}\catcode`\%=\active\def%{\%}0.13 ms}}%
\end{pgfscope}%
\begin{pgfscope}%
\definecolor{textcolor}{rgb}{0.000000,0.000000,0.000000}%
\pgfsetstrokecolor{textcolor}%
\pgfsetfillcolor{textcolor}%
\pgftext[x=4.840385in,y=0.372727in,,]{\color{textcolor}{\rmfamily\fontsize{5.790000}{6.948000}\selectfont\catcode`\^=\active\def^{\ifmmode\sp\else\^{}\fi}\catcode`\%=\active\def%{\%}0.80 ms}}%
\end{pgfscope}%
\begin{pgfscope}%
\definecolor{textcolor}{rgb}{0.000000,0.000000,0.000000}%
\pgfsetstrokecolor{textcolor}%
\pgfsetfillcolor{textcolor}%
\pgftext[x=4.840385in,y=0.639329in,,]{\color{textcolor}{\rmfamily\fontsize{5.790000}{6.948000}\selectfont\catcode`\^=\active\def^{\ifmmode\sp\else\^{}\fi}\catcode`\%=\active\def%{\%}3.77 ms}}%
\end{pgfscope}%
\begin{pgfscope}%
\definecolor{textcolor}{rgb}{0.000000,0.000000,0.000000}%
\pgfsetstrokecolor{textcolor}%
\pgfsetfillcolor{textcolor}%
\pgftext[x=4.840385in,y=0.894717in,,]{\color{textcolor}{\rmfamily\fontsize{5.790000}{6.948000}\selectfont\catcode`\^=\active\def^{\ifmmode\sp\else\^{}\fi}\catcode`\%=\active\def%{\%}0.13 ms}}%
\end{pgfscope}%
\begin{pgfscope}%
\definecolor{textcolor}{rgb}{0.000000,0.000000,0.000000}%
\pgfsetstrokecolor{textcolor}%
\pgfsetfillcolor{textcolor}%
\pgftext[x=4.840385in,y=1.112196in,,]{\color{textcolor}{\rmfamily\fontsize{5.790000}{6.948000}\selectfont\catcode`\^=\active\def^{\ifmmode\sp\else\^{}\fi}\catcode`\%=\active\def%{\%}4.08 ms}}%
\end{pgfscope}%
\begin{pgfscope}%
\definecolor{textcolor}{rgb}{0.000000,0.000000,0.000000}%
\pgfsetstrokecolor{textcolor}%
\pgfsetfillcolor{textcolor}%
\pgftext[x=4.840385in,y=1.626644in,,]{\color{textcolor}{\rmfamily\fontsize{5.790000}{6.948000}\selectfont\catcode`\^=\active\def^{\ifmmode\sp\else\^{}\fi}\catcode`\%=\active\def%{\%}4.75 ms}}%
\end{pgfscope}%
\begin{pgfscope}%
\definecolor{textcolor}{rgb}{0.000000,0.000000,0.000000}%
\pgfsetstrokecolor{textcolor}%
\pgfsetfillcolor{textcolor}%
\pgftext[x=4.840385in,y=2.405555in,,]{\color{textcolor}{\rmfamily\fontsize{5.790000}{6.948000}\selectfont\catcode`\^=\active\def^{\ifmmode\sp\else\^{}\fi}\catcode`\%=\active\def%{\%}8.61 ms}}%
\end{pgfscope}%
\begin{pgfscope}%
\definecolor{textcolor}{rgb}{0.000000,0.000000,0.000000}%
\pgfsetstrokecolor{textcolor}%
\pgfsetfillcolor{textcolor}%
\pgftext[x=4.840385in,y=2.938406in,,]{\color{textcolor}{\rmfamily\fontsize{5.790000}{6.948000}\selectfont\catcode`\^=\active\def^{\ifmmode\sp\else\^{}\fi}\catcode`\%=\active\def%{\%}0.53 ms}}%
\end{pgfscope}%
\begin{pgfscope}%
\definecolor{textcolor}{rgb}{0.000000,0.000000,0.000000}%
\pgfsetstrokecolor{textcolor}%
\pgfsetfillcolor{textcolor}%
\pgftext[x=4.840385in,y=3.001099in,,]{\color{textcolor}{\rmfamily\fontsize{5.790000}{6.948000}\selectfont\catcode`\^=\active\def^{\ifmmode\sp\else\^{}\fi}\catcode`\%=\active\def%{\%}0.07 ms}}%
\end{pgfscope}%
\begin{pgfscope}%
\definecolor{textcolor}{rgb}{0.000000,0.000000,0.000000}%
\pgfsetstrokecolor{textcolor}%
\pgfsetfillcolor{textcolor}%
\pgftext[x=4.840385in,y=3.093244in,,bottom]{\color{textcolor}{\rmfamily\fontsize{8.330000}{9.996000}\bfseries\selectfont\catcode`\^=\active\def^{\ifmmode\sp\else\^{}\fi}\catcode`\%=\active\def%{\%}22.86 ms}}%
\end{pgfscope}%
\begin{pgfscope}%
\pgfsetbuttcap%
\pgfsetmiterjoin%
\definecolor{currentfill}{rgb}{1.000000,1.000000,1.000000}%
\pgfsetfillcolor{currentfill}%
\pgfsetfillopacity{0.800000}%
\pgfsetlinewidth{1.003750pt}%
\definecolor{currentstroke}{rgb}{0.800000,0.800000,0.800000}%
\pgfsetstrokecolor{currentstroke}%
\pgfsetstrokeopacity{0.800000}%
\pgfsetdash{}{0pt}%
\pgfpathmoveto{\pgfqpoint{0.582372in}{2.258559in}}%
\pgfpathlineto{\pgfqpoint{1.971859in}{2.258559in}}%
\pgfpathquadraticcurveto{\pgfqpoint{1.987942in}{2.258559in}}{\pgfqpoint{1.987942in}{2.274642in}}%
\pgfpathlineto{\pgfqpoint{1.987942in}{3.334591in}}%
\pgfpathquadraticcurveto{\pgfqpoint{1.987942in}{3.350674in}}{\pgfqpoint{1.971859in}{3.350674in}}%
\pgfpathlineto{\pgfqpoint{0.582372in}{3.350674in}}%
\pgfpathquadraticcurveto{\pgfqpoint{0.566288in}{3.350674in}}{\pgfqpoint{0.566288in}{3.334591in}}%
\pgfpathlineto{\pgfqpoint{0.566288in}{2.274642in}}%
\pgfpathquadraticcurveto{\pgfqpoint{0.566288in}{2.258559in}}{\pgfqpoint{0.582372in}{2.258559in}}%
\pgfpathlineto{\pgfqpoint{0.582372in}{2.258559in}}%
\pgfpathclose%
\pgfusepath{stroke,fill}%
\end{pgfscope}%
\begin{pgfscope}%
\pgfsetbuttcap%
\pgfsetmiterjoin%
\definecolor{currentfill}{rgb}{0.993725,0.850196,0.704314}%
\pgfsetfillcolor{currentfill}%
\pgfsetlinewidth{0.000000pt}%
\definecolor{currentstroke}{rgb}{0.000000,0.000000,0.000000}%
\pgfsetstrokecolor{currentstroke}%
\pgfsetstrokeopacity{0.000000}%
\pgfsetdash{}{0pt}%
\pgfpathmoveto{\pgfqpoint{0.598455in}{3.257410in}}%
\pgfpathlineto{\pgfqpoint{0.759288in}{3.257410in}}%
\pgfpathlineto{\pgfqpoint{0.759288in}{3.313702in}}%
\pgfpathlineto{\pgfqpoint{0.598455in}{3.313702in}}%
\pgfpathlineto{\pgfqpoint{0.598455in}{3.257410in}}%
\pgfpathclose%
\pgfusepath{fill}%
\end{pgfscope}%
\begin{pgfscope}%
\definecolor{textcolor}{rgb}{0.000000,0.000000,0.000000}%
\pgfsetstrokecolor{textcolor}%
\pgfsetfillcolor{textcolor}%
\pgftext[x=0.823622in,y=3.257410in,left,base]{\color{textcolor}{\rmfamily\fontsize{5.790000}{6.948000}\selectfont\catcode`\^=\active\def^{\ifmmode\sp\else\^{}\fi}\catcode`\%=\active\def%{\%}photonQueryGeneration}}%
\end{pgfscope}%
\begin{pgfscope}%
\pgfsetbuttcap%
\pgfsetmiterjoin%
\definecolor{currentfill}{rgb}{0.992157,0.710065,0.464437}%
\pgfsetfillcolor{currentfill}%
\pgfsetlinewidth{0.000000pt}%
\definecolor{currentstroke}{rgb}{0.000000,0.000000,0.000000}%
\pgfsetstrokecolor{currentstroke}%
\pgfsetstrokeopacity{0.000000}%
\pgfsetdash{}{0pt}%
\pgfpathmoveto{\pgfqpoint{0.598455in}{3.138238in}}%
\pgfpathlineto{\pgfqpoint{0.759288in}{3.138238in}}%
\pgfpathlineto{\pgfqpoint{0.759288in}{3.194530in}}%
\pgfpathlineto{\pgfqpoint{0.598455in}{3.194530in}}%
\pgfpathlineto{\pgfqpoint{0.598455in}{3.138238in}}%
\pgfpathclose%
\pgfusepath{fill}%
\end{pgfscope}%
\begin{pgfscope}%
\definecolor{textcolor}{rgb}{0.000000,0.000000,0.000000}%
\pgfsetstrokecolor{textcolor}%
\pgfsetfillcolor{textcolor}%
\pgftext[x=0.823622in,y=3.138238in,left,base]{\color{textcolor}{\rmfamily\fontsize{5.790000}{6.948000}\selectfont\catcode`\^=\active\def^{\ifmmode\sp\else\^{}\fi}\catcode`\%=\active\def%{\%}photonQueryMapBuildTime}}%
\end{pgfscope}%
\begin{pgfscope}%
\pgfsetbuttcap%
\pgfsetmiterjoin%
\definecolor{currentfill}{rgb}{0.991419,0.550727,0.232772}%
\pgfsetfillcolor{currentfill}%
\pgfsetlinewidth{0.000000pt}%
\definecolor{currentstroke}{rgb}{0.000000,0.000000,0.000000}%
\pgfsetstrokecolor{currentstroke}%
\pgfsetstrokeopacity{0.000000}%
\pgfsetdash{}{0pt}%
\pgfpathmoveto{\pgfqpoint{0.598455in}{3.019067in}}%
\pgfpathlineto{\pgfqpoint{0.759288in}{3.019067in}}%
\pgfpathlineto{\pgfqpoint{0.759288in}{3.075358in}}%
\pgfpathlineto{\pgfqpoint{0.598455in}{3.075358in}}%
\pgfpathlineto{\pgfqpoint{0.598455in}{3.019067in}}%
\pgfpathclose%
\pgfusepath{fill}%
\end{pgfscope}%
\begin{pgfscope}%
\definecolor{textcolor}{rgb}{0.000000,0.000000,0.000000}%
\pgfsetstrokecolor{textcolor}%
\pgfsetfillcolor{textcolor}%
\pgftext[x=0.823622in,y=3.019067in,left,base]{\color{textcolor}{\rmfamily\fontsize{5.790000}{6.948000}\selectfont\catcode`\^=\active\def^{\ifmmode\sp\else\^{}\fi}\catcode`\%=\active\def%{\%}photonGeneration}}%
\end{pgfscope}%
\begin{pgfscope}%
\pgfsetbuttcap%
\pgfsetmiterjoin%
\definecolor{currentfill}{rgb}{0.925536,0.384867,0.059839}%
\pgfsetfillcolor{currentfill}%
\pgfsetlinewidth{0.000000pt}%
\definecolor{currentstroke}{rgb}{0.000000,0.000000,0.000000}%
\pgfsetstrokecolor{currentstroke}%
\pgfsetstrokeopacity{0.000000}%
\pgfsetdash{}{0pt}%
\pgfpathmoveto{\pgfqpoint{0.598455in}{2.901034in}}%
\pgfpathlineto{\pgfqpoint{0.759288in}{2.901034in}}%
\pgfpathlineto{\pgfqpoint{0.759288in}{2.957325in}}%
\pgfpathlineto{\pgfqpoint{0.598455in}{2.957325in}}%
\pgfpathlineto{\pgfqpoint{0.598455in}{2.901034in}}%
\pgfpathclose%
\pgfusepath{fill}%
\end{pgfscope}%
\begin{pgfscope}%
\definecolor{textcolor}{rgb}{0.000000,0.000000,0.000000}%
\pgfsetstrokecolor{textcolor}%
\pgfsetfillcolor{textcolor}%
\pgftext[x=0.823622in,y=2.901034in,left,base]{\color{textcolor}{\rmfamily\fontsize{5.790000}{6.948000}\selectfont\catcode`\^=\active\def^{\ifmmode\sp\else\^{}\fi}\catcode`\%=\active\def%{\%}photonPostprocessing}}%
\end{pgfscope}%
\begin{pgfscope}%
\pgfsetbuttcap%
\pgfsetmiterjoin%
\definecolor{currentfill}{rgb}{0.887059,0.887059,0.887059}%
\pgfsetfillcolor{currentfill}%
\pgfsetlinewidth{0.000000pt}%
\definecolor{currentstroke}{rgb}{0.000000,0.000000,0.000000}%
\pgfsetstrokecolor{currentstroke}%
\pgfsetstrokeopacity{0.000000}%
\pgfsetdash{}{0pt}%
\pgfpathmoveto{\pgfqpoint{0.598455in}{2.781862in}}%
\pgfpathlineto{\pgfqpoint{0.759288in}{2.781862in}}%
\pgfpathlineto{\pgfqpoint{0.759288in}{2.838154in}}%
\pgfpathlineto{\pgfqpoint{0.598455in}{2.838154in}}%
\pgfpathlineto{\pgfqpoint{0.598455in}{2.781862in}}%
\pgfpathclose%
\pgfusepath{fill}%
\end{pgfscope}%
\begin{pgfscope}%
\definecolor{textcolor}{rgb}{0.000000,0.000000,0.000000}%
\pgfsetstrokecolor{textcolor}%
\pgfsetfillcolor{textcolor}%
\pgftext[x=0.823622in,y=2.781862in,left,base]{\color{textcolor}{\rmfamily\fontsize{5.790000}{6.948000}\selectfont\catcode`\^=\active\def^{\ifmmode\sp\else\^{}\fi}\catcode`\%=\active\def%{\%}training}}%
\end{pgfscope}%
\begin{pgfscope}%
\pgfsetbuttcap%
\pgfsetmiterjoin%
\definecolor{currentfill}{rgb}{0.710588,0.710588,0.710588}%
\pgfsetfillcolor{currentfill}%
\pgfsetlinewidth{0.000000pt}%
\definecolor{currentstroke}{rgb}{0.000000,0.000000,0.000000}%
\pgfsetstrokecolor{currentstroke}%
\pgfsetstrokeopacity{0.000000}%
\pgfsetdash{}{0pt}%
\pgfpathmoveto{\pgfqpoint{0.598455in}{2.662690in}}%
\pgfpathlineto{\pgfqpoint{0.759288in}{2.662690in}}%
\pgfpathlineto{\pgfqpoint{0.759288in}{2.718982in}}%
\pgfpathlineto{\pgfqpoint{0.598455in}{2.718982in}}%
\pgfpathlineto{\pgfqpoint{0.598455in}{2.662690in}}%
\pgfpathclose%
\pgfusepath{fill}%
\end{pgfscope}%
\begin{pgfscope}%
\definecolor{textcolor}{rgb}{0.000000,0.000000,0.000000}%
\pgfsetstrokecolor{textcolor}%
\pgfsetfillcolor{textcolor}%
\pgftext[x=0.823622in,y=2.662690in,left,base]{\color{textcolor}{\rmfamily\fontsize{5.790000}{6.948000}\selectfont\catcode`\^=\active\def^{\ifmmode\sp\else\^{}\fi}\catcode`\%=\active\def%{\%}pathtracing}}%
\end{pgfscope}%
\begin{pgfscope}%
\pgfsetbuttcap%
\pgfsetmiterjoin%
\definecolor{currentfill}{rgb}{0.478431,0.478431,0.478431}%
\pgfsetfillcolor{currentfill}%
\pgfsetlinewidth{0.000000pt}%
\definecolor{currentstroke}{rgb}{0.000000,0.000000,0.000000}%
\pgfsetstrokecolor{currentstroke}%
\pgfsetstrokeopacity{0.000000}%
\pgfsetdash{}{0pt}%
\pgfpathmoveto{\pgfqpoint{0.598455in}{2.543519in}}%
\pgfpathlineto{\pgfqpoint{0.759288in}{2.543519in}}%
\pgfpathlineto{\pgfqpoint{0.759288in}{2.599810in}}%
\pgfpathlineto{\pgfqpoint{0.598455in}{2.599810in}}%
\pgfpathlineto{\pgfqpoint{0.598455in}{2.543519in}}%
\pgfpathclose%
\pgfusepath{fill}%
\end{pgfscope}%
\begin{pgfscope}%
\definecolor{textcolor}{rgb}{0.000000,0.000000,0.000000}%
\pgfsetstrokecolor{textcolor}%
\pgfsetfillcolor{textcolor}%
\pgftext[x=0.823622in,y=2.543519in,left,base]{\color{textcolor}{\rmfamily\fontsize{5.790000}{6.948000}\selectfont\catcode`\^=\active\def^{\ifmmode\sp\else\^{}\fi}\catcode`\%=\active\def%{\%}inference}}%
\end{pgfscope}%
\begin{pgfscope}%
\pgfsetbuttcap%
\pgfsetmiterjoin%
\definecolor{currentfill}{rgb}{1.000000,0.752941,0.796078}%
\pgfsetfillcolor{currentfill}%
\pgfsetlinewidth{0.000000pt}%
\definecolor{currentstroke}{rgb}{0.000000,0.000000,0.000000}%
\pgfsetstrokecolor{currentstroke}%
\pgfsetstrokeopacity{0.000000}%
\pgfsetdash{}{0pt}%
\pgfpathmoveto{\pgfqpoint{0.598455in}{2.425486in}}%
\pgfpathlineto{\pgfqpoint{0.759288in}{2.425486in}}%
\pgfpathlineto{\pgfqpoint{0.759288in}{2.481777in}}%
\pgfpathlineto{\pgfqpoint{0.598455in}{2.481777in}}%
\pgfpathlineto{\pgfqpoint{0.598455in}{2.425486in}}%
\pgfpathclose%
\pgfusepath{fill}%
\end{pgfscope}%
\begin{pgfscope}%
\definecolor{textcolor}{rgb}{0.000000,0.000000,0.000000}%
\pgfsetstrokecolor{textcolor}%
\pgfsetfillcolor{textcolor}%
\pgftext[x=0.823622in,y=2.425486in,left,base]{\color{textcolor}{\rmfamily\fontsize{5.790000}{6.948000}\selectfont\catcode`\^=\active\def^{\ifmmode\sp\else\^{}\fi}\catcode`\%=\active\def%{\%}visualization}}%
\end{pgfscope}%
\begin{pgfscope}%
\pgfsetbuttcap%
\pgfsetmiterjoin%
\definecolor{currentfill}{rgb}{0.854902,0.439216,0.839216}%
\pgfsetfillcolor{currentfill}%
\pgfsetlinewidth{0.000000pt}%
\definecolor{currentstroke}{rgb}{0.000000,0.000000,0.000000}%
\pgfsetstrokecolor{currentstroke}%
\pgfsetstrokeopacity{0.000000}%
\pgfsetdash{}{0pt}%
\pgfpathmoveto{\pgfqpoint{0.598455in}{2.307453in}}%
\pgfpathlineto{\pgfqpoint{0.759288in}{2.307453in}}%
\pgfpathlineto{\pgfqpoint{0.759288in}{2.363744in}}%
\pgfpathlineto{\pgfqpoint{0.598455in}{2.363744in}}%
\pgfpathlineto{\pgfqpoint{0.598455in}{2.307453in}}%
\pgfpathclose%
\pgfusepath{fill}%
\end{pgfscope}%
\begin{pgfscope}%
\definecolor{textcolor}{rgb}{0.000000,0.000000,0.000000}%
\pgfsetstrokecolor{textcolor}%
\pgfsetfillcolor{textcolor}%
\pgftext[x=0.823622in,y=2.307453in,left,base]{\color{textcolor}{\rmfamily\fontsize{5.790000}{6.948000}\selectfont\catcode`\^=\active\def^{\ifmmode\sp\else\^{}\fi}\catcode`\%=\active\def%{\%}other}}%
\end{pgfscope}%
\end{pgfpicture}%
\makeatother%
\endgroup%

    \caption{Performance vs resolution of SPPC. For reference, the performance of PT (black line) and NRC+PT (purple line) is also shown. The resolution sequence is linear in total pixel count, going up to 1080p-equivalent $1440^2$px. Measurements are averaged over 10s. Note, that PT has a near zero constant overhead, but a significantly higher per-pixel cost, whereas both Radiance Caching techniques have a much lower per-pixel cost but higher overhead.}
    \label{fig:perres}
\end{figure}